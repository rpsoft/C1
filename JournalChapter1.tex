\documentclass[prodmode,acmtecs]{acmsmall} % Aptara syntax

% Package to generate and customize Algorithm as per ACM style
\usepackage[ruled]{algorithm2e}
\renewcommand{\algorithmcfname}{ALGORITHM}
\SetAlFnt{\small}
\SetAlCapFnt{\small}
\SetAlCapNameFnt{\small}
\SetAlCapHSkip{0pt}
\IncMargin{-\parindent}
\usepackage{bibentry}
\usepackage{graphicx}
\usepackage{caption}
\usepackage{makeidx}  % allows for indexgeneration
\let\proof\relax
\let\endproof\relax
\usepackage{amsmath, amsthm, amssymb} 
\usepackage{multirow}
\usepackage{wrapfig}
\usepackage{url}
\usepackage{underscore} 
\usepackage{longtable}
\usepackage{float}
\usepackage{enumitem}
\usepackage{bibentry}
\usepackage{tikz}
\usepackage{pgf}
\usetikzlibrary{arrows,automata,patterns,shapes,arrows}
\usepackage[latin1]{inputenc}
\usepackage{verbatim}
\usepackage{pgfplots}
\usepackage{subcaption}
\pgfplotsset{compat=newest}
\usepgfplotslibrary{ternary}
\usepackage{{pgfplotstable}}
\newcommand\abs[1]{\left|#1\right|}
\newcommand\mentalnote[1]{[MentalNote:----\textcolor{blue}{#1}----]}
\captionsetup{compatibility=false}

\newcommand{\todo}[1]{\textcolor{red}{#1}}

\newcommand{\argmax}{\operatornamewithlimits{arg\,max}}

\newcommand*\circled[1]{\tikz[baseline=(char.base)]{\node[shape=rectangle,draw,inner sep=1.5pt] (char) {#1};}}


% Metadata Information
\acmVolume{9}
\acmNumber{4}
\acmArticle{39}
\acmYear{2010}
\acmMonth{3}

% Document starts
\begin{document}

% Page heads
\markboth{J. Rodriguez Perez et al.}{Microblogs Structure. Challenges and Opportunities}

% Title portion
\title{Microblogs Structure. Challenges and Opportunities}
\author{Jesus Alberto Rodriguez Perez
\affil{The University of Glasgow}
Teerapong Leelanupab
\affil{King Mongkut's Institute of Technology Ladkrabang}
Joemon M. Jose
\affil{The University of Glasgow}
}
% NOTE! Affiliations placed here should be for the institution where the
%       BULK of the research was done. If the author has gone to a new
%       institution, before publication, the (above) affiliation should NOT be changed.
%       The authors 'current' address may be given in the "Author's addresses:" block (below).
%       So for example, Mr. Abdelzaher, the bulk of the research was done at UIUC, and he is
%       currently affiliated with NASA.

\begin{abstract}
In recent years, microblog services such as Twitter have gained increasing popularity, leading to active research on how to effectively exploit its content. Microblog documents such as `tweets' differ in morphology to more traditional documents such as web pages. Particularly, tweets are considerably shorter (140 characters) than web documents and contain contextual tags regarding the topic (hashtags), intended audience (mentions) as well as links to external content (URLs). Unfortunately, state of the art retrieval models perform rather poorly in capturing the relevance of microblogs, due to the previously unforeseen conditions in which they operate.

In this work, our main focus it to investigate the shortcomings that state of the art retrieval models suffer when dealing microblogs. To this end we simulate the behaviour of a number of state of the art retrieval models, in a microblog retrieval context. As a result of our experiments, we conclude that longer documents should be promoted over shorter documents. This is due to authors striving to fit as many terms as possible regardless of character limitations, thus producing more informative documents. On the other hand, documents with higher term frequency are deemed less valuable as they are more likely to resemble spam. Therefore, we also demonstrate that the scope hypotheses does hold for microblog documents, whereas the verbosity hypotheses does not.

Finally, based on our findings we devised a retrieval model, namely \textbf{MBRM}, which significantly outperforms the state of the retrieval models, by better capturing the informativeness of microblog documents.

%Our evaluation results show statistically significant improvements over the baseline in terms of precision at different cut-off points for both approaches. These results confirm that the relative presence of the different dimensions within a document and their ordering are connected with the relevance of microblogs.

%Furthermore we look at microblog documents as a high-dimensional entity and study the structural differences between those documents which are deemed relevant against those non-relevant. Moreover we leverage these statistical differences in experiments to enhance the behaviour of retrieval models. Additionally we study the interactions between the different dimensions in terms of their order within the documents by modelling relevant and non-relevant tweets as state machines. These state machines are then utilised to produce scores which in turn are used for re-ranking. 

\end{abstract}

\category{C.2.2}{Computer-Communication Networks}{Network Protocols}

\terms{Microblog, State machines, Classification}

\keywords{Information Retrieval, Structural Models, Document Dimentions}

\acmformat{Jesus Rodriguez et al., 2017. Microblogs Structure. Challenges and Opportunities.}
% At a minimum you need to supply the author names, year and a title.
% IMPORTANT:
% Full first names whenever they are known, surname last, followed by a period.
% In the case of two authors, 'and' is placed between them.
% In the case of three or more authors, the serial comma is used, that is, all author names
% except the last one but including the penultimate author's name are followed by a comma,
% and then 'and' is placed before the final author's name.
% If only first and middle initials are known, then each initial
% is followed by a period and they are separated by a space.
% The remaining information (journal title, volume, article number, date, etc.) is 'auto-generated'.

\begin{bottomstuff}
%This work is supported by the National Science Foundation, under
%grant CNS-0435060, grant CCR-0325197 and grant EN-CS-0329609.
%
%Author's addresses: G. Zhou, Computer Science Department,
%College of William and Mary; Y. Wu  {and} J. A. Stankovic,
%Computer Science Department, University of Virginia; T. Yan,
%Eaton Innovation Center; T. He, Computer Science Department,
%University of Minnesota; C. Huang, Google; T. F. Abdelzaher,
%(Current address) NASA Ames Research Center, Moffett Field, California 94035.
\end{bottomstuff}

\maketitle

%!TEX root = JournalChapter1.tex
\section{Introduction}
\label{introduction}

Microblogs have grown in popularity in recent years, gradually transforming the way we find out about the latest events and communicate. Twitter is the most prominent service \footnote{\url{https://twitter.com/}}, as it is used by millions, posting over 500 million tweets every day\footnote{\url{https://www.dsayce.com/social-media/tweets-day/}}. Microblog services are used for various purposes including: (i) self promotion, (ii) advertising, (iii) real-time news broadcasting, (iv) social discussions etc. It provides unique insight into real-time events, such as first hand reports of events as they are developing, along with the opinion of those discussing them. This information makes Twitter a uniquely valuable media source, and hence obtaining the attention from the research and industrial communities. 

Searching through such resources is performed by millions on a daily basis, however retrieving documents in Twitter can be extremely challenging because of their morphology. The content is limited to 140 characters per messages (known as \emph{Tweets}), which in fact leads to varied linguistic quality \cite{teevan2011twittersearch} due to colloquialisms and users efforts to fit their content within the limitations. Due to these properties tweets pose new challenges for which the state of the art retrieval models were not designed for\footnote{Models such as: Okapi BM25 \cite{robertson2009probabilistic}; Divergence From Randomness (DFR) \cite{amati2003probabilistic}; Hiemstra's Language Model (HLM) \cite{model}; and Dirichlet Language Model (DLM) \cite{zhai2001study}}. 

Whilst few recent works have identified some features as being detrimental in microblog ad-hoc retrieval \cite{naveed2011searching}, no study has been carried out to determine the concrete effect of the different features on the state of the art retrieval models. Therefore we are set to investigate the connection of the structure of microblog documents with their relevance in an ad-hoc search task. This whole work revolves around the following main question: 

\begin{quotation}\begin{quote}\textbf{What are the reasons behind the underperformance of the state of the art retrieval models in the context of microblogs?}\end{quote}\end{quotation}

To this end, firstly we observe the performance of the state of the art retrieval models in the context of Twitter corpora selecting the best retrieval model as a baseline. Then we perform a series of experiments which simulate the behaviour of a number of state of the art retrieval models in order to identify possible shortcomings in their design with respect to microblog documents. This initial experiment is completed with the creation of a retrieval model, which takes into account all previous findings, namely MBRM. MBRM demonstrates that the scope hypotheses still holds within microblog documents, and that microblog document statistics can be leveraged to significantly improve ad-hoc retrieval performance. This work will be driven by the following research questions:

\begin{itemize}
\item[] \textbf{RQ1.} What is the role of document length in connection with the informativeness of microblog documents?
\item[] \textbf{RQ2.} Does term frequency of query terms relate to the informativeness of microblogs? 
\item[] \textbf{RQ3.} Can we adapt state of the art retrieval models to better handle microblogs?
\item[] \textbf{RQ4.} Can we devise a retrieval model to better capture the relevance of microblogs?
\end{itemize}

The rest of the work is organised as follows. First, we cover literature relevant to microblog retrieval and other concepts utilised throughout this work (Section \ref{background}). Section \ref{experiment} sets the evaluation environment in which our experimentation is carried out, giving way to our analysis (Section \ref{RMinvestigation}). Section \ref{MBRM-section} then introduces our MBRM retrieval model, whereas Section \ref{conclusion} concludes the work and points to future research directions.



%!TEX root = JournalChapter1.tex
\section{Background}
\label{background}
In this Section we will introduce concepts and related literature to this work.

\subsection{Retrieval Models}
The first part of this work revolves around retrieval models and how their design affects their performance when retrieving microblogs. In our experimentation we include retrieval models such as: Okapi BM25 \cite{robertson2009probabilistic}; Divergence From Randomness (DFR) \cite{amati2003probabilistic}; Hiemstra's Language Model (HLM) \cite{model}; and Dirichlet Language Model (DLM) \cite{zhai2001study}. These models are introduced in more details in Section \ref{RMinvestigation}, and their behaviour described individually against microblog conditions. However we first introduce some basic background to ease the understanding of the following sections.

\subsubsection{Probability of Relevance Framework} For many years researchers have developed their understanding on estimating the relevance of documents, thus leading to many models and definitions of relevance. One of the most representative works in this area of research is the Probability of Relevance Framework (PRF) \cite{roelleke2013information}. PRF is formulated by \(P(r|\hat{d},q)\), where \(r\) refers to relevance, \(q\) a given query and \(\hat{d}\) represents a document as a vector of features \(\hat{d} = (f_1,...f_n)\). Note that vector features can be any imaginable data. The main importance of this framework is the formalisation of relevance as a function of a given query and document vectors. This can be utilised as a framework for any probabilistic retrieval model, thus becoming the basis of numerous research works.

\subsubsection{Document length normalization} \cite{singhal1996pivoted} has been employed by retrieval models to counterbalance the effects of longer documents, which may not necessarily add any new information to a topic, but are prone to contain higher term frequencies. In line with this effort, the design of BM25 by \cite{robertson2009probabilistic} involved the study of document characteristics, resulting in the definition of the \textbf{scope} and \textbf{verbosity} hypotheses. The \textbf{verbosity} hypotheses supports that some authors are more verbose than others, thus applying length normalization by dividing by the length of the document is beneficial to better capture relevance, as repetition of terms is superfluous. On the other hand, the \textbf{scope} hypotheses states that some authors simply have more to say, thus adding more relevant information to the topic and occupying more space. BM25 applies a soft normalisation that takes into account both cases.

\subsection{Retrieval of Microblogs is Hard}
Retrieval models are designed to rely on term frequency and document length as the variables to quantify whether a document is more important than other. From a very simplified perspective, a retrieval model will give more importance to a document that contains query terms more frequently than another document (Assuming similar document lengths). Likewise, when query terms appear the same number of times, a document will be deemed less or more informative based on the document lengths \cite{manning2008introduction}. However, microblog documents are limited in length to 140 characters in the case of Twitter. This limitation obviously challenges the abovementioned assumptions, which unfortunately form the basis of the workings of most retrieval models in a way or another. The new medium and the low retrieval performance achieved by state of the art retrieval models gave way to an extensive area of research spearheaded by the Text Retrieval Conference (TREC) through its microblog track. Over recent years, numerous approaches have been proposed which significantly improve retrieval performance in diverse ways.

\subsection{TREC Microblog Retrieval Tracks}
TREC organised a number of tracks over four consecutive years 2011-2014 in order to organise the research community and jointly address this retrieval problem. To evaluate the performance of the prospective solutions and allow for comparability they agreed on a collection of documents and a set of topics, as well as relevance judgements on those topics. To this end they sampled two collections of documents from a Twitter stream over two different periods of time. The first collection was gathered in 2011 but was used for during both the 2011 and 2012 microblog tracks. Similarly, the second collection was gathered in 2013 and was used for both the 2013 and 2014 microblog tracks. Finally the number of topics varied between 50 and 60, but are 225 in total across all years.

The summary results for each of the tracks are presented in Table \ref{summarytrec} for reference. Amongst the top performing participants we can find ~\cite{amati2011fub,li2011pris,metzler2011usc} for microblog 2011 and ~\cite{kimovercoming,younosFreq,hanhit} for 2012, which mostly employed query and document expansion techniques as well as learning to rank (L2R) approaches.
The 2013 track followed a similar trend producing works in the same categories L2R \cite{pris2013,gaoictnet}, query expansion \cite{prebjut,perezuniversity} and document expansion \cite{jabeuririt}. The work by \cite{Damak:2013:ESF:2480362.2480537} produced a comprehensive summary of the features used by different approaches, and demonstrated how to successfully combine them using naive bayes as an L2R approach combining a number of features including hashtags, mentions, url presence, recency, etc.

\begin{table}[]
	\centering
	
	\caption{TREC Tracks results in terms of precision@30}
	
	\begin{tabular}{|c|c|c|c|c|c|c|c|}
		\hline
		\multicolumn{2}{|c|}{2011} &\multicolumn{2}{|c|}{2012} & \multicolumn{2}{|c|}{2013} & \multicolumn{2}{|c|}{2014} \\
		\hline
		Best & Median & Best & Median &	Best & Median &	Best & Median \\
		0.502 & 0.298 & 0.470 & 0.362 & 0.560 & 0.370 & 0.722 & 0.629 \\
		\hline
		
	\end{tabular}
	
	\label{summarytrec}
	\vspace{0.5cm}
\end{table}

%\noindent {\bf Microblog Retrieval Issues.} 
Work by \cite{thomassearching} studied the effects that preprocessing had on retrieval performance. Their findings showed that the best performance was achieved when applying all preprocessing steps, which include (i) language detection, (ii) Emotion removal, (iii) Lexical normalization, (iv) Mention Removal and (v) Link Removal. Additionally, works by \cite{ferguson2012investigation,naveed2011searching} have identified that problems affecting retrieval models in microblogs are related to \textit{term frequency} and \textit{document length normalization}.

%Moreover, work by \cite{ferguson2012investigation} studied the effects of query length normalization and term frequency on the BM25 retrieval model whereas \cite{naveed2011searching} looked at these issues from a more generic perspective. Both their findings showed how query length normalization had an undesirable effect on retrieval performance of the systems they were evaluating, since document length in microblogs does not have the same meaning as in other context.

%, which would answer why the results we obtained for BM25 runs are consistently worse compared to IDF and DFR.

%The literature has identified two main culprits which negatively affect performance of modern retrieval models in Microblogs, that is: \textbf{term frequency} and \textbf{document length normalization} . As we can observe in Table \ref{modelfeatures} BM25 has a heavy realiance on document length as it uses both ADL and DL components for its computation. The study by  \\

%\noindent {\bf Other retrieval features.} The use of temporal evidences in conjunction with other features such as geographical locations has been studied by \cite{lappas2012spatiotemporal} and \cite{6222198} following different approaches. In addition to the temporal dimension, 

\subsection{Making Sense Of Microposts}
The MSM workshop \cite{basave2013making} presented participants with a challenge. The objective was to build systems able to identify and extract concepts from microblog documents, in a semi-supervised manner. The participant systems were to categorise concepts as belonging to the categories: person, organisation, location and miscellaneous. A similar task is that of microblog summarisation \cite{5590862} in that tweets have to be processed and made sense of in order to produce a richer representation. Amongst the works submitted to this workshop, we can highlight the work by Tao, Ke et al. \cite{tao2012makes}. In their work, they perform an in depth analysis of both topic dependent and independent features for the MSM task. Some of the topic independent features consider the presence of hashtags, urls and the length of the documents to be in connection with the relevance of documents. In our work, we pay attention to the same features, but from a different angle, by looking how much space relative to the total characters in the document is dedicated to each of the microblogs elements.

\subsection{Other Microblog retrieval features} 
Work by \cite{massoudi2011incorporating} explored the use of other features to improve ad-hoc retrieval. These features include emoticons, hyperlinks, shouting, capitalization, retweets and followers. 
Work by \cite{nagmoti2010ranking} extended the study concerning the use of social features such as the number of followers and followees to enhance ad-hoc retrieval performance.
While all these works attempt to exploit some microblog features or augment them with external resources, they do not try to explain how these features relate to the relevance of microblog documents. In our work, we consider features based purely on microblog characteristics, explain their relationship with relevance, and finally use those features that seem beneficial to improve the behaviour of a state of the art retrieval model.

\subsection{Understanding Microblogs}
We believe that no significant progress has been made to understand \textit{why are retrieval models failing} in microblogs. Due to their limited size,  document length and term frequencies are often loosely blamed with the underperformance of retrieval models. We believe it is important to explore, and properly assess the interaction of such features. Better understanding could lead to better performance of retrieval models, or new models altogether, which are the starting point for many techniques commonly used in microblog retrieval (E.g. Automatic Query Expansion).


%!TEX root = JournalChapter1.tex
\section{Experimental Setting}
\label{experiment}

\noindent {\bf Datasets.} In this evaluation we have used the four collections (2011-2014) from the TREC Microblog track. The 2011 and 2012 collections share the same corpus but have different topics and relevance assessments. On the other hand the 2013 and 2014 collections share the same corpus. 
%The later corpus is an order of magnitude bigger than previous collections. However, the 2013 and 2014 relevance assessments are statiscally comparable to the 2012 track. Moreover, the ratio of documents \(\frac{relevant}{non-relevant}\) is much higher for the 2013, which can result in generally better retrieval performance than previous tracks by default. The 2014 on the other hand is closer in this ratio to the 2012 collection. In fact it has a considerably lower number of relevant documents per topic.
In total there are 225 topics with query lengths ranging from 2 to 3 tokens, in line with the literature~\cite{teevan2011twittersearch}. Please refer to Table \ref{collections} for an extended overview of these collections. \\


%\noindent{\bf Parameter setting.} We performed parameter optimization w.r.t Precision@30 for all the retrieval models with free parameters. In Table \ref{traditional} we show the results for all considered retrieval models using the most optimal parameters across the three collections.\\


\begin{table}[]
\caption{Descriptive statistics for the collections being used in this study}
  	\centering
   \begin{tabular}{l||c|c|c|c} 
    \hline
	TREC Microblog track collection year & 2011 & 2012 & 2013 & 2014 \tabularnewline
	\hline
	Number of topics   & 50 & 60 & 60 & 55 \tabularnewline
    \hline
	\# documents & \multicolumn{2}{c|}{16M} & \multicolumn{2}{c}{260M}  \tabularnewline
    \hline
	\# assessed documents   & 40855 &  73073  & 71279 & 57985 \tabularnewline
    \hline
	\# assessed non-relevant documents & 38124 & 66893 & 62268 & 47340 \tabularnewline
	\hline
	\# assessed relevant documents  & 2731 &  6180 & 9011 & 4753\tabularnewline
	\hline
	Ratio \(\frac{Relevant~Docs}{Non-Relevant~Docs}\) & 0.07 &  0.09 & 0.14 & 0.10\tabularnewline
	\hline
	Avg. relevant documents per topic   & 58.45 &  106.54 & 150.18 & 79.22 \tabularnewline
	\hline
   \end{tabular}
   \label{collections}
   \vspace{0.5cm}
\end{table}

%!TEX root = JournalChapter1.tex
\begin{table*}[]
\caption{Evaluation results for the state of the art models considered. (Bold denotes the best performing system)}
\begin{small}


	\begin{subtable}[b]{0.5\textwidth}

	\caption{2011 collection} 

	\begin{tabular}{l|c|c|c|c|c} 	
	\cline{2- 6}
	\multicolumn{1}{c}{}&\multicolumn{5}{c}{Precision} \\ 
	\cline{2- 6} &
	\textit{\textbf{@5}} & 
	\textit{\textbf{@10}} & 
	\textit{\textbf{@15}} & 
	\textit{\textbf{@20}} & 
	\textit{\textbf{@30}} 	
	\tabularnewline
	\hline
	BM25 & 0.54 & 0.48 & 0.45 & 0.41 & 0.38\\
	DFRee & 0.61 & \textbf{0.58} & \textbf{0.54} & \textbf{0.50} & 0.45\\
	DLM & 0.50 & 0.47 & 0.45 & 0.42 & 0.37\\
	HLM & 0.54 & 0.48 & 0.45 & 0.42 & 0.38\\
	IDF & \textbf{0.63} & 0.56 & 0.52 & 0.49 & \textbf{0.46}\\
	\hline	
	\end{tabular}
	 \end{subtable}
	 \hspace{2em}
 	\begin{subtable}[b]{0.5\textwidth}
 	\caption{2012 Collection} 
	\begin{tabular}{l|c|c|c|c|c} 
	
	\cline{2- 6}
	\multicolumn{1}{c}{}&\multicolumn{5}{c}{Precision} \\ 
	\cline{2- 6} &
	\textit{\textbf{@5}} & 
	\textit{\textbf{@10}} & 
	\textit{\textbf{@15}} & 
	\textit{\textbf{@20}} & 
	\textit{\textbf{@30}} 
	
	\tabularnewline
	\hline
BM25 & 0.40 & 0.37 & 0.34 & 0.34 & 0.31 \\
DFRee & \textbf{0.46} & \textbf{0.45} & \textbf{0.42} & \textbf{0.39} & \textbf{0.36}\\
DLM & 0.34 & 0.33 & 0.32 & 0.29 & 0.27\\
HLM & 0.38 & 0.37 & 0.35 & 0.33 & 0.31\\
IDF & 0.44 & 0.39 & 0.36 & 0.36 & 0.34\\
 	\hline
 	
 	\end{tabular}
 
 	 \end{subtable} \vspace{0.6cm}

 	\begin{subtable}[b]{0.5\textwidth}
 	\caption{2013 collection} 

	\begin{tabular}{l|c|c|c|c|c} 
	
	\cline{2- 6}
	\multicolumn{1}{c}{}&\multicolumn{5}{c}{Precision} \\ 
	\cline{2- 6} &
	\textit{\textbf{@5}} & 
	\textit{\textbf{@10}} & 
	\textit{\textbf{@15}} & 
	\textit{\textbf{@20}} & 
	\textit{\textbf{@30}} 
	
	\tabularnewline
	\hline
BM25 & 0.58 & 0.51 & 0.46 & 0.42 & 0.38 \\
DFRee & \textbf{0.67} & 0.60 & 0.55 & 0.51 & \textbf{0.45} \\
DLM & 0.27 & 0.28 & 0.26 & 0.26 & 0.24 \\
HLM & 0.44 & 0.38 & 0.35 & 0.33 & 0.31 \\
IDF & 0.66 & \textbf{0.62} & \textbf{0.56} & \textbf{0.52} & \textbf{0.45} \\

 	\hline
 	\end{tabular}
 
 	\end{subtable}  
 		 \hspace{2em}
 	  	\begin{subtable}[b]{0.5\textwidth}
 	 
 	  	\centering
 	  	\caption{2014 collection} 
 	 
 	 	\begin{tabular}{l|c|c|c|c|c} 
 	 	
 	 	\cline{2- 6}
 	 	\multicolumn{1}{c}{}&\multicolumn{5}{c}{Precision} \\ 
 	 	\cline{2- 6} &
 	 	\textit{\textbf{@5}} & 
 	 	\textit{\textbf{@10}} & 
 	 	\textit{\textbf{@15}} & 
 	 	\textit{\textbf{@20}} & 
 	 	\textit{\textbf{@30}} 
 	 	
 	 	\tabularnewline
 	 	\hline
	 	 BM25 & 0.69 & 0.62 & 0.58 & 0.57 & 0.52 \\
	 	 DFRee & 0.73 & 0.68 & 0.65 & 0.63 & 0.60 \\
	 	 DLM & 0.35 & 0.35 & 0.34 & 0.34 & 0.33 \\
	 	 HLM & 0.55 & 0.49 & 0.46 & 0.44 & 0.41 \\
	 	 IDF & \textbf{0.75} & \textbf{0.73} & \textbf{0.69} & \textbf{0.67} & \textbf{0.62} \\
 	 
 	  	\hline
 	  	\end{tabular}
 	  	
 	  	%\end{center}
 	  	%\end{footnotesize}
 	  	
 	  	 \end{subtable}

 	 \vspace{0.5cm}
 	 \hspace{3.5cm}
  	  	\begin{subtable}[b]{0.5\textwidth}
 	  	 	
 	  	  	\centering
 	  	  	\caption{All collections} 
 	  	 
 	  	 	\begin{tabular}{l|c|c|c|c|c} 
 	  	 	
 	  	 	\cline{2- 6}
 	  	 	\multicolumn{1}{c}{}&\multicolumn{5}{c}{Precision} \\ 
 	  	 	\cline{2- 6} &
 	  	 	\textit{\textbf{@5}} & 
 	  	 	\textit{\textbf{@10}} & 
 	  	 	\textit{\textbf{@15}} & 
 	  	 	\textit{\textbf{@20}} & 
 	  	 	\textit{\textbf{@30}} 
 	  	 	
 	  	 	\tabularnewline
 	  	 	\hline
 	 	 	 BM25 & 0.55 & 0.49 & 0.46 & 0.43 & 0.39 \\
 	 	 	 DFRee  & \textbf{0.62} & \textbf{0.57} & \textbf{0.54} & \textbf{0.51} & \textbf{0.46} \\
 	 	 	 DLM  & 0.36 & 0.35 & 0.34 & 0.32 & 0.30 \\
 	 	 	 HLM  & 0.47 & 0.43 & 0.40 & 0.38 & 0.35 \\
 	 	 	 IDF  & \textbf{0.62} & \textbf{0.57} & 0.53 & \textbf{0.51} & \textbf{0.46} \\
 	  	 
 	  	  	\hline
 	  	  	\end{tabular}
 	  	  	
 	  	  	%\end{center}
 	  	  	%\end{footnotesize}
 	  	 
 	  	  	%\vspace{-0.50cm}
 	  	  	
 		\end{subtable}
 	 
 	 
  	\label{traditional}
  	  \vspace{0.7cm}
  	  \end{small}
 \end{table*}

\noindent{\bf Evaluation measures.} We pay attention to precision at different ranks, with a maximum cut-off point at rank 30. Future evidence is accepted only at the collection statistics level as agreed by TREC organisers disregarding any documents after the query issuing time when computing evaluation measures \footnote{https://github.com/lintool/twitter-tools/wiki/TREC-2013-Track-Guidelines}.\\

\noindent{\bf Baseline selection.} Table \ref{traditional} contains evaluation results for the considered state of the art retrieval models when applied to Twitter corpora from the 2011, 2012 and 2013 Trec microblog collections. The models considered in this evaluation are TF-IDF (IDF)\footnote{\(Where~TF=1.\) Results worsen considerably if we do not set TF to a constant.}, BM25 \cite{robertson1995okapi}, DFRee \cite{amati2003probabilistic}, Hiemstra's LM (HLM) \cite{hiemstra2001using} and Dirichlet's LM (DLM) \cite{blei2003latent} since it was the baseline for the Microblog Tracks in 2013 and 2014. Moreover, we adhere to the implementation and default settings found within the Terrier IR platform~\cite{ounis2005terrier}. Finally, since DFRee and IDF are generally the best performing models we will use them as our baselines.



%!TEX root = JournalChapter1.tex
\section{Investigating Retrieval Model Problems}
\label{RMinvestigation}
The literature has identified \textbf{document length normalization} as the main culprit for the under-performance of retrieval efforts in microblogs. The work by \cite{naveed2011searching} suggests that the \textbf{Verbosity} and \textbf{Scope} hypotheses do not hold for microblog retrieval. The \textbf{verbosity} hypothesis supports that some authors are more verbose than others, and hence applying length normalization by dividing by the length of the document is beneficial to better capture relevance, as repetition of terms is superfluous. On the other hand, the \textbf{scope} hypotheses states that some authors simply have more to say, thus naturally adding more relevant information to the topic. As a result documents are longer but more extensive and rigorous in their content than shorter ones. The added value of longer documents should be accounted for and thus promoted over shorter ones.

In the context of Microblog retrieval, \cite{naveed2011searching} carried out a number of experiments using a logistic regression model over a number of tweet features as the retrieval methodology. They showed significant improvements in performance when their algorithm did not perform document length normalization over its normalised counterpart. However, since in their work their ranking approach takes into consideration multiple other features, it is not clear if their finding about document length normalization is generalisable.

Furthermore, although it is been often assumed, it is not known if length normalisation is bad altogether for microblog retrieval, or maybe is just how it is interpreted in this particular case what makes it harmful. Intuition tell us that document length normalization as we know it does not interact well with the limitations characterised by microblogs. The \textbf{Verbosity} and \textbf{Scope} hypotheses seem not to model the behaviour of users publishing microblogs. Microblog users generally have the challenge of fitting their messages within the strict character limit. Consequently, retrieval models designed under scope and verbosity or similar premises, such as BM25 \cite{robertson2009probabilistic} are likely to exhibit unexpected behaviour, as it can be observed in Tables \ref{traditional}.

To aid in developing our understanding of the behaviour of retrieval models we formalise their composition. To this end we have compiled Table \ref{modelfeatures} to show the different components involved in the score computation of a variety of state of the art retrieval models. The top row of the table indicates whether the component relies on collection statistics (I.e. Collection feature) or the document itself (Document feature). The second row contains acronyms for each of the features, which are expanded as: 

\begin{itemize}
\item [ADL.] \textbf{AverageDocumentLength:} This is the average document length, in number of tokens, for the whole collection.
\item [DL.] \textbf{DocumentLength:} This is the document length, in number of tokens, for the document being scored.
\item [ND.] \textbf{NumberOfDocuments:} Total number of documents in the collection. 
\item [DF.] \textbf{DocumentFrequency:} Number of documents in which the term appears (I.e. A term's posting list size).
\item [NT.] \textbf{NumberOfTokens:} Number of different tokens in the collection.
\item [CTF.] \textbf{CollectionTermFrequency:} Frequency of a term in the whole collection. (I.e. Total number of occurences of a term in the collection)
\item [TF.] \textbf{TermFrequency:} Frequency of the term in the document being evaluated.
\end{itemize}

\begin{table}[]
	\caption{Features involved in the computation of retrieval models.}
	\centering
	\begin{tabular}{|l|c|c|c|c|c||c|c|} 
		\cline{2- 8}
		\multicolumn{1}{c|}{}& \multicolumn{5}{c||}{Collection Features} &  \multicolumn{2}{c|}{Document Features} \tabularnewline
		\cline{2- 8}
		\multicolumn{1}{c|}{}
		& \textit{\textbf{ND} } & \textit{\textbf{DF} } & \textit{\textbf{ADL} } & \textit{\textbf{NT} } 
		& \textit{\textbf{CTF} } & \textit{\textbf{TF} } & \textit{\textbf{DL} } \tabularnewline \hline
		\textit{IDF} 	& * &  *&  	&   &  	&	&   \tabularnewline \hline
		\textit{DFRee} 	&   &   &   & * & * &*	&* \tabularnewline \hline
		\textit{BM25}	& * &  *& * &   &  	&*	&* \tabularnewline \hline
		\textit{HLM} 	&   &   &  	& * & * &*	&* \tabularnewline \hline
		\textit{DLM} 	&   &   &  	& * & * &*	&* \tabularnewline \hline
	\end{tabular}
	\label{modelfeatures}
\end{table}

Each of the remaining rows contain the name of the retrieval model as well as whether a component involved in its computation (Denoted by *). For example, DFRee uses NumberOfTokens (NT), CollectionTermFrequency (CTF), TermFrequency (TF) and Document Length (DL).

In the following sections we investigate the behaviour of the abovementioned retrieval models in terms of these features. We perform our analysis mainly by means of simulating their behaviour with a range of different values common under microblog retrieval conditions. We then contextualise the model's actual performance with respect to its simulated behaviour, and draw generalised conclusions across these experiments.

\subsection{The BM25 Case}
\label{bm25case}
The work by \cite{ferguson2012investigation} examined the performance of BM25 when used under a microblog retrieval scenario. Their findings showed how the closer to zero the free parameters were set in BM25, the better the performance achieved. However, they did not connect this finding to the design of BM25 and what these settings meant in terms of the affected components. In this section we exemplify and connect these findings to the theory by simulating the behaviour of BM25 under microblog retrieval conditions.

% and examine how they affect other retrieval models aside from BM25.
First, we observe in Table \ref{modelfeatures} how BM25 relies on document length by using both ADL and DL components in its computation. Furthermore, BM25 has two free parameters, namely \(b\) and \(k_1\), which control the effects of the ``saturation function'' over the final score. The saturation function in BM25 encodes the document length evidence as part of the score as follows: 

The first version of the saturation function is given by:

\begin{equation}
 \text{Version 1: }\frac{f(q_i, D)}{f(q_i, D) + k_1} \text{   for some k_1 $>$ 0}
\end{equation}

Once we take into consideration the Verbosity and Scope hypotheses, the following saturation function can be derived:

\begin{equation}
 \text{Version 2: }\frac{f(q_i, D)}{f(q_i, D) + k_1*((1-b)+b*dl/avdl)} \text{   for some k_1 $>$ 0}
\end{equation}

The main difference between these equations is that \textbf{Version 2} reduces the effect of term frequency with respect to the document length and its collection average, whilst \textbf{Version 1} only relies on the \(k_1\) free parameter. Secondly, the free parameter \(b\) ponders between the Verbosity and Scope hypotheses. Setting \(b\) to 0 effectively disables the Verbose hypothesis, giving full weight to Scope, in other words, the longer the document the better. Thus when \(b\) is set to 0, \textit{Version 2} of the saturation function becomes \textit{Version 1}.

As we mentioned before, the study carried by \cite{ferguson2012investigation} explored the best parameters for \(b\) and \(k_1\) concluding that best performance is achieved as both parameters tend to 0. However, the authors did not mention is that by setting those parameters close to 0, we are disregarding the document length normalisation component altogether. Thus for all intents and purposes BM25 becomes IDF. This can be proved mathematically by substituting \(b\) and \(k_1\) by 0 as follows \ref{bm25proof}.

\begin{small}
\begin{align}
\label{bm25proof}
    \notag \text{BM25}(D,Q) &= \sum_{i=1}^{n} \text{IDF}(q_i) \cdot \frac{f(q_i, D) \cdot (k_1 + 1)}{f(q_i, D) + k_1 \cdot (1 - b + b \cdot \frac{|D|}{\text{avgdl}})} \\
  \notag&= \sum_{i=1}^{n} \text{IDF}(q_i) \cdot \frac{f(q_i, D) \cdot (0 + 1)}{f(q_i, D) + 0 \cdot (1 - 0 + 0 \cdot \frac{|D|}{\text{avgdl}})} \\
  \notag&= \sum_{i=1}^{n} \text{IDF}(q_i) \cdot \frac{f(q_i, D)}{f(q_i, D) } \\
  &= \sum_{i=1}^{n} \text{IDF}(q_i)              
\end{align}
%\end{proof}
\end{small}

Initially it would seem that the \textbf{Scope} and \textbf{Verbosity} hypotheses do not hold for microblogs. The reasoning behind being that these hypotheses were developed for documents that were unbounded in terms of their length such as web pages or books. However, since document length has an upper bound in microblogs, authors express their ideas in a very constrained space where verbosity and scope hypotheses do not seem to hold. However we will later observe that this conclusion is partially true\footnote{We later demonstrate that \textbf{scope} does hold, but not \textbf{verbosity}}.

\begin{figure}[]
  \centering
   \include{bm25TFDL}
     \caption{Term Frequency (TF) vs, Doc. Length (DL)}
  \label{bm25scoretfdl}
\end{figure}

Furthermore, terms in microblog documents have very low document frequencies and more often than not, query terms appear at most once in each document unless dealing with spam. Thus a query term appearing more than once within a document can have a dramatic effect over the score produced by BM25. In other words, the very low document frequencies result in unreliable estimations of the informativeness of a query term. Consequently, in this particular case, it is better to rely on features outside the document such as collection features.

Finally, Figure \ref{bm25scoretfdl} shows the possible BM25 scores for a range of Term Frequency (TF) and Doc. Length (DL) values.\footnote{Where \(ND=100k\) and \(DF=100\)}. We can extract two interesting behaviours which we can compare later to other retrieval models. Firstly the increase of document length is regarded as negative. In other words the more information in number of terms is encoded in the document the less relevant it is regarded. Secondly the increasing term frequency results in increased scores. This would seem counter-intuitive in a document with such a limited length, as users normally struggle to fit their messages. Additionally, there is a danger of promoting spam messages which may only contain the query terms.


%\mentalnote{Finally, by reducing the values of the b and k constants, the standard deviation of across the scores (w.r.t. tf and dl) by bm25 is also reduced, reaching 0 when b and k are 0, as tf and dl do not play any role in this case. Lower stdev. better performance}

\subsection{The Hiemstra's Language Model (HLM) Case}
In this section we study the Hiemstra's Language Model (HLM) \cite{hiemstra2001using} under Microblog conditions. Table \ref{modelfeatures} shows that HLM utilises both CollectionTermFrequency (CTF) and TermFrequency (TF) together with the total number of different tokens in the collection (NT) and document length (DL). Furthermore, if we pay attention to Table \ref{traditional} we can observe that whilst DFR and HLM utilize the same components, HLM exhibits a more erratic performance under microblog conditions. HLM's performance for the 2013 collection is considerably lower than that of DFR or IDF, whereas it remains close to the top performing models for the 2011, 2012 and 2014 collections. Let us look into the formulation of HLM: 

\begin{small}
\begin{align}
\label{hlmformula}
    \text{HLM}(D,Q) &= \sum_{i=1}^{n} \log_2 \left[ 1 + \frac{c \cdot f(q_i, D) \cdot ntoks }{ (1-c) \cdot f(q_i, C) \cdot |D|} \right]
\end{align}
\end{small}

where $ntoks$ refers to the number of unique tokens in the collection (NT), $c$ is a free parameter, and $C$ represents the set of all documents in the collection. $f(q_i, D)$ represents the TF of a query term $q_i$ in document $D$, whereas $f(q_i, C)$ is CTF of term $q_i$. The free parameter c regulates how HLM satisfies the conditions of \textbf{coordination level ranking (CLR)}) \cite{hiemstra2000relating}. CLR is a rule enforced in the design of HLM which ensures that documents containing $n$ query terms are ranked higher than those with $n-1$ terms.

Similarly to BM25, the assumption where higher term frequencies should be regarded positively, can easily result in the promotion of spam and undesired results. And this is rooted in the fact that query terms occur normally 1-2 times in a microblog document, due to length limitations.

%\begin{figure}[]
%%        
%%       \begin{subfigure}[b]{0.5\textwidth}
%%        \centering
%%        \caption{Doc. Frequency (CTF) vs, $c$}
%%         \input{hlmfigure-df-c}
%%      \end{subfigure}      
%%      ~
%       \begin{subfigure}[b]{0.5\textwidth}
%        \centering
%        \caption{Doc. Frequency (CTF) vs, Doc. Length (DL)}
%         \input{hlmfigure1}
%		\label{hlm-ctf-dl}
%      \end{subfigure} 
%    \caption{HLM analysis}
%		\label{hlmanalysis}
%\end{figure}

%\begin{figure}[]
%	\centering
%	\caption{HLM: Doc. Frequency (CTF) vs, Doc. Length (DL)}
%	\input{hlmfigure1}
%	\label{hlm-ctf-dl}
%\end{figure} 

\begin{figure}[]
     \begin{subfigure}[b]{0.5\textwidth}
      \centering
      \caption{TF vs, Doc. Length (DL)  with $c = 0.15$}
       \input{hlmfigureDLVSTF}
       	\label{cTFVSDL15}
    \end{subfigure}  
      ~
     \begin{subfigure}[b]{0.5\textwidth}
      \centering
      \caption{TF vs, Doc. Length (DL)  with $c = 0.99$}
       \input{hlmfigureDLVSTF99}
       \label{cTFVSDL99}
    \end{subfigure}  
    \caption{HLM analysis}
	\label{cTFVSDL}
\end{figure}

%Figure \ref{hlm-ctf-dl} shows HLM scores with respect to ``collection term frequency (CTF)\footnote{Also known as ``document frequency''}'' and document length (DL). 
Figure \ref{cTFVSDL15} shows a plot of the possible scores produced by HLM in its default configuration (\(c=0.15\))\footnote{Where \(ND=100k\), \(DF=100\) and \(NT=1000\)}. We can observe that for documents where the length is lower than 5 the differences between the scores are very marked. Above length 5 the progression of scores is much more subtle. In other words, shorter documents are subject to high differences between their scores due to small changes in their limited length.

Furthermore, we can observe in Formula \ref{hlmformula}, how the high sensitivity to low document length is a result of the model's design, since document length acts as a multiplier in the denominator. Additionally, term frequency can be found within the nominator as a multiplying component. Consequently, when higher than 1 it will result in an unreasonable boost of the score. In the case of microblog documents this can be problematic due to the scarce frequencies which average around 1.17 ($\pm 0.48$)\footnote{Computed for query terms in all TREC microblog topics up to 2014 and our baseline DFR}.


 
%To further illustrate these differences, we introduce Figure \ref{hlmcomp} where we show HLM scores w.r.t. term frequency ($f(q_i, D)$) within the 1 to 10 range. All other variables are kept constant\footnote{($c = 0.15$, $f(q_i, C) = 100$, $|D| = 5$ and $ntoks = 1000$)}.
%
%\begin{figure}[]
%  \centering
%   \include{hlmcomfigure}
%     \caption{TF vs HLM Score}
%  \label{hlmcomp}
%\end{figure}
%
%As we can observe in Figure \ref{hlmcomp}, the low term frequencies show substantial differences between the scores. As term frequencies grow the differences between scores become increasingly smaller. The intuition is that for documents of the same length, documents with higher query term frequency should be ranked higher. Unfortunately, for very low query term frequencies the score differences introduced by design for this purpose are too aggressive, and very unlikely correspond to the actual importance of such frequency differences.

Table \ref{traditional} shows that HLM is the second worst model overall for microblog retrieval. We hypothesise that the reason for this under-performance lies in the substantial scoring differences above-mentioned, resulting from the specific morphology of microblog documents which HLM does not account for. Thus reducing de differences in the scoring, should yield improved retrieval performance.

\subsubsection{Offsetting experiment}
In order to test this hypotheses we simulate the behaviour of longer documents with higher term frequency by offsetting the values of TF and DL. We do this by a simple addition \(TF = TF+dTF\), in this case \(dTF\) being the pondering value to offset \(TF\). Likewise, we utilise \(DL = DL+dDL\) where \(dDL\) is the variable to offset \(DL\).

Table \ref{hlmOverestimates} shows the performance of HLM measured by Precision@30 with different configurations. The first row shows the performance of HLM with a default configuration of $c = 0.15$. The second row with $dTF = 20$ so that $TF = TF+20$ which denotes the offsetting of TF by +20. As stated before, the reason behind this offsetting is to reduce the differences between possible scores with respect to the actual values of TF. As we can observe only offsetting TF does no result in any significant improvement. Similarly, the third row shows the performance of HLM when offsetting DL by +20 in order to reduce the possible score differences. Consequently the results are much better than before with a Precision@30 increase of +11.76\%. Finally, we experiment with the offsetting of TF and DL together to achieve yet another +15.79\% Precision@30 increase over the previous combination and a very substantial increase of +29.41\% over the baseline (no offsets) configuration). 

\begin{table}[]

	\caption{P@30 scores for HLM as we consider different combinations of dTF and dDL, and c (All collections together)}
	\centering
	\begin{tabular}{l|c|c|c|c} 	
	\textit{\textbf{}} &
	\textit{\textbf{c}} & 
	\textit{\textbf{dTF}} & 
	\textit{\textbf{dDL}} & 
	\textit{\textbf{P@30}} 	
	\tabularnewline
	\hline
	1 & 0.15 &    &    & 0.3475\\
	2 & 0.15 & 20 &    & 0.3486\\
	3 & 0.15 &    & 20 & 0.3839 \\
	4 & 0.15 & 20 & 20 & 0.4462 \\
	\hline
	\hline
	5 & 0.05 &  &  & 0.2824 \\
	6 & 0.40 &  &  & 0.4009 \\
	7 & 0.70 &  &  & 0.4281 \\
	8 & 0.99 &  &  & 0.4492 \\
	\hline
    \hline
	9 & 0.99 & 20 & 20 & \textbf{0.4532} \\	
	\hline
	\end{tabular}
	\label{hlmOverestimates}
\end{table}


It is interesting to notice how only the increase of TF does not help in retrieval, however only increasing DL does produce better results. Yet more importantly, by incrementing both TF and DL we obtain the best performance over all previous configurations. These results hint to a very subtle relationship between DL and TF values of microblog documents. 

Rows 5 to 8 in Table \ref{hlmOverestimates} show the performance of HLM with different values of $c$. As $c$ is increased performance increases as well, reaching comparable performance to the approach which offsets DL and TF (Row 4).

Finally, we compare Figures \ref{cTFVSDL15} and \ref{cTFVSDL99} which show scores produced by HLM w.r.t. TF and DL with different values of $c$. Figure \ref{cTFVSDL15} sets $c=0.15$ whereas Figure \ref{cTFVSDL99} sets $c=0.99$. Figure \ref{cTFVSDL15} shows more differences across the spectrum of scores with respect to TF and DL than Figure \ref{cTFVSDL99}. We can also observe how offsetting DL and TF forces the possible values of HLM to lie in the more stable area of the Figures. Furthermore, Figure \ref{cTFVSDL99} produces the most stable scores. From these experiments we can conclude that retrieval models require a conservative and delicate relationship with DL and TF, taking especial care to reduce the differences across the spectrum of possible scores, in order to reduce any unfair weighting differences due to scarcity in DL and TF.

\subsection{The DLM Case}
Dirichlet Smoothed language model (DLM), was the baseline retrieval model for the 2013 and 2014 instances of the microblog track. DLM was used within the "Microblog track as a service" client which managed a Lucene index in its core. DLM has a smoothing parameter named $\mu$, which was set to 2500 by default during the 2013 and 2014 microblog tracks. Moreover, DLM scores are produced \footnote{As implemented in the Terrier IR platform} by the following equation:

\begin{small}
\begin{align}
\label{dlmformula}
    \text{DLM}(D,Q) &= \sum_{i=1}^{n} \log_2 \left[ 1 + \frac{f(q_i, D)}{\mu \cdot \frac{ f(q_i, C) }{ ntoks }}\right] + \log_2 \left[\frac{\mu}{|D| + \mu}\right]
\end{align}
\label{dlmequation}
%\end{proof}
\end{small}

\noindent where $ntoks$ refers to the number of unique tokens in the collection (NT), $\mu$ is a free parameter, and $C$ represents the set of all documents in the collection. $f(q_i, D)$ represents the TF of a query term $q_i$ in document $D$, whereas $f(q_i, C)$ is the collection document frequency (CTF) of term $q_i$.

\begin{figure}
 		\begin{subfigure}[]{0.5\textwidth}
     	\caption{Document Frequency and $\mu$ parameter} 
    	\input{dlmfigurec2}
     	\label{dlmproofc2}
        \end{subfigure}
%        \qquad %add desired spacing between images, e. g. ~, \quad, \qquad etc.
          %(or a blank line to force the subfigure onto a new line)
        ~
		\begin{subfigure}[]{0.5\textwidth}
           \caption{Doc. length and $\mu$ parameter}
           \input{dlmcfigure-dl-c}
           \label{dlmproofcc}          
        \end{subfigure}
        
		\begin{subfigure}[]{\textwidth}
          \caption{Doc. length and Document Frequency}
          \hspace{3.9cm}
\begin{tikzpicture}[thick,scale=0.7, every node/.style={transform shape}] \begin{axis}[
 %title={},
 %y dir=reverse, 
 %x dir=reverse, 
 ylabel={docLength},
 xlabel={docFrequency},
 zlabel={DirichletLM score},
 every axis/.append style={font=\large\bfseries},
 max space between ticks=25pt
% yticklabels={0k,100k}
 ] 

		\addplot3[surf] coordinates { 
%patch,patch type=biquadratic, shader=faceted,patch refines=3
(100.00,20.00,4.60)(100.00,18.00,4.60)(100.00,16.00,4.60)(100.00,14.00,4.60)(100.00,12.00,4.60)(100.00,10.00,4.61)(100.00,8.00,4.61)(100.00,6.00,4.61)(100.00,4.00,4.61)(100.00,2.00,4.61)

(1100.00,20.00,1.64)(1100.00,18.00,1.64)(1100.00,16.00,1.64)(1100.00,14.00,1.64)(1100.00,12.00,1.64)(1100.00,10.00,1.64)(1100.00,8.00,1.64)(1100.00,6.00,1.64)(1100.00,4.00,1.64)(1100.00,2.00,1.65)

(2100.00,20.00,1.07)(2100.00,18.00,1.07)(2100.00,16.00,1.07)(2100.00,14.00,1.07)(2100.00,12.00,1.07)(2100.00,10.00,1.08)(2100.00,8.00,1.08)(2100.00,6.00,1.08)(2100.00,4.00,1.08)(2100.00,2.00,1.08)

(3100.00,20.00,0.80)(3100.00,18.00,0.80)(3100.00,16.00,0.80)(3100.00,14.00,0.80)(3100.00,12.00,0.81)(3100.00,10.00,0.81)(3100.00,8.00,0.81)(3100.00,6.00,0.81)(3100.00,4.00,0.81)(3100.00,2.00,0.81)

(4100.00,20.00,0.64)(4100.00,18.00,0.64)(4100.00,16.00,0.64)(4100.00,14.00,0.64)(4100.00,12.00,0.65)(4100.00,10.00,0.65)(4100.00,8.00,0.65)(4100.00,6.00,0.65)(4100.00,4.00,0.65)(4100.00,2.00,0.65)

(5100.00,20.00,0.53)(5100.00,18.00,0.54)(5100.00,16.00,0.54)(5100.00,14.00,0.54)(5100.00,12.00,0.54)(5100.00,10.00,0.54)(5100.00,8.00,0.54)(5100.00,6.00,0.54)(5100.00,4.00,0.54)(5100.00,2.00,0.54)

(6100.00,20.00,0.46)(6100.00,18.00,0.46)(6100.00,16.00,0.46)(6100.00,14.00,0.46)(6100.00,12.00,0.46)(6100.00,10.00,0.46)(6100.00,8.00,0.46)(6100.00,6.00,0.47)(6100.00,4.00,0.47)(6100.00,2.00,0.47)

(7100.00,20.00,0.40)(7100.00,18.00,0.40)(7100.00,16.00,0.40)(7100.00,14.00,0.40)(7100.00,12.00,0.40)(7100.00,10.00,0.41)(7100.00,8.00,0.41)(7100.00,6.00,0.41)(7100.00,4.00,0.41)(7100.00,2.00,0.41)

(8100.00,20.00,0.36)(8100.00,18.00,0.36)(8100.00,16.00,0.36)(8100.00,14.00,0.36)(8100.00,12.00,0.36)(8100.00,10.00,0.36)(8100.00,8.00,0.36)(8100.00,6.00,0.36)(8100.00,4.00,0.36)(8100.00,2.00,0.37)

(9100.00,20.00,0.32)(9100.00,18.00,0.32)(9100.00,16.00,0.32)(9100.00,14.00,0.32)(9100.00,12.00,0.32)(9100.00,10.00,0.32)(9100.00,8.00,0.33)(9100.00,6.00,0.33)(9100.00,4.00,0.33)(9100.00,2.00,0.33)

(10100.00,20.00,0.29)(10100.00,18.00,0.29)(10100.00,16.00,0.29)(10100.00,14.00,0.29)(10100.00,12.00,0.29)(10100.00,10.00,0.30)(10100.00,8.00,0.30)(10100.00,6.00,0.30)(10100.00,4.00,0.30)(10100.00,2.00,0.30)

(11100.00,20.00,0.26)(11100.00,18.00,0.27)(11100.00,16.00,0.27)(11100.00,14.00,0.27)(11100.00,12.00,0.27)(11100.00,10.00,0.27)(11100.00,8.00,0.27)(11100.00,6.00,0.27)(11100.00,4.00,0.27)(11100.00,2.00,0.28)

(12100.00,20.00,0.24)(12100.00,18.00,0.25)(12100.00,16.00,0.25)(12100.00,14.00,0.25)(12100.00,12.00,0.25)(12100.00,10.00,0.25)(12100.00,8.00,0.25)(12100.00,6.00,0.25)(12100.00,4.00,0.25)(12100.00,2.00,0.25)

(13100.00,20.00,0.23)(13100.00,18.00,0.23)(13100.00,16.00,0.23)(13100.00,14.00,0.23)(13100.00,12.00,0.23)(13100.00,10.00,0.23)(13100.00,8.00,0.23)(13100.00,6.00,0.23)(13100.00,4.00,0.24)(13100.00,2.00,0.24)

(14100.00,20.00,0.21)(14100.00,18.00,0.21)(14100.00,16.00,0.21)(14100.00,14.00,0.21)(14100.00,12.00,0.21)(14100.00,10.00,0.22)(14100.00,8.00,0.22)(14100.00,6.00,0.22)(14100.00,4.00,0.22)(14100.00,2.00,0.22)

(15100.00,20.00,0.20)(15100.00,18.00,0.20)(15100.00,16.00,0.20)(15100.00,14.00,0.20)(15100.00,12.00,0.20)(15100.00,10.00,0.20)(15100.00,8.00,0.20)(15100.00,6.00,0.20)(15100.00,4.00,0.21)(15100.00,2.00,0.21)

(16100.00,20.00,0.18)(16100.00,18.00,0.19)(16100.00,16.00,0.19)(16100.00,14.00,0.19)(16100.00,12.00,0.19)(16100.00,10.00,0.19)(16100.00,8.00,0.19)(16100.00,6.00,0.19)(16100.00,4.00,0.19)(16100.00,2.00,0.19)

(17100.00,20.00,0.17)(17100.00,18.00,0.18)(17100.00,16.00,0.18)(17100.00,14.00,0.18)(17100.00,12.00,0.18)(17100.00,10.00,0.18)(17100.00,8.00,0.18)(17100.00,6.00,0.18)(17100.00,4.00,0.18)(17100.00,2.00,0.18)

(18100.00,20.00,0.16)(18100.00,18.00,0.17)(18100.00,16.00,0.17)(18100.00,14.00,0.17)(18100.00,12.00,0.17)(18100.00,10.00,0.17)(18100.00,8.00,0.17)(18100.00,6.00,0.17)(18100.00,4.00,0.17)(18100.00,2.00,0.17)

(19100.00,20.00,0.16)(19100.00,18.00,0.16)(19100.00,16.00,0.16)(19100.00,14.00,0.16)(19100.00,12.00,0.16)(19100.00,10.00,0.16)(19100.00,8.00,0.16)(19100.00,6.00,0.16)(19100.00,4.00,0.16)(19100.00,2.00,0.17)

(20100.00,20.00,0.15)(20100.00,18.00,0.15)(20100.00,16.00,0.15)(20100.00,14.00,0.15)(20100.00,12.00,0.15)(20100.00,10.00,0.15)(20100.00,8.00,0.15)(20100.00,6.00,0.16)(20100.00,4.00,0.16)(20100.00,2.00,0.16)

(21100.00,20.00,0.14)(21100.00,18.00,0.14)(21100.00,16.00,0.14)(21100.00,14.00,0.14)(21100.00,12.00,0.15)(21100.00,10.00,0.15)(21100.00,8.00,0.15)(21100.00,6.00,0.15)(21100.00,4.00,0.15)(21100.00,2.00,0.15)

(22100.00,20.00,0.13)(22100.00,18.00,0.14)(22100.00,16.00,0.14)(22100.00,14.00,0.14)(22100.00,12.00,0.14)(22100.00,10.00,0.14)(22100.00,8.00,0.14)(22100.00,6.00,0.14)(22100.00,4.00,0.14)(22100.00,2.00,0.14)

(23100.00,20.00,0.13)(23100.00,18.00,0.13)(23100.00,16.00,0.13)(23100.00,14.00,0.13)(23100.00,12.00,0.13)(23100.00,10.00,0.13)(23100.00,8.00,0.13)(23100.00,6.00,0.14)(23100.00,4.00,0.14)(23100.00,2.00,0.14)

(24100.00,20.00,0.12)(24100.00,18.00,0.12)(24100.00,16.00,0.12)(24100.00,14.00,0.13)(24100.00,12.00,0.13)(24100.00,10.00,0.13)(24100.00,8.00,0.13)(24100.00,6.00,0.13)(24100.00,4.00,0.13)(24100.00,2.00,0.13)

(25100.00,20.00,0.12)(25100.00,18.00,0.12)(25100.00,16.00,0.12)(25100.00,14.00,0.12)(25100.00,12.00,0.12)(25100.00,10.00,0.12)(25100.00,8.00,0.12)(25100.00,6.00,0.13)(25100.00,4.00,0.13)(25100.00,2.00,0.13)

}; \end{axis} \end{tikzpicture}

          \label{dlmproof}          
        \end{subfigure}

        \caption{DLM evaluation figures}
\end{figure}

Figures \ref{dlmproofc2} and \ref{dlmproofcc} show DLM scores in terms of the $\mu$ parameter, w.r.t. document frequency and document length respectively. Figure \ref{dlmproof} on the other hand demonstrates the relation between document frequency and document length.

As we can observe from Equation \ref{dlmequation} the parameter $\mu$ is closely related to the collection statistics, and the length normalization component of the equation. Moreover the lower the values of $\mu$ the higher the score differences for similar document frequencies as shown in Figure \ref{dlmproofc2}. Similarly, we can observe in Figure \ref{dlmproofcc} how $\mu$ interacts with document length. For low values of $\mu$ we can observe how the scores are reduced at the same time that documents become larger, as expected for normal documents. Interestingly, this behaviour is dampened with higher values of $\mu$, as score differences are heavily reduced w.r.t. the different document lengths. Since the default value for $\mu$ is 2500, it is no surprise that document length has virtually no effect over the scores for DLM as seen in Figure \ref{dlmproof}, contrary to other retrieval models. 

\begin{table}[]

	\caption{P@30 scores for DLM for a range of $\mu$ values (All collections together)}
	\centering
	\begin{tabular}{l|c} 	
	\textit{\textbf{$\mu$}} & 
	\textit{\textbf{P@30}} 	
	\tabularnewline
	\hline
	1 & 0.4028 \\
	5 &  0.4164 \\
	20 & 0.4241 \\
	50 &  0.4099 \\
	100 &  0.3933 \\
	500 &  0.3396 \\
	1000 & 0.3227 \\
	2500 & 0.2988 \\
	\hline	
	\end{tabular}
	\label{drmmuvalues}
\end{table}

This could be a desired feature for microblog retrieval, however let us look at the performance achieved for a range of $\mu$ values in Table \ref{drmmuvalues}. As we can observe generally the higher the value of $mu$ the worse the performance obtained, with the exception of $\mu$ within the 1 to 20 range. 

In order to further understand the behaviour of DLM in the case of Microblog retrieval, we perform an analogous experiment to the previously performed for HLM. Since DLM was also designed for longer documents than microblogs, offsetting the statistics of TF and DL can be interesting experiment as it would better resemble its standard behaviour in term of the numerical values produced as scores. 

The results of the evaluation are presented in Table \ref{drmdtfmuvalues}. The first four lines contain the P@30 values for different combinations where $\mu$ is set to 20. As we can observe offsetting TF by +20 results in a substantial +7.47\% increase of P@30 with respect to the default configuration. On the other hand offsetting DL by +20 results in a 8.02\% decrease of performance in terms of P@30. Finally, combining the offsetting of both TF and DL results in comparable performance than that obtained by only increasing TF.

\begin{table}[]
	\caption{P@30 scores for DLM as we consider different combinations of dTF and dDL, and $\mu$, (All collections together)}
	\centering
	\begin{tabular}{l|c|c|c|c} 	
	&
	\textit{\textbf{$\mu$}} & 
	\textit{\textbf{dTF}} & 
	\textit{\textbf{dDL}} & 
	\textit{\textbf{P@30}} 	
	\tabularnewline
	\hline
	1 & 20 &    &    & 0.4241\\
	2 & 20 & 20 &    & 0.4558\\
	3 & 20 &    & 20 & 0.3901\\
	4 & 20 & 20 & 20 & 0.4547\\
	\hline	
	\hline
	5 & 2500 &    &    & 0.2988\\
	6 & 2500 & 20 &    & 0.4468\\
	7 & 2500 &    & 20 & 0.2892\\
	8 & 2500 & 20 & 20 & 0.4466\\
    \hline
	\end{tabular}
	\label{drmdtfmuvalues}
\end{table}

The same behaviour is obtained across all combinations when we set the $\mu = 2500$. To further develop our understanding of the behaviour, and to draw conclusions for such results, we devised Figures \ref{dlmfigureTFDL2500} and \ref{dlmfigureTFDL20}. Figures \ref{dlmfigureTFDL2500} and \ref{dlmfigureTFDL20} present the DLM scores produced with respect to Doc. Length (DL) and Term Frequency (TF) when $\mu=2500$ and $\mu=20$ respectively.

Let us analyse the results from Table \ref{drmdtfmuvalues} in connection with Figures \ref{dlmfigureTFDL2500} and \ref{dlmfigureTFDL20}. As we can observe incrementing DL will result in an increased differentiation of DLM scores with respect to TF as more values are closer to the minimum and maximum values. In other words there are less intermediate values (Light coloured areas), which ultimately reflects on heightened sensitivity to differences across the TF spectrum. Furthermore, we can also observe in Table \ref{drmdtfmuvalues} how incrementing DL values, results in worse performance in all cases. Consequently the increased differentiation of DLM scores with respect to the TF parameter, produced by the increment of DL is detrimental and in line with the findings in the previous section.

Additionally, Figure \ref{dlmfigureTFDL2500} shows an almost linear progression of DLM scores with respect to TF, whereas Figure \ref{dlmfigureTFDL20} ($\mu=20$) exhibits a logarithmic behaviour with respect to TF. The latter behaviour is more desirable because there should be a saturation point when incrementing TF at which there is very little value added to the score of the document, or could be even counter productive. In fact, if we take into consideration that term frequencies within microblogs are in the range 1-2, the pivoting value w.r.t TF should be very low, to avoid promoting spam microblogs.

The better behaviour with respect to TF is rewarded with increased performance whether the value of $\mu$ is 20 or 2500. In fact the offsetting of TF seems to overrule the effects of $\mu$ as similar results are obtained in both $\mu=20$ and $\mu=2500$ conditions. The effects of offsetting TF are most visually evident when looking at Figure \ref{dlmfigureTFDL20} as differences amongst the different scores become very small, when $TF > 20$. 


\begin{figure}
      	\begin{subfigure}[b]{0.5\textwidth}
          \centering
          \caption{Doc. length (DL) and Term Frequency (TF) when $\mu = 2500$}
          \input{dlmfigureTFDL2500}
          \label{dlmfigureTFDL2500}          
        \end{subfigure} 
        ~
 		\begin{subfigure}[b]{0.5\textwidth}
          \centering
          \caption{Doc. length (DL) and Term Frequency (TF) when $\mu = 20$}
          \input{dlmfigureTFDL}
          \label{dlmfigureTFDL20}          
        \end{subfigure} 
        \caption{Evaluating DLM's behaviour}
\end{figure}

Extending on the findings by \cite{naveed2011searching} who showed how length normalization was detrimental to microblog retrieval in an L2R retrieval framework. Our experiments have so far indicated the existence of a particular relationship between TF and DL that is most appropriate for Microblog retrieval. We believe that the score progressions with respect to \textit{DL should modelled by a very gentle slope}, whereas there should be a pivoting point with respect to \textit{TF where scores should decay} in order to account for spam. In the following sections these ideas will be further elaborated.

\subsection{The DFRee Case}
DFRee\footnote{http://terrier.org/docs/v2.2.1/javadoc/uk/ac/gla/terrier/matching/models/DFRee.html} is a Divergence From Randomness model implemented in the Terrier IR platform \cite{terrierir}. DFRee has been designed as a parameter-free model and adheres to the following implementation:

\begin{equation}
prior = \frac{f(q_i, D)}{|D|}, posterior = \frac{f(q_i, D)+1}{|D|+1} 
\end{equation}

\begin{equation}
InvPriorColl = \frac{ntoks}{f(q_i, C)}, norm = f(q_i, D)*log_2{\frac{posterior}{prior}}
\end{equation}

%\begin{equation}
\begin{multline}
DFRee(q_i,D,C) = norm * [                    \\
f(q_i, D)*(-log_2(prior*InvPriorColl))       \\
+(f(q_i, D)+1)*log_2(posterior*InvPriorColl) \\
+ 0.5*log_2(posterior/prior)],
\end{multline}

where \(f(q_i, D)\) represents the frequency of query term \(q_i\) within document \(D\). Similarly \(f(q_i, C)\) holds the collection \(C\) frequency for query term \(q_i\). Furthermore \(ntoks\) is the total number of unique terms within collection \(C\) and \(|D|\) represents the document length of document \(D\).

\begin{figure}
	\centering
	\caption{Evaluating DFR's behaviour: Doc. length (DL) and Term Frequency (TF)}
	%!TEX root = ./JournalChapter1.tex
\begin{tikzpicture}[thick,scale=0.8, every node/.style={transform shape}]\begin{axis}[
 %title={},
% y dir=reverse, 
 x dir=reverse, 
 ylabel={docLength (DL)},
 xlabel={term frequency (TF)},
 zlabel={DFR score},
 every axis/.append style={font=\large\bfseries},
 max space between ticks=25pt
% yticklabels={0k,100k}
 ] 


\addplot3[surf,unbounded coords=jump]
coordinates  { 
(1,15,0.602828690204566)	(2,15,0.830458458727289)	(3,15,0.934450716638443)	(4,15,0.966903402267895)	(5,15,0.953929912325151)	(6,15,0.910204250164967)	(7,15,0.844773148573639)	(8,15,0.763616521737829)	(9,15,0.670903438148194)	(10,15,0.569664557459257)	(11,15,0.462178197338477)	(12,15,0.350204856834806)	(13,15,0.235136442615999)	(14,15,0.11809496699829)	(15,15,0)

(1,16,0.578099188538087)	(2,16,0.811155500645937)	(3,16,0.92455241464552)	(4,16,0.967767908505993)	(5,16,0.966062841257201)	(6,16,0.933720545287019)	(7,16,0.87957150961847)	(8,16,0.809460625419414)	(9,16,0.72746535973258)	(10,16,0.636550382771426)	(11,16,0.538944280996693)	(12,16,0.436368745300519)	(13,16,0.330184550451496)	(14,16,0.221488171413441)	(15,16,0.111177842268355)

(1,17,0.554089938092192)	(2,17,0.791526910417312)	(3,17,0.913056948252712)	(4,17,0.965714873324416)	(5,17,0.973972028301543)	(6,17,0.951752192118292)	(7,17,0.907687908109167)	(8,17,0.847501039032133)	(9,17,0.775185911781543)	(10,17,0.693647497457941)	(11,17,0.60506951435267)	(12,17,0.51113872004729)	(13,17,0.413187919650668)	(14,17,0.312290682010694)	(15,17,0.209326130035835)

(1,18,0.530802801771364)	(2,18,0.771767857536177)	(3,18,0.900372831623677)	(4,18,0.961364141168104)	(5,18,0.978478269819011)	(6,18,0.965306243347949)	(7,18,0.930298335362267)	(8,18,0.879063614353843)	(9,18,0.815520463917971)	(10,18,0.742519493293841)	(11,18,0.662203670865781)	(12,18,0.576228097258246)	(13,18,0.485900278395978)	(14,18,0.392273096789813)	(15,18,0.296208439454882)

(1,19,0.508226789640423)	(2,19,0.752018863072771)	(3,19,0.886810317359786)	(4,19,0.955195710616974)	(5,19,0.980224486964259)	(6,19,0.975178213356986)	(7,19,0.948338450288501)	(8,19,0.905210288059222)	(9,19,0.849642347853426)	(10,19,0.784435449978092)	(11,19,0.711695334773675)	(12,19,0.633048243551952)	(13,19,0.549778660548279)	(14,19,0.462920673152469)	(15,19,0.373320528862043)

(1,20,0.48634303905334)	(2,20,0.7323812008066)	(3,20,0.872607218288516)	(4,20,0.947585137098897)	(5,20,0.97971952494445)	(6,20,0.98200290254851)	(7,20,0.962559939768004)	(8,20,0.926799408296809)	(9,20,0.878505521275719)	(10,20,0.820433262364011)	(11,20,0.754654182131245)	(12,20,0.682768072499512)	(13,20,0.606038355509857)	(14,20,0.525481948969255)	(15,20,0.441930831844985)

(1,21,0.465128061580168)	(2,21,0.712927637929877)	(3,21,0.857947247761062)	(4,21,0.93882895517277)	(5,21,0.97736988612122)	(6,21,0.986291533238511)	(7,21,0.973572063624968)	(8,21,0.944530636289626)	(9,21,0.902891730766613)	(10,21,0.851367936716025)	(11,21,0.791999256676888)	(12,21,0.726361125537109)	(13,21,0.655697602410148)	(14,21,0.581009843803962)	(15,21,0.503116746944407)

(1,22,0.444555871588363)	(2,22,0.69371003859671)	(3,22,0.842973247656932)	(4,22,0.929163209859617)	(5,22,0.973503053852954)	(6,22,0.988459253367945)	(7,22,0.981872662054963)	(8,22,0.958978737875236)	(9,22,0.923446336219237)	(10,22,0.877948672683593)	(11,22,0.824496515075059)	(12,22,0.764642770856045)	(13,22,0.699613620363273)	(14,22,0.630395687626556)	(15,22,0.557795822851653)

(1,23,0.424599383032406)	(2,23,0.674764818583908)	(3,23,0.827796862032113)	(4,23,0.918777149915091)	(5,23,0.968384861977651)	(6,23,0.988845761863186)	(7,23,0.987871568720419)	(8,23,0.970619283566948)	(9,23,0.940705773382284)	(10,23,0.900767545813065)	(11,23,0.852788172693824)	(12,23,0.798299644283731)	(13,23,0.738511566834336)	(14,23,0.674397111108444)	(15,23,0.606751988315705)

(1,24,0.40523132186617)	(2,24,0.656116904788573)	(3,24,0.812505700538763)	(4,24,0.907823472234563)	(5,24,0.962232586378857)	(6,24,0.987730954630336)	(7,24,0.991908481950635)	(8,24,0.979848390893796)	(9,24,0.955118795248774)	(10,24,0.920321862204126)	(11,24,0.87741577498763)	(12,24,0.827913032165606)	(13,24,0.773007819631001)	(14,24,0.713660808324183)	(15,24,0.650657372038589)

(1,25,0.38642481493876)	(2,25,0.637782639637063)	(3,25,0.79716870357892)	(4,25,0.896426073240635)	(5,25,0.955224921679465)	(6,25,0.985346919947008)	(7,25,0.994266744020595)	(8,25,0.986998033599149)	(9,25,0.96706304333111)	(10,25,0.937031713492213)	(11,25,0.898838449553016)	(12,25,0.853977527193232)	(13,25,0.803628737708457)	(14,25,0.748741099129492)	(15,25,0.690090359300742)

(1,26,0.36815376139952)	(2,26,0.619771931399019)	(3,26,0.781840202366091)	(4,26,0.884685975052332)	(5,26,0.947509661773085)	(6,26,0.981887225781465)	(7,26,0.995184065634633)	(8,26,0.992348018723311)	(9,26,0.976858081971228)	(10,26,0.95125386482836)	(11,26,0.917447437797707)	(12,26,0.876915988605277)	(13,26,0.830825830481115)	(14,26,0.78011508458774)	(15,26,0.725550507465349)

(1,27,0.350393057325973)	(2,27,0.60208985927249)	(3,27,0.766563020474647)	(4,27,0.872685899583419)	(5,27,0.939209668327629)	(6,27,0.977514176161289)	(7,27,0.994860946200371)	(8,27,0.996135435044116)	(9,27,0.984775730704953)	(10,27,0.963292820489072)	(11,27,0.933577738074063)	(12,27,0.897091603060034)	(13,27,0.854988073305192)	(14,27,0.808195051131356)	(15,27,0.757470870504045)

(1,28,0.333118721214438)	(2,28,0.584737880102515)	(3,28,0.751370863683325)	(4,28,0.860493830168551)	(5,28,0.930427549263709)	(6,28,0.97236452885122)	(7,28,0.993467339157226)	(8,28,0.998562165131648)	(9,28,0.99104831536007)	(10,28,0.973409701227053)	(11,28,0.947517492491807)	(12,28,0.914817660899949)	(13,28,0.876451950536151)	(14,28,0.833338655534655)	(15,28,0.786228199121581)

(1,29,0.316307952698114)	(2,29,0.567714741253384)	(3,29,0.736290176025056)	(4,29,0.848165807395434)	(5,29,0.921249355632251)	(6,29,0.966554036195814)	(7,29,0.991147968519457)	(8,29,0.999800901797173)	(9,29,0.995875303197284)	(10,29,0.981829412799183)	(11,29,0.959515600327343)	(12,29,0.930365523362214)	(13,29,0.89550968334361)	(14,29,0.855857318363783)	(15,29,0.81215140336571)

(1,30,0.299939146627551)	(2,30,0.55101717484961)	(3,30,0.721341591745516)	(4,30,0.835748140323265)	(5,30,0.911747524748403)	(6,30,0.960181077894388)	(7,30,0.988026599941357)	(8,30,1)	(9,30,0.999428673828312)	(10,30,0.98874647076145)	(11,30,0.969787928978429)	(12,30,0.943971149900242)	(13,30,0.912416001900636)	(14,30,0.876023168107538)	(15,30,0.835528594560206)

};

\addplot3 [data cs=cart,surf,domain=-10:10,samples=2, opacity=0.3,color=purple] coordinates  { 
(0,15,0) (0,15,1)

(15,15,0) (15,15,1)

};


\addplot3[surf,unbounded coords=jump]
coordinates  { 
%patch,patch type=biquadratic, shader=faceted,patch refines=3
(1,1,0)	(2,1,nan)	(3,1,nan)	(4,1,nan)	(5,1,nan)	(6,1,nan)	(7,1,nan)	(8,1,nan)	(9,1,nan)	(10,1,nan)	(11,1,nan)	(12,1,nan)	(13,1,nan)	(14,1,nan)	(15,1,nan)

(1,2,0.550946811011921)	(2,2,0)	(3,2,nan)	(4,2,nan)	(5,2,nan)	(6,2,nan)	(7,2,nan)	(8,2,nan)	(9,2,nan)	(10,2,nan)	(11,2,nan)	(12,2,nan)	(13,2,nan)	(14,2,nan)	(15,2,nan)

(1,3,0.739499035509737)	(2,3,0.447402531994322)	(3,3,0)	(4,3,nan)	(5,3,nan)	(6,3,nan)	(7,3,nan)	(8,3,nan)	(9,3,nan)	(10,3,nan)	(11,3,nan)	(12,3,nan)	(13,3,nan)	(14,3,nan)	(15,3,nan)

(1,4,0.808372355369359)	(2,4,0.674021740218476)	(3,4,0.367084989657738)	(4,4,0)	(5,4,nan)	(6,4,nan)	(7,4,nan)	(8,4,nan)	(9,4,nan)	(10,4,nan)	(11,4,nan)	(12,4,nan)	(13,4,nan)	(14,4,nan)	(15,4,nan)

(1,5,0.827712304044469)	(2,5,0.795475244633043)	(3,5,0.589020399660793)	(4,5,0.309402318963368)	(5,5,0)	(6,5,nan)	(7,5,nan)	(8,5,nan)	(9,5,nan)	(10,5,nan)	(11,5,nan)	(12,5,nan)	(13,5,nan)	(14,5,nan)	(15,5,nan)

(1,6,0.823758030119902)	(2,6,0.861291472386904)	(3,6,0.728208149339069)	(4,6,0.516703283500178)	(5,6,0.266826414784725)	(6,6,0)	(7,6,nan)	(8,6,nan)	(9,6,nan)	(10,6,nan)	(11,6,nan)	(12,6,nan)	(13,6,nan)	(14,6,nan)	(15,6,nan)

(1,7,0.808049149749894)	(2,7,0.895286730204683)	(3,7,0.81731608309985)	(4,7,0.659114482699154)	(5,7,0.458088606324659)	(6,7,0.234329891272407)	(7,7,0)	(8,7,nan)	(9,7,nan)	(10,7,nan)	(11,7,nan)	(12,7,nan)	(13,7,nan)	(14,7,nan)	(15,7,nan)

(1,8,0.786232834510155)	(2,8,0.910031065751261)	(3,8,0.874547067067584)	(4,8,0.758685635473427)	(5,8,0.597677806310571)	(6,8,0.410536626837436)	(7,8,0.208787788218013)	(8,8,0)	(9,8,nan)	(10,8,nan)	(11,8,nan)	(12,8,nan)	(13,8,nan)	(14,8,nan)	(15,8,nan)

(1,9,0.761283776662586)	(2,9,0.912649127410629)	(3,9,0.9106548363097)	(4,9,0.828974496924382)	(5,9,0.700999123980372)	(6,9,0.54484240060699)	(7,9,0.371505245812937)	(8,9,0.188214411477962)	(9,9,0)	(10,9,nan)	(11,9,nan)	(12,9,nan)	(13,9,nan)	(14,9,nan)	(15,9,nan)

(1,10,0.734851414275236)	(2,10,0.907408307407578)	(3,10,0.932312475346719)	(4,10,0.878648510130617)	(5,10,0.778218625288615)	(6,10,0.648369486864792)	(7,10,0.499658763815525)	(8,10,0.339026534140879)	(9,10,0.171302777785929)	(10,10,0)	(11,10,nan)	(12,10,nan)	(13,10,nan)	(14,10,nan)	(15,10,nan)

(1,11,0.707882534208275)	(2,11,0.896978134982265)	(3,11,0.943832686982997)	(4,11,0.91344706582647)	(5,11,0.83622890765175)	(6,11,0.728859016655641)	(7,11,0.601513634159584)	(8,11,0.460887857152159)	(9,11,0.311641587291567)	(10,11,0.157162472776661)	(11,11,0)	(12,11,nan)	(13,11,nan)	(14,11,nan)	(15,11,nan)

(1,12,0.680931777320597)	(2,12,0.883087512447978)	(3,12,0.948098319824685)	(4,12,0.937288690029642)	(5,12,0.879825806210446)	(6,12,0.791804036576653)	(7,12,0.683066380350784)	(8,12,0.560095278476284)	(9,12,0.427403757686762)	(10,12,0.288272090676837)	(11,12,0.145167857760471)	(12,12,0)	(13,12,nan)	(14,12,nan)	(15,12,nan)

(1,13,0.654326072191879)	(2,13,0.866887863527846)	(3,13,0.947091556647468)	(4,13,0.95292023017884)	(5,13,0.912425568445965)	(6,13,0.841179268774445)	(7,13,0.748729504809226)	(8,13,0.641371703296107)	(9,13,0.523490353484945)	(10,13,0.398272129069081)	(11,13,0.268113442610775)	(12,13,0.134867288166317)	(13,13,0)	(14,13,nan)	(15,13,nan)

(1,14,0.628256015834611)	(2,14,0.849163506028517)	(3,14,0.942208095803169)	(4,14,0.962311537733073)	(5,14,0.936513154948686)	(6,14,0.879914128436669)	(7,14,0.80179902209063)	(8,14,0.708296983893924)	(9,14,0.603678740961905)	(10,14,0.491048437946633)	(11,14,0.372739929982426)	(12,14,0.25055715796921)	(13,14,0.125926946814904)	(14,14,0)	(15,14,nan)

(1,15,0.602828690204566)	(2,15,0.830458458727289)	(3,15,0.934450716638443)	(4,15,0.966903402267895)	(5,15,0.953929912325151)	(6,15,0.910204250164967)	(7,15,0.844773148573639)	(8,15,0.763616521737829)	(9,15,0.670903438148194)	(10,15,0.569664557459257)	(11,15,0.462178197338477)	(12,15,0.350204856834806)	(13,15,0.235136442615999)	(14,15,0.11809496699829)	(15,15,0)

};


 \end{axis} \end{tikzpicture}

	\label{dfrTFDLcomp}
\end{figure} 

Similarly to the evaluations carried out in previous sections, we simulated the scores produced by DFRee given a range of TF and DL values. The objective is studying its behaviour in microbloging conditions, and draw conclusions about its performance. These simulated values are shown in Figure \ref{dfrTFDLcomp}.

As we traverse the Document Length axis we can observe an interesting behaviour which is not present in any model observed so far. 

For low values of TF, incrementing DL from 1 to $\sim16$ results in also a higher score. This behaviour aligns with the scope hypotheses as longer documents are regarded as more informative. However, when DL reaches high enough values the scores start to decline. The latter behaviour is in line with the verbose hypotheses which assumes the extra length is due to superfluous information. Particularly when the extended document length is not accompanied by higher query term frequencies.

When dealing with documents as short as microblogs it is very difficult to assert their informativeness or relevance in terms of the verbose or scope hypotheses. In fact all retrieval models observed so far follow these to some degree and perform worse than a simply using IDF as a retrieval model. Additionally, the premises in which they are built seem not to hold as they fail to perform better than simple IDF. However DFRee is an interesting exception as it performs better than all the studied retrieval models, and it performs better than IDF in some cases (Table \ref{traditional}).

We believe that the \textit{saturation point} observed in Figure \ref{dfrTFDLcomp} in terms of TF and DL is responsible for DFRee outperforming other retrieval models in this task (And sometimes IDF). The score produced by DFRee can only be higher if both TF and DL increase. Thus, incrementing the value of a single component will increase the score to a saturation point after which the score will then decrease. As an example, consider an average microblog document of length 15 (blue plane in Figure \ref{dfrTFDLcomp}). The score is maximised when TF approaches 3, after which higher TF values result in a significant reduction to the score.

This behaviour opposed to that of BM25, HLM and DLM which exhibit a positive correlation between TF and the score produced. Note that in this case a document made up of repeating query terms would be valued over others with richer, and more informative content. This behaviour is obviously problematic as it promotes spam-like documents. Fortunately DFRee has a pivoting point which attempts to alleviate this possibility, thus reducing the value of increasing TF in short documents.

Recall that users of microblog services such as Twitter, strive to fit their messages within the character limit. It stands to reason, that the more terms they fit within the character limit the higher the chances of it being informative. The pivoted behaviour of DFRee does not completely match this premise, however it does match it better than all other observed retrieval models (Including BM25, HLM and DLM) where longer documents are simply less relevant under microblog conditions.

Summarising, we believe that DFRee's behaviour is key to better understand why most retrieval models fail to capture the relevance of microblogs. Particularly important is the \textit{saturation point} behaviour as a function of TF and DL. We can observe that promoting documents that are longer, whilst penalising documents with higher TF values than 2 may be a better fit to capture microblogs' relevance.

\subsection{Harmonising Score differences}
So far we have introduced a set of representative retrieval models, and discussed how they behave when facing microblog-like conditions. We have mainly simulated the spectrum of scores produced w.r.t. TF and DL by each model when fixing all other parameters. Moreover we have observed that retrieval models performance seems to increase when we overestimate the values of TF and DL, thus forcing the models to return values of lesser score differences.


\begin{table}[]
	\caption{Behaviour when harmonising score differences.(All collections together.)}
	\centering
	\begin{tabular}{l|c|c|c} 	
		\textit{\textbf{Model}} & 
		\textit{\textbf{configuration}} & 
		\textit{\textbf{stdev}} & 
		\textit{\textbf{P@30}} 	
		\tabularnewline
		\hline
		DLM & \(c=2500\) & 0.2639 & 0.2988 \\
		DLM & \(c=50\) & 0.2479 & 0.4099 \\
		DLM & \(c=20\) & 0.2384 & 0.4241 \\
		\hline	
		HLM & \(c=0.15\) & 0.2553 & 0.3475\\
		HLM & \(c=0.40\) & 0.2365 & 0.4009\\
		HLM & \(c=0.99\) & 0.1135 & 0.4492\\
		\hline
		BM25 & \(b=0.75, k=1.2\) & 0.1274 & 0.3948\\
		BM25 & \(b=0.75, k=0.7\) & 0.0927 & 0.4399\\
		BM25 & \(b=0.9, k=0.1\) & 0.0181 & 0.4580\\
%		\hline
%		DFRee & NA & 0.2268 & 0.4614\\
		\hline
		\hline    
		PEARSON & \multicolumn{2}{|c}{-0.70}    \\ %(-0.58 including DFRee)
		KTau & \multicolumn{2}{|c}{-0.66}    \\ %(-0.56 including DFRee)
	\end{tabular}
	\label{stdevharmonising}
\end{table}

Table \ref{stdevharmonising} holds a summary of the results for all retrieval models with various configurations with respect to Precision@30. Additionally the third column holds the standard deviation of the simulated scores produced by the retrieval models in microblog conditions\footnote{where $DL<=30$ and $TF<=15$}. 

As it can be easily observed, the possible document scores are much closer together for those configurations that improve a retrieval model's performance. In fact there is a strong statistical correlation (last two columns) between reducing the standard deviation and improving the retrieval performance of the models. This observation motivates the following hypothesis:

\begin{quotation}
\begin{quote}
\textbf{The range of scores produced by retrieval models can be unfairly different due to its behaviour w.r.t. the scarcity of TF and DL values in microblog conditions.}
\end{quote}
\end{quotation}


If this hypothesis is true, we should be able to achieve similar positive results if we reduce the scoring differences of a retrieval model by means of any other technique. To this end we decided to apply a base two logarithm, to the scoring function of each retrieval model. As an example, the formulation of HLM would be as follows: 

\begin{small}
	\begin{align}
	\label{hlmformulalog}
	\text{HLM}(D,Q) &=  \sum_{i=1}^{n} \bf{log_2} \left[ \log_2 \left[ 1 + \frac{c \cdot f(q_i, D) \cdot ntoks }{ (1-c) \cdot f(q_i, C) \cdot |D|} \right] \right]
	\end{align}
\end{small}

\noindent where the added logarithm function can be found next to the summation sign.

Table \ref{loggedRMS} holds a comparison between the default P@30 achieved by each model and the same model with the log function applied to it. As we can observe the results for DLM, HLM and BM25 perform significantly better than their standard, whereas DFRee performs marginally worse and IDF remains unaffected.

From these experiments we can conclude that state of the art retrieval models produce unfair scores due to the scarcity of TF and DL during microblog retrieval. This effect can be mitigated by employing techniques to reduce possible score differences such as applying a log function. To conclude, when ranking microblog documents our models should consider the existing TF and DL evidence, but should also be conservative when managing the overall effects on the produced scores.
%
%\todo{We can conclude that based on this evidence most retrieval models are not prepared to effectively capture the relevance of microblogs. The verbose and scope hypotheses, which serves as inspiration to most retrieval models, do not hold for microblog documents. Additionally, the main reason points to their over-sensitiveness to low values of term frequency and document length. This sensitiveness often produces a high degree of score differences amongst the ranked documents which ultimately negatively affects performance.}

\begin{table}[]
	
	\caption{Retrieval models performance with log-smoothed scores (All collections)} 
	\centering
	\begin{tabular}{l|c|c|c|} 
		\multicolumn{1}{c}{}&\multicolumn{3}{|c|}{Precision @ 30} \\ 
		\cline{2- 4}
			& Default & $log_2(Ret. Model)$ & \% difference \\
		\hline
					 
		$DLM$ & 0.2988 & 0.3977 & +33.10\% \\
		$HLM$ & 0.3475 & 0.4489 & +29.18\%\\
		$BM25$ & 0.3948 & 0.4336 & +9.83\%\\
		$DFRee$ & 0.4614 & 0.4531 & -1.80\%\\
		$IDF$ & 0.4626 & 0.4626 & 0\%\\
		\hline
	\end{tabular}
	\label{loggedRMS}
\end{table}


\section{MBRM: A MicroBlog Retrieval Model}
\label{MBRM-section}
In the previous section, we discussed a number of problems faced by state of the art retrieval models when dealing with microblogs. We presented scarcity of TF and DL as a source of high scoring differences amongst the spectrum of possible scores for a retrieval model. Additionally we started defining the requirements for a retrieval model to effectively handle microblog documents by better capturing their informativeness. These requirements can be summarised as: 

\begin{enumerate}
\item Higher DL should be regarded positively as authors of microblogs strive to fit as much content as possible within the character limits
\item Higher TF should be regarded negatively as high TF could be a result of spam messages, and normally TF revolves around 1-2
\item Score differences with respect to DL and TF should produce gentle slopes, to not penalise/promote unfairly documents with very little differences.
\end{enumerate}

Following these premises, we have designed a ``MicroBlogs Retrieval Model'', namely MBRM. MBRM is composed of two parts to deal with document based evidence. Then we attach the aforementioned part to an IDF component which represents the collection's information. Similarly to the formulation of BM25, the two main components of MBRM deal with document length and query term frequency. The first component deals with the document length and is given by the following logistic distribution:

\begin{equation}
DLComp(DL)={\frac  {c_1}{1+{a_1\mathrm  e}^{{-b_1DL}}}}
\end{equation}

where \(a_1, b_1\) and \(c_1\) are parameters to control the growth, maximum and starting point of the distribution. Secondly, the following component given by a gaussian distribution deals with the effect of TF over the final score produced by MBRM:

\begin{equation}
TFComp\left(TF\right)=a_2e^{-{\frac {(TF-b_2)^{2}}{2c_2^{2}}}}
\end{equation}

where \(a_2, b_2\) and \(c_2\) are similar parameters to those found in the previous function. These functions were chosen as they offer good control over the curves, and their values can be bound between 0 and 1 thus we do not need to normalise them. The final formulation for MBRM is given by: 

\begin{equation}
MBRM(D,Q) = \sum_{i=1}^{|Q|} (1-\alpha)*\text{IDF}(q_i) + \alpha * DLComp(|D|) * TFComp(q_i)
\end{equation}

which can be also expressed as:

\begin{equation}
MBRM(D,Q) = \sum_{i=1}^{|Q|} (1-\alpha)*\text{IDF}(q_i) + \alpha * \left({\frac  {c_1}{1+{a_1\mathrm e}^{{-b_1DL(|D|)}}}} \right) * \left(a_2e^{-{\frac {(TF(q_i)-b_2)^{2}}{2c_2^{2}}}}\right) 
\end{equation}

\begin{table}[b]
	\caption{MBRM recommended parameter settings} 
	\centering
	\begin{tabular}{l|c} 	
		\hline
		\textbf{Parameter} & \textbf{Recommended values} \\
		\hline
		\centering					 
		$a_1$ & 1.5 \\
		$b_1$ & 0.3 \\
		$c_1$ & 1.0 \\
		\hline
		$a_2$ & 1.0 \\
		$b_2$ & 2.0 \\
		$c_2$ & 6.0 \\
		\hline
	\end{tabular}
	\label{recommended settings}
\end{table}

Figure \ref{microblogRM} shows a simulation of the behaviour of MBRM in terms of TF and DL. The parameters used to for both components (DLComp and TFComp) are shown in Table \ref{recommended settings}. In Figure \ref{microblogRM} we can observe how the values obtained on the TF axis decrease slowly for the initial values of TF, but rapidly accelerate in their descent to then settle near 0. This behaviour is similar to that of DFRee (Albeit smoother) in which the highest importance is given to low TF values $\sim2$ and then it is reduced. 
%High TF values are most likely than not associated with spam or unimportant documents, since actual users struggling to fit their content in the 140 characters limit are unlikely to repeat words. Although this is not always the case, thus the slow descent for low values of TF.

\begin{figure}
	\begin{subfigure}[]{0.5\textwidth}
		\caption{Doc. length (DL) and Term Frequency (TF)}
		\input{MBRM-GAUSS-TF}
		\label{microblogRM}
	\end{subfigure} 
	~
	\begin{subfigure}[]{0.5\textwidth}
		\vspace{-0.7cm}
		\caption{MBRM effects of $\alpha$ on each fold.}
		\hspace{0.5cm}
\begin{tikzpicture}[thick,scale=0.7, every node/.style={transform shape}]
\begin{axis}[
	xlabel={$\alpha$},
	ylabel={$P@30$}
]
\addplot coordinates {
	(0,0.4071)	(0.05,0.3895)	(0.1,0.3966)	(0.15,0.3975)	(0.2,0.3955)	(0.25,0.3918)	(0.3,0.3841)	(0.35,0.3751)	(0.4,0.3656)	(0.45,0.3539)	(0.5,0.3227)	(0.55,0.2764)	(0.6,0.2304)	(0.65,0.1951)	(0.7,0.1681)	(0.75,0.1434)	(0.8,0.1181)	(0.85,0.1043)	(0.9,0.0985)	(0.95,0.0905)
};

\addplot coordinates{	
	(0,0.2188)	(0.05,0.2241)	(0.1,0.2322)	(0.15,0.2324)	(0.2,0.2317)	(0.25,0.2281)	(0.3,0.2275)	(0.35,0.2253)	(0.4,0.2191)	(0.45,0.2102)	(0.5,0.197)	(0.55,0.1859)	(0.6,0.1599)	(0.65,0.1417)	(0.7,0.1324)	(0.75,0.113)	(0.8,0.1025)	(0.85,0.0908)	(0.9,0.0876)	(0.95,0.0832)
};

\addplot coordinates{
	(0,0.1814)	(0.05,0.1878)	(0.1,0.1978)	(0.15,0.2)	(0.2,0.2014)	(0.25,0.2026)	(0.3,0.2027)	(0.35,0.2023)	(0.4,0.1993)	(0.45,0.193)	(0.5,0.1859)	(0.55,0.1707)	(0.6,0.1521)	(0.65,0.1322)	(0.7,0.1187)	(0.75,0.1006)	(0.8,0.0853)	(0.85,0.0748)	(0.9,0.0716)	(0.95,0.0685)
};

\addplot coordinates{
	(0,0.3427)	(0.05,0.3334)	(0.1,0.3351)	(0.15,0.3361)	(0.2,0.3366)	(0.25,0.3366)	(0.3,0.3364)	(0.35,0.3339)	(0.4,0.3273)	(0.45,0.315)	(0.5,0.2927)	(0.55,0.2695)	(0.6,0.2312)	(0.65,0.1978)	(0.7,0.1686)	(0.75,0.1426)	(0.8,0.1141)	(0.85,0.0998)	(0.9,0.0929)	(0.95,0.0874)
};

\addplot coordinates{
		(0,0.3696)	(0.05,0.382)	(0.1,0.3856)	(0.15,0.3867)	(0.2,0.3857)	(0.25,0.3833)	(0.3,0.379)	(0.35,0.3763)	(0.4,0.367)	(0.45,0.3549)	(0.5,0.3346)	(0.55,0.3068)	(0.6,0.2717)	(0.65,0.2308)	(0.7,0.2057)	(0.75,0.1727)	(0.8,0.1506)	(0.85,0.1338)	(0.9,0.1251)	(0.95,0.1178)
};
\legend{$1$,$2$,$3$,$4$,$5$}
\end{axis}
\end{tikzpicture}
		\label{microblogRM-param}
	\end{subfigure} 
	\caption{MBRM: A Microblog Retrieval Model}
\end{figure} 

In terms of $DL$ we produce a soft increasing slope to account for increasing value assigned to more informative documents. Unlike $DFRee$, the slope is always incremental. The idea behind it being that the more terms in the microblog the more comprehensive it should be, as more information is encoded regardless of the character limitation.

In order to find the optimal value for the pondering value of $\alpha$ we divided the all the collections into 5 folds. For each of the folds we produced a P@30 result for a number of $\alpha$ values in the 0-1 range. These can be found in Figure \ref{microblogRM-param}. It can very easily be observed that the most optimal values for the mixing parameter $\alpha$ are near $0.20$.

Finally Table \ref{MBRMPerformance} shows the evaluation results obtained for MBRM in terms of Precision at different levels in comparison with IDF and DFRee. As it can be observed, the performance is always significantly superior than the baselines. The main difference with respect to IDF is obviously that it takes advantage of document statistics, where IDF does not. However the main difference with respect to DFRee is that documents longer than 15 terms are not penalised following the aforementioned rationale. 

These results not only demonstrate that we can make effective use of document statistics unlike previously thought by other authors \cite{naveed2011searching}, but also that the scope hypotheses still holds for small documents. In other words, the authors of the documents will attempt to encode as much information as possible even with the obvious document limitations. 

%This contradicts our findings in Subsection \ref{bm25case} however we believe that in the particular case of BM25, document length has a much more aggressive effect on the scores, thus resulting in a misleading behaviour.

\begin{table}[] 	
	  	  	\centering
	  	  	\caption{Performance of MBRM on all collections (Where * $p<0.05$ and ** $p<0.01$ respectively, with respect to IDF and DFRee)} 
	  	 	\begin{tabular}{l|c|c|c|c|c} 	  	 	
	  	 	\cline{2- 6}
	  	 	\multicolumn{1}{c}{}&\multicolumn{5}{c}{Precision} \\ 
	  	 	\cline{2- 6} &
	  	 	\textit{\textbf{@5}} & 
	  	 	\textit{\textbf{@10}} & 
	  	 	\textit{\textbf{@15}} & 
	  	 	\textit{\textbf{@20}} & 
	  	 	\textit{\textbf{@30}} 
	  	 	\tabularnewline
	  	 	\hline
	 	 	 DFRee  & 0.62 & 0.57 & 0.54 & 0.51 & 0.46 \\
	 	 	 IDF  & 0.62 & 0.57 & 0.53 & 0.51 & 0.46 \\
	 	 	 \hline
 	 	 	 \hline
  	  	 	 MBRM ($\alpha=0.20$)  & \textbf{0.64*} & \textbf{0.59*} & \textbf{0.56**} & \textbf{0.53**} & \textbf{0.48*} \\
	  	  	\hline
	  	  	\end{tabular}
	  	  	\label{MBRMPerformance}	
\end{table}


However, based on the statistically significantly better performance achieved by MBRM the verbose hypotheses seems not to hold, as authors capped by the character limitation. Thus documents are not generally longer due to style differences, or the verbosity of the author, but it is rather a reflection of the author's capacity to encode rich information in such limited constraints, which again aligns better with the scope hypotheses. And this is what we ultimately aimed to capture with our MBRM retrieval model.


%\input{MBInformativeness}
%
%\input{MBStructures}

\input{Conclusions}


% Bibliography
\bibliographystyle{ACM-Reference-Format-Journals}
\bibliography{bibtexshort}
                             
\received{x}{y}{z}


\end{document}
