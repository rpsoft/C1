\documentclass[prodmode,acmtecs]{acmsmall} % Aptara syntax

% Package to generate and customize Algorithm as per ACM style
\usepackage[ruled]{algorithm2e}
\renewcommand{\algorithmcfname}{ALGORITHM}
\SetAlFnt{\small}
\SetAlCapFnt{\small}
\SetAlCapNameFnt{\small}
\SetAlCapHSkip{0pt}
\IncMargin{-\parindent}
\usepackage{bibentry}
\usepackage{graphicx}
\usepackage{caption}
\usepackage{makeidx}  % allows for indexgeneration
\let\proof\relax
\let\endproof\relax
\usepackage{amsmath, amsthm, amssymb} 
\usepackage{multirow}
\usepackage{wrapfig}
\usepackage{url}
\usepackage{underscore} 
\usepackage{longtable}
\usepackage{float}
\usepackage{enumitem}
\usepackage{bibentry}
\usepackage{tikz}
\usepackage{pgf}
\usetikzlibrary{arrows,automata,patterns,shapes,arrows}
\usepackage[latin1]{inputenc}
\usepackage{verbatim}
\usepackage{pgfplots}
\usepackage{subcaption}
\pgfplotsset{compat=newest}
\usepgfplotslibrary{ternary}
\usepackage{{pgfplotstable}}
\newcommand\abs[1]{\left|#1\right|}
\newcommand\mentalnote[1]{[MentalNote:----\textcolor{blue}{#1}----]}
\captionsetup{compatibility=false}

\newcommand{\todo}[1]{\textcolor{red}{#1}}

\newcommand{\argmax}{\operatornamewithlimits{arg\,max}}

\newcommand*\circled[1]{\tikz[baseline=(char.base)]{\node[shape=rectangle,draw,inner sep=1.5pt] (char) {#1};}}


% Metadata Information
\acmVolume{9}
\acmNumber{4}
\acmArticle{39}
\acmYear{2010}
\acmMonth{3}

% Document starts
\begin{document}

% Page heads
\markboth{J. Rodriguez Perez et al.}{Microblogs Structure. Challenges and Opportunities}

% Title portion
\title{Microblogs Structure. Challenges and Opportunities}
\author{Jesus Alberto Rodriguez Perez
\affil{The University of Glasgow}
Teerapong Leelanupab
\affil{King Mongkut's Institute of Technology}
Joemon M. Jose
\affil{The University of Glasgow}
}
% NOTE! Affiliations placed here should be for the institution where the
%       BULK of the research was done. If the author has gone to a new
%       institution, before publication, the (above) affiliation should NOT be changed.
%       The authors 'current' address may be given in the "Author's addresses:" block (below).
%       So for example, Mr. Abdelzaher, the bulk of the research was done at UIUC, and he is
%       currently affiliated with NASA.

\begin{abstract}
In recent years, microblog services such as Twitter have gained increasing popularity, leading to active research on how to effectively exploit its content. Microblog documents such as `tweets' differ in morphology to more traditional documents such as web pages. Particularly, tweets are considerably shorter (140 characters) than web documents and contain contextual tags regarding the topic (hashtags), intended audience (mentions) as well as links to external content (URLs). Unfortunately, state of the art retrieval models perform rather poorly in capturing the relevance of microblogs, due to the previously unforeseen conditions in which they operate.

In this work, our main focus it to investigate the shortcomings that state of the art retrieval models suffer when dealing microblogs. To this end we simulate the behaviour of a number of state of the art retrieval models, in a microblog retrieval context. As a result of our experiments, we conclude that longer documents should be promoted over shorter documents. This is due to authors striving to fit as many terms as possible regardless of character limitations, thus producing more informative documents. On the other hand, documents with higher term frequency are deemed less valuable as they are more likely to resemble spam. Therefore, we also demonstrate that the scope hypotheses does hold for microblog documents, whereas the verbosity hypotheses does not.

Finally, based on our findings we devised a retrieval model, namely \textbf{MBRM}, which significantly outperforms the state of the retrieval models, by better capturing the informativeness of microblog documents.

%Our evaluation results show statistically significant improvements over the baseline in terms of precision at different cut-off points for both approaches. These results confirm that the relative presence of the different dimensions within a document and their ordering are connected with the relevance of microblogs.

%Furthermore we look at microblog documents as a high-dimensional entity and study the structural differences between those documents which are deemed relevant against those non-relevant. Moreover we leverage these statistical differences in experiments to enhance the behaviour of retrieval models. Additionally we study the interactions between the different dimensions in terms of their order within the documents by modelling relevant and non-relevant tweets as state machines. These state machines are then utilised to produce scores which in turn are used for re-ranking. 

\end{abstract}

\category{C.2.2}{Computer-Communication Networks}{Network Protocols}

\terms{Microblog, State machines, Classification}

\keywords{Information Retrieval, Structural Models, Document Dimentions}

\acmformat{Jesus Rodriguez et al., 2017. Microblogs Structure. Challenges and Opportunities.}
% At a minimum you need to supply the author names, year and a title.
% IMPORTANT:
% Full first names whenever they are known, surname last, followed by a period.
% In the case of two authors, 'and' is placed between them.
% In the case of three or more authors, the serial comma is used, that is, all author names
% except the last one but including the penultimate author's name are followed by a comma,
% and then 'and' is placed before the final author's name.
% If only first and middle initials are known, then each initial
% is followed by a period and they are separated by a space.
% The remaining information (journal title, volume, article number, date, etc.) is 'auto-generated'.

\begin{bottomstuff}
%This work is supported by the National Science Foundation, under
%grant CNS-0435060, grant CCR-0325197 and grant EN-CS-0329609.
%
%Author's addresses: G. Zhou, Computer Science Department,
%College of William and Mary; Y. Wu  {and} J. A. Stankovic,
%Computer Science Department, University of Virginia; T. Yan,
%Eaton Innovation Center; T. He, Computer Science Department,
%University of Minnesota; C. Huang, Google; T. F. Abdelzaher,
%(Current address) NASA Ames Research Center, Moffett Field, California 94035.
\end{bottomstuff}

\maketitle

%!TEX root = JournalChapter1.tex
\section{Introduction}
\label{introduction}

Microblogs have grown in popularity in recent years, gradually transforming the way we find out about the latest events and communicate. Twitter is the most prominent service \footnote{\url{https://twitter.com/}}, as it is used by millions, posting over 340 million tweets every day\footnote{\url{http://blog.twitter.com/2012/03/twitter-turns-six.html}}. Microblog services are used for various purposes including: (i) self promotion, (ii) advertising, (iii) real-time news broadcasting, (iv) social discussions etc. The most important aspect of Twitter is that it provides unique insight into real-time events, such as first hand reports of events as they are developing, along with the opinion of those discussing them.
This information makes Twitter a uniquely valuable media source, which led to obtaining much attention by research and industrial communities.

Retrieving documents in Twitter can be extremely challenging because of their morphology. The content is limited to 140 characters per messages (known as \emph{Tweets}). This constraint leads to varied linguistic quality \cite{teevan2011twittersearch} due to colloquialisms and users efforts to fit their content within the limitations. More importantly tweets pose new challenges for which state of the art retrieval models were not designed for\footnote{Models such as: Okapi BM25 \cite{robertson2009probabilistic}; Divergence From Randomness (DFR) \cite{amati2003probabilistic}; Hiemstra's Language Model (HLM) \cite{model}; and Dirichlet Language Model (DLM) \cite{zhai2001study}}. 

%To the best of our knowledge, although some work has been done in this area \cite{singhal1996pivoted} \cite{naveed2011searching} nobody has reliably assessed the effect that these constraints actually have on the current relevance assumptions in which retrieval models are based.
%\cite{singhal1996pivoted}
Whilst few recent works have identified some features as possibly being detrimental in microblog ad-hoc retrieval \cite{naveed2011searching}, no study has been carried out to determine the concrete effect of the different features on state of the art retrieval models. 
%The main objective of this work is not the improvement over the very best scores achieved for each of the microblog collections taking into consideration all available approaches to the day. 
Therefore we are set to investigate the connection of the structure of microblog documents with their relevance during an ad-hoc search task. This whole work revolves around the following main question: 

\begin{quotation}\begin{quote}\textbf{What are the reasons behind the underperformance of state of the art retrieval models in the context of microblogs? And what can we do about it?}\end{quote}\end{quotation}

To this end, firstly we observe the performance of state of the art retrieval models in the context of Twitter corpora selecting the best retrieval model as a baseline. Then we perform a series of experiments which simulate the behaviour of a number of state of the art retrieval models in order to identify possible shortcomings in their design with respect to microblog documents. This initial experiment is completed with the creation of a retrieval model which takes into account all previous findings, namely MBRM. MBRM demonstrates that the scope hypotheses still holds within microblog documents, and that microblog document statistics can be leveraged to significantly improve ad-hoc retrieval performance.

%Secondly, we study the behaviour of inherent features of microblog documents and evaluate their suitability for enhancing the behaviour of state of the art retrieval models. Moreover, we demonstrate which microblog specific features are most indicative of the relevance of microblogs by reporting statistically significantly improved retrieval performance for ad-hoc search when taking them into account.
%
%Finally, we extend our analysis by considering the ordering of the different component that make up microblog documents. In order to do so, we encode the structure of observed relevant and non-relevant documents into state machines. These are in turn used to produce scores for re-ranking. We utilise the 2013 microblog collection to construct such state machines, and we test on the 2011 and 2012 microblog collections combined. Our results show statistically significantly improved results over the selected baseline, demonstrating the connection of microblog structure with relevance.

This work will be driven by the following research questions:
%With the objective of studying microblog document's structures, we set the focus of this work in the context of these research questions: 

\begin{itemize}
%
%\item[] \textbf{RQ1.} Are there structural differences between relevant and non-relevant microblog documents? Can we exploit them for ad-hoc retrieval?

\item[] \textbf{RQ1.} What is the role of document length in connection with the informativeness of microblog documents?
\item[] \textbf{RQ2.} Does term frequency of query terms relate to the informativeness of microblogs? 
\item[] \textbf{RQ3.} Can we adapt state of the art retrieval models to better handle microblogs?
\item[] \textbf{RQ4.} Can we devise a retrieval model to better capture the relevance of microblogs?

%\item[] \textbf{RQ2.} Can microblog features be exploited to help retrieval models better capture relevance than current retrieval models?
%\item[] \textbf{RQ3.} Is the order of the different elements in a microblog document connected with relevance? Can it be utilised for ad-hoc retrieval?

\end{itemize}

The rest of the work is organised as follows. First, we cover literature relevant to microblog retrieval and other concepts utilised throughout this work (Section \ref{background}). Section \ref{experiment} sets the evaluation environment in which our experimentation is carried out, giving way to our analysis (Section \ref{RMinvestigation}). Section \ref{MBRM-section} then introduces our MBRM retrieval model, whereas Section \ref{conclusion} concludes the work and points to future research directions.



%!TEX root = JournalChapter1.tex
\section{Background}
\label{background}
In this Section we will introduce concepts and related literature to this work.

\subsection{Retrieval of Microblogs is Hard}
Retrieval models are designed to rely on term frequency and document length as the variables to quantify whether a document is more important than other. From a very simplified perspective, a retrieval model will give more importance to a document that contains query terms more frequently than another document (Assuming similar document lengths). Likewise, when query terms appear the same number of times, a document will be deemed less or more informative based on the document lengths.

However, as stated before, microblog documents are limited in length to 140 characters in the case of Twitter. This limitation obviously challenges the abovementioned assumptions, which unfortunately form the basis of the workings of most retrieval models in a way or another.

The new medium and the low retrieval performance achieved by state of the art retrieval models gave way to an extensive area of research spearheaded by the Text Retrieval Conference (TREC) through its microblog track. Over recent years, numerous approaches have been proposed which significantly improve retrieval performance in diverse ways. \\

\subsubsection{Microblog retrieval tracks}
TREC organised a number of tracks over four consecutive years 2011-2014 in order to organise the research community and jointly address this retrieval problem. In order to evaluate the performance of the prospective solutions and allow for comparability they agreed on a collection of documents and a set of topics, as well as relevance judgements on those topics provided by NIST obtained through pooling.

To this end they sampled two collections of documents from a Twitter stream over two different periods of time. The first collection was gathered in 2011 but was used for during both the 2011 and 2012 microblog tracks. Similarly, the second collection was gathered in 2013 and was used for both the 2013 and 2014 microblog tracks. Finally the number of topics varied between 50 and 60, but are 225 in total.


\subsubsection{Microblog Tracks Results}
The summary results for each of the tracks are presented in Table \ref{summarytrec} for reference. Amongst the top performing participants we can find ~\cite{amati2011fub,li2011pris,metzler2011usc} for microblog 2011 and ~\cite{kimovercoming,younosFreq,hanhit} for 2012, which mostly employed query and document expansion techniques as well as learning to rank (L2R) approaches.
Additionally, the 2013 track followed a similar trend producing works in the same categories L2R \cite{pris2013,gaoictnet}, query expansion \cite{prebjut,perezuniversity} and document expansion \cite{jabeuririt}.

Moreover, the work by \cite{Damak:2013:ESF:2480362.2480537} produced a comprehensive summary of the features used by different approaches, and demonstrated how to successfully combine them using naive bayes as an L2R approach combining a number of features including hashtags, mentions, url presence, recency, etc.


\begin{table}[]
	\centering
	
	\caption{TREC Tracks results in terms of precision@30}
	
	\begin{tabular}{|c|c|c|c|c|c|c|c|}
		\hline
		\multicolumn{2}{|c|}{2011} &\multicolumn{2}{|c|}{2012} & \multicolumn{2}{|c|}{2013} & \multicolumn{2}{|c|}{2014} \\
		\hline
		Best & Median & Best & Median &	Best & Median &	Best & Median \\
		0.502 & 0.298 & 0.470 & 0.362 & 0.560 & 0.370 & 0.722 & 0.629 \\
		\hline
		
	\end{tabular}
	
	\label{summarytrec}
	\vspace{0.5cm}
\end{table}

%\noindent {\bf Microblog Retrieval Issues.} 
Work by \cite{thomassearching} studied the effects that preprocessing had on retrieval performance. Their findings showed that the best performance was achieved when applying all preprocessing steps, which include (i) language detection, (ii) Emotion removal, (iii) Lexical normalization, (iv) Mention Removal and (v) Link Removal. 

Moreover, works by \cite{ferguson2012investigation,naveed2011searching} have identified that problems affecting retrieval models in microblogs are related to \textit{term frequency} and \textit{document length normalization}.

However, no significant effort has been paid to understanding \textit{why are retrieval models really failing}. Reduced document length, and reduced term frequencies are often loosely blamed as we lack solid evidence of their interaction with the retrieval model estimations. We believe it is important to explore the interaction of such features within the retrieval models, not only because the better understanding would lead to better retrieval models, but also the domino effect it would produce on techniques which rely on retrieval models that could also benefit (E.g. Automatic Query Expansion).\\

\noindent \textbf{Probability of Relevance Framework}. For many years researchers have developed their understanding on estimating the relevance of documents, thus leading to many works with many definitions. One of the most representative works in this area of research is the Probability of Relevance Framework (PRF) \cite{roelleke2013information}. PRF is given by \(P(r|\hat{d},q)\), where \(r\) refers to relevance, \(q\) a given query and \(\hat{d}\) represents a document as a vector of features \(\hat{d} = (f_1,...f_n)\). Note that vector features can be any imaginable data. The main importance of this framework is the formalisation of relevance w.r.t. a document which is the basis of numerous research works.\\

\noindent \textbf{Document length normalization} \cite{singhal1996pivoted} has been employed by retrieval models to counterbalance the effects of longer documents, which may not necessarily add any new information to a topic, but are prone to contain higher term frequencies. 
%An important study connecting document length with relevance was that by \cite{singhal1996pivoted}. In their work it was  concluded that retrieval performance improves if documents are given similar chances regardless of their document length. In fact most retrieval models deal with document length in order to not over-promote documents that contain higher term frequencies as a result of their length. 
In line with this effort, the design of BM25 by \cite{robertson2009probabilistic} involved the study of document characteristics, resulting in the definition of the \textbf{scope} and \textbf{verbosity} hypotheses. The \textbf{verbosity} hypotheses supports that some authors are more verbose than others, thus applying length normalization by dividing by the length of the document is beneficial to better capture relevance, as repetition of terms is superfluous. On the other hand, the \textbf{scope} hypotheses states that some authors simply have more to say, thus adding more relevant information to the topic and occupying more space. BM25 applies a soft normalisation that takes into account both cases.\\ %These hypotheses are of utmost importance as they highlight the reasons behind longer documents.\\




%Moreover, work by \cite{ferguson2012investigation} studied the effects of query length normalization and term frequency on the BM25 retrieval model whereas \cite{naveed2011searching} looked at these issues from a more generic perspective. Both their findings showed how query length normalization had an undesirable effect on retrieval performance of the systems they were evaluating, since document length in microblogs does not have the same meaning as in other context.

%, which would answer why the results we obtained for BM25 runs are consistently worse compared to IDF and DFR.

%The literature has identified two main culprits which negatively affect performance of modern retrieval models in Microblogs, that is: \textbf{term frequency} and \textbf{document length normalization} . As we can observe in Table \ref{modelfeatures} BM25 has a heavy realiance on document length as it uses both ADL and DL components for its computation. The study by  \\

%\noindent {\bf Other retrieval features.} The use of temporal evidences in conjunction with other features such as geographical locations has been studied by \cite{lappas2012spatiotemporal} and \cite{6222198} following different approaches. In addition to the temporal dimension, 

\noindent \textbf{Making Sense Of Microposts.} The MSM workshop \cite{basave2013making} presented participants with a challenge. The objective was to build systems able to identify and extract concepts from microblog documents, in a semi-supervised manner. The participant systems were to categorise concepts as belonging to the categories: person, organisation, location and miscellaneous. A similar task is that of microblog summarisation \cite{5590862} in that tweets have to be processed and made sense of in order to produce a richer representation.

Amongst the works submitted to this workshop, we can highlight the work by Tao, Ke et al. \cite{tao2012makes}. In their work they perform an in depth analysis of both topic dependent and independent features for the MSM task. Some of the topic independent features consider the presence of hashtags, urls and the length of the documents to be in connection with the relevance of documents. In our work, we pay attention to the same features, but from a different angle, by looking how much space relative to the total characters in the document is dedicated to each of the microblogs elements. \\


\noindent {\bf Other Microblog retrieval features.} 
Work by \cite{massoudi2011incorporating} explored the use of other features to improve ad-hoc retrieval. These features include emoticons, hyperlinks, shouting, capitalization, retweets and followers. 
Work by \cite{nagmoti2010ranking} extended the study concerning the use of social features such as the number of followers and followees to enhance ad-hoc retrieval performance.
%Weng et al. \cite{EventDetection} proposed that clusters of features which show bursty behaviour in close temporal proximity suggest an event. Their system builds signals for individual features using wavelet analysis for each of the terms. Events are then formed by clustering terms with similar behaviour. 
While all these works attempt to exploit some microblog features or augment them with external resources, they do not try to explain how these features relate to the relevance of microblog documents. In our work, we consider features based purely on microblog characteristics, explain their relationship with relevance, and finally use those features that seem beneficial to improve the behaviour of a state of the art retrieval model.


%!TEX root = JournalChapter1.tex
\section{Experimental Setting}
\label{experiment}

\noindent {\bf Datasets.} In this evaluation we have used the four collections from the TREC Microblog track. The 2011 and 2012 collections share the same corpus but have different topics and relevance assessments. On the other hand the 2013 and 2014 collections share the same corpus. The later corpus is an order of magnitude bigger than previous collections. However, the 2013 and 2014 relevance assessments are statiscally comparable to the 2012 track. Moreover, the ratio of documents \(\frac{relevant}{non-relevant}\) is much higher for the 2013, which can result in generally better retrieval performance than previous tracks by default. The 2014 on the other hand is closer in this ratio to the 2012 collection. In fact it has a considerably lower number of relevant documents per topic.

In total there are 225 topics with query lengths ranging from 2 to 3 tokens, in line with the literature~\cite{teevan2011twittersearch}. Refer to Table \ref{collections} for an extended overview of these collections. \\


%\noindent{\bf Parameter setting.} We performed parameter optimization w.r.t Precision@30 for all the retrieval models with free parameters. In Table \ref{traditional} we show the results for all considered retrieval models using the most optimal parameters across the three collections.\\





\begin{table}[]



\caption{Descriptive statistics for the collections being used in this study}

 

  	\centering

   \begin{tabular}{l||c|c|c|c} 



    \hline

    

TREC Microblog track collection year & 2011 & 2012 & 2013 & 2014 \tabularnewline



	\hline

Number of topics   & 50 & 60 & 60 & 55 \tabularnewline

        \hline

\# documents & \multicolumn{2}{c|}{16M} & \multicolumn{2}{c}{260M}  \tabularnewline

        \hline

\# assessed documents   & 40855 &  73073  & 71279 & 57985 \tabularnewline

        \hline

\# assessed non-relevant documents & 38124 & 66893 & 62268 & 47340 \tabularnewline

	\hline

\# assessed relevant documents  & 2731 &  6180 & 9011 & 4753\tabularnewline

	\hline

 Ratio \(\frac{Relevant~Docs}{Non-Relevant~Docs}\) & 0.07 &  0.09 & 0.14 & 0.10\tabularnewline

	\hline

 Avg. relevant documents per topic   & 58.45 &  106.54 & 150.18 & 79.22 \tabularnewline

	\hline

   \end{tabular}


   \label{collections}

     \vspace{0.5cm}

 \end{table}

%!TEX root = JournalChapter1.tex
\begin{table*}[]
\caption{Evaluation results for the state of the art models considered. (Bold denotes the best performing system)}
\begin{small}


	\begin{subtable}[b]{0.44\textwidth}

	\caption{2011 collection} 

	\begin{tabular}{l|c|c|c|c|c} 	
	\cline{2- 6}
	\multicolumn{1}{c}{}&\multicolumn{5}{c}{Precision} \\ 
	\cline{2- 6} &
	\textit{\textbf{@5}} & 
	\textit{\textbf{@10}} & 
	\textit{\textbf{@15}} & 
	\textit{\textbf{@20}} & 
	\textit{\textbf{@30}} 	
	\tabularnewline
	\hline
	BM25 & 0.54 & 0.48 & 0.45 & 0.41 & 0.38\\
	DFRee & 0.61 & \textbf{0.58} & \textbf{0.54} & \textbf{0.50} & 0.45\\
	DLM & 0.50 & 0.47 & 0.45 & 0.42 & 0.37\\
	HLM & 0.54 & 0.48 & 0.45 & 0.42 & 0.38\\
	IDF & \textbf{0.63} & 0.56 & 0.52 & 0.49 & \textbf{0.46}\\
	\hline	
	\end{tabular}
	 \end{subtable}
	 \hspace{2em}
 	\begin{subtable}[b]{0.44\textwidth}
 	\caption{2012 Collection} 
	\begin{tabular}{l|c|c|c|c|c} 
	
	\cline{2- 6}
	\multicolumn{1}{c}{}&\multicolumn{5}{c}{Precision} \\ 
	\cline{2- 6} &
	\textit{\textbf{@5}} & 
	\textit{\textbf{@10}} & 
	\textit{\textbf{@15}} & 
	\textit{\textbf{@20}} & 
	\textit{\textbf{@30}} 
	
	\tabularnewline
	\hline
BM25 & 0.40 & 0.37 & 0.34 & 0.34 & 0.31 \\
DFRee & \textbf{0.46} & \textbf{0.45} & \textbf{0.42} & \textbf{0.39} & \textbf{0.36}\\
DLM & 0.34 & 0.33 & 0.32 & 0.29 & 0.27\\
HLM & 0.38 & 0.37 & 0.35 & 0.33 & 0.31\\
IDF & 0.44 & 0.39 & 0.36 & 0.36 & 0.34\\
 	\hline
 	
 	\end{tabular}
 
 	 \end{subtable} \vspace{0.6cm}

 	\begin{subtable}[b]{0.44\textwidth}
 	\caption{2013 collection} 

	\begin{tabular}{l|c|c|c|c|c} 
	
	\cline{2- 6}
	\multicolumn{1}{c}{}&\multicolumn{5}{c}{Precision} \\ 
	\cline{2- 6} &
	\textit{\textbf{@5}} & 
	\textit{\textbf{@10}} & 
	\textit{\textbf{@15}} & 
	\textit{\textbf{@20}} & 
	\textit{\textbf{@30}} 
	
	\tabularnewline
	\hline
BM25 & 0.58 & 0.51 & 0.46 & 0.42 & 0.38 \\
DFRee & \textbf{0.67} & 0.60 & 0.55 & 0.51 & \textbf{0.45} \\
DLM & 0.27 & 0.28 & 0.26 & 0.26 & 0.24 \\
HLM & 0.44 & 0.38 & 0.35 & 0.33 & 0.31 \\
IDF & 0.66 & \textbf{0.62} & \textbf{0.56} & \textbf{0.52} & \textbf{0.45} \\

 	\hline
 	\end{tabular}
 
 	\end{subtable}  
 		 \hspace{2em}
 	  	\begin{subtable}[b]{0.44\textwidth}
 	 
 	  	\centering
 	  	\caption{2014 collection} 
 	 
 	 	\begin{tabular}{l|c|c|c|c|c} 
 	 	
 	 	\cline{2- 6}
 	 	\multicolumn{1}{c}{}&\multicolumn{5}{c}{Precision} \\ 
 	 	\cline{2- 6} &
 	 	\textit{\textbf{@5}} & 
 	 	\textit{\textbf{@10}} & 
 	 	\textit{\textbf{@15}} & 
 	 	\textit{\textbf{@20}} & 
 	 	\textit{\textbf{@30}} 
 	 	
 	 	\tabularnewline
 	 	\hline
	 	 BM25 & 0.69 & 0.62 & 0.58 & 0.57 & 0.52 \\
	 	 DFRee & 0.73 & 0.68 & 0.65 & 0.63 & 0.60 \\
	 	 DLM & 0.35 & 0.35 & 0.34 & 0.34 & 0.33 \\
	 	 HLM & 0.55 & 0.49 & 0.46 & 0.44 & 0.41 \\
	 	 IDF & \textbf{0.75} & \textbf{0.73} & \textbf{0.69} & \textbf{0.67} & \textbf{0.62} \\
 	 
 	  	\hline
 	  	\end{tabular}
 	  	
 	  	%\end{center}
 	  	%\end{footnotesize}
 	  	
 	  	 \end{subtable}

 	 \vspace{0.5cm}
  	  	\begin{subtable}[b]{0.44\textwidth}
 	  	 	
 	  	  	\centering
 	  	  	\caption{All collections} 
 	  	 
 	  	 	\begin{tabular}{l|c|c|c|c|c} 
 	  	 	
 	  	 	\cline{2- 6}
 	  	 	\multicolumn{1}{c}{}&\multicolumn{5}{c}{Precision} \\ 
 	  	 	\cline{2- 6} &
 	  	 	\textit{\textbf{@5}} & 
 	  	 	\textit{\textbf{@10}} & 
 	  	 	\textit{\textbf{@15}} & 
 	  	 	\textit{\textbf{@20}} & 
 	  	 	\textit{\textbf{@30}} 
 	  	 	
 	  	 	\tabularnewline
 	  	 	\hline
 	 	 	 BM25 & 0.55 & 0.49 & 0.46 & 0.43 & 0.39 \\
 	 	 	 DFRee  & \textbf{0.62} & \textbf{0.57} & \textbf{0.54} & \textbf{0.51} & \textbf{0.46} \\
 	 	 	 DLM  & 0.36 & 0.35 & 0.34 & 0.32 & 0.30 \\
 	 	 	 HLM  & 0.47 & 0.43 & 0.40 & 0.38 & 0.35 \\
 	 	 	 IDF  & \textbf{0.62} & \textbf{0.57} & 0.53 & \textbf{0.51} & \textbf{0.46} \\
 	  	 
 	  	  	\hline
 	  	  	\end{tabular}
 	  	  	
 	  	  	%\end{center}
 	  	  	%\end{footnotesize}
 	  	 
 	  	  	%\vspace{-0.50cm}
 	  	  	
 	\end{subtable}
 	 
 	 
  	\label{traditional}
  	  \vspace{0.7cm}
  	  \end{small}
 \end{table*}

\noindent{\bf Evaluation measures.} We pay attention to precision at different ranks, with a maximum cut-off point at rank 100. Future evidence is accepted only at the collection statistics level as agreed by TREC organisers disregarding any documents after the query issuing time when computing evaluation measures \footnote{https://github.com/lintool/twitter-tools/wiki/TREC-2013-Track-Guidelines}.\\

\noindent{\bf Baseline selection.} Table \ref{traditional} contains evaluation results for the considered state of the art retrieval models when applied to Twitter corpora from the 2011, 2012 and 2013 Trec microblog collections. The models considered in this evaluation are TF-IDF (IDF)\footnote{\(Where~TF=1.\) Results worsen considerably if we do not set TF to a constant.}, BM25, DFRee, Hiemstra's LM (HLM) and Dirichlet's LM (DLM) since it was the baseline for the Microblog Tracks in 2013 and 2014. Moreover, we adhere to the implementation and default settings found within the Terrier IR platform~\cite{ounis2005terrier}. Finally, since DFRee and IDF are generally the best performing models we will use them as our main point of reference.

%In the literature, models such as BM25 or Hiemstra's Language Model(HLM) are known to outperform simpler models such as TF-IDF \cite{robertson1995okapi}. 

%It is interesting to observe how these models seem to behave unexpectedly and are more often than not outperformed by the simpler IDF. An exception is DFR which remains amongst the best performing models across all Twitter collections, as shown in Table \ref{traditional}. 

%



%!TEX root = JournalChapter1.tex
\section{Investigating Retrieval Model Problems}
%In this section we will study problems in state of the art retrieval models when utilised in the context of microblogs.
The literature has identified \textbf{document length normalization} as the main culprit for the under-performance of retrieval efforts in microblogs. The work by \cite{naveed2011searching} suggests that the \textbf{Verbosity} and \textbf{Scope} hypotheses do not hold for microblog retrieval.

The \textbf{verbosity} hypothesis supports that some authors are more verbose than others, thus applying length normalization by dividing by the length of the document is beneficial to better capture relevance, as repetition of terms is superfluous. On the other hand, the \textbf{scope} hypotheses states that some authors simply have more to say, thus naturally adding more relevant information to the topic. As a result documents are longer but more extensive and rigorous in their content than shorter ones. The added value should be accounted for and thus the documents should promoted over shorter ones should not be normalised w.r.t their length.

In the context of Microblog retrieval, \cite{naveed2011searching} carried out a number of experiments using a logistic regression model over a number of tweet features as the retrieval methodology. They showed significant improvements in performance when their algorithm did not perform document length normalization over its normalised counterpart. However, since in their work their ranking approach takes into consideration multiple other features, it is not clear if their finding about document length normalization is generalisable. Furthermore, although it is been often assumed, it is not known if length normalisation is bad altogether for microblog retrieval, or maybe is just how it is interpreted in this particular case what makes it harmful.

Intuition makes us believe that document length normalization is in direct conflict with the limitations which characterise microblogs. The \textbf{Verbosity} and \textbf{Scope} hypotheses seem not to hold as users strive to contain their messages and content within the character limit. Consequently, retrieval models designed under scope and verbosity or similar premises, such as BM25 \cite{robertson2009probabilistic}, are likely to exhibit unexpected behaviour.

Our first step in order to help in the understanding of the behaviour of retrieval models with respect to microblogs, is to observe their composition. To this end we have compiled Table \ref{modelfeatures}. This table shows the different components involved in the score computation of a variety of retrieval models. The top row of the table indicates whether the component is measured from collection statistics (I.e. Collection feature) or the document itself (Document feature). The second row contains acronyms for each of the features, which are expanded as: 

\begin{itemize}
\item [$\bullet$] \textbf{AverageDocumentLength (ADL):} This is the average document length, in number of tokens, for the whole collection.
\item [$\bullet$] \textbf{DocumentLength (DL):} This is the document length, in number of tokens, for the document being scored.
%% what about key frequency (shouldnt matter, as terms appear only once in the queries, but good to mention)
%\item \textbf{KeyFrequency (KF):} This is the term frequency of the query term within the query itself.
\item [$\bullet$] \textbf{NumberOfDocuments (ND):} Total number of documents in the collection. 
\item [$\bullet$] \textbf{DocumentFrequency (DF):} Number of documents in which the term appears (I.e. A term's posting list size).
\item [$\bullet$] \textbf{NumberOfTokens (NT):} Number of different tokens in the collection.
\item [$\bullet$] \textbf{CollectionTermFrequency (CTF):} Frequency of a term in the whole collection. (I.e. Total number of occurences of a term in the collection)
\item [$\bullet$] \textbf{TermFrequency (TF):} Frequency of the term in the document being evaluated.
\end{itemize}

\begin{table}[h!]
	\caption{Features involved in the computation of retrieval models.}
	\centering
	\begin{tabular}{|l|c|c|c|c|c||c|c|} 
		\cline{2- 8}
		\multicolumn{1}{c|}{}& \multicolumn{5}{c||}{Collection Features} &  \multicolumn{2}{c|}{Document Features} \tabularnewline
		\cline{2- 8}
		\multicolumn{1}{c|}{}
		& \textit{\textbf{ND} } & \textit{\textbf{DF} } & \textit{\textbf{ADL} } & \textit{\textbf{NT} } 
		& \textit{\textbf{CTF} } & \textit{\textbf{TF} } & \textit{\textbf{DL} } \tabularnewline \hline
		\textit{IDF} 	& * &  *&  	&   &  	&	&   \tabularnewline \hline
		\textit{DFRee} 	&   &   &   & * & * &*	&* \tabularnewline \hline
		\textit{BM25}	& * &  *& * &   &  	&*	&* \tabularnewline \hline
		\textit{HLM} 	&   &   &  	& * & * &*	&* \tabularnewline \hline
		\textit{DLM} 	&   &   &  	& * & * &*	&* \tabularnewline \hline
	\end{tabular}
	\label{modelfeatures}
\end{table}

Each of the remaining rows contain the name of the retrieval model as well as the components involved in its computation. For example, DFRee uses NumberOfTokens (NT), CollectionTermFrequency (CTF), TermFrequency (TF) and Document Length (DL).
This table will aid in investigating the behaviour of each retrieval model in the context of microblog retrieval conditions.


\subsection{The BM25 Case}
\label{bm25case}
The work by \cite{ferguson2012investigation} examined the performance of BM25 when used under a microblog retrieval scenario. Their findings showed how the closer to zero the free parameters were set in BM25, the better the performance achieved. However, they did not connect this finding to the design of BM25 and what these settings meant in terms of the affected components. In this section we exemplify and connect these findings to the theory by simulating the behaviour of BM25 under microblog retrieval conditions.

% and examine how they affect other retrieval models aside from BM25.
First, we observe in Table \ref{modelfeatures} how BM25 relies on document length by using both ADL and DL components in its computation. Furthermore, BM25 has two free parameters, namely \(b\) and \(k_1\), which control the effects of the ``saturation function'' over the final score. The saturation function in BM25 encodes the document length evidence as part of the score as follows: 

The first version of the saturation function is given by:

\begin{equation}
 \text{Version 1: }\frac{f(q_i, D)}{f(q_i, D) + k_1} \text{   for some k_1 $>$ 0}
\end{equation}

Once we take into consideration the Verbosity and Scope hypotheses, we derive the following saturation function:

\begin{equation}
 \text{Version 2: }\frac{f(q_i, D)}{f(q_i, D) + k_1*((1-b)+b*dl/avdl)} \text{   for some k_1 $>$ 0}
\end{equation}

The main difference between these equations is that \textbf{Version 2} reduces the effect of term frequency with respect to the document length and its collection average, whilst \textbf{Version 1} only relies on the \(k_1\) free parameter. Secondly, the free parameter \(b\) ponders between the Verbosity and Scope hypotheses. Setting \(b\) to 0 effectively disables the Verbose hypothesis, giving full weight to Scope, in other words, the longer the document the better. Thus when \(b\) is set to 0, \textit{Version 2} of the saturation function becomes \textit{Version 1}.

As we introduced before, the study carried by \cite{ferguson2012investigation} explored the best parameters for \(b\) and \(k_1\) concluding that best performance is achieved as both parameters tend to 0. However, the authors did not mention is that by setting those parameters close to 0, we are disregarding the document length normalisation component altogether. Thus for all intents and purposes BM25 becomes IDF. This can be proved mathematically by substituting \(b\) and \(k_1\) by 0 as follows \ref{bm25proof}.

\begin{small}
\begin{align}
\label{bm25proof}
    \notag \text{BM25}(D,Q) &= \sum_{i=1}^{n} \text{IDF}(q_i) \cdot \frac{f(q_i, D) \cdot (k_1 + 1)}{f(q_i, D) + k_1 \cdot (1 - b + b \cdot \frac{|D|}{\text{avgdl}})} \\
  \notag&= \sum_{i=1}^{n} \text{IDF}(q_i) \cdot \frac{f(q_i, D) \cdot (0 + 1)}{f(q_i, D) + 0 \cdot (1 - 0 + 0 \cdot \frac{|D|}{\text{avgdl}})} \\
  \notag&= \sum_{i=1}^{n} \text{IDF}(q_i) \cdot \frac{f(q_i, D)}{f(q_i, D) } \\
  &= \sum_{i=1}^{n} \text{IDF}(q_i)              
\end{align}
%\end{proof}
\end{small}

We can reach an important conclusion from this proof. The \textbf{Scope} and \textbf{Verbosity} hypotheses do not seem to hold when BM25 works with microblogs. These hypotheses were developed for documents that were unbounded in terms of their length such as web pages or books. However, since document length has an upper bound in microblogs, authors express their ideas in a very constrained space where verbosity and scope hypotheses do not seem to hold.

Furthermore, terms in microblog documents have very low document frequencies. In fact, more often than not, query terms appear at most once in each document. Thus a query term appearing more than once within a document can have a dramatic effect over the score produced by BM25. In other words, the very low document frequencies result in unreliable estimations of the informativeness of a query term. Consequently, in this particular case, it is better to rely on features outside the document such as collection features.

\begin{figure}[]
  \centering
   
\begin{tikzpicture}[thick,scale=0.7, every node/.style={transform shape}]\begin{axis}[
 %title={},
 %y dir=reverse, 
 x dir=reverse, 
 ylabel={docLength (DL)},
 xlabel={term frequency (TF)},
 zlabel={BM25 score},
 every axis/.append style={font=\large\bfseries},
 max space between ticks=25pt
 view={55}{80}
% yticklabels={0k,100k}
 ] 

\addplot3[surf,unbounded coords=jump] coordinates { 
%patch,patch type=biquadratic, shader=faceted,patch refines=3
(1,1,0.865013774104683)	(2,1,nan)	(3,1,nan)	(4,1,nan)	(5,1,nan)	(6,1,nan)	(7,1,nan)	(8,1,nan)	(9,1,nan)	(10,1,nan)	(11,1,nan)	(12,1,nan)	(13,1,nan)	(14,1,nan)	(15,1,nan)

(1,2,0.840696117804551)	(2,2,0.932442464736451)	(3,2,nan)	(4,2,nan)	(5,2,nan)	(6,2,nan)	(7,2,nan)	(8,2,nan)	(9,2,nan)	(10,2,nan)	(11,2,nan)	(12,2,nan)	(13,2,nan)	(14,2,nan)	(15,2,nan)

(1,3,0.817708333333333)	(2,3,0.91812865497076)	(3,3,0.957317073170732)	(4,3,nan)	(5,3,nan)	(6,3,nan)	(7,3,nan)	(8,3,nan)	(9,3,nan)	(10,3,nan)	(11,3,nan)	(12,3,nan)	(13,3,nan)	(14,3,nan)	(15,3,nan)

(1,4,0.79594423320659)	(2,4,0.904247660187184)	(3,4,0.947209653092005)	(4,4,0.970258787176516)	(5,4,nan)	(6,4,nan)	(7,4,nan)	(8,4,nan)	(9,4,nan)	(10,4,nan)	(11,4,nan)	(12,4,nan)	(13,4,nan)	(14,4,nan)	(15,4,nan)

(1,5,0.775308641975309)	(2,5,0.890780141843972)	(3,5,0.937313432835821)	(4,5,0.962452107279693)	(5,5,0.978193146417446)	(6,5,nan)	(7,5,nan)	(8,5,nan)	(9,5,nan)	(10,5,nan)	(11,5,nan)	(12,5,nan)	(13,5,nan)	(14,5,nan)	(15,5,nan)

(1,6,0.755716004813477)	(2,6,0.87770789657582)	(3,6,0.927621861152142)	(4,6,0.95477004941087)	(5,6,0.971835345094398)	(6,6,0.98355520751762)	(7,6,nan)	(8,6,nan)	(9,6,nan)	(10,6,nan)	(11,6,nan)	(12,6,nan)	(13,6,nan)	(14,6,nan)	(15,6,nan)

(1,7,0.737089201877934)	(2,7,0.865013774104683)	(3,7,0.91812865497076)	(4,7,0.947209653092005)	(5,7,0.965559655596556)	(6,7,0.978193146417446)	(7,7,0.987421383647798)	(8,7,nan)	(9,7,nan)	(10,7,nan)	(11,7,nan)	(12,7,nan)	(13,7,nan)	(14,7,nan)	(15,7,nan)

(1,8,0.719358533791524)	(2,8,0.852681602172437)	(3,8,0.908827785817656)	(4,8,0.939768050879162)	(5,8,0.959364497402994)	(6,8,0.972889233152595)	(7,8,0.982785602503913)	(8,8,0.990341021092055)	(9,8,nan)	(10,8,nan)	(11,8,nan)	(12,8,nan)	(13,8,nan)	(14,8,nan)	(15,8,nan)

(1,9,0.702460850111857)	(2,9,0.840696117804551)	(3,9,0.899713467048711)	(4,9,0.932442464736451)	(5,9,0.953248330297511)	(6,9,0.967642526964561)	(7,9,0.978193146417446)	(8,9,0.986258343148802)	(9,9,0.992623814541623)	(10,9,nan)	(11,9,nan)	(12,9,nan)	(13,9,nan)	(14,9,nan)	(15,9,nan)

(1,10,0.686338797814207)	(2,10,0.829042904290429)	(3,10,0.890780141843972)	(4,10,0.925230202578269)	(5,10,0.947209653092005)	(6,10,0.962452107279693)	(7,10,0.973643410852713)	(8,10,0.982209188660801)	(9,10,0.988976377952755)	(10,10,0.9944576405384)	(11,10,nan)	(12,10,nan)	(13,10,nan)	(14,10,nan)	(15,10,nan)

(1,11,0.670940170940171)	(2,11,0.817708333333333)	(3,11,0.882022471910112)	(4,11,0.91812865497076)	(5,11,0.941247002398082)	(6,11,0.957317073170732)	(7,11,0.969135802469136)	(8,11,0.978193146417446)	(9,11,0.985355648535565)	(10,11,0.991161616161616)	(11,11,0.995963091118801)	(12,11,nan)	(13,11,nan)	(14,11,nan)	(15,11,nan)

(1,12,0.656217345872518)	(2,12,0.806679511881823)	(3,12,0.873435326842837)	(4,12,0.911135291984041)	(5,12,0.935358951444742)	(6,12,0.952236542835481)	(7,12,0.964669738863287)	(8,12,0.974209811906147)	(9,12,0.981761334028139)	(10,12,0.987887368255466)	(11,12,0.992956734224522)	(12,12,0.997221119491862)	(13,12,nan)	(14,12,nan)	(15,12,nan)

(1,13,0.642126789366053)	(2,13,0.79594423320659)	(3,13,0.865013774104683)	(4,13,0.904247660187184)	(5,13,0.929544108940201)	(6,13,0.947209653092005)	(7,13,0.960244648318043)	(8,13,0.970258787176516)	(9,13,0.978193146417446)	(10,13,0.984634681718407)	(11,13,0.989968472341645)	(12,13,0.9944576405384)	(13,13,0.998288089997554)	(14,13,nan)	(15,13,nan)

(1,14,0.628628628628628)	(2,14,0.785490931832395)	(3,14,0.856753069577081)	(4,14,0.897463379778491)	(5,14,0.923801117975875)	(6,14,0.942235558889722)	(7,14,0.955859969558599)	(8,14,0.966339680707828)	(9,14,0.97465080186239)	(10,14,0.981403344272543)	(11,14,0.986998142591798)	(12,14,0.991709435452033)	(13,14,0.995731186730088)	(14,14,0.99920445505171)	(15,14,nan)

(1,15,0.615686274509804)	(2,15,0.775308641975309)	(3,15,0.848648648648648)	(4,15,0.890780141843972)	(5,15,0.91812865497076)	(6,15,0.937313432835821)	(7,15,0.951515151515151)	(8,15,0.962452107279693)	(9,15,0.971134020618556)	(10,15,0.978193146417446)	(11,15,0.984045584045584)	(12,15,0.988976377952755)	(13,15,0.993187347931874)	(14,15,0.996825396825397)	(15,15,1)

(1,16,0.603266090297791)	(2,16,0.765386959171236)	(3,16,0.840696117804551)	(4,16,0.884195705737416)	(5,16,0.91252542865446)	(6,16,0.932442464736451)	(7,16,0.947209653092005)	(8,16,0.95859568784583)	(9,16,0.967642526964561)	(10,16,0.975003881384878)	(11,16,0.981110637693509)	(12,16,0.986258343148802)	(13,16,0.990656473728916)	(14,16,0.9944576405384)	(15,16,0.997775659358118)

(1,17,0.591337099811676)	(2,17,0.755716004813477)	(3,17,0.832891246684349)	(4,17,0.87770789657582)	(5,17,0.906990179087232)	(6,17,0.927621861152142)	(7,17,0.942942942942943)	(8,17,0.95477004941087)	(9,17,0.96417604912999)	(10,17,0.971835345094398)	(11,17,0.978193146417446)	(12,17,0.98355520751762)	(13,17,0.988138465262648)	(14,17,0.992101105845181)	(15,17,0.995561192136969)

(1,18,0.579870729455217)	(2,18,0.746286393345217)	(3,18,0.825229960578186)	(4,18,0.87131460284426)	(5,18,0.901521676715475)	(6,18,0.922850844966936)	(7,18,0.938714499252616)	(8,18,0.950974824910089)	(9,18,0.960734319224884)	(10,18,0.968687336109825)	(11,18,0.975292954962587)	(12,18,0.980866848887153)	(13,18,0.985633224677049)	(14,18,0.989755713159969)	(15,18,0.993356532742803)

(1,19,0.568840579710145)	(2,19,0.737089201877934)	(3,19,0.817708333333333)	(4,19,0.865013774104683)	(5,19,0.896118721461187)	(6,19,0.91812865497076)	(7,19,0.934523809523809)	(8,19,0.947209653092005)	(9,19,0.957317073170732)	(10,19,0.965559655596556)	(11,19,0.972409909909909)	(12,19,0.978193146417446)	(13,19,0.983140655105973)	(14,19,0.987421383647798)	(15,19,0.991161616161616)

(1,20,0.558222222222222)	(2,20,0.728115942028985)	(3,20,0.810322580645161)	(4,20,0.858803418803419)	(5,20,0.890780141843972)	(6,20,0.913454545454545)	(7,20,0.93037037037037)	(8,20,0.943474178403755)	(9,20,0.95392405063291)	(10,20,0.962452107279693)	(11,20,0.969543859649122)	(12,20,0.975533980582524)	(13,20,0.980660660660661)	(14,20,0.985098039215686)	(15,20,0.988976377952755)

(1,21,0.547993019197208)	(2,21,0.719358533791524)	(3,21,0.803069053708439)	(4,21,0.852681602172437)	(5,21,0.885504794134235)	(6,21,0.908827785817656)	(7,21,0.926253687315634)	(8,21,0.939768050879162)	(9,21,0.950554994954591)	(10,21,0.959364497402994)	(11,21,0.966694654352085)	(12,21,0.972889233152595)	(13,21,0.978193146417446)	(14,21,0.982785602503913)	(15,21,0.986800754242615)

(1,22,0.538131962296487)	(2,22,0.710809281267685)	(3,22,0.79594423320659)	(4,22,0.84664644421975)	(5,22,0.880291561536304)	(6,22,0.904247660187184)	(7,22,0.922173274596182)	(8,22,0.936090926029439)	(9,22,0.947209653092005)	(10,22,0.956296634688594)	(11,22,0.9638621459467)	(12,22,0.970258787176516)	(13,22,0.975738018405641)	(14,22,0.980483996877439)	(15,22,0.984634681718407)

(1,23,0.528619528619528)	(2,23,0.702460850111857)	(3,23,0.78894472361809)	(4,23,0.840696117804551)	(5,23,0.875139353400222)	(6,23,0.899713467048711)	(7,23,0.91812865497076)	(8,23,0.932442464736451)	(9,23,0.943887775551102)	(10,23,0.953248330297511)	(11,23,0.961046188091263)	(12,23,0.967642526964561)	(13,23,0.973295183595613)	(14,23,0.978193146417446)	(15,23,0.982478097622027)

(1,24,0.519437551695616)	(2,24,0.694306246545052)	(3,24,0.782067247820672)	(4,24,0.834828846792954)	(5,24,0.870047104461069)	(6,24,0.895224518888096)	(7,24,0.914119359534206)	(8,24,0.928822333148456)	(9,24,0.940589116325512)	(10,24,0.950219397790891)	(11,24,0.958246636149258)	(12,24,0.965040338071456)	(13,24,0.970864549887025)	(14,24,0.975912975912976)	(15,24,0.98033093974399)

(1,25,0.510569105691057)	(2,25,0.686338797814207)	(3,25,0.775308641975309)	(4,25,0.829042904290429)	(5,25,0.865013774104683)	(6,25,0.890780141843972)	(7,25,0.910144927536232)	(8,25,0.925230202578269)	(9,25,0.937313432835821)	(10,25,0.947209653092005)	(11,25,0.955463347164592)	(12,25,0.962452107279693)	(13,25,0.968446026097272)	(14,25,0.973643410852713)	(15,25,0.978193146417446)

(1,26,0.501998401278977)	(2,26,0.678552133981632)	(3,26,0.768665850673194)	(4,26,0.823336610947231)	(5,26,0.860038345658723)	(6,26,0.886379675370501)	(7,26,0.906204906204906)	(8,26,0.921665749403778)	(9,26,0.934060485870104)	(10,26,0.944218914448954)	(11,26,0.952696179837263)	(12,26,0.95987772258311)	(13,26,0.966039521950065)	(14,26,0.97138437741686)	(15,26,0.976064656512278)

(1,27,0.493710691823899)	(2,27,0.670940170940171)	(3,27,0.762135922330097)	(4,27,0.817708333333333)	(5,27,0.855119825708061)	(6,27,0.882022471910112)	(7,27,0.902298850574712)	(8,27,0.91812865497076)	(9,27,0.930830039525691)	(10,27,0.941247002398082)	(11,27,0.94994499449945)	(12,27,0.957317073170732)	(13,27,0.963644948064211)	(14,27,0.969135802469136)	(15,27,0.97394540942928)

(1,28,0.48569218870843)	(2,28,0.663497094558901)	(3,28,0.755716004813477)	(4,28,0.812156482379566)	(5,28,0.850257243433523)	(6,28,0.87770789657582)	(7,28,0.898426323319027)	(8,28,0.914618605497906)	(9,28,0.927621861152142)	(10,28,0.938293739728074)	(11,28,0.947209653092005)	(12,28,0.95477004941087)	(13,28,0.961262215942541)	(14,28,0.966897613548883)	(15,28,0.971835345094398)



}; \end{axis} \end{tikzpicture}

     \caption{Term Frequency (TF) vs, Doc. Length (DL)}
  \label{bm25scoretfdl}
\end{figure}

%\mentalnote{Finally, by reducing the values of the b and k constants, the standard deviation of across the scores (w.r.t. tf and dl) by bm25 is also reduced, reaching 0 when b and k are 0, as tf and dl do not play any role in this case. Lower stdev. better performance}

\subsection{The Hiemstra's Language Model (HLM) Case}
In this section we study Hiemstra's Language Model (HLM) with respect to Microblog conditions. Table \ref{modelfeatures} shows that HLM utilises both CollectionTermFrequency (CTF) and TermFrequency (TF) together with the total number of different tokens in the collection (NT) and document length (DL). Furthermore, if we pay attention to Table \ref{traditional} we can observe that whilst DFR and HLM utilize the same components, HLM exhibits a more erratic behaviour in terms of its performance under microblog conditions. HLM's performance for the 2013 collection is considerably lower than that of DFR, whereas it remains close to the top models for the 2011, 2012 and 2014 collections. HLM's formulation is as follows: 

\begin{small}
\begin{align}
\label{hlmformula}
    \text{HLM}(D,Q) &= \sum_{i=1}^{n} \log_2 \left[ 1 + \frac{c \cdot f(q_i, D) \cdot ntoks }{ (1-c) \cdot f(q_i, C) \cdot |D|} \right]
\end{align}
\end{small}

where $ntoks$ refers to the number of unique tokens in the collection (NT), $c$ is a free parameter, and $C$ represents the set of all documents in the collection. $f(q_i, D)$ represents the TF of a query term $q_i$ in document $D$, whereas $f(q_i, C)$ is CTF of term $q_i$. The free parameter c regulates how HLM satisfies the conditions of \textbf{coordination level ranking (CLR)}) \cite{hiemstra2000relating}. CLR is a rule enforced in the design of HLM which ensures that documents containing $n$ query terms are ranked higher than those with $n-1$ terms.

\begin{figure}[]
        
       \begin{subfigure}[b]{0.5\textwidth}
        \centering
        \caption{Doc. Frequency (CTF) vs, $c$}
         
\begin{tikzpicture}[thick,scale=0.7, every node/.style={transform shape}]\begin{axis}[
 %title={},
 %y dir=reverse, 
 %x dir=reverse, 
 ylabel={$c$},
 xlabel={docFrequency (CTF)},
 zlabel={HLM score},
 max space between ticks=16pt
% yticklabels={0k,100k}
 ] 

		\addplot3[surf] coordinates { 
%patch,patch type=biquadratic, shader=faceted,patch refines=3

(100.00,0.05,7.28)(100.00,0.10,8.35)(100.00,0.15,9.02)(100.00,0.20,9.52)(100.00,0.25,9.93)(100.00,0.30,10.30)(100.00,0.35,10.62)(100.00,0.40,10.93)(100.00,0.45,11.23)(100.00,0.50,11.52)(100.00,0.55,11.81)(100.00,0.60,12.10)(100.00,0.65,12.41)(100.00,0.70,12.74)(100.00,0.75,13.10)(100.00,0.80,13.52)(100.00,0.85,14.02)(100.00,0.90,14.69)(100.00,0.95,15.76)

(1100.00,0.05,3.91)(1100.00,0.10,4.94)(1100.00,0.15,5.59)(1100.00,0.20,6.08)(1100.00,0.25,6.49)(1100.00,0.30,6.85)(1100.00,0.35,7.17)(1100.00,0.40,7.48)(1100.00,0.45,7.77)(1100.00,0.50,8.06)(1100.00,0.55,8.35)(1100.00,0.60,8.65)(1100.00,0.65,8.95)(1100.00,0.70,9.28)(1100.00,0.75,9.64)(1100.00,0.80,10.06)(1100.00,0.85,10.56)(1100.00,0.90,11.23)(1100.00,0.95,12.31)

(2100.00,0.05,3.06)(2100.00,0.10,4.04)(2100.00,0.15,4.68)(2100.00,0.20,5.17)(2100.00,0.25,5.57)(2100.00,0.30,5.93)(2100.00,0.35,6.25)(2100.00,0.40,6.56)(2100.00,0.45,6.85)(2100.00,0.50,7.13)(2100.00,0.55,7.42)(2100.00,0.60,7.72)(2100.00,0.65,8.02)(2100.00,0.70,8.35)(2100.00,0.75,8.71)(2100.00,0.80,9.13)(2100.00,0.85,9.63)(2100.00,0.90,10.30)(2100.00,0.95,11.37)

(3100.00,0.05,2.58)(3100.00,0.10,3.52)(3100.00,0.15,4.14)(3100.00,0.20,4.62)(3100.00,0.25,5.02)(3100.00,0.30,5.38)(3100.00,0.35,5.70)(3100.00,0.40,6.00)(3100.00,0.45,6.29)(3100.00,0.50,6.58)(3100.00,0.55,6.86)(3100.00,0.60,7.16)(3100.00,0.65,7.46)(3100.00,0.70,7.79)(3100.00,0.75,8.15)(3100.00,0.80,8.57)(3100.00,0.85,9.07)(3100.00,0.90,9.73)(3100.00,0.95,10.81)

(4100.00,0.05,2.25)(4100.00,0.10,3.16)(4100.00,0.15,3.77)(4100.00,0.20,4.24)(4100.00,0.25,4.63)(4100.00,0.30,4.98)(4100.00,0.35,5.30)(4100.00,0.40,5.60)(4100.00,0.45,5.89)(4100.00,0.50,6.18)(4100.00,0.55,6.47)(4100.00,0.60,6.76)(4100.00,0.65,7.06)(4100.00,0.70,7.39)(4100.00,0.75,7.75)(4100.00,0.80,8.16)(4100.00,0.85,8.67)(4100.00,0.90,9.33)(4100.00,0.95,10.41)

(5100.00,0.05,2.01)(5100.00,0.10,2.88)(5100.00,0.15,3.48)(5100.00,0.20,3.94)(5100.00,0.25,4.33)(5100.00,0.30,4.68)(5100.00,0.35,5.00)(5100.00,0.40,5.30)(5100.00,0.45,5.59)(5100.00,0.50,5.87)(5100.00,0.55,6.15)(5100.00,0.60,6.45)(5100.00,0.65,6.75)(5100.00,0.70,7.08)(5100.00,0.75,7.44)(5100.00,0.80,7.85)(5100.00,0.85,8.35)(5100.00,0.90,9.02)(5100.00,0.95,10.09)

(6100.00,0.05,1.82)(6100.00,0.10,2.66)(6100.00,0.15,3.24)(6100.00,0.20,3.70)(6100.00,0.25,4.09)(6100.00,0.30,4.43)(6100.00,0.35,4.75)(6100.00,0.40,5.05)(6100.00,0.45,5.33)(6100.00,0.50,5.62)(6100.00,0.55,5.90)(6100.00,0.60,6.19)(6100.00,0.65,6.50)(6100.00,0.70,6.82)(6100.00,0.75,7.18)(6100.00,0.80,7.59)(6100.00,0.85,8.09)(6100.00,0.90,8.76)(6100.00,0.95,9.84)

(7100.00,0.05,1.67)(7100.00,0.10,2.48)(7100.00,0.15,3.05)(7100.00,0.20,3.50)(7100.00,0.25,3.88)(7100.00,0.30,4.22)(7100.00,0.35,4.54)(7100.00,0.40,4.83)(7100.00,0.45,5.12)(7100.00,0.50,5.40)(7100.00,0.55,5.69)(7100.00,0.60,5.98)(7100.00,0.65,6.28)(7100.00,0.70,6.60)(7100.00,0.75,6.96)(7100.00,0.80,7.38)(7100.00,0.85,7.88)(7100.00,0.90,8.54)(7100.00,0.95,9.62)

(8100.00,0.05,1.54)(8100.00,0.10,2.33)(8100.00,0.15,2.88)(8100.00,0.20,3.33)(8100.00,0.25,3.71)(8100.00,0.30,4.04)(8100.00,0.35,4.36)(8100.00,0.40,4.65)(8100.00,0.45,4.94)(8100.00,0.50,5.22)(8100.00,0.55,5.50)(8100.00,0.60,5.79)(8100.00,0.65,6.09)(8100.00,0.70,6.42)(8100.00,0.75,6.78)(8100.00,0.80,7.19)(8100.00,0.85,7.69)(8100.00,0.90,8.35)(8100.00,0.95,9.43)

(9100.00,0.05,1.43)(9100.00,0.10,2.19)(9100.00,0.15,2.74)(9100.00,0.20,3.18)(9100.00,0.25,3.55)(9100.00,0.30,3.89)(9100.00,0.35,4.20)(9100.00,0.40,4.49)(9100.00,0.45,4.77)(9100.00,0.50,5.05)(9100.00,0.55,5.33)(9100.00,0.60,5.62)(9100.00,0.65,5.93)(9100.00,0.70,6.25)(9100.00,0.75,6.61)(9100.00,0.80,7.02)(9100.00,0.85,7.52)(9100.00,0.90,8.18)(9100.00,0.95,9.26)

(10100.00,0.05,1.34)(10100.00,0.10,2.08)(10100.00,0.15,2.61)(10100.00,0.20,3.05)(10100.00,0.25,3.42)(10100.00,0.30,3.75)(10100.00,0.35,4.06)(10100.00,0.40,4.35)(10100.00,0.45,4.63)(10100.00,0.50,4.91)(10100.00,0.55,5.19)(10100.00,0.60,5.48)(10100.00,0.65,5.78)(10100.00,0.70,6.10)(10100.00,0.75,6.46)(10100.00,0.80,6.87)(10100.00,0.85,7.37)(10100.00,0.90,8.03)(10100.00,0.95,9.11)

(11100.00,0.05,1.26)(11100.00,0.10,1.98)(11100.00,0.15,2.50)(11100.00,0.20,2.93)(11100.00,0.25,3.29)(11100.00,0.30,3.62)(11100.00,0.35,3.93)(11100.00,0.40,4.22)(11100.00,0.45,4.50)(11100.00,0.50,4.78)(11100.00,0.55,5.06)(11100.00,0.60,5.34)(11100.00,0.65,5.64)(11100.00,0.70,5.97)(11100.00,0.75,6.33)(11100.00,0.80,6.74)(11100.00,0.85,7.23)(11100.00,0.90,7.90)(11100.00,0.95,8.97)

(12100.00,0.05,1.19)(12100.00,0.10,1.88)(12100.00,0.15,2.40)(12100.00,0.20,2.82)(12100.00,0.25,3.18)(12100.00,0.30,3.51)(12100.00,0.35,3.81)(12100.00,0.40,4.10)(12100.00,0.45,4.38)(12100.00,0.50,4.66)(12100.00,0.55,4.94)(12100.00,0.60,5.22)(12100.00,0.65,5.52)(12100.00,0.70,5.85)(12100.00,0.75,6.20)(12100.00,0.80,6.61)(12100.00,0.85,7.11)(12100.00,0.90,7.77)(12100.00,0.95,8.85)

(13100.00,0.05,1.12)(13100.00,0.10,1.80)(13100.00,0.15,2.31)(13100.00,0.20,2.72)(13100.00,0.25,3.08)(13100.00,0.30,3.40)(13100.00,0.35,3.71)(13100.00,0.40,3.99)(13100.00,0.45,4.27)(13100.00,0.50,4.55)(13100.00,0.55,4.82)(13100.00,0.60,5.11)(13100.00,0.65,5.41)(13100.00,0.70,5.73)(13100.00,0.75,6.09)(13100.00,0.80,6.50)(13100.00,0.85,7.00)(13100.00,0.90,7.66)(13100.00,0.95,8.73)

(14100.00,0.05,1.07)(14100.00,0.10,1.73)(14100.00,0.15,2.22)(14100.00,0.20,2.63)(14100.00,0.25,2.99)(14100.00,0.30,3.31)(14100.00,0.35,3.61)(14100.00,0.40,3.89)(14100.00,0.45,4.17)(14100.00,0.50,4.45)(14100.00,0.55,4.72)(14100.00,0.60,5.01)(14100.00,0.65,5.31)(14100.00,0.70,5.63)(14100.00,0.75,5.99)(14100.00,0.80,6.39)(14100.00,0.85,6.89)(14100.00,0.90,7.56)(14100.00,0.95,8.63)

(15100.00,0.05,1.02)(15100.00,0.10,1.66)(15100.00,0.15,2.15)(15100.00,0.20,2.55)(15100.00,0.25,2.90)(15100.00,0.30,3.22)(15100.00,0.35,3.52)(15100.00,0.40,3.80)(15100.00,0.45,4.08)(15100.00,0.50,4.35)(15100.00,0.55,4.63)(15100.00,0.60,4.91)(15100.00,0.65,5.21)(15100.00,0.70,5.53)(15100.00,0.75,5.89)(15100.00,0.80,6.30)(15100.00,0.85,6.79)(15100.00,0.90,7.46)(15100.00,0.95,8.53)

(16100.00,0.05,0.97)(16100.00,0.10,1.60)(16100.00,0.15,2.07)(16100.00,0.20,2.47)(16100.00,0.25,2.82)(16100.00,0.30,3.14)(16100.00,0.35,3.43)(16100.00,0.40,3.72)(16100.00,0.45,3.99)(16100.00,0.50,4.26)(16100.00,0.55,4.54)(16100.00,0.60,4.82)(16100.00,0.65,5.12)(16100.00,0.70,5.44)(16100.00,0.75,5.80)(16100.00,0.80,6.21)(16100.00,0.85,6.70)(16100.00,0.90,7.36)(16100.00,0.95,8.44)

(17100.00,0.05,0.93)(17100.00,0.10,1.54)(17100.00,0.15,2.01)(17100.00,0.20,2.40)(17100.00,0.25,2.75)(17100.00,0.30,3.06)(17100.00,0.35,3.35)(17100.00,0.40,3.64)(17100.00,0.45,3.91)(17100.00,0.50,4.18)(17100.00,0.55,4.46)(17100.00,0.60,4.74)(17100.00,0.65,5.04)(17100.00,0.70,5.36)(17100.00,0.75,5.71)(17100.00,0.80,6.12)(17100.00,0.85,6.62)(17100.00,0.90,7.28)(17100.00,0.95,8.35)

(18100.00,0.05,0.89)(18100.00,0.10,1.48)(18100.00,0.15,1.95)(18100.00,0.20,2.34)(18100.00,0.25,2.68)(18100.00,0.30,2.99)(18100.00,0.35,3.28)(18100.00,0.40,3.56)(18100.00,0.45,3.83)(18100.00,0.50,4.10)(18100.00,0.55,4.38)(18100.00,0.60,4.66)(18100.00,0.65,4.96)(18100.00,0.70,5.28)(18100.00,0.75,5.63)(18100.00,0.80,6.04)(18100.00,0.85,6.54)(18100.00,0.90,7.20)(18100.00,0.95,8.27)

(19100.00,0.05,0.85)(19100.00,0.10,1.44)(19100.00,0.15,1.89)(19100.00,0.20,2.27)(19100.00,0.25,2.61)(19100.00,0.30,2.92)(19100.00,0.35,3.21)(19100.00,0.40,3.49)(19100.00,0.45,3.76)(19100.00,0.50,4.03)(19100.00,0.55,4.30)(19100.00,0.60,4.59)(19100.00,0.65,4.88)(19100.00,0.70,5.20)(19100.00,0.75,5.56)(19100.00,0.80,5.96)(19100.00,0.85,6.46)(19100.00,0.90,7.12)(19100.00,0.95,8.19)

(20100.00,0.05,0.82)(20100.00,0.10,1.39)(20100.00,0.15,1.84)(20100.00,0.20,2.22)(20100.00,0.25,2.55)(20100.00,0.30,2.86)(20100.00,0.35,3.15)(20100.00,0.40,3.42)(20100.00,0.45,3.69)(20100.00,0.50,3.96)(20100.00,0.55,4.23)(20100.00,0.60,4.52)(20100.00,0.65,4.81)(20100.00,0.70,5.13)(20100.00,0.75,5.48)(20100.00,0.80,5.89)(20100.00,0.85,6.39)(20100.00,0.90,7.05)(20100.00,0.95,8.12)
%
%(21100.00,0.05,0.79)(21100.00,0.10,1.35)(21100.00,0.15,1.79)(21100.00,0.20,2.16)(21100.00,0.25,2.49)(21100.00,0.30,2.80)(21100.00,0.35,3.08)(21100.00,0.40,3.36)(21100.00,0.45,3.63)(21100.00,0.50,3.90)(21100.00,0.55,4.17)(21100.00,0.60,4.45)(21100.00,0.65,4.74)(21100.00,0.70,5.06)(21100.00,0.75,5.42)(21100.00,0.80,5.82)(21100.00,0.85,6.32)(21100.00,0.90,6.98)(21100.00,0.95,8.05)
%
%(22100.00,0.05,0.76)(22100.00,0.10,1.31)(22100.00,0.15,1.74)(22100.00,0.20,2.11)(22100.00,0.25,2.44)(22100.00,0.30,2.74)(22100.00,0.35,3.03)(22100.00,0.40,3.30)(22100.00,0.45,3.57)(22100.00,0.50,3.83)(22100.00,0.55,4.10)(22100.00,0.60,4.38)(22100.00,0.65,4.68)(22100.00,0.70,5.00)(22100.00,0.75,5.35)(22100.00,0.80,5.76)(22100.00,0.85,6.25)(22100.00,0.90,6.91)(22100.00,0.95,7.98)
%
%(23100.00,0.05,0.74)(23100.00,0.10,1.27)(23100.00,0.15,1.70)(23100.00,0.20,2.06)(23100.00,0.25,2.39)(23100.00,0.30,2.69)(23100.00,0.35,2.97)(23100.00,0.40,3.24)(23100.00,0.45,3.51)(23100.00,0.50,3.77)(23100.00,0.55,4.04)(23100.00,0.60,4.32)(23100.00,0.65,4.62)(23100.00,0.70,4.94)(23100.00,0.75,5.29)(23100.00,0.80,5.69)(23100.00,0.85,6.19)(23100.00,0.90,6.85)(23100.00,0.95,7.92)
%
%(24100.00,0.05,0.71)(24100.00,0.10,1.23)(24100.00,0.15,1.65)(24100.00,0.20,2.01)(24100.00,0.25,2.34)(24100.00,0.30,2.63)(24100.00,0.35,2.92)(24100.00,0.40,3.19)(24100.00,0.45,3.45)(24100.00,0.50,3.72)(24100.00,0.55,3.99)(24100.00,0.60,4.27)(24100.00,0.65,4.56)(24100.00,0.70,4.88)(24100.00,0.75,5.23)(24100.00,0.80,5.63)(24100.00,0.85,6.13)(24100.00,0.90,6.79)(24100.00,0.95,7.86)
%
%(25100.00,0.05,0.69)(25100.00,0.10,1.20)(25100.00,0.15,1.61)(25100.00,0.20,1.97)(25100.00,0.25,2.29)(25100.00,0.30,2.59)(25100.00,0.35,2.87)(25100.00,0.40,3.13)(25100.00,0.45,3.40)(25100.00,0.50,3.66)(25100.00,0.55,3.93)(25100.00,0.60,4.21)(25100.00,0.65,4.50)(25100.00,0.70,4.82)(25100.00,0.75,5.17)(25100.00,0.80,5.58)(25100.00,0.85,6.07)(25100.00,0.90,6.73)(25100.00,0.95,7.80)
%
%(26100.00,0.05,0.67)(26100.00,0.10,1.17)(26100.00,0.15,1.58)(26100.00,0.20,1.93)(26100.00,0.25,2.25)(26100.00,0.30,2.54)(26100.00,0.35,2.82)(26100.00,0.40,3.08)(26100.00,0.45,3.35)(26100.00,0.50,3.61)(26100.00,0.55,3.88)(26100.00,0.60,4.16)(26100.00,0.65,4.45)(26100.00,0.70,4.77)(26100.00,0.75,5.12)(26100.00,0.80,5.52)(26100.00,0.85,6.01)(26100.00,0.90,6.67)(26100.00,0.95,7.74)
%
%(27100.00,0.05,0.65)(27100.00,0.10,1.14)(27100.00,0.15,1.54)(27100.00,0.20,1.89)(27100.00,0.25,2.20)(27100.00,0.30,2.49)(27100.00,0.35,2.77)(27100.00,0.40,3.04)(27100.00,0.45,3.30)(27100.00,0.50,3.56)(27100.00,0.55,3.83)(27100.00,0.60,4.11)(27100.00,0.65,4.40)(27100.00,0.70,4.71)(27100.00,0.75,5.06)(27100.00,0.80,5.47)(27100.00,0.85,5.96)(27100.00,0.90,6.62)(27100.00,0.95,7.69)
%
%(28100.00,0.05,0.63)(28100.00,0.10,1.11)(28100.00,0.15,1.51)(28100.00,0.20,1.85)(28100.00,0.25,2.16)(28100.00,0.30,2.45)(28100.00,0.35,2.73)(28100.00,0.40,2.99)(28100.00,0.45,3.25)(28100.00,0.50,3.51)(28100.00,0.55,3.78)(28100.00,0.60,4.06)(28100.00,0.65,4.35)(28100.00,0.70,4.66)(28100.00,0.75,5.01)(28100.00,0.80,5.42)(28100.00,0.85,5.91)(28100.00,0.90,6.57)(28100.00,0.95,7.64)
%
%(29100.00,0.05,0.61)(29100.00,0.10,1.08)(29100.00,0.15,1.47)(29100.00,0.20,1.81)(29100.00,0.25,2.12)(29100.00,0.30,2.41)(29100.00,0.35,2.68)(29100.00,0.40,2.95)(29100.00,0.45,3.21)(29100.00,0.50,3.47)(29100.00,0.55,3.73)(29100.00,0.60,4.01)(29100.00,0.65,4.30)(29100.00,0.70,4.61)(29100.00,0.75,4.96)(29100.00,0.80,5.37)(29100.00,0.85,5.86)(29100.00,0.90,6.52)(29100.00,0.95,7.59)
%
%(30100.00,0.05,0.60)(30100.00,0.10,1.06)(30100.00,0.15,1.44)(30100.00,0.20,1.78)(30100.00,0.25,2.09)(30100.00,0.30,2.37)(30100.00,0.35,2.64)(30100.00,0.40,2.91)(30100.00,0.45,3.16)(30100.00,0.50,3.42)(30100.00,0.55,3.69)(30100.00,0.60,3.96)(30100.00,0.65,4.25)(30100.00,0.70,4.57)(30100.00,0.75,4.92)(30100.00,0.80,5.32)(30100.00,0.85,5.81)(30100.00,0.90,6.47)(30100.00,0.95,7.54)
%
%(31100.00,0.05,0.58)(31100.00,0.10,1.03)(31100.00,0.15,1.41)(31100.00,0.20,1.75)(31100.00,0.25,2.05)(31100.00,0.30,2.33)(31100.00,0.35,2.60)(31100.00,0.40,2.86)(31100.00,0.45,3.12)(31100.00,0.50,3.38)(31100.00,0.55,3.65)(31100.00,0.60,3.92)(31100.00,0.65,4.21)(31100.00,0.70,4.52)(31100.00,0.75,4.87)(31100.00,0.80,5.27)(31100.00,0.85,5.77)(31100.00,0.90,6.42)(31100.00,0.95,7.49)
%
%(32100.00,0.05,0.57)(32100.00,0.10,1.01)(32100.00,0.15,1.38)(32100.00,0.20,1.71)(32100.00,0.25,2.02)(32100.00,0.30,2.30)(32100.00,0.35,2.56)(32100.00,0.40,2.83)(32100.00,0.45,3.08)(32100.00,0.50,3.34)(32100.00,0.55,3.60)(32100.00,0.60,3.88)(32100.00,0.65,4.17)(32100.00,0.70,4.48)(32100.00,0.75,4.83)(32100.00,0.80,5.23)(32100.00,0.85,5.72)(32100.00,0.90,6.38)(32100.00,0.95,7.45)
%
%(33100.00,0.05,0.55)(33100.00,0.10,0.99)(33100.00,0.15,1.36)(33100.00,0.20,1.68)(33100.00,0.25,1.98)(33100.00,0.30,2.26)(33100.00,0.35,2.53)(33100.00,0.40,2.79)(33100.00,0.45,3.04)(33100.00,0.50,3.30)(33100.00,0.55,3.56)(33100.00,0.60,3.84)(33100.00,0.65,4.12)(33100.00,0.70,4.44)(33100.00,0.75,4.78)(33100.00,0.80,5.19)(33100.00,0.85,5.68)(33100.00,0.90,6.33)(33100.00,0.95,7.40)
%
%(34100.00,0.05,0.54)(34100.00,0.10,0.97)(34100.00,0.15,1.33)(34100.00,0.20,1.65)(34100.00,0.25,1.95)(34100.00,0.30,2.23)(34100.00,0.35,2.49)(34100.00,0.40,2.75)(34100.00,0.45,3.01)(34100.00,0.50,3.26)(34100.00,0.55,3.52)(34100.00,0.60,3.80)(34100.00,0.65,4.08)(34100.00,0.70,4.40)(34100.00,0.75,4.74)(34100.00,0.80,5.14)(34100.00,0.85,5.64)(34100.00,0.90,6.29)(34100.00,0.95,7.36)
%
%(35100.00,0.05,0.53)(35100.00,0.10,0.95)(35100.00,0.15,1.31)(35100.00,0.20,1.63)(35100.00,0.25,1.92)(35100.00,0.30,2.19)(35100.00,0.35,2.46)(35100.00,0.40,2.72)(35100.00,0.45,2.97)(35100.00,0.50,3.22)(35100.00,0.55,3.49)(35100.00,0.60,3.76)(35100.00,0.65,4.04)(35100.00,0.70,4.36)(35100.00,0.75,4.70)(35100.00,0.80,5.10)(35100.00,0.85,5.59)(35100.00,0.90,6.25)(35100.00,0.95,7.32)
%
%(36100.00,0.05,0.51)(36100.00,0.10,0.93)(36100.00,0.15,1.28)(36100.00,0.20,1.60)(36100.00,0.25,1.89)(36100.00,0.30,2.16)(36100.00,0.35,2.43)(36100.00,0.40,2.68)(36100.00,0.45,2.93)(36100.00,0.50,3.19)(36100.00,0.55,3.45)(36100.00,0.60,3.72)(36100.00,0.65,4.01)(36100.00,0.70,4.32)(36100.00,0.75,4.66)(36100.00,0.80,5.06)(36100.00,0.85,5.55)(36100.00,0.90,6.21)(36100.00,0.95,7.28)
%
%(37100.00,0.05,0.50)(37100.00,0.10,0.91)(37100.00,0.15,1.26)(37100.00,0.20,1.57)(37100.00,0.25,1.86)(37100.00,0.30,2.13)(37100.00,0.35,2.39)(37100.00,0.40,2.65)(37100.00,0.45,2.90)(37100.00,0.50,3.15)(37100.00,0.55,3.41)(37100.00,0.60,3.68)(37100.00,0.65,3.97)(37100.00,0.70,4.28)(37100.00,0.75,4.63)(37100.00,0.80,5.03)(37100.00,0.85,5.52)(37100.00,0.90,6.17)(37100.00,0.95,7.24)
%
%(38100.00,0.05,0.49)(38100.00,0.10,0.89)(38100.00,0.15,1.24)(38100.00,0.20,1.55)(38100.00,0.25,1.83)(38100.00,0.30,2.10)(38100.00,0.35,2.36)(38100.00,0.40,2.62)(38100.00,0.45,2.87)(38100.00,0.50,3.12)(38100.00,0.55,3.38)(38100.00,0.60,3.65)(38100.00,0.65,3.93)(38100.00,0.70,4.24)(38100.00,0.75,4.59)(38100.00,0.80,4.99)(38100.00,0.85,5.48)(38100.00,0.90,6.13)(38100.00,0.95,7.20)
%
%(39100.00,0.05,0.48)(39100.00,0.10,0.87)(39100.00,0.15,1.22)(39100.00,0.20,1.52)(39100.00,0.25,1.81)(39100.00,0.30,2.07)(39100.00,0.35,2.33)(39100.00,0.40,2.58)(39100.00,0.45,2.83)(39100.00,0.50,3.09)(39100.00,0.55,3.34)(39100.00,0.60,3.61)(39100.00,0.65,3.90)(39100.00,0.70,4.21)(39100.00,0.75,4.55)(39100.00,0.80,4.95)(39100.00,0.85,5.44)(39100.00,0.90,6.10)(39100.00,0.95,7.16)
%
%(40100.00,0.05,0.47)(40100.00,0.10,0.86)(40100.00,0.15,1.20)(40100.00,0.20,1.50)(40100.00,0.25,1.78)(40100.00,0.30,2.05)(40100.00,0.35,2.30)(40100.00,0.40,2.55)(40100.00,0.45,2.80)(40100.00,0.50,3.05)(40100.00,0.55,3.31)(40100.00,0.60,3.58)(40100.00,0.65,3.87)(40100.00,0.70,4.17)(40100.00,0.75,4.52)(40100.00,0.80,4.92)(40100.00,0.85,5.41)(40100.00,0.90,6.06)(40100.00,0.95,7.13)
%
%(41100.00,0.05,0.46)(41100.00,0.10,0.84)(41100.00,0.15,1.18)(41100.00,0.20,1.48)(41100.00,0.25,1.76)(41100.00,0.30,2.02)(41100.00,0.35,2.27)(41100.00,0.40,2.52)(41100.00,0.45,2.77)(41100.00,0.50,3.02)(41100.00,0.55,3.28)(41100.00,0.60,3.55)(41100.00,0.65,3.83)(41100.00,0.70,4.14)(41100.00,0.75,4.48)(41100.00,0.80,4.88)(41100.00,0.85,5.37)(41100.00,0.90,6.03)(41100.00,0.95,7.09)
%
%(42100.00,0.05,0.45)(42100.00,0.10,0.83)(42100.00,0.15,1.16)(42100.00,0.20,1.45)(42100.00,0.25,1.73)(42100.00,0.30,1.99)(42100.00,0.35,2.25)(42100.00,0.40,2.50)(42100.00,0.45,2.74)(42100.00,0.50,2.99)(42100.00,0.55,3.25)(42100.00,0.60,3.52)(42100.00,0.65,3.80)(42100.00,0.70,4.11)(42100.00,0.75,4.45)(42100.00,0.80,4.85)(42100.00,0.85,5.34)(42100.00,0.90,5.99)(42100.00,0.95,7.06)
%
%(43100.00,0.05,0.44)(43100.00,0.10,0.81)(43100.00,0.15,1.14)(43100.00,0.20,1.43)(43100.00,0.25,1.71)(43100.00,0.30,1.97)(43100.00,0.35,2.22)(43100.00,0.40,2.47)(43100.00,0.45,2.71)(43100.00,0.50,2.96)(43100.00,0.55,3.22)(43100.00,0.60,3.49)(43100.00,0.65,3.77)(43100.00,0.70,4.08)(43100.00,0.75,4.42)(43100.00,0.80,4.82)(43100.00,0.85,5.30)(43100.00,0.90,5.96)(43100.00,0.95,7.02)
%
%(44100.00,0.05,0.43)(44100.00,0.10,0.80)(44100.00,0.15,1.12)(44100.00,0.20,1.41)(44100.00,0.25,1.68)(44100.00,0.30,1.94)(44100.00,0.35,2.19)(44100.00,0.40,2.44)(44100.00,0.45,2.69)(44100.00,0.50,2.93)(44100.00,0.55,3.19)(44100.00,0.60,3.46)(44100.00,0.65,3.74)(44100.00,0.70,4.04)(44100.00,0.75,4.39)(44100.00,0.80,4.79)(44100.00,0.85,5.27)(44100.00,0.90,5.93)(44100.00,0.95,6.99)
%
%(45100.00,0.05,0.42)(45100.00,0.10,0.78)(45100.00,0.15,1.10)(45100.00,0.20,1.39)(45100.00,0.25,1.66)(45100.00,0.30,1.92)(45100.00,0.35,2.17)(45100.00,0.40,2.41)(45100.00,0.45,2.66)(45100.00,0.50,2.91)(45100.00,0.55,3.16)(45100.00,0.60,3.43)(45100.00,0.65,3.71)(45100.00,0.70,4.01)(45100.00,0.75,4.36)(45100.00,0.80,4.75)(45100.00,0.85,5.24)(45100.00,0.90,5.89)(45100.00,0.95,6.96)
%
%(46100.00,0.05,0.42)(46100.00,0.10,0.77)(46100.00,0.15,1.09)(46100.00,0.20,1.37)(46100.00,0.25,1.64)(46100.00,0.30,1.90)(46100.00,0.35,2.15)(46100.00,0.40,2.39)(46100.00,0.45,2.63)(46100.00,0.50,2.88)(46100.00,0.55,3.13)(46100.00,0.60,3.40)(46100.00,0.65,3.68)(46100.00,0.70,3.98)(46100.00,0.75,4.33)(46100.00,0.80,4.72)(46100.00,0.85,5.21)(46100.00,0.90,5.86)(46100.00,0.95,6.93)
%
%(47100.00,0.05,0.41)(47100.00,0.10,0.76)(47100.00,0.15,1.07)(47100.00,0.20,1.35)(47100.00,0.25,1.62)(47100.00,0.30,1.87)(47100.00,0.35,2.12)(47100.00,0.40,2.36)(47100.00,0.45,2.61)(47100.00,0.50,2.85)(47100.00,0.55,3.11)(47100.00,0.60,3.37)(47100.00,0.65,3.65)(47100.00,0.70,3.96)(47100.00,0.75,4.30)(47100.00,0.80,4.69)(47100.00,0.85,5.18)(47100.00,0.90,5.83)(47100.00,0.95,6.90)
%
%(48100.00,0.05,0.40)(48100.00,0.10,0.75)(48100.00,0.15,1.05)(48100.00,0.20,1.34)(48100.00,0.25,1.60)(48100.00,0.30,1.85)(48100.00,0.35,2.10)(48100.00,0.40,2.34)(48100.00,0.45,2.58)(48100.00,0.50,2.83)(48100.00,0.55,3.08)(48100.00,0.60,3.34)(48100.00,0.65,3.62)(48100.00,0.70,3.93)(48100.00,0.75,4.27)(48100.00,0.80,4.67)(48100.00,0.85,5.15)(48100.00,0.90,5.80)(48100.00,0.95,6.87)
%
%(49100.00,0.05,0.39)(49100.00,0.10,0.73)(49100.00,0.15,1.04)(49100.00,0.20,1.32)(49100.00,0.25,1.58)(49100.00,0.30,1.83)(49100.00,0.35,2.08)(49100.00,0.40,2.32)(49100.00,0.45,2.56)(49100.00,0.50,2.80)(49100.00,0.55,3.05)(49100.00,0.60,3.32)(49100.00,0.65,3.60)(49100.00,0.70,3.90)(49100.00,0.75,4.24)(49100.00,0.80,4.64)(49100.00,0.85,5.12)(49100.00,0.90,5.77)(49100.00,0.95,6.84)


}; \end{axis} \end{tikzpicture}

      \end{subfigure}      
      ~
       \begin{subfigure}[b]{0.5\textwidth}
        \centering
        \caption{Doc. Frequency (CTF) vs, Doc. Length (DL)}
         
\begin{tikzpicture}[thick,scale=0.7, every node/.style={transform shape}] \begin{axis}[
 %title={},
 %y dir=reverse, 
 %x dir=reverse, 
 ylabel={docLength (DL)},
 xlabel={docFrequency (CTF)},
 zlabel={HLM score},
 max space between ticks=16pt
% yticklabels={0k,100k}
 ] 

		\addplot3[surf] coordinates { 
%patch,patch type=biquadratic, shader=faceted,patch refines=3

(100.00,20.00,9.02)(100.00,19.00,9.09)(100.00,18.00,9.17)(100.00,17.00,9.25)(100.00,16.00,9.34)(100.00,15.00,9.43)(100.00,14.00,9.53)(100.00,13.00,9.64)(100.00,12.00,9.75)(100.00,11.00,9.88)(100.00,10.00,10.02)(100.00,9.00,10.17)(100.00,8.00,10.34)(100.00,7.00,10.53)(100.00,6.00,10.75)(100.00,5.00,11.02)(100.00,4.00,11.34)(100.00,3.00,11.75)(100.00,2.00,12.34)(100.00,1.00,13.34)

(1100.00,20.00,5.59)(1100.00,19.00,5.66)(1100.00,18.00,5.73)(1100.00,17.00,5.82)(1100.00,16.00,5.90)(1100.00,15.00,5.99)(1100.00,14.00,6.09)(1100.00,13.00,6.20)(1100.00,12.00,6.31)(1100.00,11.00,6.43)(1100.00,10.00,6.57)(1100.00,9.00,6.72)(1100.00,8.00,6.89)(1100.00,7.00,7.08)(1100.00,6.00,7.30)(1100.00,5.00,7.56)(1100.00,4.00,7.88)(1100.00,3.00,8.30)(1100.00,2.00,8.88)(1100.00,1.00,9.88)

(2100.00,20.00,4.68)(2100.00,19.00,4.75)(2100.00,18.00,4.83)(2100.00,17.00,4.91)(2100.00,16.00,4.99)(2100.00,15.00,5.08)(2100.00,14.00,5.18)(2100.00,13.00,5.28)(2100.00,12.00,5.39)(2100.00,11.00,5.52)(2100.00,10.00,5.65)(2100.00,9.00,5.80)(2100.00,8.00,5.97)(2100.00,7.00,6.16)(2100.00,6.00,6.38)(2100.00,5.00,6.64)(2100.00,4.00,6.96)(2100.00,3.00,7.37)(2100.00,2.00,7.95)(2100.00,1.00,8.95)

(3100.00,20.00,4.14)(3100.00,19.00,4.21)(3100.00,18.00,4.29)(3100.00,17.00,4.37)(3100.00,16.00,4.45)(3100.00,15.00,4.54)(3100.00,14.00,4.63)(3100.00,13.00,4.74)(3100.00,12.00,4.85)(3100.00,11.00,4.97)(3100.00,10.00,5.10)(3100.00,9.00,5.25)(3100.00,8.00,5.42)(3100.00,7.00,5.60)(3100.00,6.00,5.82)(3100.00,5.00,6.08)(3100.00,4.00,6.40)(3100.00,3.00,6.81)(3100.00,2.00,7.39)(3100.00,1.00,8.39)

(4100.00,20.00,3.77)(4100.00,19.00,3.84)(4100.00,18.00,3.91)(4100.00,17.00,3.99)(4100.00,16.00,4.07)(4100.00,15.00,4.16)(4100.00,14.00,4.25)(4100.00,13.00,4.35)(4100.00,12.00,4.46)(4100.00,11.00,4.58)(4100.00,10.00,4.71)(4100.00,9.00,4.86)(4100.00,8.00,5.02)(4100.00,7.00,5.21)(4100.00,6.00,5.43)(4100.00,5.00,5.69)(4100.00,4.00,6.00)(4100.00,3.00,6.41)(4100.00,2.00,6.99)(4100.00,1.00,7.98)

(5100.00,20.00,3.48)(5100.00,19.00,3.55)(5100.00,18.00,3.62)(5100.00,17.00,3.69)(5100.00,16.00,3.77)(5100.00,15.00,3.86)(5100.00,14.00,3.95)(5100.00,13.00,4.05)(5100.00,12.00,4.16)(5100.00,11.00,4.28)(5100.00,10.00,4.41)(5100.00,9.00,4.56)(5100.00,8.00,4.72)(5100.00,7.00,4.91)(5100.00,6.00,5.12)(5100.00,5.00,5.38)(5100.00,4.00,5.69)(5100.00,3.00,6.10)(5100.00,2.00,6.68)(5100.00,1.00,7.67)

(6100.00,20.00,3.24)(6100.00,19.00,3.31)(6100.00,18.00,3.38)(6100.00,17.00,3.46)(6100.00,16.00,3.54)(6100.00,15.00,3.62)(6100.00,14.00,3.71)(6100.00,13.00,3.81)(6100.00,12.00,3.92)(6100.00,11.00,4.04)(6100.00,10.00,4.17)(6100.00,9.00,4.31)(6100.00,8.00,4.47)(6100.00,7.00,4.66)(6100.00,6.00,4.87)(6100.00,5.00,5.13)(6100.00,4.00,5.44)(6100.00,3.00,5.85)(6100.00,2.00,6.42)(6100.00,1.00,7.41)

(7100.00,20.00,3.05)(7100.00,19.00,3.12)(7100.00,18.00,3.18)(7100.00,17.00,3.26)(7100.00,16.00,3.34)(7100.00,15.00,3.42)(7100.00,14.00,3.51)(7100.00,13.00,3.61)(7100.00,12.00,3.72)(7100.00,11.00,3.83)(7100.00,10.00,3.96)(7100.00,9.00,4.10)(7100.00,8.00,4.26)(7100.00,7.00,4.45)(7100.00,6.00,4.66)(7100.00,5.00,4.91)(7100.00,4.00,5.23)(7100.00,3.00,5.63)(7100.00,2.00,6.21)(7100.00,1.00,7.20)

(8100.00,20.00,2.88)(8100.00,19.00,2.95)(8100.00,18.00,3.02)(8100.00,17.00,3.09)(8100.00,16.00,3.17)(8100.00,15.00,3.25)(8100.00,14.00,3.34)(8100.00,13.00,3.44)(8100.00,12.00,3.54)(8100.00,11.00,3.66)(8100.00,10.00,3.78)(8100.00,9.00,3.92)(8100.00,8.00,4.08)(8100.00,7.00,4.27)(8100.00,6.00,4.48)(8100.00,5.00,4.73)(8100.00,4.00,5.04)(8100.00,3.00,5.45)(8100.00,2.00,6.02)(8100.00,1.00,7.01)

(9100.00,20.00,2.74)(9100.00,19.00,2.80)(9100.00,18.00,2.87)(9100.00,17.00,2.94)(9100.00,16.00,3.02)(9100.00,15.00,3.10)(9100.00,14.00,3.19)(9100.00,13.00,3.28)(9100.00,12.00,3.39)(9100.00,11.00,3.50)(9100.00,10.00,3.63)(9100.00,9.00,3.77)(9100.00,8.00,3.93)(9100.00,7.00,4.11)(9100.00,6.00,4.32)(9100.00,5.00,4.57)(9100.00,4.00,4.88)(9100.00,3.00,5.28)(9100.00,2.00,5.85)(9100.00,1.00,6.84)

(10100.00,20.00,2.61)(10100.00,19.00,2.68)(10100.00,18.00,2.74)(10100.00,17.00,2.81)(10100.00,16.00,2.89)(10100.00,15.00,2.97)(10100.00,14.00,3.06)(10100.00,13.00,3.15)(10100.00,12.00,3.25)(10100.00,11.00,3.37)(10100.00,10.00,3.49)(10100.00,9.00,3.63)(10100.00,8.00,3.79)(10100.00,7.00,3.97)(10100.00,6.00,4.18)(10100.00,5.00,4.43)(10100.00,4.00,4.73)(10100.00,3.00,5.13)(10100.00,2.00,5.71)(10100.00,1.00,6.69)

(11100.00,20.00,2.50)(11100.00,19.00,2.56)(11100.00,18.00,2.63)(11100.00,17.00,2.70)(11100.00,16.00,2.77)(11100.00,15.00,2.85)(11100.00,14.00,2.94)(11100.00,13.00,3.03)(11100.00,12.00,3.13)(11100.00,11.00,3.24)(11100.00,10.00,3.37)(11100.00,9.00,3.51)(11100.00,8.00,3.66)(11100.00,7.00,3.84)(11100.00,6.00,4.05)(11100.00,5.00,4.30)(11100.00,4.00,4.60)(11100.00,3.00,5.00)(11100.00,2.00,5.57)(11100.00,1.00,6.56)

(12100.00,20.00,2.40)(12100.00,19.00,2.46)(12100.00,18.00,2.52)(12100.00,17.00,2.59)(12100.00,16.00,2.67)(12100.00,15.00,2.74)(12100.00,14.00,2.83)(12100.00,13.00,2.92)(12100.00,12.00,3.02)(12100.00,11.00,3.13)(12100.00,10.00,3.26)(12100.00,9.00,3.39)(12100.00,8.00,3.55)(12100.00,7.00,3.72)(12100.00,6.00,3.93)(12100.00,5.00,4.18)(12100.00,4.00,4.48)(12100.00,3.00,4.88)(12100.00,2.00,5.45)(12100.00,1.00,6.43)

(13100.00,20.00,2.31)(13100.00,19.00,2.37)(13100.00,18.00,2.43)(13100.00,17.00,2.50)(13100.00,16.00,2.57)(13100.00,15.00,2.65)(13100.00,14.00,2.73)(13100.00,13.00,2.82)(13100.00,12.00,2.92)(13100.00,11.00,3.03)(13100.00,10.00,3.15)(13100.00,9.00,3.29)(13100.00,8.00,3.44)(13100.00,7.00,3.62)(13100.00,6.00,3.82)(13100.00,5.00,4.07)(13100.00,4.00,4.37)(13100.00,3.00,4.77)(13100.00,2.00,5.34)(13100.00,1.00,6.32)

(14100.00,20.00,2.22)(14100.00,19.00,2.28)(14100.00,18.00,2.34)(14100.00,17.00,2.41)(14100.00,16.00,2.48)(14100.00,15.00,2.56)(14100.00,14.00,2.64)(14100.00,13.00,2.73)(14100.00,12.00,2.83)(14100.00,11.00,2.94)(14100.00,10.00,3.06)(14100.00,9.00,3.19)(14100.00,8.00,3.35)(14100.00,7.00,3.52)(14100.00,6.00,3.73)(14100.00,5.00,3.97)(14100.00,4.00,4.27)(14100.00,3.00,4.67)(14100.00,2.00,5.24)(14100.00,1.00,6.22)

(15100.00,20.00,2.15)(15100.00,19.00,2.20)(15100.00,18.00,2.26)(15100.00,17.00,2.33)(15100.00,16.00,2.40)(15100.00,15.00,2.48)(15100.00,14.00,2.56)(15100.00,13.00,2.65)(15100.00,12.00,2.75)(15100.00,11.00,2.85)(15100.00,10.00,2.97)(15100.00,9.00,3.11)(15100.00,8.00,3.26)(15100.00,7.00,3.43)(15100.00,6.00,3.63)(15100.00,5.00,3.88)(15100.00,4.00,4.18)(15100.00,3.00,4.57)(15100.00,2.00,5.14)(15100.00,1.00,6.12)

(16100.00,20.00,2.07)(16100.00,19.00,2.13)(16100.00,18.00,2.19)(16100.00,17.00,2.26)(16100.00,16.00,2.33)(16100.00,15.00,2.40)(16100.00,14.00,2.48)(16100.00,13.00,2.57)(16100.00,12.00,2.67)(16100.00,11.00,2.77)(16100.00,10.00,2.89)(16100.00,9.00,3.02)(16100.00,8.00,3.17)(16100.00,7.00,3.35)(16100.00,6.00,3.55)(16100.00,5.00,3.79)(16100.00,4.00,4.09)(16100.00,3.00,4.49)(16100.00,2.00,5.05)(16100.00,1.00,6.03)

(17100.00,20.00,2.01)(17100.00,19.00,2.06)(17100.00,18.00,2.12)(17100.00,17.00,2.19)(17100.00,16.00,2.26)(17100.00,15.00,2.33)(17100.00,14.00,2.41)(17100.00,13.00,2.50)(17100.00,12.00,2.59)(17100.00,11.00,2.70)(17100.00,10.00,2.82)(17100.00,9.00,2.95)(17100.00,8.00,3.10)(17100.00,7.00,3.27)(17100.00,6.00,3.47)(17100.00,5.00,3.71)(17100.00,4.00,4.01)(17100.00,3.00,4.40)(17100.00,2.00,4.97)(17100.00,1.00,5.94)

(18100.00,20.00,1.95)(18100.00,19.00,2.00)(18100.00,18.00,2.06)(18100.00,17.00,2.12)(18100.00,16.00,2.19)(18100.00,15.00,2.27)(18100.00,14.00,2.35)(18100.00,13.00,2.43)(18100.00,12.00,2.53)(18100.00,11.00,2.63)(18100.00,10.00,2.75)(18100.00,9.00,2.88)(18100.00,8.00,3.03)(18100.00,7.00,3.20)(18100.00,6.00,3.40)(18100.00,5.00,3.64)(18100.00,4.00,3.93)(18100.00,3.00,4.33)(18100.00,2.00,4.89)(18100.00,1.00,5.86)

(19100.00,20.00,1.89)(19100.00,19.00,1.94)(19100.00,18.00,2.00)(19100.00,17.00,2.07)(19100.00,16.00,2.13)(19100.00,15.00,2.20)(19100.00,14.00,2.28)(19100.00,13.00,2.37)(19100.00,12.00,2.46)(19100.00,11.00,2.57)(19100.00,10.00,2.68)(19100.00,9.00,2.81)(19100.00,8.00,2.96)(19100.00,7.00,3.13)(19100.00,6.00,3.33)(19100.00,5.00,3.56)(19100.00,4.00,3.86)(19100.00,3.00,4.25)(19100.00,2.00,4.81)(19100.00,1.00,5.79)

(20100.00,20.00,1.84)(20100.00,19.00,1.89)(20100.00,18.00,1.95)(20100.00,17.00,2.01)(20100.00,16.00,2.08)(20100.00,15.00,2.15)(20100.00,14.00,2.23)(20100.00,13.00,2.31)(20100.00,12.00,2.40)(20100.00,11.00,2.51)(20100.00,10.00,2.62)(20100.00,9.00,2.75)(20100.00,8.00,2.89)(20100.00,7.00,3.06)(20100.00,6.00,3.26)(20100.00,5.00,3.50)(20100.00,4.00,3.79)(20100.00,3.00,4.18)(20100.00,2.00,4.74)(20100.00,1.00,5.71)

(21100.00,20.00,1.79)(21100.00,19.00,1.84)(21100.00,18.00,1.90)(21100.00,17.00,1.96)(21100.00,16.00,2.02)(21100.00,15.00,2.09)(21100.00,14.00,2.17)(21100.00,13.00,2.25)(21100.00,12.00,2.35)(21100.00,11.00,2.45)(21100.00,10.00,2.56)(21100.00,9.00,2.69)(21100.00,8.00,2.83)(21100.00,7.00,3.00)(21100.00,6.00,3.20)(21100.00,5.00,3.43)(21100.00,4.00,3.73)(21100.00,3.00,4.12)(21100.00,2.00,4.67)(21100.00,1.00,5.64)

(22100.00,20.00,1.74)(22100.00,19.00,1.79)(22100.00,18.00,1.85)(22100.00,17.00,1.91)(22100.00,16.00,1.97)(22100.00,15.00,2.04)(22100.00,14.00,2.12)(22100.00,13.00,2.20)(22100.00,12.00,2.29)(22100.00,11.00,2.39)(22100.00,10.00,2.51)(22100.00,9.00,2.63)(22100.00,8.00,2.78)(22100.00,7.00,2.94)(22100.00,6.00,3.14)(22100.00,5.00,3.37)(22100.00,4.00,3.67)(22100.00,3.00,4.05)(22100.00,2.00,4.61)(22100.00,1.00,5.58)

(23100.00,20.00,1.70)(23100.00,19.00,1.75)(23100.00,18.00,1.80)(23100.00,17.00,1.86)(23100.00,16.00,1.93)(23100.00,15.00,1.99)(23100.00,14.00,2.07)(23100.00,13.00,2.15)(23100.00,12.00,2.24)(23100.00,11.00,2.34)(23100.00,10.00,2.45)(23100.00,9.00,2.58)(23100.00,8.00,2.72)(23100.00,7.00,2.89)(23100.00,6.00,3.08)(23100.00,5.00,3.32)(23100.00,4.00,3.61)(23100.00,3.00,3.99)(23100.00,2.00,4.55)(23100.00,1.00,5.52)

(24100.00,20.00,1.65)(24100.00,19.00,1.70)(24100.00,18.00,1.76)(24100.00,17.00,1.82)(24100.00,16.00,1.88)(24100.00,15.00,1.95)(24100.00,14.00,2.02)(24100.00,13.00,2.10)(24100.00,12.00,2.19)(24100.00,11.00,2.29)(24100.00,10.00,2.40)(24100.00,9.00,2.53)(24100.00,8.00,2.67)(24100.00,7.00,2.83)(24100.00,6.00,3.03)(24100.00,5.00,3.26)(24100.00,4.00,3.55)(24100.00,3.00,3.94)(24100.00,2.00,4.49)(24100.00,1.00,5.46)

(25100.00,20.00,1.61)(25100.00,19.00,1.66)(25100.00,18.00,1.72)(25100.00,17.00,1.78)(25100.00,16.00,1.84)(25100.00,15.00,1.91)(25100.00,14.00,1.98)(25100.00,13.00,2.06)(25100.00,12.00,2.15)(25100.00,11.00,2.25)(25100.00,10.00,2.36)(25100.00,9.00,2.48)(25100.00,8.00,2.62)(25100.00,7.00,2.78)(25100.00,6.00,2.98)(25100.00,5.00,3.21)(25100.00,4.00,3.50)(25100.00,3.00,3.88)(25100.00,2.00,4.43)(25100.00,1.00,5.40)

(26100.00,20.00,1.58)(26100.00,19.00,1.63)(26100.00,18.00,1.68)(26100.00,17.00,1.74)(26100.00,16.00,1.80)(26100.00,15.00,1.86)(26100.00,14.00,1.94)(26100.00,13.00,2.02)(26100.00,12.00,2.11)(26100.00,11.00,2.20)(26100.00,10.00,2.31)(26100.00,9.00,2.43)(26100.00,8.00,2.57)(26100.00,7.00,2.74)(26100.00,6.00,2.93)(26100.00,5.00,3.16)(26100.00,4.00,3.45)(26100.00,3.00,3.83)(26100.00,2.00,4.38)(26100.00,1.00,5.34)

(27100.00,20.00,1.54)(27100.00,19.00,1.59)(27100.00,18.00,1.64)(27100.00,17.00,1.70)(27100.00,16.00,1.76)(27100.00,15.00,1.83)(27100.00,14.00,1.90)(27100.00,13.00,1.98)(27100.00,12.00,2.06)(27100.00,11.00,2.16)(27100.00,10.00,2.27)(27100.00,9.00,2.39)(27100.00,8.00,2.53)(27100.00,7.00,2.69)(27100.00,6.00,2.88)(27100.00,5.00,3.11)(27100.00,4.00,3.40)(27100.00,3.00,3.78)(27100.00,2.00,4.33)(27100.00,1.00,5.29)

(28100.00,20.00,1.51)(28100.00,19.00,1.55)(28100.00,18.00,1.61)(28100.00,17.00,1.66)(28100.00,16.00,1.72)(28100.00,15.00,1.79)(28100.00,14.00,1.86)(28100.00,13.00,1.94)(28100.00,12.00,2.02)(28100.00,11.00,2.12)(28100.00,10.00,2.23)(28100.00,9.00,2.35)(28100.00,8.00,2.49)(28100.00,7.00,2.65)(28100.00,6.00,2.83)(28100.00,5.00,3.06)(28100.00,4.00,3.35)(28100.00,3.00,3.73)(28100.00,2.00,4.28)(28100.00,1.00,5.24)

(29100.00,20.00,1.47)(29100.00,19.00,1.52)(29100.00,18.00,1.57)(29100.00,17.00,1.63)(29100.00,16.00,1.69)(29100.00,15.00,1.75)(29100.00,14.00,1.82)(29100.00,13.00,1.90)(29100.00,12.00,1.99)(29100.00,11.00,2.08)(29100.00,10.00,2.19)(29100.00,9.00,2.31)(29100.00,8.00,2.44)(29100.00,7.00,2.60)(29100.00,6.00,2.79)(29100.00,5.00,3.02)(29100.00,4.00,3.31)(29100.00,3.00,3.68)(29100.00,2.00,4.23)(29100.00,1.00,5.19)

(30100.00,20.00,1.44)(30100.00,19.00,1.49)(30100.00,18.00,1.54)(30100.00,17.00,1.60)(30100.00,16.00,1.65)(30100.00,15.00,1.72)(30100.00,14.00,1.79)(30100.00,13.00,1.87)(30100.00,12.00,1.95)(30100.00,11.00,2.04)(30100.00,10.00,2.15)(30100.00,9.00,2.27)(30100.00,8.00,2.40)(30100.00,7.00,2.56)(30100.00,6.00,2.75)(30100.00,5.00,2.98)(30100.00,4.00,3.26)(30100.00,3.00,3.64)(30100.00,2.00,4.18)(30100.00,1.00,5.14)

(31100.00,20.00,1.41)(31100.00,19.00,1.46)(31100.00,18.00,1.51)(31100.00,17.00,1.56)(31100.00,16.00,1.62)(31100.00,15.00,1.69)(31100.00,14.00,1.76)(31100.00,13.00,1.83)(31100.00,12.00,1.92)(31100.00,11.00,2.01)(31100.00,10.00,2.11)(31100.00,9.00,2.23)(31100.00,8.00,2.37)(31100.00,7.00,2.52)(31100.00,6.00,2.71)(31100.00,5.00,2.94)(31100.00,4.00,3.22)(31100.00,3.00,3.60)(31100.00,2.00,4.14)(31100.00,1.00,5.10)

(32100.00,20.00,1.38)(32100.00,19.00,1.43)(32100.00,18.00,1.48)(32100.00,17.00,1.53)(32100.00,16.00,1.59)(32100.00,15.00,1.65)(32100.00,14.00,1.72)(32100.00,13.00,1.80)(32100.00,12.00,1.88)(32100.00,11.00,1.97)(32100.00,10.00,2.08)(32100.00,9.00,2.20)(32100.00,8.00,2.33)(32100.00,7.00,2.49)(32100.00,6.00,2.67)(32100.00,5.00,2.90)(32100.00,4.00,3.18)(32100.00,3.00,3.55)(32100.00,2.00,4.10)(32100.00,1.00,5.05)

(33100.00,20.00,1.36)(33100.00,19.00,1.40)(33100.00,18.00,1.45)(33100.00,17.00,1.50)(33100.00,16.00,1.56)(33100.00,15.00,1.62)(33100.00,14.00,1.69)(33100.00,13.00,1.77)(33100.00,12.00,1.85)(33100.00,11.00,1.94)(33100.00,10.00,2.04)(33100.00,9.00,2.16)(33100.00,8.00,2.29)(33100.00,7.00,2.45)(33100.00,6.00,2.63)(33100.00,5.00,2.86)(33100.00,4.00,3.14)(33100.00,3.00,3.51)(33100.00,2.00,4.06)(33100.00,1.00,5.01)

(34100.00,20.00,1.33)(34100.00,19.00,1.38)(34100.00,18.00,1.42)(34100.00,17.00,1.48)(34100.00,16.00,1.53)(34100.00,15.00,1.60)(34100.00,14.00,1.66)(34100.00,13.00,1.74)(34100.00,12.00,1.82)(34100.00,11.00,1.91)(34100.00,10.00,2.01)(34100.00,9.00,2.13)(34100.00,8.00,2.26)(34100.00,7.00,2.41)(34100.00,6.00,2.60)(34100.00,5.00,2.82)(34100.00,4.00,3.10)(34100.00,3.00,3.47)(34100.00,2.00,4.01)(34100.00,1.00,4.97)

(35100.00,20.00,1.31)(35100.00,19.00,1.35)(35100.00,18.00,1.40)(35100.00,17.00,1.45)(35100.00,16.00,1.51)(35100.00,15.00,1.57)(35100.00,14.00,1.63)(35100.00,13.00,1.71)(35100.00,12.00,1.79)(35100.00,11.00,1.88)(35100.00,10.00,1.98)(35100.00,9.00,2.10)(35100.00,8.00,2.23)(35100.00,7.00,2.38)(35100.00,6.00,2.56)(35100.00,5.00,2.79)(35100.00,4.00,3.06)(35100.00,3.00,3.44)(35100.00,2.00,3.98)(35100.00,1.00,4.93)

(36100.00,20.00,1.28)(36100.00,19.00,1.33)(36100.00,18.00,1.37)(36100.00,17.00,1.43)(36100.00,16.00,1.48)(36100.00,15.00,1.54)(36100.00,14.00,1.61)(36100.00,13.00,1.68)(36100.00,12.00,1.76)(36100.00,11.00,1.85)(36100.00,10.00,1.95)(36100.00,9.00,2.06)(36100.00,8.00,2.20)(36100.00,7.00,2.35)(36100.00,6.00,2.53)(36100.00,5.00,2.75)(36100.00,4.00,3.03)(36100.00,3.00,3.40)(36100.00,2.00,3.94)(36100.00,1.00,4.89)

(37100.00,20.00,1.26)(37100.00,19.00,1.30)(37100.00,18.00,1.35)(37100.00,17.00,1.40)(37100.00,16.00,1.46)(37100.00,15.00,1.52)(37100.00,14.00,1.58)(37100.00,13.00,1.65)(37100.00,12.00,1.73)(37100.00,11.00,1.82)(37100.00,10.00,1.92)(37100.00,9.00,2.03)(37100.00,8.00,2.17)(37100.00,7.00,2.32)(37100.00,6.00,2.50)(37100.00,5.00,2.72)(37100.00,4.00,2.99)(37100.00,3.00,3.36)(37100.00,2.00,3.90)(37100.00,1.00,4.85)

(38100.00,20.00,1.24)(38100.00,19.00,1.28)(38100.00,18.00,1.33)(38100.00,17.00,1.38)(38100.00,16.00,1.43)(38100.00,15.00,1.49)(38100.00,14.00,1.56)(38100.00,13.00,1.63)(38100.00,12.00,1.71)(38100.00,11.00,1.79)(38100.00,10.00,1.89)(38100.00,9.00,2.01)(38100.00,8.00,2.14)(38100.00,7.00,2.29)(38100.00,6.00,2.47)(38100.00,5.00,2.68)(38100.00,4.00,2.96)(38100.00,3.00,3.33)(38100.00,2.00,3.87)(38100.00,1.00,4.81)

(39100.00,20.00,1.22)(39100.00,19.00,1.26)(39100.00,18.00,1.30)(39100.00,17.00,1.35)(39100.00,16.00,1.41)(39100.00,15.00,1.47)(39100.00,14.00,1.53)(39100.00,13.00,1.60)(39100.00,12.00,1.68)(39100.00,11.00,1.77)(39100.00,10.00,1.87)(39100.00,9.00,1.98)(39100.00,8.00,2.11)(39100.00,7.00,2.26)(39100.00,6.00,2.44)(39100.00,5.00,2.65)(39100.00,4.00,2.93)(39100.00,3.00,3.30)(39100.00,2.00,3.83)(39100.00,1.00,4.78)

(40100.00,20.00,1.20)(40100.00,19.00,1.24)(40100.00,18.00,1.28)(40100.00,17.00,1.33)(40100.00,16.00,1.39)(40100.00,15.00,1.44)(40100.00,14.00,1.51)(40100.00,13.00,1.58)(40100.00,12.00,1.66)(40100.00,11.00,1.74)(40100.00,10.00,1.84)(40100.00,9.00,1.95)(40100.00,8.00,2.08)(40100.00,7.00,2.23)(40100.00,6.00,2.41)(40100.00,5.00,2.62)(40100.00,4.00,2.90)(40100.00,3.00,3.26)(40100.00,2.00,3.80)(40100.00,1.00,4.74)

(41100.00,20.00,1.18)(41100.00,19.00,1.22)(41100.00,18.00,1.26)(41100.00,17.00,1.31)(41100.00,16.00,1.36)(41100.00,15.00,1.42)(41100.00,14.00,1.48)(41100.00,13.00,1.55)(41100.00,12.00,1.63)(41100.00,11.00,1.72)(41100.00,10.00,1.81)(41100.00,9.00,1.92)(41100.00,8.00,2.05)(41100.00,7.00,2.20)(41100.00,6.00,2.38)(41100.00,5.00,2.59)(41100.00,4.00,2.87)(41100.00,3.00,3.23)(41100.00,2.00,3.76)(41100.00,1.00,4.71)

(42100.00,20.00,1.16)(42100.00,19.00,1.20)(42100.00,18.00,1.24)(42100.00,17.00,1.29)(42100.00,16.00,1.34)(42100.00,15.00,1.40)(42100.00,14.00,1.46)(42100.00,13.00,1.53)(42100.00,12.00,1.61)(42100.00,11.00,1.69)(42100.00,10.00,1.79)(42100.00,9.00,1.90)(42100.00,8.00,2.03)(42100.00,7.00,2.17)(42100.00,6.00,2.35)(42100.00,5.00,2.56)(42100.00,4.00,2.84)(42100.00,3.00,3.20)(42100.00,2.00,3.73)(42100.00,1.00,4.68)

(43100.00,20.00,1.14)(43100.00,19.00,1.18)(43100.00,18.00,1.22)(43100.00,17.00,1.27)(43100.00,16.00,1.32)(43100.00,15.00,1.38)(43100.00,14.00,1.44)(43100.00,13.00,1.51)(43100.00,12.00,1.58)(43100.00,11.00,1.67)(43100.00,10.00,1.77)(43100.00,9.00,1.87)(43100.00,8.00,2.00)(43100.00,7.00,2.15)(43100.00,6.00,2.32)(43100.00,5.00,2.54)(43100.00,4.00,2.81)(43100.00,3.00,3.17)(43100.00,2.00,3.70)(43100.00,1.00,4.64)

(44100.00,20.00,1.12)(44100.00,19.00,1.16)(44100.00,18.00,1.20)(44100.00,17.00,1.25)(44100.00,16.00,1.30)(44100.00,15.00,1.36)(44100.00,14.00,1.42)(44100.00,13.00,1.49)(44100.00,12.00,1.56)(44100.00,11.00,1.65)(44100.00,10.00,1.74)(44100.00,9.00,1.85)(44100.00,8.00,1.98)(44100.00,7.00,2.12)(44100.00,6.00,2.30)(44100.00,5.00,2.51)(44100.00,4.00,2.78)(44100.00,3.00,3.14)(44100.00,2.00,3.67)(44100.00,1.00,4.61)

(45100.00,20.00,1.10)(45100.00,19.00,1.14)(45100.00,18.00,1.19)(45100.00,17.00,1.23)(45100.00,16.00,1.28)(45100.00,15.00,1.34)(45100.00,14.00,1.40)(45100.00,13.00,1.47)(45100.00,12.00,1.54)(45100.00,11.00,1.63)(45100.00,10.00,1.72)(45100.00,9.00,1.83)(45100.00,8.00,1.95)(45100.00,7.00,2.10)(45100.00,6.00,2.27)(45100.00,5.00,2.48)(45100.00,4.00,2.75)(45100.00,3.00,3.11)(45100.00,2.00,3.64)(45100.00,1.00,4.58)

(46100.00,20.00,1.09)(46100.00,19.00,1.12)(46100.00,18.00,1.17)(46100.00,17.00,1.21)(46100.00,16.00,1.26)(46100.00,15.00,1.32)(46100.00,14.00,1.38)(46100.00,13.00,1.45)(46100.00,12.00,1.52)(46100.00,11.00,1.60)(46100.00,10.00,1.70)(46100.00,9.00,1.80)(46100.00,8.00,1.93)(46100.00,7.00,2.07)(46100.00,6.00,2.24)(46100.00,5.00,2.46)(46100.00,4.00,2.72)(46100.00,3.00,3.08)(46100.00,2.00,3.61)(46100.00,1.00,4.55)

(47100.00,20.00,1.07)(47100.00,19.00,1.11)(47100.00,18.00,1.15)(47100.00,17.00,1.20)(47100.00,16.00,1.25)(47100.00,15.00,1.30)(47100.00,14.00,1.36)(47100.00,13.00,1.43)(47100.00,12.00,1.50)(47100.00,11.00,1.58)(47100.00,10.00,1.68)(47100.00,9.00,1.78)(47100.00,8.00,1.90)(47100.00,7.00,2.05)(47100.00,6.00,2.22)(47100.00,5.00,2.43)(47100.00,4.00,2.70)(47100.00,3.00,3.06)(47100.00,2.00,3.58)(47100.00,1.00,4.52)

(48100.00,20.00,1.05)(48100.00,19.00,1.09)(48100.00,18.00,1.13)(48100.00,17.00,1.18)(48100.00,16.00,1.23)(48100.00,15.00,1.28)(48100.00,14.00,1.34)(48100.00,13.00,1.41)(48100.00,12.00,1.48)(48100.00,11.00,1.56)(48100.00,10.00,1.66)(48100.00,9.00,1.76)(48100.00,8.00,1.88)(48100.00,7.00,2.03)(48100.00,6.00,2.20)(48100.00,5.00,2.41)(48100.00,4.00,2.67)(48100.00,3.00,3.03)(48100.00,2.00,3.55)(48100.00,1.00,4.49)

(49100.00,20.00,1.04)(49100.00,19.00,1.08)(49100.00,18.00,1.12)(49100.00,17.00,1.16)(49100.00,16.00,1.21)(49100.00,15.00,1.27)(49100.00,14.00,1.32)(49100.00,13.00,1.39)(49100.00,12.00,1.46)(49100.00,11.00,1.54)(49100.00,10.00,1.64)(49100.00,9.00,1.74)(49100.00,8.00,1.86)(49100.00,7.00,2.00)(49100.00,6.00,2.17)(49100.00,5.00,2.38)(49100.00,4.00,2.65)(49100.00,3.00,3.00)(49100.00,2.00,3.53)(49100.00,1.00,4.46)


}; \end{axis} \end{tikzpicture}

		\label{hlm-ctf-dl}
      \end{subfigure} 
    \caption{HLM analysis}
		\label{hlmanalysis}
\end{figure}


Figure \ref{hlm-ctf-dl} shows HLM scores with respect to ``collection term frequency (CTF)\footnote{Also known as ``document frequency''}'' and document length (DL). 

For documents where the length is lower than 5 the differences between the scores are very marked, whereas out-with that range the progression of the scores is much more subtle. In other words, short documents such as microblogs are subject to a high impact in their scores due to small changes in their already limited length. As it can be observed in Formula \ref{hlmformula}, this sensitivity to document length is a result of the model's design, since document length acts as a multiplier in the denominator. 

In addition, within the nominator we find term frequency as a multiplying component. Consequently, when higher than 1 it will result in an unreasonable boost of the score. In the case of microblog documents this can be problematic due to the scarce frequencies which average around 1.17 ($\pm 0.48$)\footnote{Computed for query terms in all TREC microblog topics up to 2014 and our best baseline DFR}.
 
To illustrate these differences, we introduce Figure \ref{hlmcomp} where we show HLM scores w.r.t. term frequency ($f(q_i, D)$) within the 1 to 10 range. All other variables are kept constant\footnote{($c = 0.15$, $f(q_i, C) = 100$, $|D| = 5$ and $ntoks = 1000$)}.

\begin{figure}[]
  \centering
   \begin{tikzpicture}[thick,scale=0.7, every node/.style={transform shape}]
	\begin{axis}[
		xlabel=Term Frequency,
		ylabel=HLM score]
	\addplot[color=red,mark=o] coordinates {
		(1,0.436)
		(2,0.770)
		(3,1.041)
		(4,1.270)
		(5,1.467)
		(6,1.640)
		(7,1.795)
		(8,1.934)
		(9,2.062)
		(10,2.179)
	};
	\addlegendentry{c = 0.15}
	\addplot[color=blue,mark=o] coordinates {
		(1,	7.63)
		(2,8.63)
		(3,9.21)
		(4,9.63)
		(5,9.95)
		(6,10.21)
		(7,10.43)
		(8,10.63)
		(9,10.80)
		(10,11.08)
		};
	\addlegendentry{c = 0.99}
	\end{axis}
%	7.6366
%	8.6330
%	9.2167
%	9.6312
%	9.9527
%	10.2155
%	10.4378
%	10.6303
%	10.8001
%	10.9520
%	11.0895
	
\end{tikzpicture}

     \caption{TF vs HLM Score}
  \label{hlmcomp}
\end{figure}

As we can observe in Figure \ref{hlmcomp}, the low term frequencies show substantial differences between the scores. As term frequencies grow the differences between scores become increasingly smaller. The intuition is that for documents of the same length, with a higher frequency of query terms should be ranked higher. Unfortunately, for very low query term frequencies the score differences introduced by design for this purpose are too aggressive, and very unlikely correspond to the actual importance of such frequency differences.

As we can recall from Table \ref{traditional} HLM is not amongst the best performing models. \textbf{We hypothesise that the reason for this under-performance lies in the substantial scoring differences above-mentioned, which results from the specific morphology of microblog documents. }

In order to test this hypotheses we set to overestimate the vales for within document query term frequency (TF) as well as the actual document length (DL). We do this by a simple addition \(TF = TF+dTF\), in this case \(dTF\) being the pondering value to overestimate \(TF\). Likewise, we utilise \(DL = DL+dDL\) where \(dDL\) is the variable to over-estimate \(DL\).

\begin{table}[]

	\caption{P@30 scores for HLM as we consider different combinations of dTF and dDL, and c}
	\centering
	\begin{tabular}{l|c|c|c} 	
	\textit{\textbf{c}} & 
	\textit{\textbf{dTF}} & 
	\textit{\textbf{dDL}} & 
	\textit{\textbf{P@30}} 	
	\tabularnewline
	\hline
	0.15 &    &    & 0.3475\\
	0.15 & 20 &    & 0.3486\\
	0.15 &    & 20 & \textbf{0.3839} \\
	0.15 & 20 & 20 & \textbf{0.4462} \\
	\hline
	\hline
	0.05 &  &  & \textbf{0.2824} \\
	0.40 &  &  & \textbf{0.4009} \\
	0.70 &  &  & \textbf{0.4281} \\
	0.99 &  &  & \textbf{0.4492} \\
	\hline
    \hline
	0.99 & 20 & 20 & \textbf{0.4532} \\	
	\hline
	\end{tabular}
	\label{hlmOverestimates}
\end{table}

Table \ref{hlmOverestimates} shows the performance of HLM measured by Precision@30 with different configurations. The first row shows the performance of HLM with a default configuration of $c = 0.15$. 

The second row with $dTF = 20$ so that $TF = TF+20$ which denotes the overestimation of TF by +20. As stated before, the reason behind this overestimation is to reduce the differences between scores with respect to the different real values of TF. As we can observe only overestimating TF does no result in any significant improvement.

Similarly, the third row shows the performance of HLM when overestimating DL by +20 in order to reduce the effects in the score due to DL differences. As consequence the results are much better than before with a Precision@30 increase of +11.76\%. 

Finally, we test the overestimation of TF and DL together to achieve yet another +15.79\% Precision@30 increase over the previous combination and a very substantial increase of +29.41\% over the baseline configuration. It is interesting to notice how only the increase of TF does not help in retrieval, however only increasing DL does produce better results. More importantly, by incrementing both TF and DL we obtain the best performance over all previous configurations.

These results hint to a very subtle relationship between DL and TF values of microblog documents. Rows 5 to 8 in Table \ref{hlmOverestimates} show the performance of HLM with different values of $c$. As $c$ is increased performance increases as well, reaching comparable performance to the approach which overestimates DL and TF. If we look back at Figure \ref{hlmcomp} we find that as $c$ is becomes higher so do the differences in score with respect to TF. This finding on its own contradicts our hypotheses, however this is not the whole picture.

\begin{figure}[]
     \begin{subfigure}[b]{0.5\textwidth}
      \centering
      \caption{TF vs, Doc. Length (DL)  with $c = 0.15$}
       
\begin{tikzpicture}[thick,scale=0.7, every node/.style={transform shape}]\begin{axis}[
 %title={},
 %y dir=reverse, 
 x dir=reverse, 
 ylabel={docLength (DL)},
 xlabel={term frequency (TF)},
 zlabel={HLM score},
 every axis/.append style={font=\large\bfseries},
 max space between ticks=25pt,
 view={55}{45}
% yticklabels={0k,100k}
 ] 

\addplot3[surf,unbounded coords=jump]
coordinates  { 
(1,1,1)	(1,2,0.621989626156165)	(1,3,0.454919792962258)	(1,4,0.359374040891899)	(1,5,0.297247211014724)	(1,6,0.253535670926196)	(1,7,0.221079004018033)	(1,8,0.196013558516919)	(1,9,0.176066673043413)	(1,10,0.159812621390807)	(1,11,0.146311067373364)	(1,12,0.134916412227099)	(1,13,0.125170566982079)	(1,14,0.116739411565573)	(1,15,0.109373476479032)	(1,16,0.102882708780262)	(1,17,0.0971197635727905)	(1,18,0.0919686374088689)	(1,19,0.0873367506721093)	(1,20,0.0831493160857265)	(1,21,0.0793452582041037)	(1,22,0.0758742071730685)	(1,23,0.0726942505286112)	(1,24,0.0697702289388661)	(1,25,0.0670724282451611)	(1,26,0.0645755642644606)	(1,27,0.0622579866277628)	(1,28,0.0601010484168879)	(1,29,0.0580886026580694)	(1,30,0.0562065968470943)

(2,1,nan)	(2,2,1)	(2,3,0.764754026855425)	(2,4,0.621989626156165)	(2,5,0.525188803416293)	(2,6,0.454919792962258)	(2,7,0.401463820946619)	(2,8,0.359374040891899)	(2,9,0.325344772119586)	(2,10,0.297247211014724)	(2,11,0.273645732434471)	(2,12,0.253535670926196)	(2,13,0.236192227775016)	(2,14,0.221079004018033)	(2,15,0.207790318869497)	(2,16,0.196013558516919)	(2,17,0.185503863863546)	(2,18,0.176066673043413)	(2,19,0.167545408836041)	(2,20,0.159812621390807)	(2,21,0.152763503490943)	(2,22,0.146311067373364)	(2,23,0.140382505990773)	(2,24,0.134916412227099)	(2,25,0.129860628668111)	(2,26,0.125170566982079)	(2,27,0.120807881316515)	(2,28,0.116739411565573)	(2,29,0.112936334492862)	(2,30,0.109373476479032)

(3,1,nan)	(3,2,nan)	(3,3,1)	(3,4,0.829048015154822)	(3,5,0.710109539528373)	(3,6,0.621989626156165)	(3,7,0.553832024650894)	(3,8,0.499422759842862)	(3,9,0.454919792962257)	(3,10,0.417807926881635)	(3,11,0.386366066808711)	(3,12,0.359374040891899)	(3,13,0.335941294228451)	(3,14,0.315401838711434)	(3,15,0.297247211014724)	(3,16,0.281082226681238)	(3,17,0.266594929740559)	(3,18,0.253535670926196)	(3,19,0.24170222203493)	(3,20,0.230928980226506)	(3,21,0.221079004018033)	(3,22,0.212038047876564)	(3,23,0.203710031952637)	(3,24,0.196013558516919)	(3,25,0.188879202656089)	(3,26,0.182247383121429)	(3,27,0.176066673043413)	(3,28,0.170292447784477)	(3,29,0.164885793792266)	(3,30,0.159812621390807)

(4,1,nan)	(4,2,nan)	(4,3,nan)	(4,4,1)	(4,5,0.865698756849)	(4,6,0.764754026855425)	(4,7,0.685727768114339)	(4,8,0.621989626156165)	(4,9,0.569391648904037)	(4,10,0.525188803416293)	(4,11,0.487483951889205)	(4,12,0.454919792962258)	(4,13,0.426497359589787)	(4,14,0.401463820946619)	(4,15,0.379240327395148)	(4,16,0.359374040891899)	(4,17,0.341505328703937)	(4,18,0.325344772119586)	(4,19,0.310656706119163)	(4,20,0.297247211014724)	(4,21,0.284955204259961)	(4,22,0.273645732434471)	(4,23,0.263204851448174)	(4,24,0.253535670926196)	(4,25,0.244555263889519)	(4,26,0.236192227775016)	(4,27,0.228384741456421)	(4,28,0.221079004018033)	(4,29,0.21422797024867)	(4,30,0.207790318869497)

(5,1,nan)	(5,2,nan)	(5,3,nan)	(5,4,nan)	(5,5,1)	(5,6,0.889395029639195)	(5,7,0.802006501749625)	(5,8,0.730958682421364)	(5,9,0.671916477367333)	(5,10,0.621989626156165)	(5,11,0.579166557783776)	(5,12,0.541998575759629)	(5,13,0.509412832924761)	(5,14,0.480596127304584)	(5,15,0.454919792962258)	(5,16,0.431889519341492)	(5,17,0.411110867606398)	(5,18,0.392264991133567)	(5,19,0.37509117467367)	(5,20,0.359374040891899)	(5,21,0.344934020239237)	(5,22,0.331620145912424)	(5,23,0.319304533618775)	(5,24,0.307878100890691)	(5,25,0.297247211014724)	(5,26,0.287331015360257)	(5,27,0.278059329323532)	(5,28,0.269370920299868)	(5,29,0.261212116904955)	(5,30,0.253535670926196)

(6,1,nan)	(6,2,nan)	(6,3,nan)	(6,4,nan)	(6,5,nan)	(6,6,1)	(6,7,0.905978015142025)	(6,8,0.829048015154822)	(6,9,0.764754026855425)	(6,10,0.710109539528373)	(6,11,0.663025388441083)	(6,12,0.621989626156165)	(6,13,0.585877328160206)	(6,14,0.553832024650894)	(6,15,0.525188803416293)	(6,16,0.499422759842862)	(6,17,0.476113452457472)	(6,18,0.454919792962257)	(6,19,0.435561928948071)	(6,20,0.417807926881635)	(6,21,0.401463820946619)	(6,22,0.386366066808711)	(6,23,0.372375742896396)	(6,24,0.359374040891899)	(6,25,0.347258720461584)	(6,26,0.335941294228451)	(6,27,0.325344772119586)	(6,28,0.315401838711434)	(6,29,0.306053368995639)	(6,30,0.297247211014724)

(7,1,nan)	(7,2,nan)	(7,3,nan)	(7,4,nan)	(7,5,nan)	(7,6,nan)	(7,7,1)	(7,8,0.918234273546841)	(7,9,0.84957794112046)	(7,10,0.79097818724873)	(7,11,0.740291378883669)	(7,12,0.695960407129887)	(7,13,0.656822585687106)	(7,14,0.621989626156165)	(7,15,0.590769615545226)	(7,16,0.562614587372231)	(7,17,0.537084283992134)	(7,18,0.513820494881042)	(7,19,0.492528496481904)	(7,20,0.472963376930536)	(7,21,0.454919792962258)	(7,22,0.438224184168835)	(7,23,0.422728776530383)	(7,24,0.408306908660429)	(7,25,0.394849349351898)	(7,26,0.382261367362921)	(7,27,0.370460378566498)	(7,28,0.359374040891899)	(7,29,0.348938699918164)	(7,30,0.339098111501629)

(8,1,nan)	(8,2,nan)	(8,3,nan)	(8,4,nan)	(8,5,nan)	(8,6,nan)	(8,7,nan)	(8,8,1)	(8,9,0.927662416106689)	(8,10,0.865698756849)	(8,11,0.811926031756413)	(8,12,0.764754026855425)	(8,13,0.722992008524866)	(8,14,0.685727768114339)	(8,15,0.652248867041359)	(8,16,0.621989626156165)	(8,17,0.594494422029448)	(8,18,0.569391648904037)	(8,19,0.546374852391064)	(8,20,0.525188803416293)	(8,21,0.505619048373122)	(8,22,0.487483951889205)	(8,23,0.470628557307152)	(8,24,0.454919792962258)	(8,25,0.440242688612257)	(8,26,0.426497359589787)	(8,27,0.4135965811041)	(8,28,0.401463820946619)	(8,29,0.390031631699611)	(8,30,0.379240327395148)

(9,1,nan)	(9,2,nan)	(9,3,nan)	(9,4,nan)	(9,5,nan)	(9,6,nan)	(9,7,nan)	(9,8,nan)	(9,9,1)	(9,10,0.935140381695894)	(9,11,0.878694731064694)	(9,12,0.829048015154822)	(9,13,0.784988233279183)	(9,14,0.745584847988371)	(9,15,0.710109539528373)	(9,16,0.677982799728178)	(9,17,0.648736907108141)	(9,18,0.621989626156165)	(9,19,0.597425124783112)	(9,20,0.574779869158577)	(9,21,0.553832024650894)	(9,22,0.53439337361256)	(9,23,0.516303070640622)	(9,24,0.499422759842862)	(9,25,0.483632715624168)	(9,26,0.468828762278324)	(9,27,0.454919792962257)	(9,28,0.441825754802413)	(9,29,0.429476000001598)	(9,30,0.417807926881635)

(10,1,nan)	(10,2,nan)	(10,3,nan)	(10,4,nan)	(10,5,nan)	(10,6,nan)	(10,7,nan)	(10,8,nan)	(10,9,nan)	(10,10,1)	(10,11,0.941216700120071)	(10,12,0.889395029639195)	(10,13,0.843306569506414)	(10,14,0.802006501749625)	(10,15,0.764754026855425)	(10,16,0.730958682421364)	(10,17,0.700143092024258)	(10,18,0.671916477367333)	(10,19,0.645955420048794)	(10,20,0.621989626156165)	(10,21,0.599791217686162)	(10,22,0.579166557783776)	(10,23,0.559949927437822)	(10,24,0.541998575759629)	(10,25,0.525188803416293)	(10,26,0.509412832924761)	(10,27,0.494576285093826)	(10,28,0.480596127304584)	(10,29,0.467398992623918)	(10,30,0.454919792962258)

(11,1,nan)	(11,2,nan)	(11,3,nan)	(11,4,nan)	(11,5,nan)	(11,6,nan)	(11,7,nan)	(11,8,nan)	(11,9,nan)	(11,10,nan)	(11,11,1)	(11,12,0.946251773773493)	(11,13,0.898359063783389)	(11,14,0.855365760866882)	(11,15,0.816521122996797)	(11,16,0.781225898364072)	(11,17,0.748994911755086)	(11,18,0.719430440936898)	(11,19,0.692202864399309)	(11,20,0.667036329703138)	(11,21,0.643697963281252)	(11,22,0.621989626156165)	(11,23,0.601741531169737)	(11,24,0.582807242194464)	(11,25,0.56505971354562)	(11,26,0.548388122191579)	(11,27,0.532695311135566)	(11,28,0.517895708902828)	(11,29,0.50391362349877)	(11,30,0.490681833524032)

(12,1,nan)	(12,2,nan)	(12,3,nan)	(12,4,nan)	(12,5,nan)	(12,6,nan)	(12,7,nan)	(12,8,nan)	(12,9,nan)	(12,10,nan)	(12,11,nan)	(12,12,1)	(12,13,0.950492175929015)	(12,14,0.905978015142025)	(12,15,0.865698756849)	(12,16,0.829048015154822)	(12,17,0.795534246685824)	(12,18,0.764754026855425)	(12,19,0.736372612793849)	(12,20,0.710109539528373)	(12,21,0.685727768114339)	(12,22,0.663025388441083)	(12,23,0.641829190890472)	(12,24,0.621989626156165)	(12,25,0.603376810486853)	(12,26,0.585877328160206)	(12,27,0.569391648904037)	(12,28,0.553832024650894)	(12,29,0.539120763532788)	(12,30,0.525188803416293)

(13,1,nan)	(13,2,nan)	(13,3,nan)	(13,4,nan)	(13,5,nan)	(13,6,nan)	(13,7,nan)	(13,8,nan)	(13,9,nan)	(13,10,nan)	(13,11,nan)	(13,12,nan)	(13,13,1)	(13,14,0.954112309123327)	(13,15,0.912533690996205)	(13,16,0.874651908858968)	(13,17,0.839970168196483)	(13,18,0.808080318070828)	(13,19,0.778643350080918)	(13,20,0.751374939708565)	(13,21,0.726034547264215)	(13,22,0.702417079939828)	(13,23,0.680346428146221)	(13,24,0.659670394616598)	(13,25,0.640256672853748)	(13,26,0.621989626156165)	(13,27,0.604767684461415)	(13,28,0.588501222991279)	(13,29,0.573110820266669)	(13,30,0.55852581750597)

(14,1,nan)	(14,2,nan)	(14,3,nan)	(14,4,nan)	(14,5,nan)	(14,6,nan)	(14,7,nan)	(14,8,nan)	(14,9,nan)	(14,10,nan)	(14,11,nan)	(14,12,nan)	(14,13,nan)	(14,14,1)	(14,15,0.957239013621133)	(14,16,0.918234273546841)	(14,17,0.882484616547527)	(14,18,0.84957794112046)	(14,19,0.819171657582869)	(14,20,0.79097818724873)	(14,21,0.764754026855425)	(14,22,0.740291378883669)	(14,23,0.717411660231811)	(14,24,0.695960407129887)	(14,25,0.675803232372462)	(14,26,0.656822585687106)	(14,27,0.638915134124429)	(14,28,0.621989626156165)	(14,29,0.60596513679652)	(14,30,0.590769615545226)

(15,1,nan)	(15,2,nan)	(15,3,nan)	(15,4,nan)	(15,5,nan)	(15,6,nan)	(15,7,nan)	(15,8,nan)	(15,9,nan)	(15,10,nan)	(15,11,nan)	(15,12,nan)	(15,13,nan)	(15,14,nan)	(15,15,1)	(15,16,0.959966748341296)	(15,17,0.92323690127406)	(15,18,0.889395029639195)	(15,19,0.85809551482511)	(15,20,0.829048015154822)	(15,21,0.802006501749625)	(15,22,0.776760863890398)	(15,23,0.753130395817862)	(15,24,0.730958682421364)	(15,25,0.710109539528373)	(15,26,0.690463759301656)	(15,27,0.671916477367333)	(15,28,0.654375025136852)	(15,29,0.637757164447431)	(15,30,0.621989626156165)

}; \end{axis} \end{tikzpicture}

       	\label{cTFVSDL15}
    \end{subfigure}  
      ~
     \begin{subfigure}[b]{0.5\textwidth}
      \centering
      \caption{TF vs, Doc. Length (DL)  with $c = 0.99$}
       
\begin{tikzpicture}[thick,scale=0.7, every node/.style={transform shape}]\begin{axis}[
%title={},
%y dir=reverse, 
x dir=reverse, 
ylabel={docLength (DL)},
xlabel={term frequency (TF)},
zlabel={HLM score},
every axis/.append style={font=\large\bfseries},
max space between ticks=25pt,
view={55}{45}
% yticklabels={0k,100k}
] 

\addplot3[surf,unbounded coords=jump]
coordinates  { 
(1,1,1)	(1,2,0.899671365707223)	(1,3,0.841043406928067)	(1,4,0.799488487034831)	(1,5,0.767288572151157)	(1,6,0.741005844337785)	(1,7,0.718806466948285)	(1,8,0.699595803046279)	(1,9,0.682667789466625)	(1,10,0.667540331328603)	(1,11,0.653869570058235)	(1,12,0.64140161327701)	(1,13,0.629943652382547)	(1,14,0.619345814263553)	(1,15,0.609489289412862)	(1,16,0.600278299357417)	(1,17,0.591634503551573)	(1,18,0.583493007386208)	(1,19,0.575799450958578)	(1,20,0.568507845449242)	(1,21,0.561578937933734)	(1,22,0.554978956939974)	(1,23,0.548678637082807)	(1,24,0.542652451434819)	(1,25,0.536878000702402)	(1,26,0.531335522273925)	(1,27,0.526007491973526)	(1,28,0.520878298276225)	(1,29,0.515933973717023)	(1,30,0.511161971852579)

(2,1,nan)	(2,2,1)	(2,3,0.941299108339227)	(2,4,0.899671365707223)	(2,5,0.867398737969139)	(2,6,0.841043406928067)	(2,7,0.818771535680754)	(2,8,0.799488487034831)	(2,9,0.782488197569988)	(2,10,0.767288572151157)	(2,11,0.753545751950832)	(2,12,0.741005844337785)	(2,13,0.729476040457734)	(2,14,0.718806466948285)	(2,15,0.708878314051983)	(2,16,0.699595803046279)	(2,17,0.690880593136834)	(2,18,0.682667789466625)	(2,19,0.674903031885802)	(2,20,0.667540331328603)	(2,21,0.660540434625025)	(2,22,0.653869570058235)	(2,23,0.647498471999103)	(2,24,0.64140161327701)	(2,25,0.635556594355922)	(2,26,0.629943652382547)	(2,27,0.624545262940128)	(2,28,0.619345814263553)	(2,29,0.61433133864845)	(2,30,0.609489289412862)

(3,1,nan)	(3,2,nan)	(3,3,1)	(3,4,0.958347925979563)	(3,5,0.926050987289633)	(3,6,0.899671365707223)	(3,7,0.877375224303047)	(3,8,0.858067925858746)	(3,9,0.841043406928067)	(3,10,0.825819572350041)	(3,11,0.812052563271309)	(3,12,0.799488487034831)	(3,13,0.78793453476056)	(3,14,0.77724083306038)	(3,15,0.767288572151157)	(3,16,0.75798197328471)	(3,17,0.749242695641108)	(3,18,0.741005844337785)	(3,19,0.733217059199386)	(3,20,0.725830351134692)	(3,21,0.718806466948285)	(3,22,0.712111634897961)	(3,23,0.705716589329259)	(3,24,0.699595803046279)	(3,25,0.693726876487745)	(3,26,0.688090046775164)	(3,27,0.682667789466625)	(3,28,0.677444492771904)	(3,29,0.672406188961558)	(3,30,0.667540331328603)

(4,1,nan)	(4,2,nan)	(4,3,nan)	(4,4,1)	(4,5,0.967690892037064)	(4,6,0.941299108339227)	(4,7,0.918990811970849)	(4,8,0.899671365707223)	(4,9,0.882634706095753)	(4,10,0.867398737969139)	(4,11,0.853619602467696)	(4,12,0.841043406928067)	(4,13,0.829477342463892)	(4,14,0.818771535680754)	(4,15,0.808807176789224)	(4,16,0.799488487034831)	(4,17,0.790737125591366)	(4,18,0.782488197569988)	(4,19,0.774687342789078)	(4,20,0.767288572151157)	(4,21,0.760252632454558)	(4,22,0.753545751950832)	(4,23,0.747138664979285)	(4,24,0.741005844337785)	(4,25,0.735124890458836)	(4,26,0.729476040457734)	(4,27,0.724041769886358)	(4,28,0.718806466948285)	(4,29,0.713756163907883)	(4,30,0.708878314051983)

(5,1,nan)	(5,2,nan)	(5,3,nan)	(5,4,nan)	(5,5,1)	(5,6,0.973600913633586)	(5,7,0.951285317910699)	(5,8,0.931958575604364)	(5,9,0.914914623259726)	(5,10,0.899671365707223)	(5,11,0.88588494408491)	(5,12,0.873301465727175)	(5,13,0.861728121745405)	(5,14,0.851015038742928)	(5,15,0.841043406928067)	(5,16,0.831717447544101)	(5,17,0.822958819762576)	(5,18,0.814702628692406)	(5,19,0.80689451414973)	(5,20,0.799488487034831)	(5,21,0.792445294143802)	(5,22,0.785731163725958)	(5,23,0.779316830118372)	(5,24,0.773176766116681)	(5,25,0.767288572151157)	(5,26,0.761632485334869)	(5,27,0.756190981217471)	(5,28,0.750948448000317)	(5,29,0.745890917945551)	(5,30,0.741005844337785)

(6,1,nan)	(6,2,nan)	(6,3,nan)	(6,4,nan)	(6,5,nan)	(6,6,1)	(6,7,0.977679535381201)	(6,8,0.958347925979563)	(6,9,0.941299108339227)	(6,10,0.926050987289633)	(6,11,0.912259703967836)	(6,12,0.899671365707223)	(6,13,0.888093163618182)	(6,14,0.877375224303047)	(6,15,0.867398737969139)	(6,16,0.858067925858746)	(6,17,0.849304447142415)	(6,18,0.841043406928067)	(6,19,0.833230445030845)	(6,20,0.825819572350041)	(6,21,0.818771535680754)	(6,22,0.812052563271309)	(6,23,0.805633389457785)	(6,24,0.799488487034831)	(6,25,0.79359545643173)	(6,26,0.78793453476056)	(6,27,0.782488197569988)	(6,28,0.77724083306038)	(6,29,0.772178473492895)	(6,30,0.767288572151157)

(7,1,nan)	(7,2,nan)	(7,3,nan)	(7,4,nan)	(7,5,nan)	(7,6,nan)	(7,7,1)	(7,8,0.980664912598375)	(7,9,0.963612618043915)	(7,10,0.948361021165549)	(7,11,0.934566263099823)	(7,12,0.921974451179616)	(7,13,0.910392776514808)	(7,14,0.899671365707223)	(7,15,0.889691408963676)	(7,16,0.880357127525945)	(7,17,0.871590180564072)	(7,18,0.863325673185471)	(7,19,0.855509245204779)	(7,20,0.848094907520779)	(7,21,0.841043406928067)	(7,22,0.834320971674459)	(7,23,0.827898336095533)	(7,24,0.82174997298543)	(7,25,0.815853482772928)	(7,26,0.810189102569603)	(7,27,0.804739307923617)	(7,28,0.799488487034831)	(7,29,0.794422672163902)	(7,30,0.789529316593951)

(8,1,nan)	(8,2,nan)	(8,3,nan)	(8,4,nan)	(8,5,nan)	(8,6,nan)	(8,7,nan)	(8,8,1)	(8,9,0.982945096828066)	(8,10,0.967690892037064)	(8,11,0.953893526763254)	(8,12,0.941299108339227)	(8,13,0.92971482787458)	(8,14,0.918990811970849)	(8,15,0.909008250834566)	(8,16,0.899671365707223)	(8,17,0.890901815758575)	(8,18,0.882634706095753)	(8,19,0.874815676533108)	(8,20,0.867398737969139)	(8,21,0.860344637198157)	(8,22,0.853619602467696)	(8,23,0.847194368113046)	(8,24,0.841043406928067)	(8,25,0.835144319341251)	(8,26,0.829477342463892)	(8,27,0.824024951843867)	(8,28,0.818771535680754)	(8,29,0.813703126234927)	(8,30,0.808807176789224)

(9,1,nan)	(9,2,nan)	(9,3,nan)	(9,4,nan)	(9,5,nan)	(9,6,nan)	(9,7,nan)	(9,8,nan)	(9,9,1)	(9,10,0.984743766215252)	(9,11,0.970944372430966)	(9,12,0.958347925979563)	(9,13,0.946761617970464)	(9,14,0.936035575005035)	(9,15,0.926050987289633)	(9,16,0.91671207606558)	(9,17,0.907940500502457)	(9,18,0.899671365707223)	(9,19,0.891850311494057)	(9,20,0.884431348761285)	(9,21,0.877375224303047)	(9,22,0.870648166366704)	(9,23,0.864220909287374)	(9,24,0.858067925858746)	(9,25,0.852166816509141)	(9,26,0.846497818349679)	(9,27,0.841043406928067)	(9,28,0.835787970443711)	(9,29,0.830717541156813)	(9,30,0.825819572350041)

(10,1,nan)	(10,2,nan)	(10,3,nan)	(10,4,nan)	(10,5,nan)	(10,6,nan)	(10,7,nan)	(10,8,nan)	(10,9,nan)	(10,10,1)	(10,11,0.986198982977501)	(10,12,0.973600913633586)	(10,13,0.962012983077567)	(10,14,0.951285317910699)	(10,15,0.941299108339227)	(10,16,0.931958575604364)	(10,17,0.923185378875581)	(10,18,0.914914623259726)	(10,19,0.907091948570867)	(10,20,0.899671365707223)	(10,21,0.892613621462819)	(10,22,0.88588494408491)	(10,23,0.879456067908503)	(10,24,0.873301465727175)	(10,25,0.867398737969139)	(10,26,0.861728121745405)	(10,27,0.856272092603569)	(10,28,0.851015038742928)	(10,29,0.845942992423575)	(10,30,0.841043406928067)

(11,1,nan)	(11,2,nan)	(11,3,nan)	(11,4,nan)	(11,5,nan)	(11,6,nan)	(11,7,nan)	(11,8,nan)	(11,9,nan)	(11,10,nan)	(11,11,1)	(11,12,0.987400602523669)	(11,13,0.975811344091026)	(11,14,0.965082351303253)	(11,15,0.955094814366521)	(11,16,0.945752954521968)	(11,17,0.936978430938991)	(11,18,0.928706348724365)	(11,19,0.920882347692086)	(11,20,0.913460438740296)	(11,21,0.906401368662948)	(11,22,0.899671365707223)	(11,23,0.893241164208054)	(11,24,0.887085236958944)	(11,25,0.881181184388034)	(11,26,0.875509243606258)	(11,27,0.870051890161141)	(11,28,0.864793512251906)	(11,29,0.859720142138569)	(11,30,0.854819233103617)

(12,1,nan)	(12,2,nan)	(12,3,nan)	(12,4,nan)	(12,5,nan)	(12,6,nan)	(12,7,nan)	(12,8,nan)	(12,9,nan)	(12,10,nan)	(12,11,nan)	(12,12,1)	(12,13,0.988409634770886)	(12,14,0.977679535381201)	(12,15,0.967690892037064)	(12,16,0.958347925979563)	(12,17,0.949572296378043)	(12,18,0.941299108339227)	(12,19,0.93347400167706)	(12,20,0.926050987289633)	(12,21,0.918990811970849)	(12,22,0.912259703967836)	(12,23,0.905828397615476)	(12,24,0.899671365707223)	(12,25,0.893766208671163)	(12,26,0.888093163618182)	(12,27,0.882634706095753)	(12,28,0.877375224303047)	(12,29,0.872300750500031)	(12,30,0.867398737969139)

(13,1,nan)	(13,2,nan)	(13,3,nan)	(13,4,nan)	(13,5,nan)	(13,6,nan)	(13,7,nan)	(13,8,nan)	(13,9,nan)	(13,10,nan)	(13,11,nan)	(13,12,nan)	(13,13,1)	(13,14,0.989268964076458)	(13,15,0.979279384349882)	(13,16,0.969935482061322)	(13,17,0.961158916380088)	(13,18,0.952884792412865)	(13,19,0.945058749973562)	(13,20,0.937634799960232)	(13,21,0.930573689166742)	(13,22,0.923841645840184)	(13,23,0.917409404315403)	(13,24,0.911251437385816)	(13,25,0.905345345479473)	(13,26,0.899671365707223)	(13,27,0.894211973616501)	(13,28,0.888951557406445)	(13,29,0.883876149336982)	(13,30,0.878973202690511)

(14,1,nan)	(14,2,nan)	(14,3,nan)	(14,4,nan)	(14,5,nan)	(14,6,nan)	(14,7,nan)	(14,8,nan)	(14,9,nan)	(14,10,nan)	(14,11,nan)	(14,12,nan)	(14,13,nan)	(14,14,1)	(14,15,0.990009617520106)	(14,16,0.980664912598375)	(14,17,0.971887544404093)	(14,18,0.963612618043915)	(14,19,0.955785773331725)	(14,20,0.948361021165549)	(14,21,0.941299108339227)	(14,22,0.934566263099823)	(14,23,0.928133219782156)	(14,24,0.921974451179616)	(14,25,0.916067557720226)	(14,26,0.910392776514808)	(14,27,0.90493258311077)	(14,28,0.899671365707223)	(14,29,0.894595156564068)	(14,30,0.889691408963676)

(15,1,nan)	(15,2,nan)	(15,3,nan)	(15,4,nan)	(15,5,nan)	(15,6,nan)	(15,7,nan)	(15,8,nan)	(15,9,nan)	(15,10,nan)	(15,11,nan)	(15,12,nan)	(15,13,nan)	(15,14,nan)	(15,15,1)	(15,16,0.99065459935127)	(15,17,0.981876535526919)	(15,18,0.973600913633586)	(15,19,0.965773373485132)	(15,20,0.958347925979563)	(15,21,0.951285317910699)	(15,22,0.944551777525583)	(15,23,0.938118039159015)	(15,24,0.931958575604364)	(15,25,0.926050987289633)	(15,26,0.920375511325624)	(15,27,0.914914623259726)	(15,28,0.909652711291026)	(15,29,0.904575807679408)	(15,30,0.899671365707223)

}; \end{axis} \end{tikzpicture}

       \label{cTFVSDL99}
    \end{subfigure}  
    \caption{HLM analysis}
	\label{cTFVSDL}
\end{figure}

Figures \ref{cTFVSDL15} and \ref{cTFVSDL99} show scores produced by HLM w.r.t. TF and DL. Figure \ref{cTFVSDL15} sets $c=0.15$ whereas Figure \ref{cTFVSDL99} sets $c=0.99$. It is easily observed how the HLM scores differ between the two figures. Moreover Figure \ref{cTFVSDL15} shows more differences across the spectrum of scores with respect to TF and DL than Figure \ref{cTFVSDL99}. We can also observe how over-estimating DL and TF forces the values of HLM to lie in the more stable area of the Figures. Furthermore, Figure \ref{cTFVSDL99} produces the most stable scores, even when the progression of values with respect to TF may be slightly more abrupt at lower TF points.

We can conclude from these results that retrieving microblogs requires a conservative, delicate and balanced estimation of the importance of TF and DL.

\subsection{The DLM Case}
Dirichlet Smoothed language model (DLM), was the baseline retrieval model for the 2013 and 2014 iterations of the microblog track, within the "Microblog track as a service" client. DLM has a smoothing parameter named $\mu$, which was set to 2500 by default during the 2013 and 2014 microblog tracks. Moreover, DLM scores are produced \footnote{As implemented in the Terrier IR platform} by the following equation:

\begin{small}
\begin{align}
\label{dlmformula}
    \text{DLM}(D,Q) &= \sum_{i=1}^{n} \log_2 \left[ 1 + \frac{f(q_i, D)}{\mu \cdot \frac{ f(q_i, C) }{ ntoks }}\right] + \log_2 \left[\frac{\mu}{|D| + \mu}\right]
\end{align}
\label{dlmequation}
%\end{proof}
\end{small}

\noindent where $ntoks$ refers to the number of unique tokens in the collection (NT), $\mu$ is a free parameter, and $C$ represents the set of all documents in the collection. $f(q_i, D)$ represents the TF of a query term $q_i$ in document $D$, whereas $f(q_i, C)$ is the collection document frequency (CTF) of term $q_i$.

\begin{figure}
 		\begin{subfigure}[]{0.5\textwidth}
     	\caption{Document Frequency and $\mu$ parameter} 
    	
\begin{tikzpicture}[thick,scale=0.7, every node/.style={transform shape}]\begin{axis}[
 %title={},
 %y dir=reverse, 
 %x dir=reverse, 
 ylabel={$\mu$},
 xlabel={docFrequency},
 zlabel={DirichletLM score},
 every axis/.append style={font=\large\bfseries},
 max space between ticks=25pt
% yticklabels={0k,100k}
 ] 

		\addplot3[surf] coordinates { 
%patch,patch type=biquadratic, shader=faceted,patch refines=3
(100.00,1.00,11.45)(100.00,501.00,6.83)(100.00,1001.00,5.87)(100.00,1501.00,5.30)(100.00,2001.00,4.91)(100.00,2501.00,4.60)(100.00,3001.00,4.35)(100.00,3501.00,4.14)(100.00,4001.00,3.96)(100.00,4501.00,3.80)

(1100.00,1.00,7.99)(1100.00,501.00,3.48)(1100.00,1001.00,2.63)(1100.00,1501.00,2.17)(1100.00,2001.00,1.86)(1100.00,2501.00,1.63)(1100.00,3001.00,1.46)(1100.00,3501.00,1.33)(1100.00,4001.00,1.21)(1100.00,4501.00,1.12)

(2100.00,1.00,7.05)(2100.00,501.00,2.66)(2100.00,1001.00,1.89)(2100.00,1501.00,1.50)(2100.00,2001.00,1.25)(2100.00,2501.00,1.07)(2100.00,3001.00,0.94)(2100.00,3501.00,0.84)(2100.00,4001.00,0.76)(2100.00,4501.00,0.69)

(3100.00,1.00,6.49)(3100.00,501.00,2.20)(3100.00,1001.00,1.50)(3100.00,1501.00,1.16)(3100.00,2001.00,0.95)(3100.00,2501.00,0.80)(3100.00,3001.00,0.70)(3100.00,3501.00,0.61)(3100.00,4001.00,0.55)(3100.00,4501.00,0.50)

(4100.00,1.00,6.09)(4100.00,501.00,1.89)(4100.00,1001.00,1.25)(4100.00,1501.00,0.95)(4100.00,2001.00,0.76)(4100.00,2501.00,0.64)(4100.00,3001.00,0.55)(4100.00,3501.00,0.49)(4100.00,4001.00,0.43)(4100.00,4501.00,0.39)

(5100.00,1.00,5.78)(5100.00,501.00,1.66)(5100.00,1001.00,1.07)(5100.00,1501.00,0.80)(5100.00,2001.00,0.64)(5100.00,2501.00,0.53)(5100.00,3001.00,0.46)(5100.00,3501.00,0.40)(5100.00,4001.00,0.36)(5100.00,4501.00,0.32)

(6100.00,1.00,5.52)(6100.00,501.00,1.49)(6100.00,1001.00,0.94)(6100.00,1501.00,0.69)(6100.00,2001.00,0.55)(6100.00,2501.00,0.46)(6100.00,3001.00,0.39)(6100.00,3501.00,0.34)(6100.00,4001.00,0.30)(6100.00,4501.00,0.27)

(7100.00,1.00,5.30)(7100.00,501.00,1.35)(7100.00,1001.00,0.84)(7100.00,1501.00,0.61)(7100.00,2001.00,0.48)(7100.00,2501.00,0.40)(7100.00,3001.00,0.34)(7100.00,3501.00,0.30)(7100.00,4001.00,0.26)(7100.00,4501.00,0.24)

(8100.00,1.00,5.11)(8100.00,501.00,1.23)(8100.00,1001.00,0.76)(8100.00,1501.00,0.55)(8100.00,2001.00,0.43)(8100.00,2501.00,0.36)(8100.00,3001.00,0.30)(8100.00,3501.00,0.26)(8100.00,4001.00,0.23)(8100.00,4501.00,0.21)

(9100.00,1.00,4.94)(9100.00,501.00,1.14)(9100.00,1001.00,0.69)(9100.00,1501.00,0.50)(9100.00,2001.00,0.39)(9100.00,2501.00,0.32)(9100.00,3001.00,0.27)(9100.00,3501.00,0.24)(9100.00,4001.00,0.21)(9100.00,4501.00,0.19)

(10100.00,1.00,4.79)(10100.00,501.00,1.05)(10100.00,1001.00,0.63)(10100.00,1501.00,0.45)(10100.00,2001.00,0.35)(10100.00,2501.00,0.29)(10100.00,3001.00,0.25)(10100.00,3501.00,0.21)(10100.00,4001.00,0.19)(10100.00,4501.00,0.17)

(11100.00,1.00,4.65)(11100.00,501.00,0.98)(11100.00,1001.00,0.58)(11100.00,1501.00,0.42)(11100.00,2001.00,0.32)(11100.00,2501.00,0.26)(11100.00,3001.00,0.22)(11100.00,3501.00,0.19)(11100.00,4001.00,0.17)(11100.00,4501.00,0.15)

(12100.00,1.00,4.53)(12100.00,501.00,0.92)(12100.00,1001.00,0.54)(12100.00,1501.00,0.38)(12100.00,2001.00,0.30)(12100.00,2501.00,0.24)(12100.00,3001.00,0.21)(12100.00,3501.00,0.18)(12100.00,4001.00,0.16)(12100.00,4501.00,0.14)

(13100.00,1.00,4.42)(13100.00,501.00,0.86)(13100.00,1001.00,0.50)(13100.00,1501.00,0.36)(13100.00,2001.00,0.28)(13100.00,2501.00,0.23)(13100.00,3001.00,0.19)(13100.00,3501.00,0.17)(13100.00,4001.00,0.15)(13100.00,4501.00,0.13)

(14100.00,1.00,4.31)(14100.00,501.00,0.82)(14100.00,1001.00,0.47)(14100.00,1501.00,0.33)(14100.00,2001.00,0.26)(14100.00,2501.00,0.21)(14100.00,3001.00,0.18)(14100.00,3501.00,0.15)(14100.00,4001.00,0.14)(14100.00,4501.00,0.12)

(15100.00,1.00,4.21)(15100.00,501.00,0.77)(15100.00,1001.00,0.44)(15100.00,1501.00,0.31)(15100.00,2001.00,0.24)(15100.00,2501.00,0.20)(15100.00,3001.00,0.17)(15100.00,3501.00,0.14)(15100.00,4001.00,0.13)(15100.00,4501.00,0.11)

(16100.00,1.00,4.12)(16100.00,501.00,0.73)(16100.00,1001.00,0.42)(16100.00,1501.00,0.29)(16100.00,2001.00,0.23)(16100.00,2501.00,0.18)(16100.00,3001.00,0.16)(16100.00,3501.00,0.13)(16100.00,4001.00,0.12)(16100.00,4501.00,0.11)

(17100.00,1.00,4.03)(17100.00,501.00,0.70)(17100.00,1001.00,0.40)(17100.00,1501.00,0.28)(17100.00,2001.00,0.21)(17100.00,2501.00,0.17)(17100.00,3001.00,0.15)(17100.00,3501.00,0.13)(17100.00,4001.00,0.11)(17100.00,4501.00,0.10)

(18100.00,1.00,3.95)(18100.00,501.00,0.66)(18100.00,1001.00,0.38)(18100.00,1501.00,0.26)(18100.00,2001.00,0.20)(18100.00,2501.00,0.16)(18100.00,3001.00,0.14)(18100.00,3501.00,0.12)(18100.00,4001.00,0.11)(18100.00,4501.00,0.09)

(19100.00,1.00,3.87)(19100.00,501.00,0.63)(19100.00,1001.00,0.36)(19100.00,1501.00,0.25)(19100.00,2001.00,0.19)(19100.00,2501.00,0.16)(19100.00,3001.00,0.13)(19100.00,3501.00,0.11)(19100.00,4001.00,0.10)(19100.00,4501.00,0.09)

(20100.00,1.00,3.80)(20100.00,501.00,0.61)(20100.00,1001.00,0.34)(20100.00,1501.00,0.24)(20100.00,2001.00,0.18)(20100.00,2501.00,0.15)(20100.00,3001.00,0.12)(20100.00,3501.00,0.11)(20100.00,4001.00,0.09)(20100.00,4501.00,0.08)

(21100.00,1.00,3.73)(21100.00,501.00,0.58)(21100.00,1001.00,0.32)(21100.00,1501.00,0.23)(21100.00,2001.00,0.17)(21100.00,2501.00,0.14)(21100.00,3001.00,0.12)(21100.00,3501.00,0.10)(21100.00,4001.00,0.09)(21100.00,4501.00,0.08)

(22100.00,1.00,3.66)(22100.00,501.00,0.56)(22100.00,1001.00,0.31)(22100.00,1501.00,0.22)(22100.00,2001.00,0.17)(22100.00,2501.00,0.13)(22100.00,3001.00,0.11)(22100.00,3501.00,0.10)(22100.00,4001.00,0.09)(22100.00,4501.00,0.08)

(23100.00,1.00,3.60)(23100.00,501.00,0.53)(23100.00,1001.00,0.30)(23100.00,1501.00,0.21)(23100.00,2001.00,0.16)(23100.00,2501.00,0.13)(23100.00,3001.00,0.11)(23100.00,3501.00,0.09)(23100.00,4001.00,0.08)(23100.00,4501.00,0.07)

(24100.00,1.00,3.54)(24100.00,501.00,0.51)(24100.00,1001.00,0.29)(24100.00,1501.00,0.20)(24100.00,2001.00,0.15)(24100.00,2501.00,0.12)(24100.00,3001.00,0.10)(24100.00,3501.00,0.09)(24100.00,4001.00,0.08)(24100.00,4501.00,0.07)

(25100.00,1.00,3.48)(25100.00,501.00,0.50)(25100.00,1001.00,0.27)(25100.00,1501.00,0.19)(25100.00,2001.00,0.14)(25100.00,2501.00,0.12)(25100.00,3001.00,0.10)(25100.00,3501.00,0.08)(25100.00,4001.00,0.07)(25100.00,4501.00,0.07)

%(26100.00,1.00,3.43)(26100.00,501.00,0.48)(26100.00,1001.00,0.26)(26100.00,1501.00,0.18)(26100.00,2001.00,0.14)(26100.00,2501.00,0.11)(26100.00,3001.00,0.09)(26100.00,3501.00,0.08)(26100.00,4001.00,0.07)(26100.00,4501.00,0.06)
%
%(27100.00,1.00,3.37)(27100.00,501.00,0.46)(27100.00,1001.00,0.25)(27100.00,1501.00,0.18)(27100.00,2001.00,0.13)(27100.00,2501.00,0.11)(27100.00,3001.00,0.09)(27100.00,3501.00,0.08)(27100.00,4001.00,0.07)(27100.00,4501.00,0.06)
%
%(28100.00,1.00,3.32)(28100.00,501.00,0.45)(28100.00,1001.00,0.24)(28100.00,1501.00,0.17)(28100.00,2001.00,0.13)(28100.00,2501.00,0.10)(28100.00,3001.00,0.09)(28100.00,3501.00,0.08)(28100.00,4001.00,0.07)(28100.00,4501.00,0.06)
%
%(29100.00,1.00,3.27)(29100.00,501.00,0.43)(29100.00,1001.00,0.24)(29100.00,1501.00,0.16)(29100.00,2001.00,0.12)(29100.00,2501.00,0.10)(29100.00,3001.00,0.08)(29100.00,3501.00,0.07)(29100.00,4001.00,0.06)(29100.00,4501.00,0.06)
%
%(30100.00,1.00,3.22)(30100.00,501.00,0.42)(30100.00,1001.00,0.23)(30100.00,1501.00,0.16)(30100.00,2001.00,0.12)(30100.00,2501.00,0.10)(30100.00,3001.00,0.08)(30100.00,3501.00,0.07)(30100.00,4001.00,0.06)(30100.00,4501.00,0.05)
%
%(31100.00,1.00,3.17)(31100.00,501.00,0.40)(31100.00,1001.00,0.22)(31100.00,1501.00,0.15)(31100.00,2001.00,0.12)(31100.00,2501.00,0.09)(31100.00,3001.00,0.08)(31100.00,3501.00,0.07)(31100.00,4001.00,0.06)(31100.00,4501.00,0.05)
%
%(32100.00,1.00,3.13)(32100.00,501.00,0.39)(32100.00,1001.00,0.21)(32100.00,1501.00,0.15)(32100.00,2001.00,0.11)(32100.00,2501.00,0.09)(32100.00,3001.00,0.08)(32100.00,3501.00,0.07)(32100.00,4001.00,0.06)(32100.00,4501.00,0.05)
%
%(33100.00,1.00,3.08)(33100.00,501.00,0.38)(33100.00,1001.00,0.21)(33100.00,1501.00,0.14)(33100.00,2001.00,0.11)(33100.00,2501.00,0.09)(33100.00,3001.00,0.07)(33100.00,3501.00,0.06)(33100.00,4001.00,0.06)(33100.00,4501.00,0.05)
%
%(34100.00,1.00,3.04)(34100.00,501.00,0.37)(34100.00,1001.00,0.20)(34100.00,1501.00,0.14)(34100.00,2001.00,0.10)(34100.00,2501.00,0.08)(34100.00,3001.00,0.07)(34100.00,3501.00,0.06)(34100.00,4001.00,0.05)(34100.00,4501.00,0.05)
%
%(35100.00,1.00,3.00)(35100.00,501.00,0.36)(35100.00,1001.00,0.19)(35100.00,1501.00,0.13)(35100.00,2001.00,0.10)(35100.00,2501.00,0.08)(35100.00,3001.00,0.07)(35100.00,3501.00,0.06)(35100.00,4001.00,0.05)(35100.00,4501.00,0.05)
%
%(36100.00,1.00,2.96)(36100.00,501.00,0.35)(36100.00,1001.00,0.19)(36100.00,1501.00,0.13)(36100.00,2001.00,0.10)(36100.00,2501.00,0.08)(36100.00,3001.00,0.07)(36100.00,3501.00,0.06)(36100.00,4001.00,0.05)(36100.00,4501.00,0.04)
%
%(37100.00,1.00,2.92)(37100.00,501.00,0.34)(37100.00,1001.00,0.18)(37100.00,1501.00,0.13)(37100.00,2001.00,0.10)(37100.00,2501.00,0.08)(37100.00,3001.00,0.06)(37100.00,3501.00,0.06)(37100.00,4001.00,0.05)(37100.00,4501.00,0.04)
%
%(38100.00,1.00,2.88)(38100.00,501.00,0.33)(38100.00,1001.00,0.18)(38100.00,1501.00,0.12)(38100.00,2001.00,0.09)(38100.00,2501.00,0.07)(38100.00,3001.00,0.06)(38100.00,3501.00,0.05)(38100.00,4001.00,0.05)(38100.00,4501.00,0.04)
%
%(39100.00,1.00,2.85)(39100.00,501.00,0.32)(39100.00,1001.00,0.17)(39100.00,1501.00,0.12)(39100.00,2001.00,0.09)(39100.00,2501.00,0.07)(39100.00,3001.00,0.06)(39100.00,3501.00,0.05)(39100.00,4001.00,0.05)(39100.00,4501.00,0.04)
%
%(40100.00,1.00,2.81)(40100.00,501.00,0.31)(40100.00,1001.00,0.17)(40100.00,1501.00,0.11)(40100.00,2001.00,0.09)(40100.00,2501.00,0.07)(40100.00,3001.00,0.06)(40100.00,3501.00,0.05)(40100.00,4001.00,0.04)(40100.00,4501.00,0.04)
%
%(41100.00,1.00,2.77)(41100.00,501.00,0.30)(41100.00,1001.00,0.16)(41100.00,1501.00,0.11)(41100.00,2001.00,0.08)(41100.00,2501.00,0.07)(41100.00,3001.00,0.06)(41100.00,3501.00,0.05)(41100.00,4001.00,0.04)(41100.00,4501.00,0.04)
%
%(42100.00,1.00,2.74)(42100.00,501.00,0.30)(42100.00,1001.00,0.16)(42100.00,1501.00,0.11)(42100.00,2001.00,0.08)(42100.00,2501.00,0.07)(42100.00,3001.00,0.06)(42100.00,3501.00,0.05)(42100.00,4001.00,0.04)(42100.00,4501.00,0.04)
%
%(43100.00,1.00,2.71)(43100.00,501.00,0.29)(43100.00,1001.00,0.16)(43100.00,1501.00,0.11)(43100.00,2001.00,0.08)(43100.00,2501.00,0.06)(43100.00,3001.00,0.05)(43100.00,3501.00,0.05)(43100.00,4001.00,0.04)(43100.00,4501.00,0.04)
%
%(44100.00,1.00,2.67)(44100.00,501.00,0.28)(44100.00,1001.00,0.15)(44100.00,1501.00,0.10)(44100.00,2001.00,0.08)(44100.00,2501.00,0.06)(44100.00,3001.00,0.05)(44100.00,3501.00,0.05)(44100.00,4001.00,0.04)(44100.00,4501.00,0.04)
%
%(45100.00,1.00,2.64)(45100.00,501.00,0.28)(45100.00,1001.00,0.15)(45100.00,1501.00,0.10)(45100.00,2001.00,0.08)(45100.00,2501.00,0.06)(45100.00,3001.00,0.05)(45100.00,3501.00,0.04)(45100.00,4001.00,0.04)(45100.00,4501.00,0.03)
%
%(46100.00,1.00,2.61)(46100.00,501.00,0.27)(46100.00,1001.00,0.14)(46100.00,1501.00,0.10)(46100.00,2001.00,0.07)(46100.00,2501.00,0.06)(46100.00,3001.00,0.05)(46100.00,3501.00,0.04)(46100.00,4001.00,0.04)(46100.00,4501.00,0.03)
%
%(47100.00,1.00,2.58)(47100.00,501.00,0.26)(47100.00,1001.00,0.14)(47100.00,1501.00,0.10)(47100.00,2001.00,0.07)(47100.00,2501.00,0.06)(47100.00,3001.00,0.05)(47100.00,3501.00,0.04)(47100.00,4001.00,0.04)(47100.00,4501.00,0.03)
%
%(48100.00,1.00,2.55)(48100.00,501.00,0.26)(48100.00,1001.00,0.14)(48100.00,1501.00,0.09)(48100.00,2001.00,0.07)(48100.00,2501.00,0.06)(48100.00,3001.00,0.05)(48100.00,3501.00,0.04)(48100.00,4001.00,0.04)(48100.00,4501.00,0.03)
%
%(49100.00,1.00,2.52)(49100.00,501.00,0.25)(49100.00,1001.00,0.13)(49100.00,1501.00,0.09)(49100.00,2001.00,0.07)(49100.00,2501.00,0.06)(49100.00,3001.00,0.05)(49100.00,3501.00,0.04)(49100.00,4001.00,0.04)(49100.00,4501.00,0.03)


}; \end{axis} \end{tikzpicture}

     	\label{dlmproofc2}
        \end{subfigure}
%        \qquad %add desired spacing between images, e. g. ~, \quad, \qquad etc.
          %(or a blank line to force the subfigure onto a new line)
        ~
		\begin{subfigure}[]{0.5\textwidth}
           \caption{Doc. length and $\mu$ parameter}
           
\begin{tikzpicture}[thick,scale=0.7, every node/.style={transform shape}]\begin{axis}[
 %title={},
 %y dir=reverse, 
 %x dir=reverse, 
 ylabel={$\mu$},
 xlabel={docLength},
 zlabel={DirichletLM score},
 every axis/.append style={font=\large\bfseries},
 max space between ticks=25pt
% yticklabels={0k,100k}
 ] 

		\addplot3[surf] coordinates { 
%patch,patch type=biquadratic, shader=faceted,patch refines=3
(1.00,1.00,12.52)(2.00,1.00,11.93)(3.00,1.00,11.52)(4.00,1.00,11.20)(5.00,1.00,10.93)(6.00,1.00,10.71)(7.00,1.00,10.52)(8.00,1.00,10.35)(9.00,1.00,10.20)(10.00,1.00,10.06)(11.00,1.00,9.93)(12.00,1.00,9.82)(13.00,1.00,9.71)(14.00,1.00,9.61)(15.00,1.00,9.52)(16.00,1.00,9.43)(17.00,1.00,9.35)(18.00,1.00,9.27)(19.00,1.00,9.20)(20.00,1.00,9.12)(21.00,1.00,9.06)(22.00,1.00,8.99)(23.00,1.00,8.93)(24.00,1.00,8.87)(25.00,1.00,8.82)(26.00,1.00,8.76)(27.00,1.00,8.71)(28.00,1.00,8.66)(29.00,1.00,8.61)(30.00,1.00,8.56)

(1.00,51.00,7.82)(2.00,51.00,7.80)(3.00,51.00,7.77)(4.00,51.00,7.74)(5.00,51.00,7.72)(6.00,51.00,7.69)(7.00,51.00,7.67)(8.00,51.00,7.64)(9.00,51.00,7.62)(10.00,51.00,7.59)(11.00,51.00,7.57)(12.00,51.00,7.55)(13.00,51.00,7.52)(14.00,51.00,7.50)(15.00,51.00,7.48)(16.00,51.00,7.46)(17.00,51.00,7.44)(18.00,51.00,7.41)(19.00,51.00,7.39)(20.00,51.00,7.37)(21.00,51.00,7.35)(22.00,51.00,7.33)(23.00,51.00,7.31)(24.00,51.00,7.29)(25.00,51.00,7.28)(26.00,51.00,7.26)(27.00,51.00,7.24)(28.00,51.00,7.22)(29.00,51.00,7.20)(30.00,51.00,7.18)

(1.00,101.00,6.86)(2.00,101.00,6.84)(3.00,101.00,6.83)(4.00,101.00,6.82)(5.00,101.00,6.80)(6.00,101.00,6.79)(7.00,101.00,6.77)(8.00,101.00,6.76)(9.00,101.00,6.75)(10.00,101.00,6.73)(11.00,101.00,6.72)(12.00,101.00,6.71)(13.00,101.00,6.70)(14.00,101.00,6.68)(15.00,101.00,6.67)(16.00,101.00,6.66)(17.00,101.00,6.65)(18.00,101.00,6.63)(19.00,101.00,6.62)(20.00,101.00,6.61)(21.00,101.00,6.60)(22.00,101.00,6.59)(23.00,101.00,6.58)(24.00,101.00,6.56)(25.00,101.00,6.55)(26.00,101.00,6.54)(27.00,101.00,6.53)(28.00,101.00,6.52)(29.00,101.00,6.51)(30.00,101.00,6.50)

(1.00,151.00,6.29)(2.00,151.00,6.28)(3.00,151.00,6.27)(4.00,151.00,6.26)(5.00,151.00,6.25)(6.00,151.00,6.24)(7.00,151.00,6.23)(8.00,151.00,6.22)(9.00,151.00,6.21)(10.00,151.00,6.20)(11.00,151.00,6.20)(12.00,151.00,6.19)(13.00,151.00,6.18)(14.00,151.00,6.17)(15.00,151.00,6.16)(16.00,151.00,6.15)(17.00,151.00,6.14)(18.00,151.00,6.13)(19.00,151.00,6.13)(20.00,151.00,6.12)(21.00,151.00,6.11)(22.00,151.00,6.10)(23.00,151.00,6.09)(24.00,151.00,6.08)(25.00,151.00,6.08)(26.00,151.00,6.07)(27.00,151.00,6.06)(28.00,151.00,6.05)(29.00,151.00,6.04)(30.00,151.00,6.04)

(1.00,201.00,5.88)(2.00,201.00,5.88)(3.00,201.00,5.87)(4.00,201.00,5.86)(5.00,201.00,5.85)(6.00,201.00,5.85)(7.00,201.00,5.84)(8.00,201.00,5.83)(9.00,201.00,5.83)(10.00,201.00,5.82)(11.00,201.00,5.81)(12.00,201.00,5.81)(13.00,201.00,5.80)(14.00,201.00,5.79)(15.00,201.00,5.79)(16.00,201.00,5.78)(17.00,201.00,5.77)(18.00,201.00,5.77)(19.00,201.00,5.76)(20.00,201.00,5.75)(21.00,201.00,5.75)(22.00,201.00,5.74)(23.00,201.00,5.73)(24.00,201.00,5.73)(25.00,201.00,5.72)(26.00,201.00,5.71)(27.00,201.00,5.71)(28.00,201.00,5.70)(29.00,201.00,5.70)(30.00,201.00,5.69)

(1.00,251.00,5.57)(2.00,251.00,5.56)(3.00,251.00,5.56)(4.00,251.00,5.55)(5.00,251.00,5.55)(6.00,251.00,5.54)(7.00,251.00,5.54)(8.00,251.00,5.53)(9.00,251.00,5.53)(10.00,251.00,5.52)(11.00,251.00,5.51)(12.00,251.00,5.51)(13.00,251.00,5.50)(14.00,251.00,5.50)(15.00,251.00,5.49)(16.00,251.00,5.49)(17.00,251.00,5.48)(18.00,251.00,5.48)(19.00,251.00,5.47)(20.00,251.00,5.47)(21.00,251.00,5.46)(22.00,251.00,5.45)(23.00,251.00,5.45)(24.00,251.00,5.44)(25.00,251.00,5.44)(26.00,251.00,5.43)(27.00,251.00,5.43)(28.00,251.00,5.42)(29.00,251.00,5.42)(30.00,251.00,5.41)

(1.00,301.00,5.32)(2.00,301.00,5.31)(3.00,301.00,5.31)(4.00,301.00,5.30)(5.00,301.00,5.30)(6.00,301.00,5.29)(7.00,301.00,5.29)(8.00,301.00,5.28)(9.00,301.00,5.28)(10.00,301.00,5.27)(11.00,301.00,5.27)(12.00,301.00,5.26)(13.00,301.00,5.26)(14.00,301.00,5.25)(15.00,301.00,5.25)(16.00,301.00,5.25)(17.00,301.00,5.24)(18.00,301.00,5.24)(19.00,301.00,5.23)(20.00,301.00,5.23)(21.00,301.00,5.22)(22.00,301.00,5.22)(23.00,301.00,5.21)(24.00,301.00,5.21)(25.00,301.00,5.20)(26.00,301.00,5.20)(27.00,301.00,5.20)(28.00,301.00,5.19)(29.00,301.00,5.19)(30.00,301.00,5.18)

(1.00,351.00,5.10)(2.00,351.00,5.10)(3.00,351.00,5.09)(4.00,351.00,5.09)(5.00,351.00,5.08)(6.00,351.00,5.08)(7.00,351.00,5.08)(8.00,351.00,5.07)(9.00,351.00,5.07)(10.00,351.00,5.06)(11.00,351.00,5.06)(12.00,351.00,5.06)(13.00,351.00,5.05)(14.00,351.00,5.05)(15.00,351.00,5.04)(16.00,351.00,5.04)(17.00,351.00,5.04)(18.00,351.00,5.03)(19.00,351.00,5.03)(20.00,351.00,5.02)(21.00,351.00,5.02)(22.00,351.00,5.02)(23.00,351.00,5.01)(24.00,351.00,5.01)(25.00,351.00,5.00)(26.00,351.00,5.00)(27.00,351.00,5.00)(28.00,351.00,4.99)(29.00,351.00,4.99)(30.00,351.00,4.99)

(1.00,401.00,4.91)(2.00,401.00,4.91)(3.00,401.00,4.91)(4.00,401.00,4.90)(5.00,401.00,4.90)(6.00,401.00,4.90)(7.00,401.00,4.89)(8.00,401.00,4.89)(9.00,401.00,4.89)(10.00,401.00,4.88)(11.00,401.00,4.88)(12.00,401.00,4.88)(13.00,401.00,4.87)(14.00,401.00,4.87)(15.00,401.00,4.87)(16.00,401.00,4.86)(17.00,401.00,4.86)(18.00,401.00,4.85)(19.00,401.00,4.85)(20.00,401.00,4.85)(21.00,401.00,4.84)(22.00,401.00,4.84)(23.00,401.00,4.84)(24.00,401.00,4.83)(25.00,401.00,4.83)(26.00,401.00,4.83)(27.00,401.00,4.82)(28.00,401.00,4.82)(29.00,401.00,4.82)(30.00,401.00,4.81)

(1.00,451.00,4.75)(2.00,451.00,4.75)(3.00,451.00,4.74)(4.00,451.00,4.74)(5.00,451.00,4.74)(6.00,451.00,4.74)(7.00,451.00,4.73)(8.00,451.00,4.73)(9.00,451.00,4.73)(10.00,451.00,4.72)(11.00,451.00,4.72)(12.00,451.00,4.72)(13.00,451.00,4.71)(14.00,451.00,4.71)(15.00,451.00,4.71)(16.00,451.00,4.70)(17.00,451.00,4.70)(18.00,451.00,4.70)(19.00,451.00,4.69)(20.00,451.00,4.69)(21.00,451.00,4.69)(22.00,451.00,4.69)(23.00,451.00,4.68)(24.00,451.00,4.68)(25.00,451.00,4.68)(26.00,451.00,4.67)(27.00,451.00,4.67)(28.00,451.00,4.67)(29.00,451.00,4.66)(30.00,451.00,4.66)

(1.00,501.00,4.61)(2.00,501.00,4.60)(3.00,501.00,4.60)(4.00,501.00,4.60)(5.00,501.00,4.59)(6.00,501.00,4.59)(7.00,501.00,4.59)(8.00,501.00,4.59)(9.00,501.00,4.58)(10.00,501.00,4.58)(11.00,501.00,4.58)(12.00,501.00,4.57)(13.00,501.00,4.57)(14.00,501.00,4.57)(15.00,501.00,4.57)(16.00,501.00,4.56)(17.00,501.00,4.56)(18.00,501.00,4.56)(19.00,501.00,4.55)(20.00,501.00,4.55)(21.00,501.00,4.55)(22.00,501.00,4.55)(23.00,501.00,4.54)(24.00,501.00,4.54)(25.00,501.00,4.54)(26.00,501.00,4.54)(27.00,501.00,4.53)(28.00,501.00,4.53)(29.00,501.00,4.53)(30.00,501.00,4.52)

(1.00,551.00,4.47)(2.00,551.00,4.47)(3.00,551.00,4.47)(4.00,551.00,4.47)(5.00,551.00,4.46)(6.00,551.00,4.46)(7.00,551.00,4.46)(8.00,551.00,4.46)(9.00,551.00,4.45)(10.00,551.00,4.45)(11.00,551.00,4.45)(12.00,551.00,4.45)(13.00,551.00,4.44)(14.00,551.00,4.44)(15.00,551.00,4.44)(16.00,551.00,4.44)(17.00,551.00,4.43)(18.00,551.00,4.43)(19.00,551.00,4.43)(20.00,551.00,4.43)(21.00,551.00,4.42)(22.00,551.00,4.42)(23.00,551.00,4.42)(24.00,551.00,4.42)(25.00,551.00,4.41)(26.00,551.00,4.41)(27.00,551.00,4.41)(28.00,551.00,4.41)(29.00,551.00,4.40)(30.00,551.00,4.40)

(1.00,601.00,4.36)(2.00,601.00,4.35)(3.00,601.00,4.35)(4.00,601.00,4.35)(5.00,601.00,4.35)(6.00,601.00,4.34)(7.00,601.00,4.34)(8.00,601.00,4.34)(9.00,601.00,4.34)(10.00,601.00,4.33)(11.00,601.00,4.33)(12.00,601.00,4.33)(13.00,601.00,4.33)(14.00,601.00,4.32)(15.00,601.00,4.32)(16.00,601.00,4.32)(17.00,601.00,4.32)(18.00,601.00,4.32)(19.00,601.00,4.31)(20.00,601.00,4.31)(21.00,601.00,4.31)(22.00,601.00,4.31)(23.00,601.00,4.30)(24.00,601.00,4.30)(25.00,601.00,4.30)(26.00,601.00,4.30)(27.00,601.00,4.29)(28.00,601.00,4.29)(29.00,601.00,4.29)(30.00,601.00,4.29)

(1.00,651.00,4.25)(2.00,651.00,4.24)(3.00,651.00,4.24)(4.00,651.00,4.24)(5.00,651.00,4.24)(6.00,651.00,4.24)(7.00,651.00,4.23)(8.00,651.00,4.23)(9.00,651.00,4.23)(10.00,651.00,4.23)(11.00,651.00,4.22)(12.00,651.00,4.22)(13.00,651.00,4.22)(14.00,651.00,4.22)(15.00,651.00,4.22)(16.00,651.00,4.21)(17.00,651.00,4.21)(18.00,651.00,4.21)(19.00,651.00,4.21)(20.00,651.00,4.20)(21.00,651.00,4.20)(22.00,651.00,4.20)(23.00,651.00,4.20)(24.00,651.00,4.20)(25.00,651.00,4.19)(26.00,651.00,4.19)(27.00,651.00,4.19)(28.00,651.00,4.19)(29.00,651.00,4.19)(30.00,651.00,4.18)

(1.00,701.00,4.15)(2.00,701.00,4.14)(3.00,701.00,4.14)(4.00,701.00,4.14)(5.00,701.00,4.14)(6.00,701.00,4.14)(7.00,701.00,4.13)(8.00,701.00,4.13)(9.00,701.00,4.13)(10.00,701.00,4.13)(11.00,701.00,4.13)(12.00,701.00,4.12)(13.00,701.00,4.12)(14.00,701.00,4.12)(15.00,701.00,4.12)(16.00,701.00,4.11)(17.00,701.00,4.11)(18.00,701.00,4.11)(19.00,701.00,4.11)(20.00,701.00,4.11)(21.00,701.00,4.10)(22.00,701.00,4.10)(23.00,701.00,4.10)(24.00,701.00,4.10)(25.00,701.00,4.10)(26.00,701.00,4.09)(27.00,701.00,4.09)(28.00,701.00,4.09)(29.00,701.00,4.09)(30.00,701.00,4.09)

(1.00,751.00,4.05)(2.00,751.00,4.05)(3.00,751.00,4.05)(4.00,751.00,4.05)(5.00,751.00,4.04)(6.00,751.00,4.04)(7.00,751.00,4.04)(8.00,751.00,4.04)(9.00,751.00,4.04)(10.00,751.00,4.03)(11.00,751.00,4.03)(12.00,751.00,4.03)(13.00,751.00,4.03)(14.00,751.00,4.03)(15.00,751.00,4.03)(16.00,751.00,4.02)(17.00,751.00,4.02)(18.00,751.00,4.02)(19.00,751.00,4.02)(20.00,751.00,4.02)(21.00,751.00,4.01)(22.00,751.00,4.01)(23.00,751.00,4.01)(24.00,751.00,4.01)(25.00,751.00,4.01)(26.00,751.00,4.00)(27.00,751.00,4.00)(28.00,751.00,4.00)(29.00,751.00,4.00)(30.00,751.00,4.00)

(1.00,801.00,3.96)(2.00,801.00,3.96)(3.00,801.00,3.96)(4.00,801.00,3.96)(5.00,801.00,3.96)(6.00,801.00,3.96)(7.00,801.00,3.95)(8.00,801.00,3.95)(9.00,801.00,3.95)(10.00,801.00,3.95)(11.00,801.00,3.95)(12.00,801.00,3.95)(13.00,801.00,3.94)(14.00,801.00,3.94)(15.00,801.00,3.94)(16.00,801.00,3.94)(17.00,801.00,3.94)(18.00,801.00,3.93)(19.00,801.00,3.93)(20.00,801.00,3.93)(21.00,801.00,3.93)(22.00,801.00,3.93)(23.00,801.00,3.93)(24.00,801.00,3.92)(25.00,801.00,3.92)(26.00,801.00,3.92)(27.00,801.00,3.92)(28.00,801.00,3.92)(29.00,801.00,3.92)(30.00,801.00,3.91)

(1.00,851.00,3.88)(2.00,851.00,3.88)(3.00,851.00,3.88)(4.00,851.00,3.88)(5.00,851.00,3.88)(6.00,851.00,3.87)(7.00,851.00,3.87)(8.00,851.00,3.87)(9.00,851.00,3.87)(10.00,851.00,3.87)(11.00,851.00,3.87)(12.00,851.00,3.86)(13.00,851.00,3.86)(14.00,851.00,3.86)(15.00,851.00,3.86)(16.00,851.00,3.86)(17.00,851.00,3.86)(18.00,851.00,3.85)(19.00,851.00,3.85)(20.00,851.00,3.85)(21.00,851.00,3.85)(22.00,851.00,3.85)(23.00,851.00,3.85)(24.00,851.00,3.84)(25.00,851.00,3.84)(26.00,851.00,3.84)(27.00,851.00,3.84)(28.00,851.00,3.84)(29.00,851.00,3.84)(30.00,851.00,3.84)

(1.00,901.00,3.81)(2.00,901.00,3.81)(3.00,901.00,3.80)(4.00,901.00,3.80)(5.00,901.00,3.80)(6.00,901.00,3.80)(7.00,901.00,3.80)(8.00,901.00,3.80)(9.00,901.00,3.79)(10.00,901.00,3.79)(11.00,901.00,3.79)(12.00,901.00,3.79)(13.00,901.00,3.79)(14.00,901.00,3.79)(15.00,901.00,3.78)(16.00,901.00,3.78)(17.00,901.00,3.78)(18.00,901.00,3.78)(19.00,901.00,3.78)(20.00,901.00,3.78)(21.00,901.00,3.78)(22.00,901.00,3.77)(23.00,901.00,3.77)(24.00,901.00,3.77)(25.00,901.00,3.77)(26.00,901.00,3.77)(27.00,901.00,3.77)(28.00,901.00,3.76)(29.00,901.00,3.76)(30.00,901.00,3.76)

(1.00,951.00,3.73)(2.00,951.00,3.73)(3.00,951.00,3.73)(4.00,951.00,3.73)(5.00,951.00,3.73)(6.00,951.00,3.73)(7.00,951.00,3.73)(8.00,951.00,3.72)(9.00,951.00,3.72)(10.00,951.00,3.72)(11.00,951.00,3.72)(12.00,951.00,3.72)(13.00,951.00,3.72)(14.00,951.00,3.72)(15.00,951.00,3.71)(16.00,951.00,3.71)(17.00,951.00,3.71)(18.00,951.00,3.71)(19.00,951.00,3.71)(20.00,951.00,3.71)(21.00,951.00,3.70)(22.00,951.00,3.70)(23.00,951.00,3.70)(24.00,951.00,3.70)(25.00,951.00,3.70)(26.00,951.00,3.70)(27.00,951.00,3.70)(28.00,951.00,3.69)(29.00,951.00,3.69)(30.00,951.00,3.69)

(1.00,1001.00,3.67)(2.00,1001.00,3.67)(3.00,1001.00,3.66)(4.00,1001.00,3.66)(5.00,1001.00,3.66)(6.00,1001.00,3.66)(7.00,1001.00,3.66)(8.00,1001.00,3.66)(9.00,1001.00,3.66)(10.00,1001.00,3.65)(11.00,1001.00,3.65)(12.00,1001.00,3.65)(13.00,1001.00,3.65)(14.00,1001.00,3.65)(15.00,1001.00,3.65)(16.00,1001.00,3.65)(17.00,1001.00,3.64)(18.00,1001.00,3.64)(19.00,1001.00,3.64)(20.00,1001.00,3.64)(21.00,1001.00,3.64)(22.00,1001.00,3.64)(23.00,1001.00,3.64)(24.00,1001.00,3.63)(25.00,1001.00,3.63)(26.00,1001.00,3.63)(27.00,1001.00,3.63)(28.00,1001.00,3.63)(29.00,1001.00,3.63)(30.00,1001.00,3.63)

(1.00,1051.00,3.60)(2.00,1051.00,3.60)(3.00,1051.00,3.60)(4.00,1051.00,3.60)(5.00,1051.00,3.60)(6.00,1051.00,3.60)(7.00,1051.00,3.59)(8.00,1051.00,3.59)(9.00,1051.00,3.59)(10.00,1051.00,3.59)(11.00,1051.00,3.59)(12.00,1051.00,3.59)(13.00,1051.00,3.59)(14.00,1051.00,3.58)(15.00,1051.00,3.58)(16.00,1051.00,3.58)(17.00,1051.00,3.58)(18.00,1051.00,3.58)(19.00,1051.00,3.58)(20.00,1051.00,3.58)(21.00,1051.00,3.57)(22.00,1051.00,3.57)(23.00,1051.00,3.57)(24.00,1051.00,3.57)(25.00,1051.00,3.57)(26.00,1051.00,3.57)(27.00,1051.00,3.57)(28.00,1051.00,3.57)(29.00,1051.00,3.56)(30.00,1051.00,3.56)

(1.00,1101.00,3.54)(2.00,1101.00,3.54)(3.00,1101.00,3.54)(4.00,1101.00,3.54)(5.00,1101.00,3.54)(6.00,1101.00,3.53)(7.00,1101.00,3.53)(8.00,1101.00,3.53)(9.00,1101.00,3.53)(10.00,1101.00,3.53)(11.00,1101.00,3.53)(12.00,1101.00,3.53)(13.00,1101.00,3.52)(14.00,1101.00,3.52)(15.00,1101.00,3.52)(16.00,1101.00,3.52)(17.00,1101.00,3.52)(18.00,1101.00,3.52)(19.00,1101.00,3.52)(20.00,1101.00,3.52)(21.00,1101.00,3.51)(22.00,1101.00,3.51)(23.00,1101.00,3.51)(24.00,1101.00,3.51)(25.00,1101.00,3.51)(26.00,1101.00,3.51)(27.00,1101.00,3.51)(28.00,1101.00,3.51)(29.00,1101.00,3.50)(30.00,1101.00,3.50)

(1.00,1151.00,3.48)(2.00,1151.00,3.48)(3.00,1151.00,3.48)(4.00,1151.00,3.48)(5.00,1151.00,3.48)(6.00,1151.00,3.48)(7.00,1151.00,3.47)(8.00,1151.00,3.47)(9.00,1151.00,3.47)(10.00,1151.00,3.47)(11.00,1151.00,3.47)(12.00,1151.00,3.47)(13.00,1151.00,3.47)(14.00,1151.00,3.47)(15.00,1151.00,3.46)(16.00,1151.00,3.46)(17.00,1151.00,3.46)(18.00,1151.00,3.46)(19.00,1151.00,3.46)(20.00,1151.00,3.46)(21.00,1151.00,3.46)(22.00,1151.00,3.46)(23.00,1151.00,3.45)(24.00,1151.00,3.45)(25.00,1151.00,3.45)(26.00,1151.00,3.45)(27.00,1151.00,3.45)(28.00,1151.00,3.45)(29.00,1151.00,3.45)(30.00,1151.00,3.45)

(1.00,1201.00,3.43)(2.00,1201.00,3.43)(3.00,1201.00,3.42)(4.00,1201.00,3.42)(5.00,1201.00,3.42)(6.00,1201.00,3.42)(7.00,1201.00,3.42)(8.00,1201.00,3.42)(9.00,1201.00,3.42)(10.00,1201.00,3.42)(11.00,1201.00,3.41)(12.00,1201.00,3.41)(13.00,1201.00,3.41)(14.00,1201.00,3.41)(15.00,1201.00,3.41)(16.00,1201.00,3.41)(17.00,1201.00,3.41)(18.00,1201.00,3.41)(19.00,1201.00,3.40)(20.00,1201.00,3.40)(21.00,1201.00,3.40)(22.00,1201.00,3.40)(23.00,1201.00,3.40)(24.00,1201.00,3.40)(25.00,1201.00,3.40)(26.00,1201.00,3.40)(27.00,1201.00,3.40)(28.00,1201.00,3.39)(29.00,1201.00,3.39)(30.00,1201.00,3.39)

(1.00,1251.00,3.37)(2.00,1251.00,3.37)(3.00,1251.00,3.37)(4.00,1251.00,3.37)(5.00,1251.00,3.37)(6.00,1251.00,3.37)(7.00,1251.00,3.37)(8.00,1251.00,3.37)(9.00,1251.00,3.36)(10.00,1251.00,3.36)(11.00,1251.00,3.36)(12.00,1251.00,3.36)(13.00,1251.00,3.36)(14.00,1251.00,3.36)(15.00,1251.00,3.36)(16.00,1251.00,3.36)(17.00,1251.00,3.35)(18.00,1251.00,3.35)(19.00,1251.00,3.35)(20.00,1251.00,3.35)(21.00,1251.00,3.35)(22.00,1251.00,3.35)(23.00,1251.00,3.35)(24.00,1251.00,3.35)(25.00,1251.00,3.35)(26.00,1251.00,3.34)(27.00,1251.00,3.34)(28.00,1251.00,3.34)(29.00,1251.00,3.34)(30.00,1251.00,3.34)

(1.00,1301.00,3.32)(2.00,1301.00,3.32)(3.00,1301.00,3.32)(4.00,1301.00,3.32)(5.00,1301.00,3.32)(6.00,1301.00,3.32)(7.00,1301.00,3.32)(8.00,1301.00,3.31)(9.00,1301.00,3.31)(10.00,1301.00,3.31)(11.00,1301.00,3.31)(12.00,1301.00,3.31)(13.00,1301.00,3.31)(14.00,1301.00,3.31)(15.00,1301.00,3.31)(16.00,1301.00,3.31)(17.00,1301.00,3.30)(18.00,1301.00,3.30)(19.00,1301.00,3.30)(20.00,1301.00,3.30)(21.00,1301.00,3.30)(22.00,1301.00,3.30)(23.00,1301.00,3.30)(24.00,1301.00,3.30)(25.00,1301.00,3.30)(26.00,1301.00,3.29)(27.00,1301.00,3.29)(28.00,1301.00,3.29)(29.00,1301.00,3.29)(30.00,1301.00,3.29)

(1.00,1351.00,3.27)(2.00,1351.00,3.27)(3.00,1351.00,3.27)(4.00,1351.00,3.27)(5.00,1351.00,3.27)(6.00,1351.00,3.27)(7.00,1351.00,3.27)(8.00,1351.00,3.27)(9.00,1351.00,3.26)(10.00,1351.00,3.26)(11.00,1351.00,3.26)(12.00,1351.00,3.26)(13.00,1351.00,3.26)(14.00,1351.00,3.26)(15.00,1351.00,3.26)(16.00,1351.00,3.26)(17.00,1351.00,3.26)(18.00,1351.00,3.26)(19.00,1351.00,3.25)(20.00,1351.00,3.25)(21.00,1351.00,3.25)(22.00,1351.00,3.25)(23.00,1351.00,3.25)(24.00,1351.00,3.25)(25.00,1351.00,3.25)(26.00,1351.00,3.25)(27.00,1351.00,3.25)(28.00,1351.00,3.24)(29.00,1351.00,3.24)(30.00,1351.00,3.24)

(1.00,1401.00,3.23)(2.00,1401.00,3.23)(3.00,1401.00,3.22)(4.00,1401.00,3.22)(5.00,1401.00,3.22)(6.00,1401.00,3.22)(7.00,1401.00,3.22)(8.00,1401.00,3.22)(9.00,1401.00,3.22)(10.00,1401.00,3.22)(11.00,1401.00,3.22)(12.00,1401.00,3.22)(13.00,1401.00,3.21)(14.00,1401.00,3.21)(15.00,1401.00,3.21)(16.00,1401.00,3.21)(17.00,1401.00,3.21)(18.00,1401.00,3.21)(19.00,1401.00,3.21)(20.00,1401.00,3.21)(21.00,1401.00,3.21)(22.00,1401.00,3.21)(23.00,1401.00,3.20)(24.00,1401.00,3.20)(25.00,1401.00,3.20)(26.00,1401.00,3.20)(27.00,1401.00,3.20)(28.00,1401.00,3.20)(29.00,1401.00,3.20)(30.00,1401.00,3.20)

(1.00,1451.00,3.18)(2.00,1451.00,3.18)(3.00,1451.00,3.18)(4.00,1451.00,3.18)(5.00,1451.00,3.18)(6.00,1451.00,3.18)(7.00,1451.00,3.18)(8.00,1451.00,3.17)(9.00,1451.00,3.17)(10.00,1451.00,3.17)(11.00,1451.00,3.17)(12.00,1451.00,3.17)(13.00,1451.00,3.17)(14.00,1451.00,3.17)(15.00,1451.00,3.17)(16.00,1451.00,3.17)(17.00,1451.00,3.17)(18.00,1451.00,3.16)(19.00,1451.00,3.16)(20.00,1451.00,3.16)(21.00,1451.00,3.16)(22.00,1451.00,3.16)(23.00,1451.00,3.16)(24.00,1451.00,3.16)(25.00,1451.00,3.16)(26.00,1451.00,3.16)(27.00,1451.00,3.16)(28.00,1451.00,3.15)(29.00,1451.00,3.15)(30.00,1451.00,3.15)

(1.00,1501.00,3.14)(2.00,1501.00,3.14)(3.00,1501.00,3.14)(4.00,1501.00,3.14)(5.00,1501.00,3.13)(6.00,1501.00,3.13)(7.00,1501.00,3.13)(8.00,1501.00,3.13)(9.00,1501.00,3.13)(10.00,1501.00,3.13)(11.00,1501.00,3.13)(12.00,1501.00,3.13)(13.00,1501.00,3.13)(14.00,1501.00,3.13)(15.00,1501.00,3.12)(16.00,1501.00,3.12)(17.00,1501.00,3.12)(18.00,1501.00,3.12)(19.00,1501.00,3.12)(20.00,1501.00,3.12)(21.00,1501.00,3.12)(22.00,1501.00,3.12)(23.00,1501.00,3.12)(24.00,1501.00,3.12)(25.00,1501.00,3.12)(26.00,1501.00,3.11)(27.00,1501.00,3.11)(28.00,1501.00,3.11)(29.00,1501.00,3.11)(30.00,1501.00,3.11)

(1.00,1551.00,3.10)(2.00,1551.00,3.10)(3.00,1551.00,3.09)(4.00,1551.00,3.09)(5.00,1551.00,3.09)(6.00,1551.00,3.09)(7.00,1551.00,3.09)(8.00,1551.00,3.09)(9.00,1551.00,3.09)(10.00,1551.00,3.09)(11.00,1551.00,3.09)(12.00,1551.00,3.09)(13.00,1551.00,3.09)(14.00,1551.00,3.08)(15.00,1551.00,3.08)(16.00,1551.00,3.08)(17.00,1551.00,3.08)(18.00,1551.00,3.08)(19.00,1551.00,3.08)(20.00,1551.00,3.08)(21.00,1551.00,3.08)(22.00,1551.00,3.08)(23.00,1551.00,3.08)(24.00,1551.00,3.08)(25.00,1551.00,3.07)(26.00,1551.00,3.07)(27.00,1551.00,3.07)(28.00,1551.00,3.07)(29.00,1551.00,3.07)(30.00,1551.00,3.07)

(1.00,1601.00,3.06)(2.00,1601.00,3.06)(3.00,1601.00,3.05)(4.00,1601.00,3.05)(5.00,1601.00,3.05)(6.00,1601.00,3.05)(7.00,1601.00,3.05)(8.00,1601.00,3.05)(9.00,1601.00,3.05)(10.00,1601.00,3.05)(11.00,1601.00,3.05)(12.00,1601.00,3.05)(13.00,1601.00,3.05)(14.00,1601.00,3.04)(15.00,1601.00,3.04)(16.00,1601.00,3.04)(17.00,1601.00,3.04)(18.00,1601.00,3.04)(19.00,1601.00,3.04)(20.00,1601.00,3.04)(21.00,1601.00,3.04)(22.00,1601.00,3.04)(23.00,1601.00,3.04)(24.00,1601.00,3.04)(25.00,1601.00,3.03)(26.00,1601.00,3.03)(27.00,1601.00,3.03)(28.00,1601.00,3.03)(29.00,1601.00,3.03)(30.00,1601.00,3.03)

(1.00,1651.00,3.02)(2.00,1651.00,3.02)(3.00,1651.00,3.02)(4.00,1651.00,3.01)(5.00,1651.00,3.01)(6.00,1651.00,3.01)(7.00,1651.00,3.01)(8.00,1651.00,3.01)(9.00,1651.00,3.01)(10.00,1651.00,3.01)(11.00,1651.00,3.01)(12.00,1651.00,3.01)(13.00,1651.00,3.01)(14.00,1651.00,3.01)(15.00,1651.00,3.00)(16.00,1651.00,3.00)(17.00,1651.00,3.00)(18.00,1651.00,3.00)(19.00,1651.00,3.00)(20.00,1651.00,3.00)(21.00,1651.00,3.00)(22.00,1651.00,3.00)(23.00,1651.00,3.00)(24.00,1651.00,3.00)(25.00,1651.00,3.00)(26.00,1651.00,3.00)(27.00,1651.00,2.99)(28.00,1651.00,2.99)(29.00,1651.00,2.99)(30.00,1651.00,2.99)

(1.00,1701.00,2.98)(2.00,1701.00,2.98)(3.00,1701.00,2.98)(4.00,1701.00,2.98)(5.00,1701.00,2.98)(6.00,1701.00,2.98)(7.00,1701.00,2.97)(8.00,1701.00,2.97)(9.00,1701.00,2.97)(10.00,1701.00,2.97)(11.00,1701.00,2.97)(12.00,1701.00,2.97)(13.00,1701.00,2.97)(14.00,1701.00,2.97)(15.00,1701.00,2.97)(16.00,1701.00,2.97)(17.00,1701.00,2.97)(18.00,1701.00,2.97)(19.00,1701.00,2.96)(20.00,1701.00,2.96)(21.00,1701.00,2.96)(22.00,1701.00,2.96)(23.00,1701.00,2.96)(24.00,1701.00,2.96)(25.00,1701.00,2.96)(26.00,1701.00,2.96)(27.00,1701.00,2.96)(28.00,1701.00,2.96)(29.00,1701.00,2.96)(30.00,1701.00,2.96)

(1.00,1751.00,2.94)(2.00,1751.00,2.94)(3.00,1751.00,2.94)(4.00,1751.00,2.94)(5.00,1751.00,2.94)(6.00,1751.00,2.94)(7.00,1751.00,2.94)(8.00,1751.00,2.94)(9.00,1751.00,2.94)(10.00,1751.00,2.94)(11.00,1751.00,2.93)(12.00,1751.00,2.93)(13.00,1751.00,2.93)(14.00,1751.00,2.93)(15.00,1751.00,2.93)(16.00,1751.00,2.93)(17.00,1751.00,2.93)(18.00,1751.00,2.93)(19.00,1751.00,2.93)(20.00,1751.00,2.93)(21.00,1751.00,2.93)(22.00,1751.00,2.93)(23.00,1751.00,2.93)(24.00,1751.00,2.92)(25.00,1751.00,2.92)(26.00,1751.00,2.92)(27.00,1751.00,2.92)(28.00,1751.00,2.92)(29.00,1751.00,2.92)(30.00,1751.00,2.92)

(1.00,1801.00,2.91)(2.00,1801.00,2.91)(3.00,1801.00,2.91)(4.00,1801.00,2.91)(5.00,1801.00,2.90)(6.00,1801.00,2.90)(7.00,1801.00,2.90)(8.00,1801.00,2.90)(9.00,1801.00,2.90)(10.00,1801.00,2.90)(11.00,1801.00,2.90)(12.00,1801.00,2.90)(13.00,1801.00,2.90)(14.00,1801.00,2.90)(15.00,1801.00,2.90)(16.00,1801.00,2.90)(17.00,1801.00,2.90)(18.00,1801.00,2.89)(19.00,1801.00,2.89)(20.00,1801.00,2.89)(21.00,1801.00,2.89)(22.00,1801.00,2.89)(23.00,1801.00,2.89)(24.00,1801.00,2.89)(25.00,1801.00,2.89)(26.00,1801.00,2.89)(27.00,1801.00,2.89)(28.00,1801.00,2.89)(29.00,1801.00,2.89)(30.00,1801.00,2.88)

(1.00,1851.00,2.87)(2.00,1851.00,2.87)(3.00,1851.00,2.87)(4.00,1851.00,2.87)(5.00,1851.00,2.87)(6.00,1851.00,2.87)(7.00,1851.00,2.87)(8.00,1851.00,2.87)(9.00,1851.00,2.87)(10.00,1851.00,2.87)(11.00,1851.00,2.87)(12.00,1851.00,2.87)(13.00,1851.00,2.86)(14.00,1851.00,2.86)(15.00,1851.00,2.86)(16.00,1851.00,2.86)(17.00,1851.00,2.86)(18.00,1851.00,2.86)(19.00,1851.00,2.86)(20.00,1851.00,2.86)(21.00,1851.00,2.86)(22.00,1851.00,2.86)(23.00,1851.00,2.86)(24.00,1851.00,2.86)(25.00,1851.00,2.86)(26.00,1851.00,2.85)(27.00,1851.00,2.85)(28.00,1851.00,2.85)(29.00,1851.00,2.85)(30.00,1851.00,2.85)

(1.00,1901.00,2.84)(2.00,1901.00,2.84)(3.00,1901.00,2.84)(4.00,1901.00,2.84)(5.00,1901.00,2.84)(6.00,1901.00,2.84)(7.00,1901.00,2.84)(8.00,1901.00,2.84)(9.00,1901.00,2.83)(10.00,1901.00,2.83)(11.00,1901.00,2.83)(12.00,1901.00,2.83)(13.00,1901.00,2.83)(14.00,1901.00,2.83)(15.00,1901.00,2.83)(16.00,1901.00,2.83)(17.00,1901.00,2.83)(18.00,1901.00,2.83)(19.00,1901.00,2.83)(20.00,1901.00,2.83)(21.00,1901.00,2.83)(22.00,1901.00,2.82)(23.00,1901.00,2.82)(24.00,1901.00,2.82)(25.00,1901.00,2.82)(26.00,1901.00,2.82)(27.00,1901.00,2.82)(28.00,1901.00,2.82)(29.00,1901.00,2.82)(30.00,1901.00,2.82)

(1.00,1951.00,2.81)(2.00,1951.00,2.81)(3.00,1951.00,2.81)(4.00,1951.00,2.81)(5.00,1951.00,2.81)(6.00,1951.00,2.80)(7.00,1951.00,2.80)(8.00,1951.00,2.80)(9.00,1951.00,2.80)(10.00,1951.00,2.80)(11.00,1951.00,2.80)(12.00,1951.00,2.80)(13.00,1951.00,2.80)(14.00,1951.00,2.80)(15.00,1951.00,2.80)(16.00,1951.00,2.80)(17.00,1951.00,2.80)(18.00,1951.00,2.80)(19.00,1951.00,2.80)(20.00,1951.00,2.79)(21.00,1951.00,2.79)(22.00,1951.00,2.79)(23.00,1951.00,2.79)(24.00,1951.00,2.79)(25.00,1951.00,2.79)(26.00,1951.00,2.79)(27.00,1951.00,2.79)(28.00,1951.00,2.79)(29.00,1951.00,2.79)(30.00,1951.00,2.79)

(1.00,2001.00,2.78)(2.00,2001.00,2.78)(3.00,2001.00,2.78)(4.00,2001.00,2.77)(5.00,2001.00,2.77)(6.00,2001.00,2.77)(7.00,2001.00,2.77)(8.00,2001.00,2.77)(9.00,2001.00,2.77)(10.00,2001.00,2.77)(11.00,2001.00,2.77)(12.00,2001.00,2.77)(13.00,2001.00,2.77)(14.00,2001.00,2.77)(15.00,2001.00,2.77)(16.00,2001.00,2.77)(17.00,2001.00,2.77)(18.00,2001.00,2.76)(19.00,2001.00,2.76)(20.00,2001.00,2.76)(21.00,2001.00,2.76)(22.00,2001.00,2.76)(23.00,2001.00,2.76)(24.00,2001.00,2.76)(25.00,2001.00,2.76)(26.00,2001.00,2.76)(27.00,2001.00,2.76)(28.00,2001.00,2.76)(29.00,2001.00,2.76)(30.00,2001.00,2.76)

(1.00,2051.00,2.75)(2.00,2051.00,2.75)(3.00,2051.00,2.75)(4.00,2051.00,2.74)(5.00,2051.00,2.74)(6.00,2051.00,2.74)(7.00,2051.00,2.74)(8.00,2051.00,2.74)(9.00,2051.00,2.74)(10.00,2051.00,2.74)(11.00,2051.00,2.74)(12.00,2051.00,2.74)(13.00,2051.00,2.74)(14.00,2051.00,2.74)(15.00,2051.00,2.74)(16.00,2051.00,2.74)(17.00,2051.00,2.74)(18.00,2051.00,2.73)(19.00,2051.00,2.73)(20.00,2051.00,2.73)(21.00,2051.00,2.73)(22.00,2051.00,2.73)(23.00,2051.00,2.73)(24.00,2051.00,2.73)(25.00,2051.00,2.73)(26.00,2051.00,2.73)(27.00,2051.00,2.73)(28.00,2051.00,2.73)(29.00,2051.00,2.73)(30.00,2051.00,2.73)

(1.00,2101.00,2.72)(2.00,2101.00,2.72)(3.00,2101.00,2.72)(4.00,2101.00,2.72)(5.00,2101.00,2.71)(6.00,2101.00,2.71)(7.00,2101.00,2.71)(8.00,2101.00,2.71)(9.00,2101.00,2.71)(10.00,2101.00,2.71)(11.00,2101.00,2.71)(12.00,2101.00,2.71)(13.00,2101.00,2.71)(14.00,2101.00,2.71)(15.00,2101.00,2.71)(16.00,2101.00,2.71)(17.00,2101.00,2.71)(18.00,2101.00,2.71)(19.00,2101.00,2.70)(20.00,2101.00,2.70)(21.00,2101.00,2.70)(22.00,2101.00,2.70)(23.00,2101.00,2.70)(24.00,2101.00,2.70)(25.00,2101.00,2.70)(26.00,2101.00,2.70)(27.00,2101.00,2.70)(28.00,2101.00,2.70)(29.00,2101.00,2.70)(30.00,2101.00,2.70)

(1.00,2151.00,2.69)(2.00,2151.00,2.69)(3.00,2151.00,2.69)(4.00,2151.00,2.69)(5.00,2151.00,2.69)(6.00,2151.00,2.69)(7.00,2151.00,2.68)(8.00,2151.00,2.68)(9.00,2151.00,2.68)(10.00,2151.00,2.68)(11.00,2151.00,2.68)(12.00,2151.00,2.68)(13.00,2151.00,2.68)(14.00,2151.00,2.68)(15.00,2151.00,2.68)(16.00,2151.00,2.68)(17.00,2151.00,2.68)(18.00,2151.00,2.68)(19.00,2151.00,2.68)(20.00,2151.00,2.68)(21.00,2151.00,2.68)(22.00,2151.00,2.67)(23.00,2151.00,2.67)(24.00,2151.00,2.67)(25.00,2151.00,2.67)(26.00,2151.00,2.67)(27.00,2151.00,2.67)(28.00,2151.00,2.67)(29.00,2151.00,2.67)(30.00,2151.00,2.67)

(1.00,2201.00,2.66)(2.00,2201.00,2.66)(3.00,2201.00,2.66)(4.00,2201.00,2.66)(5.00,2201.00,2.66)(6.00,2201.00,2.66)(7.00,2201.00,2.66)(8.00,2201.00,2.66)(9.00,2201.00,2.66)(10.00,2201.00,2.65)(11.00,2201.00,2.65)(12.00,2201.00,2.65)(13.00,2201.00,2.65)(14.00,2201.00,2.65)(15.00,2201.00,2.65)(16.00,2201.00,2.65)(17.00,2201.00,2.65)(18.00,2201.00,2.65)(19.00,2201.00,2.65)(20.00,2201.00,2.65)(21.00,2201.00,2.65)(22.00,2201.00,2.65)(23.00,2201.00,2.65)(24.00,2201.00,2.65)(25.00,2201.00,2.64)(26.00,2201.00,2.64)(27.00,2201.00,2.64)(28.00,2201.00,2.64)(29.00,2201.00,2.64)(30.00,2201.00,2.64)

(1.00,2251.00,2.63)(2.00,2251.00,2.63)(3.00,2251.00,2.63)(4.00,2251.00,2.63)(5.00,2251.00,2.63)(6.00,2251.00,2.63)(7.00,2251.00,2.63)(8.00,2251.00,2.63)(9.00,2251.00,2.63)(10.00,2251.00,2.63)(11.00,2251.00,2.63)(12.00,2251.00,2.63)(13.00,2251.00,2.63)(14.00,2251.00,2.63)(15.00,2251.00,2.62)(16.00,2251.00,2.62)(17.00,2251.00,2.62)(18.00,2251.00,2.62)(19.00,2251.00,2.62)(20.00,2251.00,2.62)(21.00,2251.00,2.62)(22.00,2251.00,2.62)(23.00,2251.00,2.62)(24.00,2251.00,2.62)(25.00,2251.00,2.62)(26.00,2251.00,2.62)(27.00,2251.00,2.62)(28.00,2251.00,2.62)(29.00,2251.00,2.62)(30.00,2251.00,2.61)

(1.00,2301.00,2.61)(2.00,2301.00,2.61)(3.00,2301.00,2.61)(4.00,2301.00,2.60)(5.00,2301.00,2.60)(6.00,2301.00,2.60)(7.00,2301.00,2.60)(8.00,2301.00,2.60)(9.00,2301.00,2.60)(10.00,2301.00,2.60)(11.00,2301.00,2.60)(12.00,2301.00,2.60)(13.00,2301.00,2.60)(14.00,2301.00,2.60)(15.00,2301.00,2.60)(16.00,2301.00,2.60)(17.00,2301.00,2.60)(18.00,2301.00,2.60)(19.00,2301.00,2.60)(20.00,2301.00,2.59)(21.00,2301.00,2.59)(22.00,2301.00,2.59)(23.00,2301.00,2.59)(24.00,2301.00,2.59)(25.00,2301.00,2.59)(26.00,2301.00,2.59)(27.00,2301.00,2.59)(28.00,2301.00,2.59)(29.00,2301.00,2.59)(30.00,2301.00,2.59)

(1.00,2351.00,2.58)(2.00,2351.00,2.58)(3.00,2351.00,2.58)(4.00,2351.00,2.58)(5.00,2351.00,2.58)(6.00,2351.00,2.58)(7.00,2351.00,2.58)(8.00,2351.00,2.58)(9.00,2351.00,2.58)(10.00,2351.00,2.58)(11.00,2351.00,2.57)(12.00,2351.00,2.57)(13.00,2351.00,2.57)(14.00,2351.00,2.57)(15.00,2351.00,2.57)(16.00,2351.00,2.57)(17.00,2351.00,2.57)(18.00,2351.00,2.57)(19.00,2351.00,2.57)(20.00,2351.00,2.57)(21.00,2351.00,2.57)(22.00,2351.00,2.57)(23.00,2351.00,2.57)(24.00,2351.00,2.57)(25.00,2351.00,2.57)(26.00,2351.00,2.57)(27.00,2351.00,2.57)(28.00,2351.00,2.56)(29.00,2351.00,2.56)(30.00,2351.00,2.56)

(1.00,2401.00,2.56)(2.00,2401.00,2.56)(3.00,2401.00,2.55)(4.00,2401.00,2.55)(5.00,2401.00,2.55)(6.00,2401.00,2.55)(7.00,2401.00,2.55)(8.00,2401.00,2.55)(9.00,2401.00,2.55)(10.00,2401.00,2.55)(11.00,2401.00,2.55)(12.00,2401.00,2.55)(13.00,2401.00,2.55)(14.00,2401.00,2.55)(15.00,2401.00,2.55)(16.00,2401.00,2.55)(17.00,2401.00,2.55)(18.00,2401.00,2.55)(19.00,2401.00,2.54)(20.00,2401.00,2.54)(21.00,2401.00,2.54)(22.00,2401.00,2.54)(23.00,2401.00,2.54)(24.00,2401.00,2.54)(25.00,2401.00,2.54)(26.00,2401.00,2.54)(27.00,2401.00,2.54)(28.00,2401.00,2.54)(29.00,2401.00,2.54)(30.00,2401.00,2.54)

(1.00,2451.00,2.53)(2.00,2451.00,2.53)(3.00,2451.00,2.53)(4.00,2451.00,2.53)(5.00,2451.00,2.53)(6.00,2451.00,2.53)(7.00,2451.00,2.53)(8.00,2451.00,2.53)(9.00,2451.00,2.53)(10.00,2451.00,2.53)(11.00,2451.00,2.53)(12.00,2451.00,2.52)(13.00,2451.00,2.52)(14.00,2451.00,2.52)(15.00,2451.00,2.52)(16.00,2451.00,2.52)(17.00,2451.00,2.52)(18.00,2451.00,2.52)(19.00,2451.00,2.52)(20.00,2451.00,2.52)(21.00,2451.00,2.52)(22.00,2451.00,2.52)(23.00,2451.00,2.52)(24.00,2451.00,2.52)(25.00,2451.00,2.52)(26.00,2451.00,2.52)(27.00,2451.00,2.52)(28.00,2451.00,2.52)(29.00,2451.00,2.51)(30.00,2451.00,2.51)

(1.00,2501.00,2.51)(2.00,2501.00,2.51)(3.00,2501.00,2.51)(4.00,2501.00,2.51)(5.00,2501.00,2.50)(6.00,2501.00,2.50)(7.00,2501.00,2.50)(8.00,2501.00,2.50)(9.00,2501.00,2.50)(10.00,2501.00,2.50)(11.00,2501.00,2.50)(12.00,2501.00,2.50)(13.00,2501.00,2.50)(14.00,2501.00,2.50)(15.00,2501.00,2.50)(16.00,2501.00,2.50)(17.00,2501.00,2.50)(18.00,2501.00,2.50)(19.00,2501.00,2.50)(20.00,2501.00,2.50)(21.00,2501.00,2.50)(22.00,2501.00,2.50)(23.00,2501.00,2.49)(24.00,2501.00,2.49)(25.00,2501.00,2.49)(26.00,2501.00,2.49)(27.00,2501.00,2.49)(28.00,2501.00,2.49)(29.00,2501.00,2.49)(30.00,2501.00,2.49)

(1.00,2551.00,2.48)(2.00,2551.00,2.48)(3.00,2551.00,2.48)(4.00,2551.00,2.48)(5.00,2551.00,2.48)(6.00,2551.00,2.48)(7.00,2551.00,2.48)(8.00,2551.00,2.48)(9.00,2551.00,2.48)(10.00,2551.00,2.48)(11.00,2551.00,2.48)(12.00,2551.00,2.48)(13.00,2551.00,2.48)(14.00,2551.00,2.48)(15.00,2551.00,2.48)(16.00,2551.00,2.48)(17.00,2551.00,2.47)(18.00,2551.00,2.47)(19.00,2551.00,2.47)(20.00,2551.00,2.47)(21.00,2551.00,2.47)(22.00,2551.00,2.47)(23.00,2551.00,2.47)(24.00,2551.00,2.47)(25.00,2551.00,2.47)(26.00,2551.00,2.47)(27.00,2551.00,2.47)(28.00,2551.00,2.47)(29.00,2551.00,2.47)(30.00,2551.00,2.47)

(1.00,2601.00,2.46)(2.00,2601.00,2.46)(3.00,2601.00,2.46)(4.00,2601.00,2.46)(5.00,2601.00,2.46)(6.00,2601.00,2.46)(7.00,2601.00,2.46)(8.00,2601.00,2.46)(9.00,2601.00,2.46)(10.00,2601.00,2.46)(11.00,2601.00,2.46)(12.00,2601.00,2.45)(13.00,2601.00,2.45)(14.00,2601.00,2.45)(15.00,2601.00,2.45)(16.00,2601.00,2.45)(17.00,2601.00,2.45)(18.00,2601.00,2.45)(19.00,2601.00,2.45)(20.00,2601.00,2.45)(21.00,2601.00,2.45)(22.00,2601.00,2.45)(23.00,2601.00,2.45)(24.00,2601.00,2.45)(25.00,2601.00,2.45)(26.00,2601.00,2.45)(27.00,2601.00,2.45)(28.00,2601.00,2.45)(29.00,2601.00,2.45)(30.00,2601.00,2.44)

(1.00,2651.00,2.44)(2.00,2651.00,2.44)(3.00,2651.00,2.44)(4.00,2651.00,2.44)(5.00,2651.00,2.44)(6.00,2651.00,2.44)(7.00,2651.00,2.43)(8.00,2651.00,2.43)(9.00,2651.00,2.43)(10.00,2651.00,2.43)(11.00,2651.00,2.43)(12.00,2651.00,2.43)(13.00,2651.00,2.43)(14.00,2651.00,2.43)(15.00,2651.00,2.43)(16.00,2651.00,2.43)(17.00,2651.00,2.43)(18.00,2651.00,2.43)(19.00,2651.00,2.43)(20.00,2651.00,2.43)(21.00,2651.00,2.43)(22.00,2651.00,2.43)(23.00,2651.00,2.43)(24.00,2651.00,2.43)(25.00,2651.00,2.43)(26.00,2651.00,2.42)(27.00,2651.00,2.42)(28.00,2651.00,2.42)(29.00,2651.00,2.42)(30.00,2651.00,2.42)

(1.00,2701.00,2.42)(2.00,2701.00,2.42)(3.00,2701.00,2.42)(4.00,2701.00,2.41)(5.00,2701.00,2.41)(6.00,2701.00,2.41)(7.00,2701.00,2.41)(8.00,2701.00,2.41)(9.00,2701.00,2.41)(10.00,2701.00,2.41)(11.00,2701.00,2.41)(12.00,2701.00,2.41)(13.00,2701.00,2.41)(14.00,2701.00,2.41)(15.00,2701.00,2.41)(16.00,2701.00,2.41)(17.00,2701.00,2.41)(18.00,2701.00,2.41)(19.00,2701.00,2.41)(20.00,2701.00,2.41)(21.00,2701.00,2.41)(22.00,2701.00,2.41)(23.00,2701.00,2.40)(24.00,2701.00,2.40)(25.00,2701.00,2.40)(26.00,2701.00,2.40)(27.00,2701.00,2.40)(28.00,2701.00,2.40)(29.00,2701.00,2.40)(30.00,2701.00,2.40)

(1.00,2751.00,2.39)(2.00,2751.00,2.39)(3.00,2751.00,2.39)(4.00,2751.00,2.39)(5.00,2751.00,2.39)(6.00,2751.00,2.39)(7.00,2751.00,2.39)(8.00,2751.00,2.39)(9.00,2751.00,2.39)(10.00,2751.00,2.39)(11.00,2751.00,2.39)(12.00,2751.00,2.39)(13.00,2751.00,2.39)(14.00,2751.00,2.39)(15.00,2751.00,2.39)(16.00,2751.00,2.39)(17.00,2751.00,2.39)(18.00,2751.00,2.39)(19.00,2751.00,2.39)(20.00,2751.00,2.38)(21.00,2751.00,2.38)(22.00,2751.00,2.38)(23.00,2751.00,2.38)(24.00,2751.00,2.38)(25.00,2751.00,2.38)(26.00,2751.00,2.38)(27.00,2751.00,2.38)(28.00,2751.00,2.38)(29.00,2751.00,2.38)(30.00,2751.00,2.38)

(1.00,2801.00,2.37)(2.00,2801.00,2.37)(3.00,2801.00,2.37)(4.00,2801.00,2.37)(5.00,2801.00,2.37)(6.00,2801.00,2.37)(7.00,2801.00,2.37)(8.00,2801.00,2.37)(9.00,2801.00,2.37)(10.00,2801.00,2.37)(11.00,2801.00,2.37)(12.00,2801.00,2.37)(13.00,2801.00,2.37)(14.00,2801.00,2.37)(15.00,2801.00,2.37)(16.00,2801.00,2.37)(17.00,2801.00,2.37)(18.00,2801.00,2.37)(19.00,2801.00,2.36)(20.00,2801.00,2.36)(21.00,2801.00,2.36)(22.00,2801.00,2.36)(23.00,2801.00,2.36)(24.00,2801.00,2.36)(25.00,2801.00,2.36)(26.00,2801.00,2.36)(27.00,2801.00,2.36)(28.00,2801.00,2.36)(29.00,2801.00,2.36)(30.00,2801.00,2.36)

(1.00,2851.00,2.35)(2.00,2851.00,2.35)(3.00,2851.00,2.35)(4.00,2851.00,2.35)(5.00,2851.00,2.35)(6.00,2851.00,2.35)(7.00,2851.00,2.35)(8.00,2851.00,2.35)(9.00,2851.00,2.35)(10.00,2851.00,2.35)(11.00,2851.00,2.35)(12.00,2851.00,2.35)(13.00,2851.00,2.35)(14.00,2851.00,2.35)(15.00,2851.00,2.35)(16.00,2851.00,2.35)(17.00,2851.00,2.35)(18.00,2851.00,2.34)(19.00,2851.00,2.34)(20.00,2851.00,2.34)(21.00,2851.00,2.34)(22.00,2851.00,2.34)(23.00,2851.00,2.34)(24.00,2851.00,2.34)(25.00,2851.00,2.34)(26.00,2851.00,2.34)(27.00,2851.00,2.34)(28.00,2851.00,2.34)(29.00,2851.00,2.34)(30.00,2851.00,2.34)

(1.00,2901.00,2.33)(2.00,2901.00,2.33)(3.00,2901.00,2.33)(4.00,2901.00,2.33)(5.00,2901.00,2.33)(6.00,2901.00,2.33)(7.00,2901.00,2.33)(8.00,2901.00,2.33)(9.00,2901.00,2.33)(10.00,2901.00,2.33)(11.00,2901.00,2.33)(12.00,2901.00,2.33)(13.00,2901.00,2.33)(14.00,2901.00,2.33)(15.00,2901.00,2.33)(16.00,2901.00,2.33)(17.00,2901.00,2.33)(18.00,2901.00,2.32)(19.00,2901.00,2.32)(20.00,2901.00,2.32)(21.00,2901.00,2.32)(22.00,2901.00,2.32)(23.00,2901.00,2.32)(24.00,2901.00,2.32)(25.00,2901.00,2.32)(26.00,2901.00,2.32)(27.00,2901.00,2.32)(28.00,2901.00,2.32)(29.00,2901.00,2.32)(30.00,2901.00,2.32)

(1.00,2951.00,2.31)(2.00,2951.00,2.31)(3.00,2951.00,2.31)(4.00,2951.00,2.31)(5.00,2951.00,2.31)(6.00,2951.00,2.31)(7.00,2951.00,2.31)(8.00,2951.00,2.31)(9.00,2951.00,2.31)(10.00,2951.00,2.31)(11.00,2951.00,2.31)(12.00,2951.00,2.31)(13.00,2951.00,2.31)(14.00,2951.00,2.31)(15.00,2951.00,2.31)(16.00,2951.00,2.31)(17.00,2951.00,2.31)(18.00,2951.00,2.31)(19.00,2951.00,2.30)(20.00,2951.00,2.30)(21.00,2951.00,2.30)(22.00,2951.00,2.30)(23.00,2951.00,2.30)(24.00,2951.00,2.30)(25.00,2951.00,2.30)(26.00,2951.00,2.30)(27.00,2951.00,2.30)(28.00,2951.00,2.30)(29.00,2951.00,2.30)(30.00,2951.00,2.30)

(1.00,3001.00,2.29)(2.00,3001.00,2.29)(3.00,3001.00,2.29)(4.00,3001.00,2.29)(5.00,3001.00,2.29)(6.00,3001.00,2.29)(7.00,3001.00,2.29)(8.00,3001.00,2.29)(9.00,3001.00,2.29)(10.00,3001.00,2.29)(11.00,3001.00,2.29)(12.00,3001.00,2.29)(13.00,3001.00,2.29)(14.00,3001.00,2.29)(15.00,3001.00,2.29)(16.00,3001.00,2.29)(17.00,3001.00,2.29)(18.00,3001.00,2.29)(19.00,3001.00,2.29)(20.00,3001.00,2.28)(21.00,3001.00,2.28)(22.00,3001.00,2.28)(23.00,3001.00,2.28)(24.00,3001.00,2.28)(25.00,3001.00,2.28)(26.00,3001.00,2.28)(27.00,3001.00,2.28)(28.00,3001.00,2.28)(29.00,3001.00,2.28)(30.00,3001.00,2.28)

(1.00,3051.00,2.28)(2.00,3051.00,2.27)(3.00,3051.00,2.27)(4.00,3051.00,2.27)(5.00,3051.00,2.27)(6.00,3051.00,2.27)(7.00,3051.00,2.27)(8.00,3051.00,2.27)(9.00,3051.00,2.27)(10.00,3051.00,2.27)(11.00,3051.00,2.27)(12.00,3051.00,2.27)(13.00,3051.00,2.27)(14.00,3051.00,2.27)(15.00,3051.00,2.27)(16.00,3051.00,2.27)(17.00,3051.00,2.27)(18.00,3051.00,2.27)(19.00,3051.00,2.27)(20.00,3051.00,2.27)(21.00,3051.00,2.27)(22.00,3051.00,2.27)(23.00,3051.00,2.26)(24.00,3051.00,2.26)(25.00,3051.00,2.26)(26.00,3051.00,2.26)(27.00,3051.00,2.26)(28.00,3051.00,2.26)(29.00,3051.00,2.26)(30.00,3051.00,2.26)

(1.00,3101.00,2.26)(2.00,3101.00,2.26)(3.00,3101.00,2.26)(4.00,3101.00,2.26)(5.00,3101.00,2.25)(6.00,3101.00,2.25)(7.00,3101.00,2.25)(8.00,3101.00,2.25)(9.00,3101.00,2.25)(10.00,3101.00,2.25)(11.00,3101.00,2.25)(12.00,3101.00,2.25)(13.00,3101.00,2.25)(14.00,3101.00,2.25)(15.00,3101.00,2.25)(16.00,3101.00,2.25)(17.00,3101.00,2.25)(18.00,3101.00,2.25)(19.00,3101.00,2.25)(20.00,3101.00,2.25)(21.00,3101.00,2.25)(22.00,3101.00,2.25)(23.00,3101.00,2.25)(24.00,3101.00,2.25)(25.00,3101.00,2.25)(26.00,3101.00,2.24)(27.00,3101.00,2.24)(28.00,3101.00,2.24)(29.00,3101.00,2.24)(30.00,3101.00,2.24)

(1.00,3151.00,2.24)(2.00,3151.00,2.24)(3.00,3151.00,2.24)(4.00,3151.00,2.24)(5.00,3151.00,2.24)(6.00,3151.00,2.24)(7.00,3151.00,2.24)(8.00,3151.00,2.24)(9.00,3151.00,2.23)(10.00,3151.00,2.23)(11.00,3151.00,2.23)(12.00,3151.00,2.23)(13.00,3151.00,2.23)(14.00,3151.00,2.23)(15.00,3151.00,2.23)(16.00,3151.00,2.23)(17.00,3151.00,2.23)(18.00,3151.00,2.23)(19.00,3151.00,2.23)(20.00,3151.00,2.23)(21.00,3151.00,2.23)(22.00,3151.00,2.23)(23.00,3151.00,2.23)(24.00,3151.00,2.23)(25.00,3151.00,2.23)(26.00,3151.00,2.23)(27.00,3151.00,2.23)(28.00,3151.00,2.23)(29.00,3151.00,2.23)(30.00,3151.00,2.23)

(1.00,3201.00,2.22)(2.00,3201.00,2.22)(3.00,3201.00,2.22)(4.00,3201.00,2.22)(5.00,3201.00,2.22)(6.00,3201.00,2.22)(7.00,3201.00,2.22)(8.00,3201.00,2.22)(9.00,3201.00,2.22)(10.00,3201.00,2.22)(11.00,3201.00,2.22)(12.00,3201.00,2.22)(13.00,3201.00,2.22)(14.00,3201.00,2.21)(15.00,3201.00,2.21)(16.00,3201.00,2.21)(17.00,3201.00,2.21)(18.00,3201.00,2.21)(19.00,3201.00,2.21)(20.00,3201.00,2.21)(21.00,3201.00,2.21)(22.00,3201.00,2.21)(23.00,3201.00,2.21)(24.00,3201.00,2.21)(25.00,3201.00,2.21)(26.00,3201.00,2.21)(27.00,3201.00,2.21)(28.00,3201.00,2.21)(29.00,3201.00,2.21)(30.00,3201.00,2.21)

(1.00,3251.00,2.20)(2.00,3251.00,2.20)(3.00,3251.00,2.20)(4.00,3251.00,2.20)(5.00,3251.00,2.20)(6.00,3251.00,2.20)(7.00,3251.00,2.20)(8.00,3251.00,2.20)(9.00,3251.00,2.20)(10.00,3251.00,2.20)(11.00,3251.00,2.20)(12.00,3251.00,2.20)(13.00,3251.00,2.20)(14.00,3251.00,2.20)(15.00,3251.00,2.20)(16.00,3251.00,2.20)(17.00,3251.00,2.20)(18.00,3251.00,2.20)(19.00,3251.00,2.20)(20.00,3251.00,2.19)(21.00,3251.00,2.19)(22.00,3251.00,2.19)(23.00,3251.00,2.19)(24.00,3251.00,2.19)(25.00,3251.00,2.19)(26.00,3251.00,2.19)(27.00,3251.00,2.19)(28.00,3251.00,2.19)(29.00,3251.00,2.19)(30.00,3251.00,2.19)

(1.00,3301.00,2.19)(2.00,3301.00,2.19)(3.00,3301.00,2.18)(4.00,3301.00,2.18)(5.00,3301.00,2.18)(6.00,3301.00,2.18)(7.00,3301.00,2.18)(8.00,3301.00,2.18)(9.00,3301.00,2.18)(10.00,3301.00,2.18)(11.00,3301.00,2.18)(12.00,3301.00,2.18)(13.00,3301.00,2.18)(14.00,3301.00,2.18)(15.00,3301.00,2.18)(16.00,3301.00,2.18)(17.00,3301.00,2.18)(18.00,3301.00,2.18)(19.00,3301.00,2.18)(20.00,3301.00,2.18)(21.00,3301.00,2.18)(22.00,3301.00,2.18)(23.00,3301.00,2.18)(24.00,3301.00,2.18)(25.00,3301.00,2.18)(26.00,3301.00,2.17)(27.00,3301.00,2.17)(28.00,3301.00,2.17)(29.00,3301.00,2.17)(30.00,3301.00,2.17)

(1.00,3351.00,2.17)(2.00,3351.00,2.17)(3.00,3351.00,2.17)(4.00,3351.00,2.17)(5.00,3351.00,2.17)(6.00,3351.00,2.17)(7.00,3351.00,2.17)(8.00,3351.00,2.17)(9.00,3351.00,2.17)(10.00,3351.00,2.17)(11.00,3351.00,2.16)(12.00,3351.00,2.16)(13.00,3351.00,2.16)(14.00,3351.00,2.16)(15.00,3351.00,2.16)(16.00,3351.00,2.16)(17.00,3351.00,2.16)(18.00,3351.00,2.16)(19.00,3351.00,2.16)(20.00,3351.00,2.16)(21.00,3351.00,2.16)(22.00,3351.00,2.16)(23.00,3351.00,2.16)(24.00,3351.00,2.16)(25.00,3351.00,2.16)(26.00,3351.00,2.16)(27.00,3351.00,2.16)(28.00,3351.00,2.16)(29.00,3351.00,2.16)(30.00,3351.00,2.16)

(1.00,3401.00,2.15)(2.00,3401.00,2.15)(3.00,3401.00,2.15)(4.00,3401.00,2.15)(5.00,3401.00,2.15)(6.00,3401.00,2.15)(7.00,3401.00,2.15)(8.00,3401.00,2.15)(9.00,3401.00,2.15)(10.00,3401.00,2.15)(11.00,3401.00,2.15)(12.00,3401.00,2.15)(13.00,3401.00,2.15)(14.00,3401.00,2.15)(15.00,3401.00,2.15)(16.00,3401.00,2.15)(17.00,3401.00,2.15)(18.00,3401.00,2.15)(19.00,3401.00,2.14)(20.00,3401.00,2.14)(21.00,3401.00,2.14)(22.00,3401.00,2.14)(23.00,3401.00,2.14)(24.00,3401.00,2.14)(25.00,3401.00,2.14)(26.00,3401.00,2.14)(27.00,3401.00,2.14)(28.00,3401.00,2.14)(29.00,3401.00,2.14)(30.00,3401.00,2.14)

(1.00,3451.00,2.14)(2.00,3451.00,2.14)(3.00,3451.00,2.14)(4.00,3451.00,2.13)(5.00,3451.00,2.13)(6.00,3451.00,2.13)(7.00,3451.00,2.13)(8.00,3451.00,2.13)(9.00,3451.00,2.13)(10.00,3451.00,2.13)(11.00,3451.00,2.13)(12.00,3451.00,2.13)(13.00,3451.00,2.13)(14.00,3451.00,2.13)(15.00,3451.00,2.13)(16.00,3451.00,2.13)(17.00,3451.00,2.13)(18.00,3451.00,2.13)(19.00,3451.00,2.13)(20.00,3451.00,2.13)(21.00,3451.00,2.13)(22.00,3451.00,2.13)(23.00,3451.00,2.13)(24.00,3451.00,2.13)(25.00,3451.00,2.13)(26.00,3451.00,2.13)(27.00,3451.00,2.13)(28.00,3451.00,2.12)(29.00,3451.00,2.12)(30.00,3451.00,2.12)

(1.00,3501.00,2.12)(2.00,3501.00,2.12)(3.00,3501.00,2.12)(4.00,3501.00,2.12)(5.00,3501.00,2.12)(6.00,3501.00,2.12)(7.00,3501.00,2.12)(8.00,3501.00,2.12)(9.00,3501.00,2.12)(10.00,3501.00,2.12)(11.00,3501.00,2.12)(12.00,3501.00,2.12)(13.00,3501.00,2.12)(14.00,3501.00,2.11)(15.00,3501.00,2.11)(16.00,3501.00,2.11)(17.00,3501.00,2.11)(18.00,3501.00,2.11)(19.00,3501.00,2.11)(20.00,3501.00,2.11)(21.00,3501.00,2.11)(22.00,3501.00,2.11)(23.00,3501.00,2.11)(24.00,3501.00,2.11)(25.00,3501.00,2.11)(26.00,3501.00,2.11)(27.00,3501.00,2.11)(28.00,3501.00,2.11)(29.00,3501.00,2.11)(30.00,3501.00,2.11)

(1.00,3551.00,2.10)(2.00,3551.00,2.10)(3.00,3551.00,2.10)(4.00,3551.00,2.10)(5.00,3551.00,2.10)(6.00,3551.00,2.10)(7.00,3551.00,2.10)(8.00,3551.00,2.10)(9.00,3551.00,2.10)(10.00,3551.00,2.10)(11.00,3551.00,2.10)(12.00,3551.00,2.10)(13.00,3551.00,2.10)(14.00,3551.00,2.10)(15.00,3551.00,2.10)(16.00,3551.00,2.10)(17.00,3551.00,2.10)(18.00,3551.00,2.10)(19.00,3551.00,2.10)(20.00,3551.00,2.10)(21.00,3551.00,2.10)(22.00,3551.00,2.10)(23.00,3551.00,2.10)(24.00,3551.00,2.09)(25.00,3551.00,2.09)(26.00,3551.00,2.09)(27.00,3551.00,2.09)(28.00,3551.00,2.09)(29.00,3551.00,2.09)(30.00,3551.00,2.09)

(1.00,3601.00,2.09)(2.00,3601.00,2.09)(3.00,3601.00,2.09)(4.00,3601.00,2.09)(5.00,3601.00,2.09)(6.00,3601.00,2.09)(7.00,3601.00,2.09)(8.00,3601.00,2.09)(9.00,3601.00,2.09)(10.00,3601.00,2.09)(11.00,3601.00,2.08)(12.00,3601.00,2.08)(13.00,3601.00,2.08)(14.00,3601.00,2.08)(15.00,3601.00,2.08)(16.00,3601.00,2.08)(17.00,3601.00,2.08)(18.00,3601.00,2.08)(19.00,3601.00,2.08)(20.00,3601.00,2.08)(21.00,3601.00,2.08)(22.00,3601.00,2.08)(23.00,3601.00,2.08)(24.00,3601.00,2.08)(25.00,3601.00,2.08)(26.00,3601.00,2.08)(27.00,3601.00,2.08)(28.00,3601.00,2.08)(29.00,3601.00,2.08)(30.00,3601.00,2.08)

(1.00,3651.00,2.07)(2.00,3651.00,2.07)(3.00,3651.00,2.07)(4.00,3651.00,2.07)(5.00,3651.00,2.07)(6.00,3651.00,2.07)(7.00,3651.00,2.07)(8.00,3651.00,2.07)(9.00,3651.00,2.07)(10.00,3651.00,2.07)(11.00,3651.00,2.07)(12.00,3651.00,2.07)(13.00,3651.00,2.07)(14.00,3651.00,2.07)(15.00,3651.00,2.07)(16.00,3651.00,2.07)(17.00,3651.00,2.07)(18.00,3651.00,2.07)(19.00,3651.00,2.07)(20.00,3651.00,2.07)(21.00,3651.00,2.07)(22.00,3651.00,2.07)(23.00,3651.00,2.06)(24.00,3651.00,2.06)(25.00,3651.00,2.06)(26.00,3651.00,2.06)(27.00,3651.00,2.06)(28.00,3651.00,2.06)(29.00,3651.00,2.06)(30.00,3651.00,2.06)

(1.00,3701.00,2.06)(2.00,3701.00,2.06)(3.00,3701.00,2.06)(4.00,3701.00,2.06)(5.00,3701.00,2.06)(6.00,3701.00,2.06)(7.00,3701.00,2.06)(8.00,3701.00,2.06)(9.00,3701.00,2.06)(10.00,3701.00,2.06)(11.00,3701.00,2.05)(12.00,3701.00,2.05)(13.00,3701.00,2.05)(14.00,3701.00,2.05)(15.00,3701.00,2.05)(16.00,3701.00,2.05)(17.00,3701.00,2.05)(18.00,3701.00,2.05)(19.00,3701.00,2.05)(20.00,3701.00,2.05)(21.00,3701.00,2.05)(22.00,3701.00,2.05)(23.00,3701.00,2.05)(24.00,3701.00,2.05)(25.00,3701.00,2.05)(26.00,3701.00,2.05)(27.00,3701.00,2.05)(28.00,3701.00,2.05)(29.00,3701.00,2.05)(30.00,3701.00,2.05)

(1.00,3751.00,2.04)(2.00,3751.00,2.04)(3.00,3751.00,2.04)(4.00,3751.00,2.04)(5.00,3751.00,2.04)(6.00,3751.00,2.04)(7.00,3751.00,2.04)(8.00,3751.00,2.04)(9.00,3751.00,2.04)(10.00,3751.00,2.04)(11.00,3751.00,2.04)(12.00,3751.00,2.04)(13.00,3751.00,2.04)(14.00,3751.00,2.04)(15.00,3751.00,2.04)(16.00,3751.00,2.04)(17.00,3751.00,2.04)(18.00,3751.00,2.04)(19.00,3751.00,2.04)(20.00,3751.00,2.04)(21.00,3751.00,2.04)(22.00,3751.00,2.04)(23.00,3751.00,2.04)(24.00,3751.00,2.04)(25.00,3751.00,2.03)(26.00,3751.00,2.03)(27.00,3751.00,2.03)(28.00,3751.00,2.03)(29.00,3751.00,2.03)(30.00,3751.00,2.03)

(1.00,3801.00,2.03)(2.00,3801.00,2.03)(3.00,3801.00,2.03)(4.00,3801.00,2.03)(5.00,3801.00,2.03)(6.00,3801.00,2.03)(7.00,3801.00,2.03)(8.00,3801.00,2.03)(9.00,3801.00,2.03)(10.00,3801.00,2.03)(11.00,3801.00,2.03)(12.00,3801.00,2.03)(13.00,3801.00,2.03)(14.00,3801.00,2.02)(15.00,3801.00,2.02)(16.00,3801.00,2.02)(17.00,3801.00,2.02)(18.00,3801.00,2.02)(19.00,3801.00,2.02)(20.00,3801.00,2.02)(21.00,3801.00,2.02)(22.00,3801.00,2.02)(23.00,3801.00,2.02)(24.00,3801.00,2.02)(25.00,3801.00,2.02)(26.00,3801.00,2.02)(27.00,3801.00,2.02)(28.00,3801.00,2.02)(29.00,3801.00,2.02)(30.00,3801.00,2.02)

(1.00,3851.00,2.02)(2.00,3851.00,2.02)(3.00,3851.00,2.01)(4.00,3851.00,2.01)(5.00,3851.00,2.01)(6.00,3851.00,2.01)(7.00,3851.00,2.01)(8.00,3851.00,2.01)(9.00,3851.00,2.01)(10.00,3851.00,2.01)(11.00,3851.00,2.01)(12.00,3851.00,2.01)(13.00,3851.00,2.01)(14.00,3851.00,2.01)(15.00,3851.00,2.01)(16.00,3851.00,2.01)(17.00,3851.00,2.01)(18.00,3851.00,2.01)(19.00,3851.00,2.01)(20.00,3851.00,2.01)(21.00,3851.00,2.01)(22.00,3851.00,2.01)(23.00,3851.00,2.01)(24.00,3851.00,2.01)(25.00,3851.00,2.01)(26.00,3851.00,2.01)(27.00,3851.00,2.01)(28.00,3851.00,2.01)(29.00,3851.00,2.00)(30.00,3851.00,2.00)

(1.00,3901.00,2.00)(2.00,3901.00,2.00)(3.00,3901.00,2.00)(4.00,3901.00,2.00)(5.00,3901.00,2.00)(6.00,3901.00,2.00)(7.00,3901.00,2.00)(8.00,3901.00,2.00)(9.00,3901.00,2.00)(10.00,3901.00,2.00)(11.00,3901.00,2.00)(12.00,3901.00,2.00)(13.00,3901.00,2.00)(14.00,3901.00,2.00)(15.00,3901.00,2.00)(16.00,3901.00,2.00)(17.00,3901.00,2.00)(18.00,3901.00,2.00)(19.00,3901.00,1.99)(20.00,3901.00,1.99)(21.00,3901.00,1.99)(22.00,3901.00,1.99)(23.00,3901.00,1.99)(24.00,3901.00,1.99)(25.00,3901.00,1.99)(26.00,3901.00,1.99)(27.00,3901.00,1.99)(28.00,3901.00,1.99)(29.00,3901.00,1.99)(30.00,3901.00,1.99)

(1.00,3951.00,1.99)(2.00,3951.00,1.99)(3.00,3951.00,1.99)(4.00,3951.00,1.99)(5.00,3951.00,1.99)(6.00,3951.00,1.99)(7.00,3951.00,1.99)(8.00,3951.00,1.99)(9.00,3951.00,1.98)(10.00,3951.00,1.98)(11.00,3951.00,1.98)(12.00,3951.00,1.98)(13.00,3951.00,1.98)(14.00,3951.00,1.98)(15.00,3951.00,1.98)(16.00,3951.00,1.98)(17.00,3951.00,1.98)(18.00,3951.00,1.98)(19.00,3951.00,1.98)(20.00,3951.00,1.98)(21.00,3951.00,1.98)(22.00,3951.00,1.98)(23.00,3951.00,1.98)(24.00,3951.00,1.98)(25.00,3951.00,1.98)(26.00,3951.00,1.98)(27.00,3951.00,1.98)(28.00,3951.00,1.98)(29.00,3951.00,1.98)(30.00,3951.00,1.98)

(1.00,4001.00,1.97)(2.00,4001.00,1.97)(3.00,4001.00,1.97)(4.00,4001.00,1.97)(5.00,4001.00,1.97)(6.00,4001.00,1.97)(7.00,4001.00,1.97)(8.00,4001.00,1.97)(9.00,4001.00,1.97)(10.00,4001.00,1.97)(11.00,4001.00,1.97)(12.00,4001.00,1.97)(13.00,4001.00,1.97)(14.00,4001.00,1.97)(15.00,4001.00,1.97)(16.00,4001.00,1.97)(17.00,4001.00,1.97)(18.00,4001.00,1.97)(19.00,4001.00,1.97)(20.00,4001.00,1.97)(21.00,4001.00,1.97)(22.00,4001.00,1.97)(23.00,4001.00,1.97)(24.00,4001.00,1.97)(25.00,4001.00,1.97)(26.00,4001.00,1.97)(27.00,4001.00,1.96)(28.00,4001.00,1.96)(29.00,4001.00,1.96)(30.00,4001.00,1.96)

(1.00,4051.00,1.96)(2.00,4051.00,1.96)(3.00,4051.00,1.96)(4.00,4051.00,1.96)(5.00,4051.00,1.96)(6.00,4051.00,1.96)(7.00,4051.00,1.96)(8.00,4051.00,1.96)(9.00,4051.00,1.96)(10.00,4051.00,1.96)(11.00,4051.00,1.96)(12.00,4051.00,1.96)(13.00,4051.00,1.96)(14.00,4051.00,1.96)(15.00,4051.00,1.96)(16.00,4051.00,1.96)(17.00,4051.00,1.96)(18.00,4051.00,1.95)(19.00,4051.00,1.95)(20.00,4051.00,1.95)(21.00,4051.00,1.95)(22.00,4051.00,1.95)(23.00,4051.00,1.95)(24.00,4051.00,1.95)(25.00,4051.00,1.95)(26.00,4051.00,1.95)(27.00,4051.00,1.95)(28.00,4051.00,1.95)(29.00,4051.00,1.95)(30.00,4051.00,1.95)

(1.00,4101.00,1.95)(2.00,4101.00,1.95)(3.00,4101.00,1.95)(4.00,4101.00,1.95)(5.00,4101.00,1.95)(6.00,4101.00,1.95)(7.00,4101.00,1.95)(8.00,4101.00,1.95)(9.00,4101.00,1.94)(10.00,4101.00,1.94)(11.00,4101.00,1.94)(12.00,4101.00,1.94)(13.00,4101.00,1.94)(14.00,4101.00,1.94)(15.00,4101.00,1.94)(16.00,4101.00,1.94)(17.00,4101.00,1.94)(18.00,4101.00,1.94)(19.00,4101.00,1.94)(20.00,4101.00,1.94)(21.00,4101.00,1.94)(22.00,4101.00,1.94)(23.00,4101.00,1.94)(24.00,4101.00,1.94)(25.00,4101.00,1.94)(26.00,4101.00,1.94)(27.00,4101.00,1.94)(28.00,4101.00,1.94)(29.00,4101.00,1.94)(30.00,4101.00,1.94)

(1.00,4151.00,1.93)(2.00,4151.00,1.93)(3.00,4151.00,1.93)(4.00,4151.00,1.93)(5.00,4151.00,1.93)(6.00,4151.00,1.93)(7.00,4151.00,1.93)(8.00,4151.00,1.93)(9.00,4151.00,1.93)(10.00,4151.00,1.93)(11.00,4151.00,1.93)(12.00,4151.00,1.93)(13.00,4151.00,1.93)(14.00,4151.00,1.93)(15.00,4151.00,1.93)(16.00,4151.00,1.93)(17.00,4151.00,1.93)(18.00,4151.00,1.93)(19.00,4151.00,1.93)(20.00,4151.00,1.93)(21.00,4151.00,1.93)(22.00,4151.00,1.93)(23.00,4151.00,1.93)(24.00,4151.00,1.93)(25.00,4151.00,1.93)(26.00,4151.00,1.93)(27.00,4151.00,1.93)(28.00,4151.00,1.93)(29.00,4151.00,1.93)(30.00,4151.00,1.92)

(1.00,4201.00,1.92)(2.00,4201.00,1.92)(3.00,4201.00,1.92)(4.00,4201.00,1.92)(5.00,4201.00,1.92)(6.00,4201.00,1.92)(7.00,4201.00,1.92)(8.00,4201.00,1.92)(9.00,4201.00,1.92)(10.00,4201.00,1.92)(11.00,4201.00,1.92)(12.00,4201.00,1.92)(13.00,4201.00,1.92)(14.00,4201.00,1.92)(15.00,4201.00,1.92)(16.00,4201.00,1.92)(17.00,4201.00,1.92)(18.00,4201.00,1.92)(19.00,4201.00,1.92)(20.00,4201.00,1.92)(21.00,4201.00,1.92)(22.00,4201.00,1.91)(23.00,4201.00,1.91)(24.00,4201.00,1.91)(25.00,4201.00,1.91)(26.00,4201.00,1.91)(27.00,4201.00,1.91)(28.00,4201.00,1.91)(29.00,4201.00,1.91)(30.00,4201.00,1.91)

(1.00,4251.00,1.91)(2.00,4251.00,1.91)(3.00,4251.00,1.91)(4.00,4251.00,1.91)(5.00,4251.00,1.91)(6.00,4251.00,1.91)(7.00,4251.00,1.91)(8.00,4251.00,1.91)(9.00,4251.00,1.91)(10.00,4251.00,1.91)(11.00,4251.00,1.91)(12.00,4251.00,1.91)(13.00,4251.00,1.91)(14.00,4251.00,1.91)(15.00,4251.00,1.90)(16.00,4251.00,1.90)(17.00,4251.00,1.90)(18.00,4251.00,1.90)(19.00,4251.00,1.90)(20.00,4251.00,1.90)(21.00,4251.00,1.90)(22.00,4251.00,1.90)(23.00,4251.00,1.90)(24.00,4251.00,1.90)(25.00,4251.00,1.90)(26.00,4251.00,1.90)(27.00,4251.00,1.90)(28.00,4251.00,1.90)(29.00,4251.00,1.90)(30.00,4251.00,1.90)

(1.00,4301.00,1.90)(2.00,4301.00,1.90)(3.00,4301.00,1.90)(4.00,4301.00,1.90)(5.00,4301.00,1.90)(6.00,4301.00,1.90)(7.00,4301.00,1.90)(8.00,4301.00,1.89)(9.00,4301.00,1.89)(10.00,4301.00,1.89)(11.00,4301.00,1.89)(12.00,4301.00,1.89)(13.00,4301.00,1.89)(14.00,4301.00,1.89)(15.00,4301.00,1.89)(16.00,4301.00,1.89)(17.00,4301.00,1.89)(18.00,4301.00,1.89)(19.00,4301.00,1.89)(20.00,4301.00,1.89)(21.00,4301.00,1.89)(22.00,4301.00,1.89)(23.00,4301.00,1.89)(24.00,4301.00,1.89)(25.00,4301.00,1.89)(26.00,4301.00,1.89)(27.00,4301.00,1.89)(28.00,4301.00,1.89)(29.00,4301.00,1.89)(30.00,4301.00,1.89)

(1.00,4351.00,1.88)(2.00,4351.00,1.88)(3.00,4351.00,1.88)(4.00,4351.00,1.88)(5.00,4351.00,1.88)(6.00,4351.00,1.88)(7.00,4351.00,1.88)(8.00,4351.00,1.88)(9.00,4351.00,1.88)(10.00,4351.00,1.88)(11.00,4351.00,1.88)(12.00,4351.00,1.88)(13.00,4351.00,1.88)(14.00,4351.00,1.88)(15.00,4351.00,1.88)(16.00,4351.00,1.88)(17.00,4351.00,1.88)(18.00,4351.00,1.88)(19.00,4351.00,1.88)(20.00,4351.00,1.88)(21.00,4351.00,1.88)(22.00,4351.00,1.88)(23.00,4351.00,1.88)(24.00,4351.00,1.88)(25.00,4351.00,1.88)(26.00,4351.00,1.88)(27.00,4351.00,1.88)(28.00,4351.00,1.88)(29.00,4351.00,1.88)(30.00,4351.00,1.88)

(1.00,4401.00,1.87)(2.00,4401.00,1.87)(3.00,4401.00,1.87)(4.00,4401.00,1.87)(5.00,4401.00,1.87)(6.00,4401.00,1.87)(7.00,4401.00,1.87)(8.00,4401.00,1.87)(9.00,4401.00,1.87)(10.00,4401.00,1.87)(11.00,4401.00,1.87)(12.00,4401.00,1.87)(13.00,4401.00,1.87)(14.00,4401.00,1.87)(15.00,4401.00,1.87)(16.00,4401.00,1.87)(17.00,4401.00,1.87)(18.00,4401.00,1.87)(19.00,4401.00,1.87)(20.00,4401.00,1.87)(21.00,4401.00,1.87)(22.00,4401.00,1.87)(23.00,4401.00,1.87)(24.00,4401.00,1.87)(25.00,4401.00,1.87)(26.00,4401.00,1.86)(27.00,4401.00,1.86)(28.00,4401.00,1.86)(29.00,4401.00,1.86)(30.00,4401.00,1.86)

(1.00,4451.00,1.86)(2.00,4451.00,1.86)(3.00,4451.00,1.86)(4.00,4451.00,1.86)(5.00,4451.00,1.86)(6.00,4451.00,1.86)(7.00,4451.00,1.86)(8.00,4451.00,1.86)(9.00,4451.00,1.86)(10.00,4451.00,1.86)(11.00,4451.00,1.86)(12.00,4451.00,1.86)(13.00,4451.00,1.86)(14.00,4451.00,1.86)(15.00,4451.00,1.86)(16.00,4451.00,1.86)(17.00,4451.00,1.86)(18.00,4451.00,1.86)(19.00,4451.00,1.86)(20.00,4451.00,1.85)(21.00,4451.00,1.85)(22.00,4451.00,1.85)(23.00,4451.00,1.85)(24.00,4451.00,1.85)(25.00,4451.00,1.85)(26.00,4451.00,1.85)(27.00,4451.00,1.85)(28.00,4451.00,1.85)(29.00,4451.00,1.85)(30.00,4451.00,1.85)

(1.00,4501.00,1.85)(2.00,4501.00,1.85)(3.00,4501.00,1.85)(4.00,4501.00,1.85)(5.00,4501.00,1.85)(6.00,4501.00,1.85)(7.00,4501.00,1.85)(8.00,4501.00,1.85)(9.00,4501.00,1.85)(10.00,4501.00,1.85)(11.00,4501.00,1.85)(12.00,4501.00,1.85)(13.00,4501.00,1.85)(14.00,4501.00,1.85)(15.00,4501.00,1.84)(16.00,4501.00,1.84)(17.00,4501.00,1.84)(18.00,4501.00,1.84)(19.00,4501.00,1.84)(20.00,4501.00,1.84)(21.00,4501.00,1.84)(22.00,4501.00,1.84)(23.00,4501.00,1.84)(24.00,4501.00,1.84)(25.00,4501.00,1.84)(26.00,4501.00,1.84)(27.00,4501.00,1.84)(28.00,4501.00,1.84)(29.00,4501.00,1.84)(30.00,4501.00,1.84)

(1.00,4551.00,1.84)(2.00,4551.00,1.84)(3.00,4551.00,1.84)(4.00,4551.00,1.84)(5.00,4551.00,1.84)(6.00,4551.00,1.84)(7.00,4551.00,1.84)(8.00,4551.00,1.84)(9.00,4551.00,1.84)(10.00,4551.00,1.84)(11.00,4551.00,1.83)(12.00,4551.00,1.83)(13.00,4551.00,1.83)(14.00,4551.00,1.83)(15.00,4551.00,1.83)(16.00,4551.00,1.83)(17.00,4551.00,1.83)(18.00,4551.00,1.83)(19.00,4551.00,1.83)(20.00,4551.00,1.83)(21.00,4551.00,1.83)(22.00,4551.00,1.83)(23.00,4551.00,1.83)(24.00,4551.00,1.83)(25.00,4551.00,1.83)(26.00,4551.00,1.83)(27.00,4551.00,1.83)(28.00,4551.00,1.83)(29.00,4551.00,1.83)(30.00,4551.00,1.83)

(1.00,4601.00,1.83)(2.00,4601.00,1.83)(3.00,4601.00,1.83)(4.00,4601.00,1.83)(5.00,4601.00,1.83)(6.00,4601.00,1.83)(7.00,4601.00,1.82)(8.00,4601.00,1.82)(9.00,4601.00,1.82)(10.00,4601.00,1.82)(11.00,4601.00,1.82)(12.00,4601.00,1.82)(13.00,4601.00,1.82)(14.00,4601.00,1.82)(15.00,4601.00,1.82)(16.00,4601.00,1.82)(17.00,4601.00,1.82)(18.00,4601.00,1.82)(19.00,4601.00,1.82)(20.00,4601.00,1.82)(21.00,4601.00,1.82)(22.00,4601.00,1.82)(23.00,4601.00,1.82)(24.00,4601.00,1.82)(25.00,4601.00,1.82)(26.00,4601.00,1.82)(27.00,4601.00,1.82)(28.00,4601.00,1.82)(29.00,4601.00,1.82)(30.00,4601.00,1.82)

(1.00,4651.00,1.82)(2.00,4651.00,1.82)(3.00,4651.00,1.81)(4.00,4651.00,1.81)(5.00,4651.00,1.81)(6.00,4651.00,1.81)(7.00,4651.00,1.81)(8.00,4651.00,1.81)(9.00,4651.00,1.81)(10.00,4651.00,1.81)(11.00,4651.00,1.81)(12.00,4651.00,1.81)(13.00,4651.00,1.81)(14.00,4651.00,1.81)(15.00,4651.00,1.81)(16.00,4651.00,1.81)(17.00,4651.00,1.81)(18.00,4651.00,1.81)(19.00,4651.00,1.81)(20.00,4651.00,1.81)(21.00,4651.00,1.81)(22.00,4651.00,1.81)(23.00,4651.00,1.81)(24.00,4651.00,1.81)(25.00,4651.00,1.81)(26.00,4651.00,1.81)(27.00,4651.00,1.81)(28.00,4651.00,1.81)(29.00,4651.00,1.81)(30.00,4651.00,1.81)

(1.00,4701.00,1.80)(2.00,4701.00,1.80)(3.00,4701.00,1.80)(4.00,4701.00,1.80)(5.00,4701.00,1.80)(6.00,4701.00,1.80)(7.00,4701.00,1.80)(8.00,4701.00,1.80)(9.00,4701.00,1.80)(10.00,4701.00,1.80)(11.00,4701.00,1.80)(12.00,4701.00,1.80)(13.00,4701.00,1.80)(14.00,4701.00,1.80)(15.00,4701.00,1.80)(16.00,4701.00,1.80)(17.00,4701.00,1.80)(18.00,4701.00,1.80)(19.00,4701.00,1.80)(20.00,4701.00,1.80)(21.00,4701.00,1.80)(22.00,4701.00,1.80)(23.00,4701.00,1.80)(24.00,4701.00,1.80)(25.00,4701.00,1.80)(26.00,4701.00,1.80)(27.00,4701.00,1.80)(28.00,4701.00,1.80)(29.00,4701.00,1.80)(30.00,4701.00,1.80)

(1.00,4751.00,1.79)(2.00,4751.00,1.79)(3.00,4751.00,1.79)(4.00,4751.00,1.79)(5.00,4751.00,1.79)(6.00,4751.00,1.79)(7.00,4751.00,1.79)(8.00,4751.00,1.79)(9.00,4751.00,1.79)(10.00,4751.00,1.79)(11.00,4751.00,1.79)(12.00,4751.00,1.79)(13.00,4751.00,1.79)(14.00,4751.00,1.79)(15.00,4751.00,1.79)(16.00,4751.00,1.79)(17.00,4751.00,1.79)(18.00,4751.00,1.79)(19.00,4751.00,1.79)(20.00,4751.00,1.79)(21.00,4751.00,1.79)(22.00,4751.00,1.79)(23.00,4751.00,1.79)(24.00,4751.00,1.79)(25.00,4751.00,1.79)(26.00,4751.00,1.79)(27.00,4751.00,1.79)(28.00,4751.00,1.79)(29.00,4751.00,1.79)(30.00,4751.00,1.78)

(1.00,4801.00,1.78)(2.00,4801.00,1.78)(3.00,4801.00,1.78)(4.00,4801.00,1.78)(5.00,4801.00,1.78)(6.00,4801.00,1.78)(7.00,4801.00,1.78)(8.00,4801.00,1.78)(9.00,4801.00,1.78)(10.00,4801.00,1.78)(11.00,4801.00,1.78)(12.00,4801.00,1.78)(13.00,4801.00,1.78)(14.00,4801.00,1.78)(15.00,4801.00,1.78)(16.00,4801.00,1.78)(17.00,4801.00,1.78)(18.00,4801.00,1.78)(19.00,4801.00,1.78)(20.00,4801.00,1.78)(21.00,4801.00,1.78)(22.00,4801.00,1.78)(23.00,4801.00,1.78)(24.00,4801.00,1.78)(25.00,4801.00,1.78)(26.00,4801.00,1.78)(27.00,4801.00,1.77)(28.00,4801.00,1.77)(29.00,4801.00,1.77)(30.00,4801.00,1.77)

(1.00,4851.00,1.77)(2.00,4851.00,1.77)(3.00,4851.00,1.77)(4.00,4851.00,1.77)(5.00,4851.00,1.77)(6.00,4851.00,1.77)(7.00,4851.00,1.77)(8.00,4851.00,1.77)(9.00,4851.00,1.77)(10.00,4851.00,1.77)(11.00,4851.00,1.77)(12.00,4851.00,1.77)(13.00,4851.00,1.77)(14.00,4851.00,1.77)(15.00,4851.00,1.77)(16.00,4851.00,1.77)(17.00,4851.00,1.77)(18.00,4851.00,1.77)(19.00,4851.00,1.77)(20.00,4851.00,1.77)(21.00,4851.00,1.77)(22.00,4851.00,1.77)(23.00,4851.00,1.77)(24.00,4851.00,1.77)(25.00,4851.00,1.77)(26.00,4851.00,1.76)(27.00,4851.00,1.76)(28.00,4851.00,1.76)(29.00,4851.00,1.76)(30.00,4851.00,1.76)

(1.00,4901.00,1.76)(2.00,4901.00,1.76)(3.00,4901.00,1.76)(4.00,4901.00,1.76)(5.00,4901.00,1.76)(6.00,4901.00,1.76)(7.00,4901.00,1.76)(8.00,4901.00,1.76)(9.00,4901.00,1.76)(10.00,4901.00,1.76)(11.00,4901.00,1.76)(12.00,4901.00,1.76)(13.00,4901.00,1.76)(14.00,4901.00,1.76)(15.00,4901.00,1.76)(16.00,4901.00,1.76)(17.00,4901.00,1.76)(18.00,4901.00,1.76)(19.00,4901.00,1.76)(20.00,4901.00,1.76)(21.00,4901.00,1.76)(22.00,4901.00,1.76)(23.00,4901.00,1.76)(24.00,4901.00,1.75)(25.00,4901.00,1.75)(26.00,4901.00,1.75)(27.00,4901.00,1.75)(28.00,4901.00,1.75)(29.00,4901.00,1.75)(30.00,4901.00,1.75)

(1.00,4951.00,1.75)(2.00,4951.00,1.75)(3.00,4951.00,1.75)(4.00,4951.00,1.75)(5.00,4951.00,1.75)(6.00,4951.00,1.75)(7.00,4951.00,1.75)(8.00,4951.00,1.75)(9.00,4951.00,1.75)(10.00,4951.00,1.75)(11.00,4951.00,1.75)(12.00,4951.00,1.75)(13.00,4951.00,1.75)(14.00,4951.00,1.75)(15.00,4951.00,1.75)(16.00,4951.00,1.75)(17.00,4951.00,1.75)(18.00,4951.00,1.75)(19.00,4951.00,1.75)(20.00,4951.00,1.75)(21.00,4951.00,1.75)(22.00,4951.00,1.75)(23.00,4951.00,1.75)(24.00,4951.00,1.74)(25.00,4951.00,1.74)(26.00,4951.00,1.74)(27.00,4951.00,1.74)(28.00,4951.00,1.74)(29.00,4951.00,1.74)(30.00,4951.00,1.74)


}; \end{axis} \end{tikzpicture}

           \label{dlmproofcc}          
        \end{subfigure}
        ~
		\begin{subfigure}[]{0.5\textwidth}
          \caption{Doc. length and Document Frequency}
          
\begin{tikzpicture}[thick,scale=0.7, every node/.style={transform shape}] \begin{axis}[
 %title={},
 %y dir=reverse, 
 %x dir=reverse, 
 ylabel={docLength},
 xlabel={docFrequency},
 zlabel={DirichletLM score},
 every axis/.append style={font=\large\bfseries},
 max space between ticks=25pt
% yticklabels={0k,100k}
 ] 

		\addplot3[surf] coordinates { 
%patch,patch type=biquadratic, shader=faceted,patch refines=3
(100.00,20.00,4.60)(100.00,18.00,4.60)(100.00,16.00,4.60)(100.00,14.00,4.60)(100.00,12.00,4.60)(100.00,10.00,4.61)(100.00,8.00,4.61)(100.00,6.00,4.61)(100.00,4.00,4.61)(100.00,2.00,4.61)

(1100.00,20.00,1.64)(1100.00,18.00,1.64)(1100.00,16.00,1.64)(1100.00,14.00,1.64)(1100.00,12.00,1.64)(1100.00,10.00,1.64)(1100.00,8.00,1.64)(1100.00,6.00,1.64)(1100.00,4.00,1.64)(1100.00,2.00,1.65)

(2100.00,20.00,1.07)(2100.00,18.00,1.07)(2100.00,16.00,1.07)(2100.00,14.00,1.07)(2100.00,12.00,1.07)(2100.00,10.00,1.08)(2100.00,8.00,1.08)(2100.00,6.00,1.08)(2100.00,4.00,1.08)(2100.00,2.00,1.08)

(3100.00,20.00,0.80)(3100.00,18.00,0.80)(3100.00,16.00,0.80)(3100.00,14.00,0.80)(3100.00,12.00,0.81)(3100.00,10.00,0.81)(3100.00,8.00,0.81)(3100.00,6.00,0.81)(3100.00,4.00,0.81)(3100.00,2.00,0.81)

(4100.00,20.00,0.64)(4100.00,18.00,0.64)(4100.00,16.00,0.64)(4100.00,14.00,0.64)(4100.00,12.00,0.65)(4100.00,10.00,0.65)(4100.00,8.00,0.65)(4100.00,6.00,0.65)(4100.00,4.00,0.65)(4100.00,2.00,0.65)

(5100.00,20.00,0.53)(5100.00,18.00,0.54)(5100.00,16.00,0.54)(5100.00,14.00,0.54)(5100.00,12.00,0.54)(5100.00,10.00,0.54)(5100.00,8.00,0.54)(5100.00,6.00,0.54)(5100.00,4.00,0.54)(5100.00,2.00,0.54)

(6100.00,20.00,0.46)(6100.00,18.00,0.46)(6100.00,16.00,0.46)(6100.00,14.00,0.46)(6100.00,12.00,0.46)(6100.00,10.00,0.46)(6100.00,8.00,0.46)(6100.00,6.00,0.47)(6100.00,4.00,0.47)(6100.00,2.00,0.47)

(7100.00,20.00,0.40)(7100.00,18.00,0.40)(7100.00,16.00,0.40)(7100.00,14.00,0.40)(7100.00,12.00,0.40)(7100.00,10.00,0.41)(7100.00,8.00,0.41)(7100.00,6.00,0.41)(7100.00,4.00,0.41)(7100.00,2.00,0.41)

(8100.00,20.00,0.36)(8100.00,18.00,0.36)(8100.00,16.00,0.36)(8100.00,14.00,0.36)(8100.00,12.00,0.36)(8100.00,10.00,0.36)(8100.00,8.00,0.36)(8100.00,6.00,0.36)(8100.00,4.00,0.36)(8100.00,2.00,0.37)

(9100.00,20.00,0.32)(9100.00,18.00,0.32)(9100.00,16.00,0.32)(9100.00,14.00,0.32)(9100.00,12.00,0.32)(9100.00,10.00,0.32)(9100.00,8.00,0.33)(9100.00,6.00,0.33)(9100.00,4.00,0.33)(9100.00,2.00,0.33)

(10100.00,20.00,0.29)(10100.00,18.00,0.29)(10100.00,16.00,0.29)(10100.00,14.00,0.29)(10100.00,12.00,0.29)(10100.00,10.00,0.30)(10100.00,8.00,0.30)(10100.00,6.00,0.30)(10100.00,4.00,0.30)(10100.00,2.00,0.30)

(11100.00,20.00,0.26)(11100.00,18.00,0.27)(11100.00,16.00,0.27)(11100.00,14.00,0.27)(11100.00,12.00,0.27)(11100.00,10.00,0.27)(11100.00,8.00,0.27)(11100.00,6.00,0.27)(11100.00,4.00,0.27)(11100.00,2.00,0.28)

(12100.00,20.00,0.24)(12100.00,18.00,0.25)(12100.00,16.00,0.25)(12100.00,14.00,0.25)(12100.00,12.00,0.25)(12100.00,10.00,0.25)(12100.00,8.00,0.25)(12100.00,6.00,0.25)(12100.00,4.00,0.25)(12100.00,2.00,0.25)

(13100.00,20.00,0.23)(13100.00,18.00,0.23)(13100.00,16.00,0.23)(13100.00,14.00,0.23)(13100.00,12.00,0.23)(13100.00,10.00,0.23)(13100.00,8.00,0.23)(13100.00,6.00,0.23)(13100.00,4.00,0.24)(13100.00,2.00,0.24)

(14100.00,20.00,0.21)(14100.00,18.00,0.21)(14100.00,16.00,0.21)(14100.00,14.00,0.21)(14100.00,12.00,0.21)(14100.00,10.00,0.22)(14100.00,8.00,0.22)(14100.00,6.00,0.22)(14100.00,4.00,0.22)(14100.00,2.00,0.22)

(15100.00,20.00,0.20)(15100.00,18.00,0.20)(15100.00,16.00,0.20)(15100.00,14.00,0.20)(15100.00,12.00,0.20)(15100.00,10.00,0.20)(15100.00,8.00,0.20)(15100.00,6.00,0.20)(15100.00,4.00,0.21)(15100.00,2.00,0.21)

(16100.00,20.00,0.18)(16100.00,18.00,0.19)(16100.00,16.00,0.19)(16100.00,14.00,0.19)(16100.00,12.00,0.19)(16100.00,10.00,0.19)(16100.00,8.00,0.19)(16100.00,6.00,0.19)(16100.00,4.00,0.19)(16100.00,2.00,0.19)

(17100.00,20.00,0.17)(17100.00,18.00,0.18)(17100.00,16.00,0.18)(17100.00,14.00,0.18)(17100.00,12.00,0.18)(17100.00,10.00,0.18)(17100.00,8.00,0.18)(17100.00,6.00,0.18)(17100.00,4.00,0.18)(17100.00,2.00,0.18)

(18100.00,20.00,0.16)(18100.00,18.00,0.17)(18100.00,16.00,0.17)(18100.00,14.00,0.17)(18100.00,12.00,0.17)(18100.00,10.00,0.17)(18100.00,8.00,0.17)(18100.00,6.00,0.17)(18100.00,4.00,0.17)(18100.00,2.00,0.17)

(19100.00,20.00,0.16)(19100.00,18.00,0.16)(19100.00,16.00,0.16)(19100.00,14.00,0.16)(19100.00,12.00,0.16)(19100.00,10.00,0.16)(19100.00,8.00,0.16)(19100.00,6.00,0.16)(19100.00,4.00,0.16)(19100.00,2.00,0.17)

(20100.00,20.00,0.15)(20100.00,18.00,0.15)(20100.00,16.00,0.15)(20100.00,14.00,0.15)(20100.00,12.00,0.15)(20100.00,10.00,0.15)(20100.00,8.00,0.15)(20100.00,6.00,0.16)(20100.00,4.00,0.16)(20100.00,2.00,0.16)

(21100.00,20.00,0.14)(21100.00,18.00,0.14)(21100.00,16.00,0.14)(21100.00,14.00,0.14)(21100.00,12.00,0.15)(21100.00,10.00,0.15)(21100.00,8.00,0.15)(21100.00,6.00,0.15)(21100.00,4.00,0.15)(21100.00,2.00,0.15)

(22100.00,20.00,0.13)(22100.00,18.00,0.14)(22100.00,16.00,0.14)(22100.00,14.00,0.14)(22100.00,12.00,0.14)(22100.00,10.00,0.14)(22100.00,8.00,0.14)(22100.00,6.00,0.14)(22100.00,4.00,0.14)(22100.00,2.00,0.14)

(23100.00,20.00,0.13)(23100.00,18.00,0.13)(23100.00,16.00,0.13)(23100.00,14.00,0.13)(23100.00,12.00,0.13)(23100.00,10.00,0.13)(23100.00,8.00,0.13)(23100.00,6.00,0.14)(23100.00,4.00,0.14)(23100.00,2.00,0.14)

(24100.00,20.00,0.12)(24100.00,18.00,0.12)(24100.00,16.00,0.12)(24100.00,14.00,0.13)(24100.00,12.00,0.13)(24100.00,10.00,0.13)(24100.00,8.00,0.13)(24100.00,6.00,0.13)(24100.00,4.00,0.13)(24100.00,2.00,0.13)

(25100.00,20.00,0.12)(25100.00,18.00,0.12)(25100.00,16.00,0.12)(25100.00,14.00,0.12)(25100.00,12.00,0.12)(25100.00,10.00,0.12)(25100.00,8.00,0.12)(25100.00,6.00,0.13)(25100.00,4.00,0.13)(25100.00,2.00,0.13)

%(26100.00,20.00,0.11)(26100.00,18.00,0.11)(26100.00,16.00,0.11)(26100.00,14.00,0.12)(26100.00,12.00,0.12)(26100.00,10.00,0.12)(26100.00,8.00,0.12)(26100.00,6.00,0.12)(26100.00,4.00,0.12)(26100.00,2.00,0.12)
%
%(27100.00,20.00,0.11)(27100.00,18.00,0.11)(27100.00,16.00,0.11)(27100.00,14.00,0.11)(27100.00,12.00,0.11)(27100.00,10.00,0.11)(27100.00,8.00,0.12)(27100.00,6.00,0.12)(27100.00,4.00,0.12)(27100.00,2.00,0.12)
%
%(28100.00,20.00,0.10)(28100.00,18.00,0.11)(28100.00,16.00,0.11)(28100.00,14.00,0.11)(28100.00,12.00,0.11)(28100.00,10.00,0.11)(28100.00,8.00,0.11)(28100.00,6.00,0.11)(28100.00,4.00,0.11)(28100.00,2.00,0.11)
%
%(29100.00,20.00,0.10)(29100.00,18.00,0.10)(29100.00,16.00,0.10)(29100.00,14.00,0.10)(29100.00,12.00,0.10)(29100.00,10.00,0.11)(29100.00,8.00,0.11)(29100.00,6.00,0.11)(29100.00,4.00,0.11)(29100.00,2.00,0.11)
%
%(30100.00,20.00,0.10)(30100.00,18.00,0.10)(30100.00,16.00,0.10)(30100.00,14.00,0.10)(30100.00,12.00,0.10)(30100.00,10.00,0.10)(30100.00,8.00,0.10)(30100.00,6.00,0.10)(30100.00,4.00,0.11)(30100.00,2.00,0.11)
%
%(31100.00,20.00,0.09)(31100.00,18.00,0.09)(31100.00,16.00,0.10)(31100.00,14.00,0.10)(31100.00,12.00,0.10)(31100.00,10.00,0.10)(31100.00,8.00,0.10)(31100.00,6.00,0.10)(31100.00,4.00,0.10)(31100.00,2.00,0.10)
%
%(32100.00,20.00,0.09)(32100.00,18.00,0.09)(32100.00,16.00,0.09)(32100.00,14.00,0.09)(32100.00,12.00,0.09)(32100.00,10.00,0.10)(32100.00,8.00,0.10)(32100.00,6.00,0.10)(32100.00,4.00,0.10)(32100.00,2.00,0.10)
%
%(33100.00,20.00,0.09)(33100.00,18.00,0.09)(33100.00,16.00,0.09)(33100.00,14.00,0.09)(33100.00,12.00,0.09)(33100.00,10.00,0.09)(33100.00,8.00,0.09)(33100.00,6.00,0.10)(33100.00,4.00,0.10)(33100.00,2.00,0.10)
%
%(34100.00,20.00,0.08)(34100.00,18.00,0.09)(34100.00,16.00,0.09)(34100.00,14.00,0.09)(34100.00,12.00,0.09)(34100.00,10.00,0.09)(34100.00,8.00,0.09)(34100.00,6.00,0.09)(34100.00,4.00,0.09)(34100.00,2.00,0.09)
%
%(35100.00,20.00,0.08)(35100.00,18.00,0.08)(35100.00,16.00,0.08)(35100.00,14.00,0.09)(35100.00,12.00,0.09)(35100.00,10.00,0.09)(35100.00,8.00,0.09)(35100.00,6.00,0.09)(35100.00,4.00,0.09)(35100.00,2.00,0.09)
%
%(36100.00,20.00,0.08)(36100.00,18.00,0.08)(36100.00,16.00,0.08)(36100.00,14.00,0.08)(36100.00,12.00,0.08)(36100.00,10.00,0.09)(36100.00,8.00,0.09)(36100.00,6.00,0.09)(36100.00,4.00,0.09)(36100.00,2.00,0.09)
%
%(37100.00,20.00,0.08)(37100.00,18.00,0.08)(37100.00,16.00,0.08)(37100.00,14.00,0.08)(37100.00,12.00,0.08)(37100.00,10.00,0.08)(37100.00,8.00,0.08)(37100.00,6.00,0.08)(37100.00,4.00,0.09)(37100.00,2.00,0.09)
%
%(38100.00,20.00,0.07)(38100.00,18.00,0.08)(38100.00,16.00,0.08)(38100.00,14.00,0.08)(38100.00,12.00,0.08)(38100.00,10.00,0.08)(38100.00,8.00,0.08)(38100.00,6.00,0.08)(38100.00,4.00,0.08)(38100.00,2.00,0.08)
%
%(39100.00,20.00,0.07)(39100.00,18.00,0.07)(39100.00,16.00,0.07)(39100.00,14.00,0.08)(39100.00,12.00,0.08)(39100.00,10.00,0.08)(39100.00,8.00,0.08)(39100.00,6.00,0.08)(39100.00,4.00,0.08)(39100.00,2.00,0.08)
%
%(40100.00,20.00,0.07)(40100.00,18.00,0.07)(40100.00,16.00,0.07)(40100.00,14.00,0.07)(40100.00,12.00,0.08)(40100.00,10.00,0.08)(40100.00,8.00,0.08)(40100.00,6.00,0.08)(40100.00,4.00,0.08)(40100.00,2.00,0.08)
%
%(41100.00,20.00,0.07)(41100.00,18.00,0.07)(41100.00,16.00,0.07)(41100.00,14.00,0.07)(41100.00,12.00,0.07)(41100.00,10.00,0.07)(41100.00,8.00,0.08)(41100.00,6.00,0.08)(41100.00,4.00,0.08)(41100.00,2.00,0.08)
%
%(42100.00,20.00,0.07)(42100.00,18.00,0.07)(42100.00,16.00,0.07)(42100.00,14.00,0.07)(42100.00,12.00,0.07)(42100.00,10.00,0.07)(42100.00,8.00,0.07)(42100.00,6.00,0.07)(42100.00,4.00,0.08)(42100.00,2.00,0.08)
%
%(43100.00,20.00,0.06)(43100.00,18.00,0.07)(43100.00,16.00,0.07)(43100.00,14.00,0.07)(43100.00,12.00,0.07)(43100.00,10.00,0.07)(43100.00,8.00,0.07)(43100.00,6.00,0.07)(43100.00,4.00,0.07)(43100.00,2.00,0.08)
%
%(44100.00,20.00,0.06)(44100.00,18.00,0.06)(44100.00,16.00,0.07)(44100.00,14.00,0.07)(44100.00,12.00,0.07)(44100.00,10.00,0.07)(44100.00,8.00,0.07)(44100.00,6.00,0.07)(44100.00,4.00,0.07)(44100.00,2.00,0.07)
%
%(45100.00,20.00,0.06)(45100.00,18.00,0.06)(45100.00,16.00,0.06)(45100.00,14.00,0.07)(45100.00,12.00,0.07)(45100.00,10.00,0.07)(45100.00,8.00,0.07)(45100.00,6.00,0.07)(45100.00,4.00,0.07)(45100.00,2.00,0.07)
%
%(46100.00,20.00,0.06)(46100.00,18.00,0.06)(46100.00,16.00,0.06)(46100.00,14.00,0.06)(46100.00,12.00,0.06)(46100.00,10.00,0.07)(46100.00,8.00,0.07)(46100.00,6.00,0.07)(46100.00,4.00,0.07)(46100.00,2.00,0.07)
%
%(47100.00,20.00,0.06)(47100.00,18.00,0.06)(47100.00,16.00,0.06)(47100.00,14.00,0.06)(47100.00,12.00,0.06)(47100.00,10.00,0.06)(47100.00,8.00,0.07)(47100.00,6.00,0.07)(47100.00,4.00,0.07)(47100.00,2.00,0.07)
%
%(48100.00,20.00,0.06)(48100.00,18.00,0.06)(48100.00,16.00,0.06)(48100.00,14.00,0.06)(48100.00,12.00,0.06)(48100.00,10.00,0.06)(48100.00,8.00,0.06)(48100.00,6.00,0.07)(48100.00,4.00,0.07)(48100.00,2.00,0.07)
%
%(49100.00,20.00,0.06)(49100.00,18.00,0.06)(49100.00,16.00,0.06)(49100.00,14.00,0.06)(49100.00,12.00,0.06)(49100.00,10.00,0.06)(49100.00,8.00,0.06)(49100.00,6.00,0.06)(49100.00,4.00,0.06)(49100.00,2.00,0.07)


}; \end{axis} \end{tikzpicture}

          \label{dlmproof}          
        \end{subfigure}

        \caption{DLM evaluation figures}
\end{figure}

Figures \ref{dlmproofc2} and \ref{dlmproofcc} show DLM scores in terms of the $\mu$ parameter, w.r.t. document frequency and document length respectively. Figure \ref{dlmproof} on the other hand demonstrates the relation between document frequency and document length.

As we can observe from Equation \ref{dlmequation} the parameter $\mu$ is closely related to the collection statistics of the terms, and the length normalization component of the equation. Moreover the lower the values of $\mu$ the higher the score differences for similar document frequencies as shown in Figure \ref{dlmproofc2}. Similarly, we can observe in Figure \ref{dlmproofcc} how $\mu$ interacts with document length. For low values of $\mu$ we can observe how the scores are reduced at the same time that documents become larger, as expected. Interestingly, this behaviour is dampened with higher values of $\mu$, as score differences are heavily reduced w.r.t. the different document lengths. Since the default value for $\mu$ is 2500, it is no surprise that document length has virtually no effect over the scores for DLM  as seen in Figure \ref{dlmproof}, contrary to other retrieval models. 

\begin{table}[]

	\caption{P@30 scores for DLM for a range of $\mu$ values}
	\centering
	\begin{tabular}{l|c} 	
	\textit{\textbf{$\mu$}} & 
	\textit{\textbf{P@30}} 	
	\tabularnewline
	\hline
	1 & 0.4028 \\
	5 &  0.4164 \\
	20 & 0.4241 \\
	50 &  0.4099 \\
	100 &  0.3933 \\
	500 &  0.3396 \\
	1000 & 0.3227 \\
	2500 & 0.2988 \\
	\hline	
	\end{tabular}
	\label{drmmuvalues}
\end{table}

This could be a desired feature for microblog retrieval, however let us look at the performance achieved for a range of $\mu$ values in Table \ref{drmmuvalues}. As we can observe generally the higher the value of $mu$ the worse the performance obtained, with the exception of $\mu$ within the 1 to 20 range. 

In order to further understand the behaviour of DLM in the case of Microblog retrieval, we repeat the same experiment we performed for HLM in the previous Section. We assume that since DLM was also designed for bigger documents than microblogs, overestimating the statistics of TF and DL can be interesting experiment as it would better resemble its standard behaviour in term of the numerical values produced as scores. 

The resulting evaluation metrics for such runs are presented in Table \ref{drmdtfmuvalues}. The first four lines contain the P@30 values for different combinations where $\mu$ is set to 20. As we can observe overestimating TF by +20 results in a substantial +7.47\% increase of P@30 with respect to the default configuration. On the other hand overestimating DL by +20 results in a 8.02\% decrease of performance in terms of P@30. Finally, combining both TF and DL overestimates results in comparable performance than that obtained by only increasing TF.

\begin{table}[]
	\caption{P@30 scores for DLM as we consider different combinations of dTF and dDL, and $\mu$}
	\centering
	\begin{tabular}{l|c|c|c} 	
	\textit{\textbf{$\mu$}} & 
	\textit{\textbf{dTF}} & 
	\textit{\textbf{dDL}} & 
	\textit{\textbf{P@30}} 	
	\tabularnewline
	\hline
	20 &    &    & 0.4241\\
	20 & 20 &    & 0.4558\\
	20 &    & 20 & 0.3901\\
	20 & 20 & 20 & 0.4547\\
	\hline	
	\hline
	2500 &    &    & 0.2988\\
	2500 & 20 &    & 0.4468\\
	2500 &    & 20 & 0.2892\\
	2500 & 20 & 20 & 0.4466\\
    \hline
	\end{tabular}
	\label{drmdtfmuvalues}
\end{table}

The same behaviour is obtained across all combinations when we set the $\mu = 2500$. To further develop our understanding of the behaviour, and to draw conclusions for such results, we devised Figures \ref{dlmfigureTFDL2500} and \ref{dlmfigureTFDL20}. Figures \ref{dlmfigureTFDL2500} and \ref{dlmfigureTFDL20} present the DLM scores produced with respect to Doc. Length and Term Frequency (TF) when $\mu=2500$ and $\mu=20$ respectively.

Let us analyse the results from Table \ref{drmdtfmuvalues} in connection with Figures \ref{dlmfigureTFDL2500} and \ref{dlmfigureTFDL20}. As we can observe incrementing DL will result in an increased differentiation of DLM scores with respect to TF as more values are closer to the minimum and maximum values. In other words there are less intermediate values (Light coloured areas), which ultimately reflects on heightened sensitivity to differences across the TF spectrum. Furthermore, we can also observe in Table \ref{drmdtfmuvalues} how incrementing DL values, results in worse performance in all cases. Consequently the increased differentiation of DLM scores with respect to the TF parameter, produced by the increment of DL is detrimental and in line with the findings in the previous section.

Additionally, Figure \ref{dlmfigureTFDL2500} shows an almost linear progression of DLM scores with respect to TF. Furthermore, Figure \ref{dlmfigureTFDL20} where $\mu=20$ exhibits a logarithmic behaviour with respect to TF. The latter behaviour is more desirable because there should be a saturation point when incrementing TF at which there is very little value added to the score of the document, or could be even counter productive.

The better behaviour with respect to TF is rewarded with increased performance whether the value of $\mu$ is 20 or 2500. In fact the offsetting of TF seems to overrule the effects of $\mu$ as similar results are obtained in both $\mu=20$ and $\mu=2500$ conditions. The effects of offsetting TF are most visually evident when looking at Figure \ref{dlmfigureTFDL20} as differences amongst the different scores become very small. 


\begin{figure}
      	\begin{subfigure}[b]{0.5\textwidth}
          \centering
          \caption{Doc. length (DL) and Term Frequency (TF) when $\mu = 2500$}
          
\begin{tikzpicture}[thick,scale=0.7, every node/.style={transform shape}]\begin{axis}[
 %title={},
 %y dir=reverse, 
 x dir=reverse, 
 xlabel={term frequency (TF)},
 ylabel={docLength (DL)},
 zlabel={DLM score},
 every axis/.append style={font=\large\bfseries},
 max space between ticks=25pt
% yticklabels={0k,100k}
 ] 

\addplot3[surf,unbounded coords=jump] coordinates { 
%patch,patch type=biquadratic, shader=faceted,patch refines=3
(1,1,0.111973219732543)	(2,1,nan)	(3,1,nan)	(4,1,nan)	(5,1,nan)	(6,1,nan)	(7,1,nan)	(8,1,nan)	(9,1,nan)	(10,1,nan)	(11,1,nan)	(12,1,nan)	(13,1,nan)	(14,1,nan)	(15,1,nan)	(16,1,nan)	(17,1,nan)	(18,1,nan)	(19,1,nan)	(20,1,nan)	(21,1,nan)	(22,1,nan)	(23,1,nan)	(24,1,nan)	(25,1,nan)	(26,1,nan)	(27,1,nan)	(28,1,nan)

(1,2,0.108090507727749)	(2,2,0.146709262188192)	(3,2,nan)	(4,2,nan)	(5,2,nan)	(6,2,nan)	(7,2,nan)	(8,2,nan)	(9,2,nan)	(10,2,nan)	(11,2,nan)	(12,2,nan)	(13,2,nan)	(14,2,nan)	(15,2,nan)	(16,2,nan)	(17,2,nan)	(18,2,nan)	(19,2,nan)	(20,2,nan)	(21,2,nan)	(22,2,nan)	(23,2,nan)	(24,2,nan)	(25,2,nan)	(26,2,nan)	(27,2,nan)	(28,2,nan)

(1,3,0.104209347256264)	(2,3,0.142828101716707)	(3,3,0.181293910214315)	(4,3,nan)	(5,3,nan)	(6,3,nan)	(7,3,nan)	(8,3,nan)	(9,3,nan)	(10,3,nan)	(11,3,nan)	(12,3,nan)	(13,3,nan)	(14,3,nan)	(15,3,nan)	(16,3,nan)	(17,3,nan)	(18,3,nan)	(19,3,nan)	(20,3,nan)	(21,3,nan)	(22,3,nan)	(23,3,nan)	(24,3,nan)	(25,3,nan)	(26,3,nan)	(27,3,nan)	(28,3,nan)

(1,4,0.100329737078599)	(2,4,0.138948491539042)	(3,4,0.17741430003665)	(4,4,0.2157283692504)	(5,4,nan)	(6,4,nan)	(7,4,nan)	(8,4,nan)	(9,4,nan)	(10,4,nan)	(11,4,nan)	(12,4,nan)	(13,4,nan)	(14,4,nan)	(15,4,nan)	(16,4,nan)	(17,4,nan)	(18,4,nan)	(19,4,nan)	(20,4,nan)	(21,4,nan)	(22,4,nan)	(23,4,nan)	(24,4,nan)	(25,4,nan)	(26,4,nan)	(27,4,nan)	(28,4,nan)

(1,5,0.0964516759567444)	(2,5,0.135070430417187)	(3,5,0.173536238914795)	(4,5,0.211850308128545)	(5,5,0.250013830513225)	(6,5,nan)	(7,5,nan)	(8,5,nan)	(9,5,nan)	(10,5,nan)	(11,5,nan)	(12,5,nan)	(13,5,nan)	(14,5,nan)	(15,5,nan)	(16,5,nan)	(17,5,nan)	(18,5,nan)	(19,5,nan)	(20,5,nan)	(21,5,nan)	(22,5,nan)	(23,5,nan)	(24,5,nan)	(25,5,nan)	(26,5,nan)	(27,5,nan)	(28,5,nan)

(1,6,0.0925751626541784)	(2,6,0.131193917114621)	(3,6,0.169659725612229)	(4,6,0.207973794825979)	(5,6,0.246137317210659)	(6,6,0.284151471219561)	(7,6,nan)	(8,6,nan)	(9,6,nan)	(10,6,nan)	(11,6,nan)	(12,6,nan)	(13,6,nan)	(14,6,nan)	(15,6,nan)	(16,6,nan)	(17,6,nan)	(18,6,nan)	(19,6,nan)	(20,6,nan)	(21,6,nan)	(22,6,nan)	(23,6,nan)	(24,6,nan)	(25,6,nan)	(26,6,nan)	(27,6,nan)	(28,6,nan)

(1,7,0.0887001959358549)	(2,7,0.127318950396297)	(3,7,0.165784758893906)	(4,7,0.204098828107656)	(5,7,0.242262350492336)	(6,7,0.280276504501237)	(7,7,0.318142454804501)	(8,7,nan)	(9,7,nan)	(10,7,nan)	(11,7,nan)	(12,7,nan)	(13,7,nan)	(14,7,nan)	(15,7,nan)	(16,7,nan)	(17,7,nan)	(18,7,nan)	(19,7,nan)	(20,7,nan)	(21,7,nan)	(22,7,nan)	(23,7,nan)	(24,7,nan)	(25,7,nan)	(26,7,nan)	(27,7,nan)	(28,7,nan)

(1,8,0.0848267745682088)	(2,8,0.123445529028651)	(3,8,0.16191133752626)	(4,8,0.200225406740009)	(5,8,0.23838892912469)	(6,8,0.276403083133591)	(7,8,0.314269033436855)	(8,8,0.35198793113558)	(9,8,nan)	(10,8,nan)	(11,8,nan)	(12,8,nan)	(13,8,nan)	(14,8,nan)	(15,8,nan)	(16,8,nan)	(17,8,nan)	(18,8,nan)	(19,8,nan)	(20,8,nan)	(21,8,nan)	(22,8,nan)	(23,8,nan)	(24,8,nan)	(25,8,nan)	(26,8,nan)	(27,8,nan)	(28,8,nan)

(1,9,0.0809548973191464)	(2,9,0.119573651779589)	(3,9,0.158039460277197)	(4,9,0.196353529490947)	(5,9,0.234517051875627)	(6,9,0.272531205884529)	(7,9,0.310397156187792)	(8,9,0.348116053886518)	(9,9,0.385689036722731)	(10,9,nan)	(11,9,nan)	(12,9,nan)	(13,9,nan)	(14,9,nan)	(15,9,nan)	(16,9,nan)	(17,9,nan)	(18,9,nan)	(19,9,nan)	(20,9,nan)	(21,9,nan)	(22,9,nan)	(23,9,nan)	(24,9,nan)	(25,9,nan)	(26,9,nan)	(27,9,nan)	(28,9,nan)

(1,10,0.0770845629580511)	(2,10,0.115703317418494)	(3,10,0.154169125916102)	(4,10,0.192483195129852)	(5,10,0.230646717514532)	(6,10,0.268660871523434)	(7,10,0.306526821826697)	(8,10,0.344245719525422)	(9,10,0.381818702361636)	(10,10,0.41924689492421)	(11,10,nan)	(12,10,nan)	(13,10,nan)	(14,10,nan)	(15,10,nan)	(16,10,nan)	(17,10,nan)	(18,10,nan)	(19,10,nan)	(20,10,nan)	(21,10,nan)	(22,10,nan)	(23,10,nan)	(24,10,nan)	(25,10,nan)	(26,10,nan)	(27,10,nan)	(28,10,nan)

(1,11,0.0732157702557744)	(2,11,0.111834524716217)	(3,11,0.150300333213825)	(4,11,0.188614402427575)	(5,11,0.226777924812255)	(6,11,0.264792078821157)	(7,11,0.30265802912442)	(8,11,0.340376926823146)	(9,11,0.377949909659359)	(10,11,0.415378102221933)	(11,11,0.452662616148559)	(12,11,nan)	(13,11,nan)	(14,11,nan)	(15,11,nan)	(16,11,nan)	(17,11,nan)	(18,11,nan)	(19,11,nan)	(20,11,nan)	(21,11,nan)	(22,11,nan)	(23,11,nan)	(24,11,nan)	(25,11,nan)	(26,11,nan)	(27,11,nan)	(28,11,nan)

(1,12,0.0693485179846348)	(2,12,0.107967272445077)	(3,12,0.146433080942686)	(4,12,0.184747150156436)	(5,12,0.222910672541116)	(6,12,0.260924826550017)	(7,12,0.298790776853281)	(8,12,0.336509674552006)	(9,12,0.374082657388219)	(10,12,0.411510849950794)	(11,12,0.448795363877419)	(12,12,0.48593729805271)	(13,12,nan)	(14,12,nan)	(15,12,nan)	(16,12,nan)	(17,12,nan)	(18,12,nan)	(19,12,nan)	(20,12,nan)	(21,12,nan)	(22,12,nan)	(23,12,nan)	(24,12,nan)	(25,12,nan)	(26,12,nan)	(27,12,nan)	(28,12,nan)

(1,13,0.0654828049184215)	(2,13,0.104101559378864)	(3,13,0.142567367876472)	(4,13,0.180881437090222)	(5,13,0.219044959474903)	(6,13,0.257059113483804)	(7,13,0.294925063787067)	(8,13,0.332643961485793)	(9,13,0.370216944322006)	(10,13,0.40764513688458)	(11,13,0.444929650811206)	(12,13,0.482071584986497)	(13,13,0.519072025736323)	(14,13,nan)	(15,13,nan)	(16,13,nan)	(17,13,nan)	(18,13,nan)	(19,13,nan)	(20,13,nan)	(21,13,nan)	(22,13,nan)	(23,13,nan)	(24,13,nan)	(25,13,nan)	(26,13,nan)	(27,13,nan)	(28,13,nan)

(1,14,0.0616186298323818)	(2,14,0.100237384292824)	(3,14,0.138703192790433)	(4,14,0.177017262004182)	(5,14,0.215180784388863)	(6,14,0.253194938397764)	(7,14,0.291060888701028)	(8,14,0.328779786399753)	(9,14,0.366352769235966)	(10,14,0.403780961798541)	(11,14,0.441065475725166)	(12,14,0.478207409900457)	(13,14,0.515207850650284)	(14,14,0.552067871932422)	(15,14,nan)	(16,14,nan)	(17,14,nan)	(18,14,nan)	(19,14,nan)	(20,14,nan)	(21,14,nan)	(22,14,nan)	(23,14,nan)	(24,14,nan)	(25,14,nan)	(26,14,nan)	(27,14,nan)	(28,14,nan)

(1,15,0.0577559915032278)	(2,15,0.0963747459636703)	(3,15,0.134840554461279)	(4,15,0.173154623675029)	(5,15,0.211318146059709)	(6,15,0.24933230006861)	(7,15,0.287198250371874)	(8,15,0.324917148070599)	(9,15,0.362490130906812)	(10,15,0.399918323469387)	(11,15,0.437202837396012)	(12,15,0.474344771571303)	(13,15,0.51134521232113)	(14,15,0.548205233603268)	(15,15,0.584925897194439)	(16,15,nan)	(17,15,nan)	(18,15,nan)	(19,15,nan)	(20,15,nan)	(21,15,nan)	(22,15,nan)	(23,15,nan)	(24,15,nan)	(25,15,nan)	(26,15,nan)	(27,15,nan)	(28,15,nan)

(1,16,0.053894888709128)	(2,16,0.0925136431695705)	(3,16,0.130979451667179)	(4,16,0.169293520880929)	(5,16,0.207457043265609)	(6,16,0.24547119727451)	(7,16,0.283337147577774)	(8,16,0.321056045276499)	(9,16,0.358629028112712)	(10,16,0.396057220675287)	(11,16,0.433341734601912)	(12,16,0.470483668777203)	(13,16,0.50748410952703)	(14,16,0.544344130809168)	(15,16,0.581064794400339)	(16,16,0.61764715007973)	(17,16,nan)	(18,16,nan)	(19,16,nan)	(20,16,nan)	(21,16,nan)	(22,16,nan)	(23,16,nan)	(24,16,nan)	(25,16,nan)	(26,16,nan)	(27,16,nan)	(28,16,nan)

(1,17,0.0500353202297106)	(2,17,0.0886540746901531)	(3,17,0.127119883187761)	(4,17,0.165433952401511)	(5,17,0.203597474786192)	(6,17,0.241611628795093)	(7,17,0.279477579098356)	(8,17,0.317196476797082)	(9,17,0.354769459633295)	(10,17,0.392197652195869)	(11,17,0.429482166122495)	(12,17,0.466624100297786)	(13,17,0.503624541047612)	(14,17,0.540484562329751)	(15,17,0.577205225920921)	(16,17,0.613787581600312)	(17,17,0.650232667329652)	(18,17,nan)	(19,17,nan)	(20,17,nan)	(21,17,nan)	(22,17,nan)	(23,17,nan)	(24,17,nan)	(25,17,nan)	(26,17,nan)	(27,17,nan)	(28,17,nan)

(1,18,0.0461772848460546)	(2,18,0.0847960393064971)	(3,18,0.123261847804105)	(4,18,0.161575917017855)	(5,18,0.199739439402536)	(6,18,0.237753593411437)	(7,18,0.2756195437147)	(8,18,0.313338441413426)	(9,18,0.350911424249639)	(10,18,0.388339616812213)	(11,18,0.425624130738839)	(12,18,0.46276606491413)	(13,18,0.499766505663956)	(14,18,0.536626526946095)	(15,18,0.573347190537265)	(16,18,0.609929546216656)	(17,18,0.646374631945996)	(18,18,0.682683474046268)	(19,18,nan)	(20,18,nan)	(21,18,nan)	(22,18,nan)	(23,18,nan)	(24,18,nan)	(25,18,nan)	(26,18,nan)	(27,18,nan)	(28,18,nan)

(1,19,0.0423207813406933)	(2,19,0.0809395358011359)	(3,19,0.119405344298744)	(4,19,0.157719413512494)	(5,19,0.195882935897174)	(6,19,0.233897089906076)	(7,19,0.271763040209339)	(8,19,0.309481937908065)	(9,19,0.347054920744278)	(10,19,0.384483113306852)	(11,19,0.421767627233478)	(12,19,0.458909561408768)	(13,19,0.495910002158595)	(14,19,0.532770023440733)	(15,19,0.569490687031904)	(16,19,0.606073042711295)	(17,19,0.642518128440635)	(18,19,0.678826970540907)	(19,19,0.715000583865767)	(20,19,nan)	(21,19,nan)	(22,19,nan)	(23,19,nan)	(24,19,nan)	(25,19,nan)	(26,19,nan)	(27,19,nan)	(28,19,nan)

(1,20,0.0384658084976089)	(2,20,0.0770845629580515)	(3,20,0.11555037145566)	(4,20,0.15386444066941)	(5,20,0.19202796305409)	(6,20,0.230042117062991)	(7,20,0.267908067366255)	(8,20,0.30562696506498)	(9,20,0.343199947901193)	(10,20,0.380628140463768)	(11,20,0.417912654390393)	(12,20,0.455054588565684)	(13,20,0.492055029315511)	(14,20,0.528915050597649)	(15,20,0.56563571418882)	(16,20,0.602218069868211)	(17,20,0.638663155597551)	(18,20,0.674971997697823)	(19,20,0.711145611022683)	(20,20,0.747184999128663)	(21,20,nan)	(22,20,nan)	(23,20,nan)	(24,20,nan)	(25,20,nan)	(26,20,nan)	(27,20,nan)	(28,20,nan)

(1,21,0.0346123651022313)	(2,21,0.0732311195626738)	(3,21,0.111696928060282)	(4,21,0.150010997274032)	(5,21,0.188174519658712)	(6,21,0.226188673667614)	(7,21,0.264054623970877)	(8,21,0.301773521669603)	(9,21,0.339346504505816)	(10,21,0.37677469706839)	(11,21,0.414059210995016)	(12,21,0.451201145170306)	(13,21,0.488201585920133)	(14,21,0.525061607202271)	(15,21,0.561782270793442)	(16,21,0.598364626472833)	(17,21,0.634809712202173)	(18,21,0.671118554302445)	(19,21,0.707292167627305)	(20,21,0.743331555733285)	(21,21,0.779237711046848)	(22,21,nan)	(23,21,nan)	(24,21,nan)	(25,21,nan)	(26,21,nan)	(27,21,nan)	(28,21,nan)

(1,22,0.0307604499414344)	(2,22,0.0693792044018769)	(3,22,0.107845012899485)	(4,22,0.146159082113235)	(5,22,0.184322604497915)	(6,22,0.222336758506817)	(7,22,0.26020270881008)	(8,22,0.297921606508806)	(9,22,0.335494589345019)	(10,22,0.372922781907593)	(11,22,0.410207295834219)	(12,22,0.447349230009509)	(13,22,0.484349670759336)	(14,22,0.521209692041474)	(15,22,0.557930355632645)	(16,22,0.594512711312036)	(17,22,0.630957797041376)	(18,22,0.667266639141648)	(19,22,0.703440252466508)	(20,22,0.739479640572488)	(21,22,0.775385795886051)	(22,22,0.811159699867572)	(23,22,nan)	(24,22,nan)	(25,22,nan)	(26,22,nan)	(27,22,nan)	(28,22,nan)

(1,23,0.0269100618035358)	(2,23,0.0655288162639784)	(3,23,0.103994624761587)	(4,23,0.142308693975337)	(5,23,0.180472216360017)	(6,23,0.218486370368918)	(7,23,0.256352320672182)	(8,23,0.294071218370907)	(9,23,0.33164420120712)	(10,23,0.369072393769695)	(11,23,0.40635690769632)	(12,23,0.443498841871611)	(13,23,0.480499282621438)	(14,23,0.517359303903576)	(15,23,0.554079967494747)	(16,23,0.590662323174138)	(17,23,0.627107408903478)	(18,23,0.66341625100375)	(19,23,0.69958986432861)	(20,23,0.73562925243459)	(21,23,0.771535407748152)	(22,23,0.807309311729674)	(23,23,0.842951935034412)	(24,23,nan)	(25,23,nan)	(26,23,nan)	(27,23,nan)	(28,23,nan)

(1,24,0.0230611994782949)	(2,24,0.0616799539387374)	(3,24,0.100145762436346)	(4,24,0.138459831650096)	(5,24,0.176623354034776)	(6,24,0.214637508043677)	(7,24,0.252503458346941)	(8,24,0.290222356045666)	(9,24,0.327795338881879)	(10,24,0.365223531444454)	(11,24,0.402508045371079)	(12,24,0.43964997954637)	(13,24,0.476650420296197)	(14,24,0.513510441578335)	(15,24,0.550231105169506)	(16,24,0.586813460848896)	(17,24,0.623258546578237)	(18,24,0.659567388678509)	(19,24,0.695741002003369)	(20,24,0.731780390109349)	(21,24,0.767686545422911)	(22,24,0.803460449404433)	(23,24,0.839103072709171)	(24,24,0.874615375345294)	(25,24,nan)	(26,24,nan)	(27,24,nan)	(28,24,nan)

(1,25,0.0192138617569053)	(2,25,0.0578326162173479)	(3,25,0.0962984247149561)	(4,25,0.134612493928706)	(5,25,0.172776016313386)	(6,25,0.210790170322288)	(7,25,0.248656120625551)	(8,25,0.286375018324276)	(9,25,0.32394800116049)	(10,25,0.361376193723064)	(11,25,0.39866070764969)	(12,25,0.43580264182498)	(13,25,0.472803082574807)	(14,25,0.509663103856945)	(15,25,0.546383767448116)	(16,25,0.582966123127507)	(17,25,0.619411208856847)	(18,25,0.655720050957119)	(19,25,0.691893664281979)	(20,25,0.727933052387959)	(21,25,0.763839207701522)	(22,25,0.799613111683043)	(23,25,0.835255734987781)	(24,25,0.870768037623905)	(25,25,0.906150969107645)	(26,25,nan)	(27,25,nan)	(28,25,nan)

(1,26,0.0153680474320015)	(2,26,0.053986801892444)	(3,26,0.0924526103900523)	(4,26,0.130766679603802)	(5,26,0.168930201988483)	(6,26,0.206944355997384)	(7,26,0.244810306300647)	(8,26,0.282529203999373)	(9,26,0.320102186835586)	(10,26,0.35753037939816)	(11,26,0.394814893324786)	(12,26,0.431956827500076)	(13,26,0.468957268249903)	(14,26,0.505817289532042)	(15,26,0.542537953123212)	(16,26,0.579120308802603)	(17,26,0.615565394531943)	(18,26,0.651874236632215)	(19,26,0.688047849957076)	(20,26,0.724087238063055)	(21,26,0.759993393376618)	(22,26,0.795767297358139)	(23,26,0.831409920662877)	(24,26,0.866922223299001)	(25,26,0.902305154782741)	(26,26,0.937559654290723)	(27,26,nan)	(28,26,nan)

(1,27,0.0115237552976474)	(2,27,0.0501425097580899)	(3,27,0.0886083182556982)	(4,27,0.126922387469448)	(5,27,0.165085909854128)	(6,27,0.20310006386303)	(7,27,0.240966014166293)	(8,27,0.278684911865019)	(9,27,0.316257894701232)	(10,27,0.353686087263806)	(11,27,0.390970601190432)	(12,27,0.428112535365722)	(13,27,0.465112976115549)	(14,27,0.501972997397687)	(15,27,0.538693660988858)	(16,27,0.575276016668249)	(17,27,0.611721102397589)	(18,27,0.648029944497861)	(19,27,0.684203557822721)	(20,27,0.720242945928701)	(21,27,0.756149101242264)	(22,27,0.791923005223785)	(23,27,0.827565628528523)	(24,27,0.863077931164647)	(25,27,0.898460862648387)	(26,27,0.933715362156369)	(27,27,0.96884235867519)	(28,27,nan)

(1,28,0.00768098414934081)	(2,28,0.0462997386097834)	(3,28,0.0847655471073916)	(4,28,0.123079616321142)	(5,28,0.161243138705822)	(6,28,0.199257292714723)	(7,28,0.237123243017987)	(8,28,0.274842140716712)	(9,28,0.312415123552925)	(10,28,0.3498433161155)	(11,28,0.387127830042125)	(12,28,0.424269764217416)	(13,28,0.461270204967243)	(14,28,0.498130226249381)	(15,28,0.534850889840552)	(16,28,0.571433245519943)	(17,28,0.607878331249283)	(18,28,0.644187173349555)	(19,28,0.680360786674415)	(20,28,0.716400174780395)	(21,28,0.752306330093957)	(22,28,0.788080234075479)	(23,28,0.823722857380217)	(24,28,0.85923516001634)	(25,28,0.894618091500081)	(26,28,0.929872591008062)	(27,28,0.964999587526883)	(28,28,1)

(1,29,0.00383973278400676)	(2,29,0.0424584872444493)	(3,29,0.0809242957420575)	(4,29,0.119238364955807)	(5,29,0.157401887340488)	(6,29,0.195416041349389)	(7,29,0.233281991652653)	(8,29,0.271000889351378)	(9,29,0.308573872187591)	(10,29,0.346002064750166)	(11,29,0.383286578676791)	(12,29,0.420428512852082)	(13,29,0.457428953601909)	(14,29,0.494288974884047)	(15,29,0.531009638475218)	(16,29,0.567591994154608)	(17,29,0.604037079883948)	(18,29,0.640345921984221)	(19,29,0.676519535309081)	(20,29,0.712558923415061)	(21,29,0.748465078728623)	(22,29,0.784238982710144)	(23,29,0.819881606014882)	(24,29,0.855393908651006)	(25,29,0.890776840134747)	(26,29,0.926031339642728)	(27,29,0.961158336161549)	(28,29,0.996158748634666)

(1,30,0)	(2,30,0.0386187544604425)	(3,30,0.0770845629580511)	(4,30,0.115398632171801)	(5,30,0.153562154556481)	(6,30,0.191576308565382)	(7,30,0.229442258868646)	(8,30,0.267161156567371)	(9,30,0.304734139403584)	(10,30,0.342162331966159)	(11,30,0.379446845892784)	(12,30,0.416588780068075)	(13,30,0.453589220817902)	(14,30,0.49044924210004)	(15,30,0.527169905691211)	(16,30,0.563752261370602)	(17,30,0.600197347099942)	(18,30,0.636506189200214)	(19,30,0.672679802525074)	(20,30,0.708719190631054)	(21,30,0.744625345944617)	(22,30,0.780399249926138)	(23,30,0.816041873230876)	(24,30,0.851554175867)	(25,30,0.88693710735074)	(26,30,0.922191606858721)	(27,30,0.957318603377542)	(28,30,0.992319015850659)

}; \end{axis} \end{tikzpicture}

          \label{dlmfigureTFDL2500}          
        \end{subfigure} 
        ~
 		\begin{subfigure}[b]{0.5\textwidth}
          \centering
          \caption{Doc. length (DL) and Term Frequency (TF) when $\mu = 20$}
          
\begin{tikzpicture}[thick,scale=0.7, every node/.style={transform shape}]\begin{axis}[
 %title={},
 %y dir=reverse, 
 x dir=reverse, 
 ylabel={docLength (DL)},
 xlabel={term frequency (TF)},
 zlabel={DLM score},
 every axis/.append style={font=\large\bfseries},
 max space between ticks=25pt
% yticklabels={0k,100k}
 ] 

		\addplot3[surf,unbounded coords=jump] coordinates { 
%patch,patch type=biquadratic, shader=faceted,patch refines=3
(1,1,0.370187737472834)	(2,1,nan)	(3,1,nan)	(4,1,nan)	(5,1,nan)	(6,1,nan)	(7,1,nan)	(8,1,nan)	(9,1,nan)	(10,1,nan)	(11,1,nan)	(12,1,nan)	(13,1,nan)	(14,1,nan)	(15,1,nan)	(16,1,nan)	(17,1,nan)	(18,1,nan)	(19,1,nan)	(20,1,nan)	(21,1,nan)	(22,1,nan)	(23,1,nan)	(24,1,nan)	(25,1,nan)	(26,1,nan)	(27,1,nan)	(28,1,nan)

(1,2,0.35033629301713)	(2,2,0.473098605201195)	(3,2,nan)	(4,2,nan)	(5,2,nan)	(6,2,nan)	(7,2,nan)	(8,2,nan)	(9,2,nan)	(10,2,nan)	(11,2,nan)	(12,2,nan)	(13,2,nan)	(14,2,nan)	(15,2,nan)	(16,2,nan)	(17,2,nan)	(18,2,nan)	(19,2,nan)	(20,2,nan)	(21,2,nan)	(22,2,nan)	(23,2,nan)	(24,2,nan)	(25,2,nan)	(26,2,nan)	(27,2,nan)	(28,2,nan)

(1,3,0.331367432332329)	(2,3,0.454129744516394)	(3,3,0.549351591588173)	(4,3,nan)	(5,3,nan)	(6,3,nan)	(7,3,nan)	(8,3,nan)	(9,3,nan)	(10,3,nan)	(11,3,nan)	(12,3,nan)	(13,3,nan)	(14,3,nan)	(15,3,nan)	(16,3,nan)	(17,3,nan)	(18,3,nan)	(19,3,nan)	(20,3,nan)	(21,3,nan)	(22,3,nan)	(23,3,nan)	(24,3,nan)	(25,3,nan)	(26,3,nan)	(27,3,nan)	(28,3,nan)

(1,4,0.313206006327623)	(2,4,0.435968318511688)	(3,4,0.531190165583467)	(4,4,0.608992079627922)	(5,4,nan)	(6,4,nan)	(7,4,nan)	(8,4,nan)	(9,4,nan)	(10,4,nan)	(11,4,nan)	(12,4,nan)	(13,4,nan)	(14,4,nan)	(15,4,nan)	(16,4,nan)	(17,4,nan)	(18,4,nan)	(19,4,nan)	(20,4,nan)	(21,4,nan)	(22,4,nan)	(23,4,nan)	(24,4,nan)	(25,4,nan)	(26,4,nan)	(27,4,nan)	(28,4,nan)

(1,5,0.295786073300299)	(2,5,0.418548385484364)	(3,5,0.513770232556143)	(4,5,0.591572146600598)	(5,5,0.657352727639452)	(6,5,nan)	(7,5,nan)	(8,5,nan)	(9,5,nan)	(10,5,nan)	(11,5,nan)	(12,5,nan)	(13,5,nan)	(14,5,nan)	(15,5,nan)	(16,5,nan)	(17,5,nan)	(18,5,nan)	(19,5,nan)	(20,5,nan)	(21,5,nan)	(22,5,nan)	(23,5,nan)	(24,5,nan)	(25,5,nan)	(26,5,nan)	(27,5,nan)	(28,5,nan)

(1,6,0.279049453631335)	(2,6,0.4018117658154)	(3,6,0.497033612887179)	(4,6,0.574835526931634)	(5,6,0.640616107970488)	(6,6,0.697597839115699)	(7,6,nan)	(8,6,nan)	(9,6,nan)	(10,6,nan)	(11,6,nan)	(12,6,nan)	(13,6,nan)	(14,6,nan)	(15,6,nan)	(16,6,nan)	(17,6,nan)	(18,6,nan)	(19,6,nan)	(20,6,nan)	(21,6,nan)	(22,6,nan)	(23,6,nan)	(24,6,nan)	(25,6,nan)	(26,6,nan)	(27,6,nan)	(28,6,nan)

(1,7,0.262944557395454)	(2,7,0.385706869579518)	(3,7,0.480928716651297)	(4,7,0.558730630695753)	(5,7,0.624511211734606)	(6,7,0.681492942879817)	(7,7,0.731754391811987)	(8,7,nan)	(9,7,nan)	(10,7,nan)	(11,7,nan)	(12,7,nan)	(13,7,nan)	(14,7,nan)	(15,7,nan)	(16,7,nan)	(17,7,nan)	(18,7,nan)	(19,7,nan)	(20,7,nan)	(21,7,nan)	(22,7,nan)	(23,7,nan)	(24,7,nan)	(25,7,nan)	(26,7,nan)	(27,7,nan)	(28,7,nan)

(1,8,0.247425425288769)	(2,8,0.370187737472834)	(3,8,0.465409584544613)	(4,8,0.543211498589068)	(5,8,0.608992079627922)	(6,8,0.665973810773133)	(7,8,0.716235259705302)	(8,8,0.761195657844912)	(9,8,nan)	(10,8,nan)	(11,8,nan)	(12,8,nan)	(13,8,nan)	(14,8,nan)	(15,8,nan)	(16,8,nan)	(17,8,nan)	(18,8,nan)	(19,8,nan)	(20,8,nan)	(21,8,nan)	(22,8,nan)	(23,8,nan)	(24,8,nan)	(25,8,nan)	(26,8,nan)	(27,8,nan)	(28,8,nan)

(1,9,0.232450937928789)	(2,9,0.355213250112854)	(3,9,0.450435097184633)	(4,9,0.528237011229088)	(5,9,0.594017592267942)	(6,9,0.650999323413153)	(7,9,0.701260772345323)	(8,9,0.746221170484932)	(9,9,0.78689279783988)	(10,9,nan)	(11,9,nan)	(12,9,nan)	(13,9,nan)	(14,9,nan)	(15,9,nan)	(16,9,nan)	(17,9,nan)	(18,9,nan)	(19,9,nan)	(20,9,nan)	(21,9,nan)	(22,9,nan)	(23,9,nan)	(24,9,nan)	(25,9,nan)	(26,9,nan)	(27,9,nan)	(28,9,nan)

(1,10,0.217984159255844)	(2,10,0.340746471439909)	(3,10,0.435968318511688)	(4,10,0.513770232556143)	(5,10,0.579550813594996)	(6,10,0.636532544740208)	(7,10,0.686793993672377)	(8,10,0.731754391811987)	(9,10,0.772426019166934)	(10,10,0.809556305856442)	(11,10,nan)	(12,10,nan)	(13,10,nan)	(14,10,nan)	(15,10,nan)	(16,10,nan)	(17,10,nan)	(18,10,nan)	(19,10,nan)	(20,10,nan)	(21,10,nan)	(22,10,nan)	(23,10,nan)	(24,10,nan)	(25,10,nan)	(26,10,nan)	(27,10,nan)	(28,10,nan)

(1,11,0.20399178763689)	(2,11,0.326754099820955)	(3,11,0.421975946892734)	(4,11,0.499777860937189)	(5,11,0.565558441976042)	(6,11,0.622540173121254)	(7,11,0.672801622053423)	(8,11,0.717762020193033)	(9,11,0.75843364754798)	(10,11,0.795563934237488)	(11,11,0.829720486933776)	(12,11,nan)	(13,11,nan)	(14,11,nan)	(15,11,nan)	(16,11,nan)	(17,11,nan)	(18,11,nan)	(19,11,nan)	(20,11,nan)	(21,11,nan)	(22,11,nan)	(23,11,nan)	(24,11,nan)	(25,11,nan)	(26,11,nan)	(27,11,nan)	(28,11,nan)

(1,12,0.190443694143558)	(2,12,0.313206006327623)	(3,12,0.408427853399402)	(4,12,0.486229767443857)	(5,12,0.55201034848271)	(6,12,0.608992079627922)	(7,12,0.659253528560091)	(8,12,0.704213926699701)	(9,12,0.744885554054648)	(10,12,0.782015840744156)	(11,12,0.816172393440444)	(12,12,0.84779642178301)	(13,12,nan)	(14,12,nan)	(15,12,nan)	(16,12,nan)	(17,12,nan)	(18,12,nan)	(19,12,nan)	(20,12,nan)	(21,12,nan)	(22,12,nan)	(23,12,nan)	(24,12,nan)	(25,12,nan)	(26,12,nan)	(27,12,nan)	(28,12,nan)

(1,13,0.177312531900896)	(2,13,0.300074844084961)	(3,13,0.39529669115674)	(4,13,0.473098605201195)	(5,13,0.538879186240049)	(6,13,0.59586091738526)	(7,13,0.646122366317429)	(8,13,0.691082764457039)	(9,13,0.731754391811987)	(10,13,0.768884678501494)	(11,13,0.803041231197782)	(12,13,0.834665259540348)	(13,13,0.864106525573273)	(14,13,nan)	(15,13,nan)	(16,13,nan)	(17,13,nan)	(18,13,nan)	(19,13,nan)	(20,13,nan)	(21,13,nan)	(22,13,nan)	(23,13,nan)	(24,13,nan)	(25,13,nan)	(26,13,nan)	(27,13,nan)	(28,13,nan)

(1,14,0.164573403770433)	(2,14,0.287335715954497)	(3,14,0.382557563026276)	(4,14,0.460359477070732)	(5,14,0.526140058109585)	(6,14,0.583121789254797)	(7,14,0.633383238186966)	(8,14,0.678343636326575)	(9,14,0.719015263681523)	(10,14,0.756145550371031)	(11,14,0.790302103067318)	(12,14,0.821926131409884)	(13,14,0.85136739744281)	(14,14,0.878907862555096)	(15,14,nan)	(16,14,nan)	(17,14,nan)	(18,14,nan)	(19,14,nan)	(20,14,nan)	(21,14,nan)	(22,14,nan)	(23,14,nan)	(24,14,nan)	(25,14,nan)	(26,14,nan)	(27,14,nan)	(28,14,nan)

(1,15,0.15220357821699)	(2,15,0.274965890401055)	(3,15,0.370187737472834)	(4,15,0.447989651517289)	(5,15,0.513770232556143)	(6,15,0.570751963701354)	(7,15,0.621013412633524)	(8,15,0.665973810773133)	(9,15,0.706645438128081)	(10,15,0.743775724817588)	(11,15,0.777932277513876)	(12,15,0.809556305856442)	(13,15,0.838997571889367)	(14,15,0.866538037001653)	(15,15,0.892408327374779)	(16,15,nan)	(17,15,nan)	(18,15,nan)	(19,15,nan)	(20,15,nan)	(21,15,nan)	(22,15,nan)	(23,15,nan)	(24,15,nan)	(25,15,nan)	(26,15,nan)	(27,15,nan)	(28,15,nan)

(1,16,0.140182245211389)	(2,16,0.262944557395454)	(3,16,0.358166404467232)	(4,16,0.435968318511688)	(5,16,0.501748899550541)	(6,16,0.558730630695753)	(7,16,0.608992079627922)	(8,16,0.653952477767531)	(9,16,0.694624105122479)	(10,16,0.731754391811987)	(11,16,0.765910944508274)	(12,16,0.79753497285084)	(13,16,0.826976238883766)	(14,16,0.854516703996052)	(15,16,0.880386994369177)	(16,16,0.904778152928221)	(17,16,nan)	(18,16,nan)	(19,16,nan)	(20,16,nan)	(21,16,nan)	(22,16,nan)	(23,16,nan)	(24,16,nan)	(25,16,nan)	(26,16,nan)	(27,16,nan)	(28,16,nan)

(1,17,0.128490305584639)	(2,17,0.251252617768704)	(3,17,0.346474464840483)	(4,17,0.424276378884938)	(5,17,0.490056959923792)	(6,17,0.547038691069003)	(7,17,0.597300140001172)	(8,17,0.642260538140782)	(9,17,0.68293216549573)	(10,17,0.720062452185237)	(11,17,0.754219004881525)	(12,17,0.785843033224091)	(13,17,0.815284299257016)	(14,17,0.842824764369302)	(15,17,0.868695054742427)	(16,17,0.893086213301471)	(17,17,0.916158270033887)	(18,17,nan)	(19,17,nan)	(20,17,nan)	(21,17,nan)	(22,17,nan)	(23,17,nan)	(24,17,nan)	(25,17,nan)	(26,17,nan)	(27,17,nan)	(28,17,nan)

(1,18,0.117110188478973)	(2,18,0.239872500663038)	(3,18,0.335094347734817)	(4,18,0.412896261779272)	(5,18,0.478676842818125)	(6,18,0.535658573963336)	(7,18,0.585920022895506)	(8,18,0.630880421035115)	(9,18,0.671552048390063)	(10,18,0.708682335079571)	(11,18,0.742838887775858)	(12,18,0.774462916118424)	(13,18,0.80390418215135)	(14,18,0.831444647263636)	(15,18,0.857314937636761)	(16,18,0.881706096195805)	(17,18,0.904778152928221)	(18,18,0.926666494335415)	(19,18,nan)	(20,18,nan)	(21,18,nan)	(22,18,nan)	(23,18,nan)	(24,18,nan)	(25,18,nan)	(26,18,nan)	(27,18,nan)	(28,18,nan)

(1,19,0.106025692515101)	(2,19,0.228788004699166)	(3,19,0.324009851770945)	(4,19,0.4018117658154)	(5,19,0.467592346854253)	(6,19,0.524574077999465)	(7,19,0.574835526931634)	(8,19,0.619795925071244)	(9,19,0.660467552426191)	(10,19,0.697597839115699)	(11,19,0.731754391811987)	(12,19,0.763378420154553)	(13,19,0.792819686187478)	(14,19,0.820360151299764)	(15,19,0.846230441672889)	(16,19,0.870621600231933)	(17,19,0.893693656964349)	(18,19,0.915581998371543)	(19,19,0.936402181270787)	(20,19,nan)	(21,19,nan)	(22,19,nan)	(23,19,nan)	(24,19,nan)	(25,19,nan)	(26,19,nan)	(27,19,nan)	(28,19,nan)

(1,20,0.0952218470717789)	(2,20,0.217984159255844)	(3,20,0.313206006327623)	(4,20,0.391007920372078)	(5,20,0.456788501410932)	(6,20,0.513770232556143)	(7,20,0.564031681488312)	(8,20,0.608992079627922)	(9,20,0.649663706982869)	(10,20,0.686793993672377)	(11,20,0.720950546368665)	(12,20,0.752574574711231)	(13,20,0.782015840744156)	(14,20,0.809556305856442)	(15,20,0.835426596229567)	(16,20,0.859817754788611)	(17,20,0.882889811521027)	(18,20,0.904778152928221)	(19,20,0.925598335827465)	(20,20,0.945449780283169)	(21,20,nan)	(22,20,nan)	(23,20,nan)	(24,20,nan)	(25,20,nan)	(26,20,nan)	(27,20,nan)	(28,20,nan)

(1,21,0.0846847906969246)	(2,21,0.207447102880989)	(3,21,0.302668949952768)	(4,21,0.380470863997224)	(5,21,0.446251445036077)	(6,21,0.503233176181288)	(7,21,0.553494625113458)	(8,21,0.598455023253067)	(9,21,0.639126650608015)	(10,21,0.676256937297522)	(11,21,0.71041348999381)	(12,21,0.742037518336376)	(13,21,0.771478784369301)	(14,21,0.799019249481587)	(15,21,0.824889539854713)	(16,21,0.849280698413757)	(17,21,0.872352755146173)	(18,21,0.894241096553367)	(19,21,0.91506127945261)	(20,21,0.934912723908314)	(21,21,0.953881584593115)	(22,21,nan)	(23,21,nan)	(24,21,nan)	(25,21,nan)	(26,21,nan)	(27,21,nan)	(28,21,nan)

(1,22,0.0744016641725351)	(2,22,0.1971639763566)	(3,22,0.292385823428379)	(4,22,0.370187737472834)	(5,22,0.435968318511688)	(6,22,0.492950049656899)	(7,22,0.543211498589068)	(8,22,0.588171896728678)	(9,22,0.628843524083626)	(10,22,0.665973810773133)	(11,22,0.700130363469421)	(12,22,0.731754391811987)	(13,22,0.761195657844912)	(14,22,0.788736122957198)	(15,22,0.814606413330323)	(16,22,0.838997571889367)	(17,22,0.862069628621783)	(18,22,0.883957970028977)	(19,22,0.904778152928221)	(20,22,0.924629597383925)	(21,22,0.943598458068726)	(22,22,0.961759884073432)	(23,22,nan)	(24,22,nan)	(25,22,nan)	(26,22,nan)	(27,22,nan)	(28,22,nan)

(1,23,0.0643605161639091)	(2,23,0.187122828347974)	(3,23,0.282344675419753)	(4,23,0.360146589464208)	(5,23,0.425927170503062)	(6,23,0.482908901648273)	(7,23,0.533170350580442)	(8,23,0.578130748720052)	(9,23,0.618802376075)	(10,23,0.655932662764507)	(11,23,0.690089215460795)	(12,23,0.721713243803361)	(13,23,0.751154509836286)	(14,23,0.778694974948572)	(15,23,0.804565265321697)	(16,23,0.828956423880741)	(17,23,0.852028480613157)	(18,23,0.873916822020351)	(19,23,0.894737004919595)	(20,23,0.914588449375299)	(21,23,0.9335573100601)	(22,23,0.951718736064806)	(23,23,0.96913866909213)	(24,23,nan)	(25,23,nan)	(26,23,nan)	(27,23,nan)	(28,23,nan)

(1,24,0.0545502197168313)	(2,24,0.177312531900896)	(3,24,0.272534378972675)	(4,24,0.35033629301713)	(5,24,0.416116874055984)	(6,24,0.473098605201195)	(7,24,0.523360054133364)	(8,24,0.568320452272974)	(9,24,0.608992079627922)	(10,24,0.646122366317429)	(11,24,0.680278919013717)	(12,24,0.711902947356283)	(13,24,0.741344213389208)	(14,24,0.768884678501494)	(15,24,0.79475496887462)	(16,24,0.819146127433663)	(17,24,0.842218184166079)	(18,24,0.864106525573273)	(19,24,0.884926708472517)	(20,24,0.904778152928221)	(21,24,0.923747013613022)	(22,24,0.941908439617728)	(23,24,0.959328372645052)	(24,24,0.976064992314016)	(25,24,nan)	(26,24,nan)	(27,24,nan)	(28,24,nan)

(1,25,0.0449603981396097)	(2,25,0.167722710323675)	(3,25,0.262944557395454)	(4,25,0.340746471439909)	(5,25,0.406527052478762)	(6,25,0.463508783623974)	(7,25,0.513770232556143)	(8,25,0.558730630695752)	(9,25,0.5994022580507)	(10,25,0.636532544740208)	(11,25,0.670689097436495)	(12,25,0.702313125779061)	(13,25,0.731754391811987)	(14,25,0.759294856924273)	(15,25,0.785165147297398)	(16,25,0.809556305856442)	(17,25,0.832628362588858)	(18,25,0.854516703996052)	(19,25,0.875336886895295)	(20,25,0.895188331350999)	(21,25,0.9141571920358)	(22,25,0.932318618040507)	(23,25,0.949738551067831)	(24,25,0.966475170736794)	(25,25,0.982580066972676)	(26,25,nan)	(27,25,nan)	(28,25,nan)

(1,26,0.0355813590320303)	(2,26,0.158343671216095)	(3,26,0.253565518287874)	(4,26,0.331367432332329)	(5,26,0.397148013371183)	(6,26,0.454129744516394)	(7,26,0.504391193448564)	(8,26,0.549351591588173)	(9,26,0.590023218943121)	(10,26,0.627153505632628)	(11,26,0.661310058328916)	(12,26,0.692934086671482)	(13,26,0.722375352704407)	(14,26,0.749915817816693)	(15,26,0.775786108189819)	(16,26,0.800177266748862)	(17,26,0.823249323481278)	(18,26,0.845137664888472)	(19,26,0.865957847787716)	(20,26,0.88580929224342)	(21,26,0.904778152928221)	(22,26,0.922939578932927)	(23,26,0.940359511960251)	(24,26,0.957096131629215)	(25,26,0.973201027865096)	(26,26,0.988720159971781)	(27,26,nan)	(28,26,nan)

(1,27,0.0264040354097076)	(2,27,0.149166347593773)	(3,27,0.244388194665551)	(4,27,0.322190108710007)	(5,27,0.38797068974886)	(6,27,0.444952420894072)	(7,27,0.495213869826241)	(8,27,0.54017426796585)	(9,27,0.580845895320798)	(10,27,0.617976182010306)	(11,27,0.652132734706593)	(12,27,0.683756763049159)	(13,27,0.713198029082085)	(14,27,0.740738494194371)	(15,27,0.766608784567496)	(16,27,0.79099994312654)	(17,27,0.814071999858956)	(18,27,0.83596034126615)	(19,27,0.856780524165393)	(20,27,0.876631968621097)	(21,27,0.895600829305898)	(22,27,0.913762255310605)	(23,27,0.931182188337929)	(24,27,0.947918808006892)	(25,27,0.964023704242774)	(26,27,0.979542836349458)	(27,27,0.994517323709438)	(28,27,nan)

(1,28,0.0174199330273238)	(2,28,0.140182245211389)	(3,28,0.235404092283168)	(4,28,0.313206006327623)	(5,28,0.378986587366476)	(6,28,0.435968318511688)	(7,28,0.486229767443857)	(8,28,0.531190165583467)	(9,28,0.571861792938414)	(10,28,0.608992079627922)	(11,28,0.64314863232421)	(12,28,0.674772660666775)	(13,28,0.704213926699701)	(14,28,0.731754391811987)	(15,28,0.757624682185112)	(16,28,0.782015840744156)	(17,28,0.805087897476572)	(18,28,0.826976238883766)	(19,28,0.84779642178301)	(20,28,0.867647866238714)	(21,28,0.886616726923514)	(22,28,0.904778152928221)	(23,28,0.922198085955545)	(24,28,0.938934705624509)	(25,28,0.95503960186039)	(26,28,0.970558733967074)	(27,28,0.985533221327054)	(28,28,1)

(1,29,0.00862108313368152)	(2,29,0.131383395317746)	(3,29,0.226605242389525)	(4,29,0.304407156433981)	(5,29,0.370187737472834)	(6,29,0.427169468618045)	(7,29,0.477430917550215)	(8,29,0.522391315689824)	(9,29,0.563062943044772)	(10,29,0.600193229734279)	(11,29,0.634349782430567)	(12,29,0.665973810773133)	(13,29,0.695415076806059)	(14,29,0.722955541918344)	(15,29,0.74882583229147)	(16,29,0.773216990850514)	(17,29,0.79628904758293)	(18,29,0.818177388990124)	(19,29,0.838997571889367)	(20,29,0.858849016345071)	(21,29,0.877817877029872)	(22,29,0.895979303034579)	(23,29,0.913399236061903)	(24,29,0.930135855730866)	(25,29,0.946240751966748)	(26,29,0.961759884073432)	(27,29,0.976734371433412)	(28,29,0.991201150106358)

(1,30,0)	(2,30,0.122762312184065)	(3,30,0.217984159255844)	(4,30,0.295786073300299)	(5,30,0.361566654339153)	(6,30,0.418548385484364)	(7,30,0.468809834416533)	(8,30,0.513770232556143)	(9,30,0.554441859911091)	(10,30,0.591572146600598)	(11,30,0.625728699296886)	(12,30,0.657352727639452)	(13,30,0.686793993672377)	(14,30,0.714334458784663)	(15,30,0.740204749157788)	(16,30,0.764595907716832)	(17,30,0.787667964449248)	(18,30,0.809556305856442)	(19,30,0.830376488755686)	(20,30,0.85022793321139)	(21,30,0.869196793896191)	(22,30,0.887358219900897)	(23,30,0.904778152928221)	(24,30,0.921514772597185)	(25,30,0.937619668833066)	(26,30,0.953138800939751)	(27,30,0.968113288299731)	(28,30,0.982580066972676)


}; \end{axis} \end{tikzpicture}

          \label{dlmfigureTFDL20}          
        \end{subfigure} 
        \caption{Evaluating DLM's behaviour}
\end{figure}

Extending on the findings by \cite{naveed2011searching} who showed how length normalization was detrimental to retrieval in an L2R retrieval framework. By experimenting with state of the art retrieval models so far we have found a particular relationship between TF and DL that is most appropriate for Microblog retrieval. We believe that the score progressions with respect to TF and DL should resemble a very soft slope due to the very low TF and DL values in order to not over or under represent the score of a term or document.

\subsection{The DFRee Case}
DFRee is a Divergence From Randomness model implemented in the Terrier IR platform. DFRee has been designed to be parameter-free as all components are computed online. DFRee adheres to the implementation described as follows:

\begin{equation}
prior = \frac{f(q_i, D)}{|D|}, posterior = \frac{f(q_i, D)+1}{|D|+1} 
\end{equation}

\begin{equation}
InvPriorColl = \frac{ntoks}{f(q_i, C)}, norm = f(q_i, D)*log_2{\frac{posterior}{prior}}
\end{equation}

%\begin{equation}
\begin{multline}
DFRee(q_i,D,C) = norm * [                    \\
f(q_i, D)*(-log_2(prior*InvPriorColl))       \\
+(f(q_i, D)+1)*log_2(posterior*InvPriorColl) \\
+ 0.5*log_2(posterior/prior)],
\end{multline}

where \(f(q_i, D)\) represents the frequency of query term \(q_i\) within document \(D\). Similarly \(f(q_i, C)\) gives the collection \(C\) frequency of the query term \(q_i\). Furthermore \(ntoks\) is the total number of unique terms within the collection \(C\) and \(|D|\) gives the document length of document \(D\).

\begin{figure}
	\centering
	\caption{Evaluating DFR's behaviour: Doc. length (DL) and Term Frequency (TF)}
	%!TEX root = ./JournalChapter1.tex
\begin{tikzpicture}[thick,scale=1.0, every node/.style={transform shape}]\begin{axis}[
 %title={},
% y dir=reverse, 
 x dir=reverse, 
 ylabel={docLength (DL)},
 xlabel={term frequency (TF)},
 zlabel={DFR score},
 every axis/.append style={font=\large\bfseries},
 max space between ticks=25pt
% yticklabels={0k,100k}
 ] 


\addplot3[surf,unbounded coords=jump]
coordinates  { 
(1,15,0.602828690204566)	(2,15,0.830458458727289)	(3,15,0.934450716638443)	(4,15,0.966903402267895)	(5,15,0.953929912325151)	(6,15,0.910204250164967)	(7,15,0.844773148573639)	(8,15,0.763616521737829)	(9,15,0.670903438148194)	(10,15,0.569664557459257)	(11,15,0.462178197338477)	(12,15,0.350204856834806)	(13,15,0.235136442615999)	(14,15,0.11809496699829)	(15,15,0)

(1,16,0.578099188538087)	(2,16,0.811155500645937)	(3,16,0.92455241464552)	(4,16,0.967767908505993)	(5,16,0.966062841257201)	(6,16,0.933720545287019)	(7,16,0.87957150961847)	(8,16,0.809460625419414)	(9,16,0.72746535973258)	(10,16,0.636550382771426)	(11,16,0.538944280996693)	(12,16,0.436368745300519)	(13,16,0.330184550451496)	(14,16,0.221488171413441)	(15,16,0.111177842268355)

(1,17,0.554089938092192)	(2,17,0.791526910417312)	(3,17,0.913056948252712)	(4,17,0.965714873324416)	(5,17,0.973972028301543)	(6,17,0.951752192118292)	(7,17,0.907687908109167)	(8,17,0.847501039032133)	(9,17,0.775185911781543)	(10,17,0.693647497457941)	(11,17,0.60506951435267)	(12,17,0.51113872004729)	(13,17,0.413187919650668)	(14,17,0.312290682010694)	(15,17,0.209326130035835)

(1,18,0.530802801771364)	(2,18,0.771767857536177)	(3,18,0.900372831623677)	(4,18,0.961364141168104)	(5,18,0.978478269819011)	(6,18,0.965306243347949)	(7,18,0.930298335362267)	(8,18,0.879063614353843)	(9,18,0.815520463917971)	(10,18,0.742519493293841)	(11,18,0.662203670865781)	(12,18,0.576228097258246)	(13,18,0.485900278395978)	(14,18,0.392273096789813)	(15,18,0.296208439454882)

(1,19,0.508226789640423)	(2,19,0.752018863072771)	(3,19,0.886810317359786)	(4,19,0.955195710616974)	(5,19,0.980224486964259)	(6,19,0.975178213356986)	(7,19,0.948338450288501)	(8,19,0.905210288059222)	(9,19,0.849642347853426)	(10,19,0.784435449978092)	(11,19,0.711695334773675)	(12,19,0.633048243551952)	(13,19,0.549778660548279)	(14,19,0.462920673152469)	(15,19,0.373320528862043)

(1,20,0.48634303905334)	(2,20,0.7323812008066)	(3,20,0.872607218288516)	(4,20,0.947585137098897)	(5,20,0.97971952494445)	(6,20,0.98200290254851)	(7,20,0.962559939768004)	(8,20,0.926799408296809)	(9,20,0.878505521275719)	(10,20,0.820433262364011)	(11,20,0.754654182131245)	(12,20,0.682768072499512)	(13,20,0.606038355509857)	(14,20,0.525481948969255)	(15,20,0.441930831844985)

(1,21,0.465128061580168)	(2,21,0.712927637929877)	(3,21,0.857947247761062)	(4,21,0.93882895517277)	(5,21,0.97736988612122)	(6,21,0.986291533238511)	(7,21,0.973572063624968)	(8,21,0.944530636289626)	(9,21,0.902891730766613)	(10,21,0.851367936716025)	(11,21,0.791999256676888)	(12,21,0.726361125537109)	(13,21,0.655697602410148)	(14,21,0.581009843803962)	(15,21,0.503116746944407)

(1,22,0.444555871588363)	(2,22,0.69371003859671)	(3,22,0.842973247656932)	(4,22,0.929163209859617)	(5,22,0.973503053852954)	(6,22,0.988459253367945)	(7,22,0.981872662054963)	(8,22,0.958978737875236)	(9,22,0.923446336219237)	(10,22,0.877948672683593)	(11,22,0.824496515075059)	(12,22,0.764642770856045)	(13,22,0.699613620363273)	(14,22,0.630395687626556)	(15,22,0.557795822851653)

(1,23,0.424599383032406)	(2,23,0.674764818583908)	(3,23,0.827796862032113)	(4,23,0.918777149915091)	(5,23,0.968384861977651)	(6,23,0.988845761863186)	(7,23,0.987871568720419)	(8,23,0.970619283566948)	(9,23,0.940705773382284)	(10,23,0.900767545813065)	(11,23,0.852788172693824)	(12,23,0.798299644283731)	(13,23,0.738511566834336)	(14,23,0.674397111108444)	(15,23,0.606751988315705)

(1,24,0.40523132186617)	(2,24,0.656116904788573)	(3,24,0.812505700538763)	(4,24,0.907823472234563)	(5,24,0.962232586378857)	(6,24,0.987730954630336)	(7,24,0.991908481950635)	(8,24,0.979848390893796)	(9,24,0.955118795248774)	(10,24,0.920321862204126)	(11,24,0.87741577498763)	(12,24,0.827913032165606)	(13,24,0.773007819631001)	(14,24,0.713660808324183)	(15,24,0.650657372038589)

(1,25,0.38642481493876)	(2,25,0.637782639637063)	(3,25,0.79716870357892)	(4,25,0.896426073240635)	(5,25,0.955224921679465)	(6,25,0.985346919947008)	(7,25,0.994266744020595)	(8,25,0.986998033599149)	(9,25,0.96706304333111)	(10,25,0.937031713492213)	(11,25,0.898838449553016)	(12,25,0.853977527193232)	(13,25,0.803628737708457)	(14,25,0.748741099129492)	(15,25,0.690090359300742)

(1,26,0.36815376139952)	(2,26,0.619771931399019)	(3,26,0.781840202366091)	(4,26,0.884685975052332)	(5,26,0.947509661773085)	(6,26,0.981887225781465)	(7,26,0.995184065634633)	(8,26,0.992348018723311)	(9,26,0.976858081971228)	(10,26,0.95125386482836)	(11,26,0.917447437797707)	(12,26,0.876915988605277)	(13,26,0.830825830481115)	(14,26,0.78011508458774)	(15,26,0.725550507465349)

(1,27,0.350393057325973)	(2,27,0.60208985927249)	(3,27,0.766563020474647)	(4,27,0.872685899583419)	(5,27,0.939209668327629)	(6,27,0.977514176161289)	(7,27,0.994860946200371)	(8,27,0.996135435044116)	(9,27,0.984775730704953)	(10,27,0.963292820489072)	(11,27,0.933577738074063)	(12,27,0.897091603060034)	(13,27,0.854988073305192)	(14,27,0.808195051131356)	(15,27,0.757470870504045)

(1,28,0.333118721214438)	(2,28,0.584737880102515)	(3,28,0.751370863683325)	(4,28,0.860493830168551)	(5,28,0.930427549263709)	(6,28,0.97236452885122)	(7,28,0.993467339157226)	(8,28,0.998562165131648)	(9,28,0.99104831536007)	(10,28,0.973409701227053)	(11,28,0.947517492491807)	(12,28,0.914817660899949)	(13,28,0.876451950536151)	(14,28,0.833338655534655)	(15,28,0.786228199121581)

(1,29,0.316307952698114)	(2,29,0.567714741253384)	(3,29,0.736290176025056)	(4,29,0.848165807395434)	(5,29,0.921249355632251)	(6,29,0.966554036195814)	(7,29,0.991147968519457)	(8,29,0.999800901797173)	(9,29,0.995875303197284)	(10,29,0.981829412799183)	(11,29,0.959515600327343)	(12,29,0.930365523362214)	(13,29,0.89550968334361)	(14,29,0.855857318363783)	(15,29,0.81215140336571)

(1,30,0.299939146627551)	(2,30,0.55101717484961)	(3,30,0.721341591745516)	(4,30,0.835748140323265)	(5,30,0.911747524748403)	(6,30,0.960181077894388)	(7,30,0.988026599941357)	(8,30,1)	(9,30,0.999428673828312)	(10,30,0.98874647076145)	(11,30,0.969787928978429)	(12,30,0.943971149900242)	(13,30,0.912416001900636)	(14,30,0.876023168107538)	(15,30,0.835528594560206)

};

\addplot3 [data cs=cart,surf,domain=-10:10,samples=2, opacity=0.3,color=purple] coordinates  { 
(0,15,0) (0,15,1)

(15,15,0) (15,15,1)

};


\addplot3[surf,unbounded coords=jump]
coordinates  { 
%patch,patch type=biquadratic, shader=faceted,patch refines=3
(1,1,0)	(2,1,nan)	(3,1,nan)	(4,1,nan)	(5,1,nan)	(6,1,nan)	(7,1,nan)	(8,1,nan)	(9,1,nan)	(10,1,nan)	(11,1,nan)	(12,1,nan)	(13,1,nan)	(14,1,nan)	(15,1,nan)

(1,2,0.550946811011921)	(2,2,0)	(3,2,nan)	(4,2,nan)	(5,2,nan)	(6,2,nan)	(7,2,nan)	(8,2,nan)	(9,2,nan)	(10,2,nan)	(11,2,nan)	(12,2,nan)	(13,2,nan)	(14,2,nan)	(15,2,nan)

(1,3,0.739499035509737)	(2,3,0.447402531994322)	(3,3,0)	(4,3,nan)	(5,3,nan)	(6,3,nan)	(7,3,nan)	(8,3,nan)	(9,3,nan)	(10,3,nan)	(11,3,nan)	(12,3,nan)	(13,3,nan)	(14,3,nan)	(15,3,nan)

(1,4,0.808372355369359)	(2,4,0.674021740218476)	(3,4,0.367084989657738)	(4,4,0)	(5,4,nan)	(6,4,nan)	(7,4,nan)	(8,4,nan)	(9,4,nan)	(10,4,nan)	(11,4,nan)	(12,4,nan)	(13,4,nan)	(14,4,nan)	(15,4,nan)

(1,5,0.827712304044469)	(2,5,0.795475244633043)	(3,5,0.589020399660793)	(4,5,0.309402318963368)	(5,5,0)	(6,5,nan)	(7,5,nan)	(8,5,nan)	(9,5,nan)	(10,5,nan)	(11,5,nan)	(12,5,nan)	(13,5,nan)	(14,5,nan)	(15,5,nan)

(1,6,0.823758030119902)	(2,6,0.861291472386904)	(3,6,0.728208149339069)	(4,6,0.516703283500178)	(5,6,0.266826414784725)	(6,6,0)	(7,6,nan)	(8,6,nan)	(9,6,nan)	(10,6,nan)	(11,6,nan)	(12,6,nan)	(13,6,nan)	(14,6,nan)	(15,6,nan)

(1,7,0.808049149749894)	(2,7,0.895286730204683)	(3,7,0.81731608309985)	(4,7,0.659114482699154)	(5,7,0.458088606324659)	(6,7,0.234329891272407)	(7,7,0)	(8,7,nan)	(9,7,nan)	(10,7,nan)	(11,7,nan)	(12,7,nan)	(13,7,nan)	(14,7,nan)	(15,7,nan)

(1,8,0.786232834510155)	(2,8,0.910031065751261)	(3,8,0.874547067067584)	(4,8,0.758685635473427)	(5,8,0.597677806310571)	(6,8,0.410536626837436)	(7,8,0.208787788218013)	(8,8,0)	(9,8,nan)	(10,8,nan)	(11,8,nan)	(12,8,nan)	(13,8,nan)	(14,8,nan)	(15,8,nan)

(1,9,0.761283776662586)	(2,9,0.912649127410629)	(3,9,0.9106548363097)	(4,9,0.828974496924382)	(5,9,0.700999123980372)	(6,9,0.54484240060699)	(7,9,0.371505245812937)	(8,9,0.188214411477962)	(9,9,0)	(10,9,nan)	(11,9,nan)	(12,9,nan)	(13,9,nan)	(14,9,nan)	(15,9,nan)

(1,10,0.734851414275236)	(2,10,0.907408307407578)	(3,10,0.932312475346719)	(4,10,0.878648510130617)	(5,10,0.778218625288615)	(6,10,0.648369486864792)	(7,10,0.499658763815525)	(8,10,0.339026534140879)	(9,10,0.171302777785929)	(10,10,0)	(11,10,nan)	(12,10,nan)	(13,10,nan)	(14,10,nan)	(15,10,nan)

(1,11,0.707882534208275)	(2,11,0.896978134982265)	(3,11,0.943832686982997)	(4,11,0.91344706582647)	(5,11,0.83622890765175)	(6,11,0.728859016655641)	(7,11,0.601513634159584)	(8,11,0.460887857152159)	(9,11,0.311641587291567)	(10,11,0.157162472776661)	(11,11,0)	(12,11,nan)	(13,11,nan)	(14,11,nan)	(15,11,nan)

(1,12,0.680931777320597)	(2,12,0.883087512447978)	(3,12,0.948098319824685)	(4,12,0.937288690029642)	(5,12,0.879825806210446)	(6,12,0.791804036576653)	(7,12,0.683066380350784)	(8,12,0.560095278476284)	(9,12,0.427403757686762)	(10,12,0.288272090676837)	(11,12,0.145167857760471)	(12,12,0)	(13,12,nan)	(14,12,nan)	(15,12,nan)

(1,13,0.654326072191879)	(2,13,0.866887863527846)	(3,13,0.947091556647468)	(4,13,0.95292023017884)	(5,13,0.912425568445965)	(6,13,0.841179268774445)	(7,13,0.748729504809226)	(8,13,0.641371703296107)	(9,13,0.523490353484945)	(10,13,0.398272129069081)	(11,13,0.268113442610775)	(12,13,0.134867288166317)	(13,13,0)	(14,13,nan)	(15,13,nan)

(1,14,0.628256015834611)	(2,14,0.849163506028517)	(3,14,0.942208095803169)	(4,14,0.962311537733073)	(5,14,0.936513154948686)	(6,14,0.879914128436669)	(7,14,0.80179902209063)	(8,14,0.708296983893924)	(9,14,0.603678740961905)	(10,14,0.491048437946633)	(11,14,0.372739929982426)	(12,14,0.25055715796921)	(13,14,0.125926946814904)	(14,14,0)	(15,14,nan)

(1,15,0.602828690204566)	(2,15,0.830458458727289)	(3,15,0.934450716638443)	(4,15,0.966903402267895)	(5,15,0.953929912325151)	(6,15,0.910204250164967)	(7,15,0.844773148573639)	(8,15,0.763616521737829)	(9,15,0.670903438148194)	(10,15,0.569664557459257)	(11,15,0.462178197338477)	(12,15,0.350204856834806)	(13,15,0.235136442615999)	(14,15,0.11809496699829)	(15,15,0)

};


 \end{axis} \end{tikzpicture}

	\label{dfrTFDLcomp}
\end{figure} 

Similarly to previous sections, we simulated the scores produced by DFRee given a range of TF and DL values with the objective of studying its behaviour in microbloging conditions. The simulated values are shown in Figure \ref{dfrTFDLcomp}.

As we traverse the Document Length axis we can observe an interesting behaviour. For low values of TF, incrementing DL values from 1 to approximately 16 results in a growing score. This behaviour aligns with the scope hypotheses as longer documents are regarded as more informative. However, for values of DL higher than 16 the scores experience a slow decline. The latter behaviour is in line with the verbose hypotheses which assumes the extra length is due to superfluous information, when the extended length is not accompanied by higher query term frequencies.

When dealing with documents as short as microblogs it is very difficult assert their informativeness or relevance in terms of the verbose or scope hypotheses. In fact all retrieval models studied so far in this work follow them to some degree and perform worse than a simply using IDF as a retrieval model. The premises in which they are built seem to fail to effectively measure the informativeness of microblog documents. However DFRee is an interesting exception as it seems to capture the relevance of microblog documents better than any of the previously studied retrieval models, and it consistently outperforms them in almost all cases (Table \ref{traditional}).

We believe that the \textit{saturation point} behaviour observed in Figure \ref{dfrTFDLcomp} both in terms of TF and DL is responsible for DFRee outperforming other retrieval models in this task. The behaviour can be summarised as a dependency of TF and DL. In other words, the score of a document can only be higher if both TF and DL increase. Thus, incrementing the value of a single component will increase the score to a saturation point after which the score will decrease. This saturation point is given by the relationship between TF and DL. As an example, consider an average microblog document of length 15 (blue square in Figure \ref{dfrTFDLcomp}), the score is maximised when TF approaches 3, after which higher TF values result in a significant reduction to the score. As a side note, this is a very effective way of getting rid of uninformative documents such as spam in microblogs. Furthermore, consider the values of DL for TF=2 in Figure \ref{dfrTFDLcomp}. The scores grow as the values for DL approach $\simeq 9$, after which it follows a slow descent. This behaviour does not completely match the premise which assumes users will attempt to encode their messages within the character limit, therefore longer should be more informative. However the score differences in the descent are very small and may not adversely affect the results significantly.

The opposite behaviour can be observed in previous retrieval models HLM and DLM. These retrieval models consider that the longer the document the less relevant they are under microblog conditions. This behaviour goes against a basic idea of microblog writing as users try to encode as much information as possible into the character limit. 
Moreover, the behaviour of HLM and DLM exhibits a positive correlation between TF and the score produced, thus the higher the TF the higher the score. However it is important to note under this premise that a document containing only of query terms would be valued over others with richer, and more informative content. This behaviour is obviously problematic as it promotes uninformative documents, particularly spam.

Summarising, we believe that DFRee's behaviour is key to understand how to achieve the best possible results for a retrieval model under microblog conditions. Particularly important is the \textit{saturation point} behaviour as a function of TF and DL. This mechanism ensures no over/under estimations of the value of microblog documents when TF is disproportionately high with respect to DL and vice-versa.

%DFRee behaves in a different way which better models the relevance of microblog documents. As previously stated DFRee promotes longer documents up to a maximum point, which is conditioned by the values of TF and viceversa. This can be easily observed in Figure \ref{dfrTFDLcomp} for DL values under 4, as the increment of TF over 1 only decreases the retrieval model score. On the contrary given a TF of 2 or 3, a document is increasingly interesting up until a DL value of 10.

%Finally, if we consider the scores produced around the $DL=15$ area, we can see that the differences are quite subtle compared to other areas of the Figure \ref{dfrTFDLcomp}, and particularly along the DL axis. This is a good property since the value of documents should not be greatly affected when the differences in length are so small.

%If an important query term appears in a document twice, and another document once together with another term, the second document should be more important. However, this may not be true if the score differences are so great.

%
%In this section we have evidenced some of the problems faced by popular retrieval models when dealing with microblog documents. In the following section we will propose a different theory to understand the relevance on microblog documents.
\section{Towards a microblog retrieval model}
In this section firstly we further extend our experimentation on the above-mentioned retrieval models. Secondly we gather all our findings to produce a retrieval model specifically tailored to microblog retrieval.

\subsection{Score differences and Harmonisation}
So far we have introduced a set of representative retrieval models, and discussed how they behave when facing microblog-like conditions. We have mainly done so by simulating the scores produced by each model, when fixing all parameters except TF and DL which are the variables to be considered. In the above-mentioned experiments, we have observed a very interesting relationship between TF, DL and the score given to the terms. In most retrieval models performance seems to increase when we overestimate the values of TF and DL, thus forcing the models into an area of values where the score differences with respect to TF and DL are much lower.


Table \ref{stdevharmonising} holds a summary of the results for all retrieval models in their various configurations with respect to Precision@30. Additionally the third column holds the standard deviation of the simulated scores produced by the retrieval models. As it can be easily observed that the possible document scores are much closer together for those configurations that improve a retrieval model's performance. In fact there seems to be a strong negative statistical correlation between the standard deviation and the performance achieved by the retrieval models.

In other words, reducing the standard deviation of the possible scores for most of the retrieval models is connected with significantly better performance. These observations motivate the following hypothesis:\\

\textbf{\textquotedblleft The range of scores produced by retrieval models when ranking microblogs can be unfairly different due to the retrieval model's behaviour with respect to scarce TF and DL values\textquotedblright}.\\



\begin{table}[]
	\caption{Behaviour when harmonising score differences.}
	\centering
	\begin{tabular}{l|c|c|c} 	
		\textit{\textbf{Model}} & 
		\textit{\textbf{configuration}} & 
		\textit{\textbf{stdev}} & 
		\textit{\textbf{P@30}} 	
		\tabularnewline
		\hline
		DLM & \(c=2500\) & 0.2639 & 0.2988 \\
		DLM & \(c=50\) & 0.2479 & 0.4099 \\
		DLM & \(c=20\) & 0.2384 & 0.4241 \\
		\hline	
		HLM & \(c=0.15\) & 0.2553 & 0.3475\\
		HLM & \(c=0.40\) & 0.2365 & 0.4009\\
		HLM & \(c=0.99\) & 0.1135 & 0.4492\\
		\hline
		BM25 & \(b=0.75, k=1.2\) & 0.1274 & 0.3948\\
		BM25 & \(b=0.75, k=0.7\) & 0.0927 & 0.4399\\
		BM25 & \(b=0.9, k=0.1\) & 0.0181 & 0.4580\\
		\hline
		DFRee & NA & 0.2268 & 0.4614\\
		\hline
		\hline    
		PEARSON & \multicolumn{2}{|c}{-0.70 or -0.58}    \\
		KTau & \multicolumn{2}{|c}{-0.66 or -0.5555}    \\
	\end{tabular}
	\label{stdevharmonising}
\end{table}

If this hypothesis is true, we should be able to achieve similar results using a different technique to reduce the standard deviation of the scores produced by the different retrieval models. To this end we produced the results in Table \ref{loggedRMS}. This Table holds performance metrics for all retrieval models with their standard configurations, however each of the scores computed for each document have been normalised using a logarithm base 2.

As an example, the formulation for HLM would look as follows: 

\begin{small}
	\begin{align}
	\label{hlmformulalog}
	\text{HLM}(D,Q) &=  \sum_{i=1}^{n} log_2 \left[ \log_2 \left[ 1 + \frac{c \cdot f(q_i, D) \cdot ntoks }{ (1-c) \cdot f(q_i, C) \cdot |D|} \right] \right]
	\end{align}
\end{small}

As we can observe in Table \ref{loggedRMS} the results for DLM, HLM and BM25 are significantly better than standard (Table \ref{stdevharmonising}), whereas DFRee performs marginally worse than its default form and IDF remains unaffected. 

We can conclude that based on the empirical evidence presented in this work, most retrieval models are not prepared to effectively capture the relevance of microblogs. The verbose and scope hypotheses, which serves as inspiration to most retrieval models, do not hold for microblog documents. Additionally, the main reason points to their over-sensitiveness to low values of term frequency and document length. This sensitiveness often produces a high degree of score differences amongst the ranked documents which ultimately negatively affects performance.

\begin{table}[]
	
	\caption{Retrieval models performance with log-smoothed scores (All collections)} 
	\centering
	\begin{tabular}{l|c|c|c|} 
		
%		\cline{2- 6}
		\multicolumn{1}{c}{}&\multicolumn{3}{|c|}{Precision @ 30} \\ 
		\cline{2- 4}
		
			& Default & $log_2(Ret. Model)$ & \% difference \\
		\hline
					 
		$DLM$ & 0.2988 & 0.3977 & +33.10\% \\
		$HLM$ & 0.3475 & 0.4489 & +29.18\%\\
		$BM25$ & 0.3948 & 0.4336 & +9.83\%\\
		$DFRee$ & 0.4614 & 0.4531 & -1.80\%\\
		$IDF$ & 0.4626 & 0.4626 & 0\%\\
		\hline
	\end{tabular}
	\label{loggedRMS}
\end{table}

%
%\subsection{Conclusions about retrieval models}
%
%RM and us do not understand microblogs. In the following chapters we will try and develop this understanding.
%

\subsection{MBRM: A Microblogs Retrieval Model}

In previous sections, we have presented a number of problems faced by retrieval models when dealing with microblogs. We have shown empirical evidence of their existence by improving the performance of state of the art retrieval models. However we can further investigate these issues by devising a retrieval model, which relies on what we have learnt so far about microblogs. To this end, we would like to introduce a ``MicroBlogs Retrieval Model'', namely MBRM.

MBRM is made up of two main components which deal with document based information attached to an IDF component which represents the collection's information. Similarly to the formulation of BM25, the two main components of MBRM deal with document length and query term frequency. In fact we came up with a formulation to represent the behaviour we observed as being the best for microblog retrieval. The first component deals with the document length and is given by by the following logistic distribution:

\begin{equation}
DLComp(DL)={\frac  {c_1}{1+{a_1\mathrm  e}^{{-b_1DL}}}}
\end{equation}

where \(a_1, b_1\) and \(c_1\) are parameters to control the growth, maximum and starting point of the distribution. Secondly, the following component given by a gaussian distribution deals with the effect of TF over the final score produced by MBRM:

\begin{equation}
TFComp\left(TF\right)=a_2e^{-{\frac {(TF-b_2)^{2}}{2c_2^{2}}}}
\end{equation}

where \(a_2, b_2\) and \(c_2\) are parameters similar parameters to those found in the previous function. These functions were chosen as they offer good control over the curves, and their values can be bound between 1 and 0 so we do not need to normalise values when combining them. The final formulation for MBRM is given by: 

\begin{equation}
MBRM(D,Q) = \sum_{i=1}^{|Q|} (1-\alpha)*\text{IDF}(q_i) + \alpha * DLComp(|D|) * TFComp(q_i)
\end{equation}

which can be also expressed as:

\begin{equation}
MBRM(D,Q) = \sum_{i=1}^{|Q|} (1-\alpha)*\text{IDF}(q_i) + \alpha * \left({\frac  {c_1}{1+{a_1\mathrm e}^{{-b_1DL(|D|)}}}} \right) * \left(a_2e^{-{\frac {(TF(q_i)-b_2)^{2}}{2c_2^{2}}}}\right) 
\end{equation}

\begin{table}[]
	\caption{MBRM recommended parameter settings} 
	\centering
	\begin{tabular}{l|c} 	
		\hline
		\textbf{Parameter} & \textbf{Recommended values} \\
		\hline
		\centering					 
		$a_1$ & 1.5 \\
		$b_1$ & 0.3 \\
		$c_1$ & 1.0 \\
		\hline
		$a_2$ & 1.0 \\
		$b_2$ & 2.0 \\
		$c_2$ & 6.0 \\
		\hline
	\end{tabular}
	\label{recommended settings}
\end{table}

Figure \ref{microblogRM} shows a simulation of the behaviour of MBRM in terms of TF and DL. The parameters used to for both components (DLComp and TFComp) are shown in Table \ref{recommended settings}. In Figure \ref{microblogRM} we can observe how the values obtained on the TF axis decrease slowly for the initial values of TF, but rapidly accelerate in their descent to then settle near 0. This behaviour is similar to that of DFRee in which the highest importance is given to TF when near the average ~2 and then it is reduced as it increases. High TF values are most likely than not associated with spam or unimportant documents, since actual users struggling to fit their content in the 140 characters limit are unlikely to repeat words. Although this is not always the case, thus the slow descent for low values of TF.


\begin{figure}
	\centering
	\caption{MBRM: A Microblog Retrieval Model}
	%!TEX root = ./JournalChapter1.tex
\begin{tikzpicture}[thick,scale=0.70, every node/.style={transform shape}]\begin{axis}[
 %title={},
% y dir=reverse, 
 x dir=reverse, 
 ylabel={docLength (DL)},
 xlabel={term frequency (TF)},
 zlabel={MBRM score},
 every axis/.append style={font=\large\bfseries},
  max space between ticks=25pt,
  view={20}{45}
% yticklabels={0k,100k}
 ] 

\addplot3[surf,unbounded coords=jump]
coordinates  { 
%patch,patch type=biquadratic, shader=faceted,patch refines=3
(1,1,0.473658084609043)	(2,1,nan)	(3,1,nan)	(4,1,nan)	(5,1,nan)	(6,1,nan)	(7,1,nan)	(8,1,nan)	(9,1,nan)	(10,1,nan)	(11,1,nan)	(12,1,nan)	(13,1,nan)	(14,1,nan)	(15,1,nan)	(16,1,nan)	(17,1,nan)	(18,1,nan)	(19,1,nan)	(20,1,nan)

(1,2,0.548480827396313)	(2,2,0.540915700429286)	(3,2,nan)	(4,2,nan)	(5,2,nan)	(6,2,nan)	(7,2,nan)	(8,2,nan)	(9,2,nan)	(10,2,nan)	(11,2,nan)	(12,2,nan)	(13,2,nan)	(14,2,nan)	(15,2,nan)	(16,2,nan)	(17,2,nan)	(18,2,nan)	(19,2,nan)	(20,2,nan)

(1,3,0.621174011140144)	(2,3,0.612606236245987)	(3,3,0.587605459635394)	(4,3,nan)	(5,3,nan)	(6,3,nan)	(7,3,nan)	(8,3,nan)	(9,3,nan)	(10,3,nan)	(11,3,nan)	(12,3,nan)	(13,3,nan)	(14,3,nan)	(15,3,nan)	(16,3,nan)	(17,3,nan)	(18,3,nan)	(19,3,nan)	(20,3,nan)

(1,4,0.688804056409613)	(2,4,0.679303468819562)	(3,4,0.651580743731372)	(4,4,0.607867497379652)	(5,4,nan)	(6,4,nan)	(7,4,nan)	(8,4,nan)	(9,4,nan)	(10,4,nan)	(11,4,nan)	(12,4,nan)	(13,4,nan)	(14,4,nan)	(15,4,nan)	(16,4,nan)	(17,4,nan)	(18,4,nan)	(19,4,nan)	(20,4,nan)

(1,5,0.749234519496894)	(2,5,0.738900422141148)	(3,5,0.70874551463543)	(4,5,0.661197198359976)	(5,5,0.599940192995648)	(6,5,nan)	(7,5,nan)	(8,5,nan)	(9,5,nan)	(10,5,nan)	(11,5,nan)	(12,5,nan)	(13,5,nan)	(14,5,nan)	(15,5,nan)	(16,5,nan)	(17,5,nan)	(18,5,nan)	(19,5,nan)	(20,5,nan)

(1,6,0.801315033266165)	(2,6,0.790262595943928)	(3,6,0.758011571622054)	(4,6,0.707158094310837)	(5,6,0.641643014567497)	(6,6,0.56624802053189)	(7,6,nan)	(8,6,nan)	(9,6,nan)	(10,6,nan)	(11,6,nan)	(12,6,nan)	(13,6,nan)	(14,6,nan)	(15,6,nan)	(16,6,nan)	(17,6,nan)	(18,6,nan)	(19,6,nan)	(20,6,nan)

(1,7,0.844819426554692)	(2,7,0.833166938615529)	(3,7,0.799164965930266)	(4,7,0.745550589864886)	(5,7,0.676478614671992)	(6,7,0.596990332308645)	(7,7,0.512409056462909)	(8,7,nan)	(9,7,nan)	(10,7,nan)	(11,7,nan)	(12,7,nan)	(13,7,nan)	(14,7,nan)	(15,7,nan)	(16,7,nan)	(17,7,nan)	(18,7,nan)	(19,7,nan)	(20,7,nan)

(1,8,0.880221911285893)	(2,8,0.86808112133404)	(3,8,0.832654282836054)	(4,8,0.776793175610896)	(5,8,0.704826712589884)	(6,8,0.622007443018828)	(7,8,0.533881756104225)	(8,8,0.445687908263982)	(9,8,nan)	(10,8,nan)	(11,8,nan)	(12,8,nan)	(13,8,nan)	(14,8,nan)	(15,8,nan)	(16,8,nan)	(17,8,nan)	(18,8,nan)	(19,8,nan)	(20,8,nan)

(1,9,0.908423257476761)	(2,9,0.895893489909067)	(3,9,0.859331614298145)	(4,9,0.801680778365641)	(5,9,0.727408588673019)	(6,9,0.641935880392358)	(7,9,0.550986742967013)	(8,9,0.459967260814607)	(9,9,0.373464189430496)	(10,9,nan)	(11,9,nan)	(12,9,nan)	(13,9,nan)	(14,9,nan)	(15,9,nan)	(16,9,nan)	(17,9,nan)	(18,9,nan)	(19,9,nan)	(20,9,nan)

(1,10,0.930508894824686)	(2,10,0.917674502842903)	(3,10,0.880223732854985)	(4,10,0.821171286555585)	(5,10,0.745093387208254)	(6,10,0.657542661634779)	(7,10,0.564382363663124)	(8,10,0.471150010739445)	(9,10,0.382543871816776)	(10,10,0.302092237207199)	(11,10,nan)	(12,10,nan)	(13,10,nan)	(14,10,nan)	(15,10,nan)	(16,10,nan)	(17,10,nan)	(18,10,nan)	(19,10,nan)	(20,10,nan)

(1,11,0.947575504683797)	(2,11,0.934505715101894)	(3,11,0.896368054656656)	(4,11,0.836232518160242)	(5,11,0.758759262106185)	(6,11,0.669602754917344)	(7,11,0.574733789281485)	(8,11,0.479791447122406)	(9,11,0.389560169082285)	(10,11,0.307632958400254)	(11,11,0.236280265818511)	(12,11,nan)	(13,11,nan)	(14,11,nan)	(15,11,nan)	(16,11,nan)	(17,11,nan)	(18,11,nan)	(19,11,nan)	(20,11,nan)

(1,12,0.960628005176078)	(2,12,0.947378184098945)	(3,12,0.908715191551647)	(4,12,0.847751310384182)	(5,12,0.769210889014237)	(6,12,0.678826284066198)	(7,12,0.582650533678577)	(8,12,0.486400396033392)	(9,12,0.394926215664944)	(10,12,0.311870488096952)	(11,12,0.239534938686972)	(12,12,0.178936826240635)	(13,12,nan)	(14,12,nan)	(15,12,nan)	(16,12,nan)	(17,12,nan)	(18,12,nan)	(19,12,nan)	(20,12,nan)

(1,13,0.970531794728468)	(2,13,0.957145371929495)	(3,13,0.91808377540689)	(4,13,0.856491374722895)	(5,13,0.777141224924866)	(6,13,0.685824781532228)	(7,13,0.58865748770974)	(8,13,0.491415039719145)	(9,13,0.398997787706975)	(10,13,0.315085780244454)	(11,13,0.242004472794262)	(12,13,0.180781611798323)	(13,13,0.13134737203193)	(14,13,nan)	(15,13,nan)	(16,13,nan)	(17,13,nan)	(18,13,nan)	(19,13,nan)	(20,13,nan)

(1,14,0.978001395270939)	(2,14,0.964511945212535)	(3,14,0.925149715033043)	(4,14,0.863083274619427)	(5,14,0.783122414358129)	(6,14,0.691103163124669)	(7,14,0.593188030978296)	(8,14,0.495197166247305)	(9,14,0.402068634131264)	(10,14,0.317510806325848)	(11,14,0.243867035928292)	(12,14,0.182172979327849)	(13,14,0.132358273896981)	(14,14,0.0935307681545622)	(15,14,nan)	(16,14,nan)	(17,14,nan)	(18,14,nan)	(19,14,nan)	(20,14,nan)

(1,15,0.983609577303426)	(2,15,0.97004277429664)	(3,15,0.930454828128277)	(4,15,0.868032478308364)	(5,15,0.787613096145167)	(6,15,0.69506617622544)	(7,15,0.596589566470275)	(8,15,0.498036789854914)	(9,15,0.404374228070767)	(10,15,0.319331517837884)	(11,15,0.245265449811771)	(12,15,0.183217619176433)	(13,15,0.133117259821857)	(14,15,0.094067104376561)	(15,15,0.0646513286328791)	(16,15,nan)	(17,15,nan)	(18,15,nan)	(19,15,nan)	(20,15,nan)

(1,16,0.987805871939086)	(2,16,0.974181189968945)	(3,16,0.934424352921542)	(4,16,0.871735695638039)	(5,16,0.790973226715866)	(6,16,0.698031481295689)	(7,16,0.599134748680002)	(8,16,0.500161524259554)	(9,16,0.406099377401566)	(10,16,0.320693856276042)	(11,16,0.2463118061254)	(12,16,0.183999265807626)	(13,16,0.133685167308927)	(14,16,0.0944684153180115)	(15,16,0.0649271454099729)	(16,16,0.043401252259261)	(17,16,nan)	(18,16,nan)	(19,16,nan)	(20,16,nan)

(1,17,0.990937724657488)	(2,17,0.977269845437283)	(3,17,0.937386958766447)	(4,17,0.874499546193769)	(5,17,0.793481018702778)	(6,17,0.700244600142548)	(7,17,0.601034314014291)	(8,17,0.501747293562906)	(9,17,0.407386921315999)	(10,17,0.321710620757894)	(11,17,0.247092741247879)	(12,17,0.184582638125177)	(13,17,0.134109018053835)	(14,17,0.0947679287869243)	(15,17,0.0651329978579353)	(16,17,0.0435388565535171)	(17,17,0.0283067000508986)	(18,17,nan)	(19,17,nan)	(20,17,nan)

(1,18,0.993270694323764)	(2,18,0.979570636746807)	(3,18,0.939593853595404)	(4,18,0.876558384871192)	(5,18,0.795349115053675)	(6,18,0.70189318952456)	(7,18,0.602449332120974)	(8,18,0.502928559738258)	(9,18,0.408346034392652)	(10,18,0.322468025689477)	(11,18,0.247674472930652)	(12,18,0.185017201957951)	(13,18,0.134424751589161)	(14,18,0.0949910413980342)	(15,18,0.0652863408021942)	(16,18,0.0436413603023564)	(17,18,0.0283733426571168)	(18,18,0.0179415103222197)	(19,18,nan)	(20,18,nan)

(1,19,0.995006096807971)	(2,19,0.981282103043136)	(3,19,0.94123547406905)	(4,19,0.878089872317012)	(5,19,0.79673871693961)	(6,19,0.703119508987834)	(7,19,0.603501907288596)	(8,19,0.503807256227479)	(9,19,0.409059480109465)	(10,19,0.323031428814181)	(11,19,0.248107199777478)	(12,19,0.185340456548803)	(13,19,0.134659613091851)	(14,19,0.0951570059132078)	(15,19,0.0654004064629052)	(16,19,0.043717608726392)	(17,19,0.0284229154167016)	(18,19,0.0179728570052151)	(19,19,0.0110535527433294)	(20,19,nan)

(1,20,0.996295630106907)	(2,20,0.982553849971706)	(3,20,0.942455320349234)	(4,20,0.879227881554788)	(5,20,0.797771294638723)	(6,20,0.704030755685598)	(7,20,0.604284049034176)	(8,20,0.504460193164484)	(9,20,0.409589623414655)	(10,20,0.323450079298228)	(11,20,0.248428748054262)	(12,20,0.185580658785881)	(13,20,0.134834132680887)	(14,20,0.0952803299090564)	(15,20,0.0654851657444497)	(16,20,0.0437742669844503)	(17,20,0.0284597516692636)	(18,20,0.0179961499253888)	(19,20,0.0110678782076455)	(20,20,0.0066204180310302)

};


 \end{axis} \end{tikzpicture}

	\label{microblogRM}
\end{figure} 

Now we pay attention to the values in terms of the $DL$ axis. We can observe that they increase in a soft slope as we traverse $DL$. Unlike $DFRee$, the slope is always incremental. The idea behind is that no document is significantly more important when it contains a single word more. However it should be more important since the more terms in a document the more comprehensive the document should be, particularly in terms of the amount of information encoded in it, regardless of the character limitation.


\begin{figure}
	\centering
	\caption{MBRM setting $\alpha$ per fold.}
	%
%\begin{tikzpicture}[thick,scale=0.7, every node/.style={transform shape}] \begin{axis}[
% %title={},
% %y dir=reverse, 
% %x dir=reverse, 
% ylabel={Split},
% xlabel={Alpha},
% zlabel={P@30},
% every axis/.append style={font=\large\bfseries},
% max space between ticks=25pt,
% view={-50}{20}
%% yticklabels={0k,100k}
% ] 
%
%		\addplot3[surf] coordinates { 
%%patch,patch type=biquadratic, shader=faceted,patch refines=3
%(0,1,0.4071)	(0.05,1,0.3895)	(0.1,1,0.3966)	(0.15,1,0.3975)	(0.2,1,0.3955)	(0.25,1,0.3918)	(0.3,1,0.3841)	(0.35,1,0.3751)	(0.4,1,0.3656)	(0.45,1,0.3539)	(0.5,1,0.3227)	(0.55,1,0.2764)	(0.6,1,0.2304)	(0.65,1,0.1951)	(0.7,1,0.1681)	(0.75,1,0.1434)	(0.8,1,0.1181)	(0.85,1,0.1043)	(0.9,1,0.0985)	(0.95,1,0.0905)
%
%(0,2,0.2188)	(0.05,2,0.2241)	(0.1,2,0.2322)	(0.15,2,0.2324)	(0.2,2,0.2317)	(0.25,2,0.2281)	(0.3,2,0.2275)	(0.35,2,0.2253)	(0.4,2,0.2191)	(0.45,2,0.2102)	(0.5,2,0.197)	(0.55,2,0.1859)	(0.6,2,0.1599)	(0.65,2,0.1417)	(0.7,2,0.1324)	(0.75,2,0.113)	(0.8,2,0.1025)	(0.85,2,0.0908)	(0.9,2,0.0876)	(0.95,2,0.0832)
%
%(0,3,0.1814)	(0.05,3,0.1878)	(0.1,3,0.1978)	(0.15,3,0.2)	(0.2,3,0.2014)	(0.25,3,0.2026)	(0.3,3,0.2027)	(0.35,3,0.2023)	(0.4,3,0.1993)	(0.45,3,0.193)	(0.5,3,0.1859)	(0.55,3,0.1707)	(0.6,3,0.1521)	(0.65,3,0.1322)	(0.7,3,0.1187)	(0.75,3,0.1006)	(0.8,3,0.0853)	(0.85,3,0.0748)	(0.9,3,0.0716)	(0.95,3,0.0685)
%
%(0,4,0.3427)	(0.05,4,0.3334)	(0.1,4,0.3351)	(0.15,4,0.3361)	(0.2,4,0.3366)	(0.25,4,0.3366)	(0.3,4,0.3364)	(0.35,4,0.3339)	(0.4,4,0.3273)	(0.45,4,0.315)	(0.5,4,0.2927)	(0.55,4,0.2695)	(0.6,4,0.2312)	(0.65,4,0.1978)	(0.7,4,0.1686)	(0.75,4,0.1426)	(0.8,4,0.1141)	(0.85,4,0.0998)	(0.9,4,0.0929)	(0.95,4,0.0874)
%
%(0,5,0.3696)	(0.05,5,0.382)	(0.1,5,0.3856)	(0.15,5,0.3867)	(0.2,5,0.3857)	(0.25,5,0.3833)	(0.3,5,0.379)	(0.35,5,0.3763)	(0.4,5,0.367)	(0.45,5,0.3549)	(0.5,5,0.3346)	(0.55,5,0.3068)	(0.6,5,0.2717)	(0.65,5,0.2308)	(0.7,5,0.2057)	(0.75,5,0.1727)	(0.8,5,0.1506)	(0.85,5,0.1338)	(0.9,5,0.1251)	(0.95,5,0.1178)
%
%
%}; \end{axis} \end{tikzpicture}
\begin{tikzpicture}[thick,scale=0.7, every node/.style={transform shape}]
\begin{axis}[
	xlabel={$\alpha$},
	ylabel={$P@30$}
]
\addplot coordinates {
	(0,0.4071)	(0.05,0.3895)	(0.1,0.3966)	(0.15,0.3975)	(0.2,0.3955)	(0.25,0.3918)	(0.3,0.3841)	(0.35,0.3751)	(0.4,0.3656)	(0.45,0.3539)	(0.5,0.3227)	(0.55,0.2764)	(0.6,0.2304)	(0.65,0.1951)	(0.7,0.1681)	(0.75,0.1434)	(0.8,0.1181)	(0.85,0.1043)	(0.9,0.0985)	(0.95,0.0905)
};

\addplot coordinates{	
	(0,0.2188)	(0.05,0.2241)	(0.1,0.2322)	(0.15,0.2324)	(0.2,0.2317)	(0.25,0.2281)	(0.3,0.2275)	(0.35,0.2253)	(0.4,0.2191)	(0.45,0.2102)	(0.5,0.197)	(0.55,0.1859)	(0.6,0.1599)	(0.65,0.1417)	(0.7,0.1324)	(0.75,0.113)	(0.8,0.1025)	(0.85,0.0908)	(0.9,0.0876)	(0.95,0.0832)
};

\addplot coordinates{
	(0,0.1814)	(0.05,0.1878)	(0.1,0.1978)	(0.15,0.2)	(0.2,0.2014)	(0.25,0.2026)	(0.3,0.2027)	(0.35,0.2023)	(0.4,0.1993)	(0.45,0.193)	(0.5,0.1859)	(0.55,0.1707)	(0.6,0.1521)	(0.65,0.1322)	(0.7,0.1187)	(0.75,0.1006)	(0.8,0.0853)	(0.85,0.0748)	(0.9,0.0716)	(0.95,0.0685)
};

\addplot coordinates{
	(0,0.3427)	(0.05,0.3334)	(0.1,0.3351)	(0.15,0.3361)	(0.2,0.3366)	(0.25,0.3366)	(0.3,0.3364)	(0.35,0.3339)	(0.4,0.3273)	(0.45,0.315)	(0.5,0.2927)	(0.55,0.2695)	(0.6,0.2312)	(0.65,0.1978)	(0.7,0.1686)	(0.75,0.1426)	(0.8,0.1141)	(0.85,0.0998)	(0.9,0.0929)	(0.95,0.0874)
};

\addplot coordinates{
		(0,0.3696)	(0.05,0.382)	(0.1,0.3856)	(0.15,0.3867)	(0.2,0.3857)	(0.25,0.3833)	(0.3,0.379)	(0.35,0.3763)	(0.4,0.367)	(0.45,0.3549)	(0.5,0.3346)	(0.55,0.3068)	(0.6,0.2717)	(0.65,0.2308)	(0.7,0.2057)	(0.75,0.1727)	(0.8,0.1506)	(0.85,0.1338)	(0.9,0.1251)	(0.95,0.1178)
};
\legend{$1$,$2$,$3$,$4$,$5$}
\end{axis}
\end{tikzpicture}
	\label{microblogRM-param}
\end{figure} 

Figure \ref{microblogRM-param} shows the results of parameter optimisation by means of a 5-fold cross-validation. The whole set of topics is subdivided into 5 groups where 4 are used for training and one for testing. The roles of each group are alternated resulting in 5 different experiments. It can very easily be observed that the most optimal values for the mixing parameter $\alpha$ are near $0.20$.


Table \ref{MBRMPerformance} shows the performance results obtained for MBRM in terms of Precision at different levels with respect to IDF and DFRee. As it can be observed, the performance is always significantly superior than the baselines. The main difference with respect to IDF is obviously that it takes advantage of document statistics, where IDF does not. However the main difference with respect to DFRee is that documents longer than 15 terms are not penalised following the aforementioned rationale. These results not only demonstrate that we can make effective use of document statistics unlike previously thought by other authors \cite{naveed2011searching}, but also that the scope hypotheses still holds for small documents. In other words, the authors of the documents will attempt to encode as much information as possible even with the obvious document limitations. This contradicts our findings in Subsection \ref{bm25case} however we believe that in the particular case of BM25, document length has a much more aggressive effect on the scores, thus resulting in a misleading behaviour.

\begin{table}[b] 	
	  	  	\centering
	  	  	\caption{Performance of MBRM on all collections (Where * $p<0.05$ and ** $p<0.01$ respectively, with respect to IDF and DFRee)} 
	  	 	\begin{tabular}{l|c|c|c|c|c} 	  	 	
	  	 	\cline{2- 6}
	  	 	\multicolumn{1}{c}{}&\multicolumn{5}{c}{Precision} \\ 
	  	 	\cline{2- 6} &
	  	 	\textit{\textbf{@5}} & 
	  	 	\textit{\textbf{@10}} & 
	  	 	\textit{\textbf{@15}} & 
	  	 	\textit{\textbf{@20}} & 
	  	 	\textit{\textbf{@30}} 
	  	 	\tabularnewline
	  	 	\hline
	 	 	 DFRee  & 0.62 & 0.57 & 0.54 & 0.51 & 0.46 \\
	 	 	 IDF  & 0.62 & 0.57 & 0.53 & 0.51 & 0.46 \\
	 	 	 \hline
 	 	 	 \hline
  	  	 	 MBRM ($\alpha=0.20$)  & \textbf{0.64*} & \textbf{0.59*} & \textbf{0.56**} & \textbf{0.53**} & \textbf{0.48*} \\
	  	  	\hline
	  	  	\end{tabular}
	  	  	\label{MBRMPerformance}	
\end{table}


The verbose hypotheses however seems not to hold, as authors are very careful to come up with specific words to effectively encode their message. Thus documents are not generally longer due to style differences, or the verbosity of the author, but it is rather a reflection of the author's capacity to encode rich information in such limited constraints. And this is what is ultimately captured by our MBRM retrieval model.


Final note, the previous experiment where we apply a logarithmic function to the scores of the retrieval models, reduce the effect of the possible values of DL. This can also be interpreted as being closer to the behaviour of MBRM where increasing DL also increases the score, which provides the best experimental results.


\section{MBRM: A MicroBlog Retrieval Model}
\label{MBRM-section}
In the previous section, we discussed a number of problems faced by state of the art retrieval models when dealing with microblogs. We presented scarcity of TF and DL as a source of high scoring differences amongst the spectrum of possible scores for a retrieval model. Additionally we started defining the requirements for a retrieval model to effectively handle microblog documents by better capturing their informativeness. These requirements can be summarised as: 

\begin{enumerate}
\item Higher DL should be regarded positively as authors of microblogs strive to fit as much content as possible within the character limits
\item Higher TF should be regarded negatively as high TF could be a result of spam messages, and normally TF revolves around 1-2
\item Score differences with respect to DL and TF should produce gentle slopes, to not penalise/promote unfairly documents with very little differences.
\end{enumerate}

Following these premises, we have designed a ``MicroBlogs Retrieval Model'', namely MBRM. MBRM is composed of two parts to deal with document based evidence. Then we attach the aforementioned part to an IDF component which represents the collection's information. Similarly to the formulation of BM25, the two main components of MBRM deal with document length and query term frequency. The first component deals with the document length and is given by the following logistic distribution:

\begin{equation}
DLComp(DL)={\frac  {c_1}{1+{a_1\mathrm  e}^{{-b_1DL}}}}
\end{equation}

where \(a_1, b_1\) and \(c_1\) are parameters to control the growth, maximum and starting point of the distribution. Secondly, the following component given by a gaussian distribution deals with the effect of TF over the final score produced by MBRM:

\begin{equation}
TFComp\left(TF\right)=a_2e^{-{\frac {(TF-b_2)^{2}}{2c_2^{2}}}}
\end{equation}

where \(a_2, b_2\) and \(c_2\) are similar parameters to those found in the previous function. These functions were chosen as they offer good control over the curves, and their values can be bound between 1 and 0 and we do not need to normalise them. The final formulation for MBRM is given by: 

\begin{equation}
MBRM(D,Q) = \sum_{i=1}^{|Q|} (1-\alpha)*\text{IDF}(q_i) + \alpha * DLComp(|D|) * TFComp(q_i)
\end{equation}

which can be also expressed as:

\begin{equation}
MBRM(D,Q) = \sum_{i=1}^{|Q|} (1-\alpha)*\text{IDF}(q_i) + \alpha * \left({\frac  {c_1}{1+{a_1\mathrm e}^{{-b_1DL(|D|)}}}} \right) * \left(a_2e^{-{\frac {(TF(q_i)-b_2)^{2}}{2c_2^{2}}}}\right) 
\end{equation}

\begin{table}[b]
	\caption{MBRM recommended parameter settings} 
	\centering
	\begin{tabular}{l|c} 	
		\hline
		\textbf{Parameter} & \textbf{Recommended values} \\
		\hline
		\centering					 
		$a_1$ & 1.5 \\
		$b_1$ & 0.3 \\
		$c_1$ & 1.0 \\
		\hline
		$a_2$ & 1.0 \\
		$b_2$ & 2.0 \\
		$c_2$ & 6.0 \\
		\hline
	\end{tabular}
	\label{recommended settings}
\end{table}

Figure \ref{microblogRM} shows a simulation of the behaviour of MBRM in terms of TF and DL. The parameters used to for both components (DLComp and TFComp) are shown in Table \ref{recommended settings}. In Figure \ref{microblogRM} we can observe how the values obtained on the TF axis decrease slowly for the initial values of TF, but rapidly accelerate in their descent to then settle near 0. This behaviour is similar to that of DFRee (Albeit smoother) in which the highest importance is given to low TF values $\sim2$ and then it is reduced. 
%High TF values are most likely than not associated with spam or unimportant documents, since actual users struggling to fit their content in the 140 characters limit are unlikely to repeat words. Although this is not always the case, thus the slow descent for low values of TF.

\begin{figure}
	\begin{subfigure}[]{0.5\textwidth}
		\caption{Doc. length (DL) and Term Frequency (TF)}
		%!TEX root = ./JournalChapter1.tex
\begin{tikzpicture}[thick,scale=0.70, every node/.style={transform shape}]\begin{axis}[
 %title={},
% y dir=reverse, 
 x dir=reverse, 
 ylabel={docLength (DL)},
 xlabel={term frequency (TF)},
 zlabel={MBRM score},
 every axis/.append style={font=\large\bfseries},
  max space between ticks=25pt,
  view={20}{45}
% yticklabels={0k,100k}
 ] 

\addplot3[surf,unbounded coords=jump]
coordinates  { 
%patch,patch type=biquadratic, shader=faceted,patch refines=3
(1,1,0.473658084609043)	(2,1,nan)	(3,1,nan)	(4,1,nan)	(5,1,nan)	(6,1,nan)	(7,1,nan)	(8,1,nan)	(9,1,nan)	(10,1,nan)	(11,1,nan)	(12,1,nan)	(13,1,nan)	(14,1,nan)	(15,1,nan)	(16,1,nan)	(17,1,nan)	(18,1,nan)	(19,1,nan)	(20,1,nan)

(1,2,0.548480827396313)	(2,2,0.540915700429286)	(3,2,nan)	(4,2,nan)	(5,2,nan)	(6,2,nan)	(7,2,nan)	(8,2,nan)	(9,2,nan)	(10,2,nan)	(11,2,nan)	(12,2,nan)	(13,2,nan)	(14,2,nan)	(15,2,nan)	(16,2,nan)	(17,2,nan)	(18,2,nan)	(19,2,nan)	(20,2,nan)

(1,3,0.621174011140144)	(2,3,0.612606236245987)	(3,3,0.587605459635394)	(4,3,nan)	(5,3,nan)	(6,3,nan)	(7,3,nan)	(8,3,nan)	(9,3,nan)	(10,3,nan)	(11,3,nan)	(12,3,nan)	(13,3,nan)	(14,3,nan)	(15,3,nan)	(16,3,nan)	(17,3,nan)	(18,3,nan)	(19,3,nan)	(20,3,nan)

(1,4,0.688804056409613)	(2,4,0.679303468819562)	(3,4,0.651580743731372)	(4,4,0.607867497379652)	(5,4,nan)	(6,4,nan)	(7,4,nan)	(8,4,nan)	(9,4,nan)	(10,4,nan)	(11,4,nan)	(12,4,nan)	(13,4,nan)	(14,4,nan)	(15,4,nan)	(16,4,nan)	(17,4,nan)	(18,4,nan)	(19,4,nan)	(20,4,nan)

(1,5,0.749234519496894)	(2,5,0.738900422141148)	(3,5,0.70874551463543)	(4,5,0.661197198359976)	(5,5,0.599940192995648)	(6,5,nan)	(7,5,nan)	(8,5,nan)	(9,5,nan)	(10,5,nan)	(11,5,nan)	(12,5,nan)	(13,5,nan)	(14,5,nan)	(15,5,nan)	(16,5,nan)	(17,5,nan)	(18,5,nan)	(19,5,nan)	(20,5,nan)

(1,6,0.801315033266165)	(2,6,0.790262595943928)	(3,6,0.758011571622054)	(4,6,0.707158094310837)	(5,6,0.641643014567497)	(6,6,0.56624802053189)	(7,6,nan)	(8,6,nan)	(9,6,nan)	(10,6,nan)	(11,6,nan)	(12,6,nan)	(13,6,nan)	(14,6,nan)	(15,6,nan)	(16,6,nan)	(17,6,nan)	(18,6,nan)	(19,6,nan)	(20,6,nan)

(1,7,0.844819426554692)	(2,7,0.833166938615529)	(3,7,0.799164965930266)	(4,7,0.745550589864886)	(5,7,0.676478614671992)	(6,7,0.596990332308645)	(7,7,0.512409056462909)	(8,7,nan)	(9,7,nan)	(10,7,nan)	(11,7,nan)	(12,7,nan)	(13,7,nan)	(14,7,nan)	(15,7,nan)	(16,7,nan)	(17,7,nan)	(18,7,nan)	(19,7,nan)	(20,7,nan)

(1,8,0.880221911285893)	(2,8,0.86808112133404)	(3,8,0.832654282836054)	(4,8,0.776793175610896)	(5,8,0.704826712589884)	(6,8,0.622007443018828)	(7,8,0.533881756104225)	(8,8,0.445687908263982)	(9,8,nan)	(10,8,nan)	(11,8,nan)	(12,8,nan)	(13,8,nan)	(14,8,nan)	(15,8,nan)	(16,8,nan)	(17,8,nan)	(18,8,nan)	(19,8,nan)	(20,8,nan)

(1,9,0.908423257476761)	(2,9,0.895893489909067)	(3,9,0.859331614298145)	(4,9,0.801680778365641)	(5,9,0.727408588673019)	(6,9,0.641935880392358)	(7,9,0.550986742967013)	(8,9,0.459967260814607)	(9,9,0.373464189430496)	(10,9,nan)	(11,9,nan)	(12,9,nan)	(13,9,nan)	(14,9,nan)	(15,9,nan)	(16,9,nan)	(17,9,nan)	(18,9,nan)	(19,9,nan)	(20,9,nan)

(1,10,0.930508894824686)	(2,10,0.917674502842903)	(3,10,0.880223732854985)	(4,10,0.821171286555585)	(5,10,0.745093387208254)	(6,10,0.657542661634779)	(7,10,0.564382363663124)	(8,10,0.471150010739445)	(9,10,0.382543871816776)	(10,10,0.302092237207199)	(11,10,nan)	(12,10,nan)	(13,10,nan)	(14,10,nan)	(15,10,nan)	(16,10,nan)	(17,10,nan)	(18,10,nan)	(19,10,nan)	(20,10,nan)

(1,11,0.947575504683797)	(2,11,0.934505715101894)	(3,11,0.896368054656656)	(4,11,0.836232518160242)	(5,11,0.758759262106185)	(6,11,0.669602754917344)	(7,11,0.574733789281485)	(8,11,0.479791447122406)	(9,11,0.389560169082285)	(10,11,0.307632958400254)	(11,11,0.236280265818511)	(12,11,nan)	(13,11,nan)	(14,11,nan)	(15,11,nan)	(16,11,nan)	(17,11,nan)	(18,11,nan)	(19,11,nan)	(20,11,nan)

(1,12,0.960628005176078)	(2,12,0.947378184098945)	(3,12,0.908715191551647)	(4,12,0.847751310384182)	(5,12,0.769210889014237)	(6,12,0.678826284066198)	(7,12,0.582650533678577)	(8,12,0.486400396033392)	(9,12,0.394926215664944)	(10,12,0.311870488096952)	(11,12,0.239534938686972)	(12,12,0.178936826240635)	(13,12,nan)	(14,12,nan)	(15,12,nan)	(16,12,nan)	(17,12,nan)	(18,12,nan)	(19,12,nan)	(20,12,nan)

(1,13,0.970531794728468)	(2,13,0.957145371929495)	(3,13,0.91808377540689)	(4,13,0.856491374722895)	(5,13,0.777141224924866)	(6,13,0.685824781532228)	(7,13,0.58865748770974)	(8,13,0.491415039719145)	(9,13,0.398997787706975)	(10,13,0.315085780244454)	(11,13,0.242004472794262)	(12,13,0.180781611798323)	(13,13,0.13134737203193)	(14,13,nan)	(15,13,nan)	(16,13,nan)	(17,13,nan)	(18,13,nan)	(19,13,nan)	(20,13,nan)

(1,14,0.978001395270939)	(2,14,0.964511945212535)	(3,14,0.925149715033043)	(4,14,0.863083274619427)	(5,14,0.783122414358129)	(6,14,0.691103163124669)	(7,14,0.593188030978296)	(8,14,0.495197166247305)	(9,14,0.402068634131264)	(10,14,0.317510806325848)	(11,14,0.243867035928292)	(12,14,0.182172979327849)	(13,14,0.132358273896981)	(14,14,0.0935307681545622)	(15,14,nan)	(16,14,nan)	(17,14,nan)	(18,14,nan)	(19,14,nan)	(20,14,nan)

(1,15,0.983609577303426)	(2,15,0.97004277429664)	(3,15,0.930454828128277)	(4,15,0.868032478308364)	(5,15,0.787613096145167)	(6,15,0.69506617622544)	(7,15,0.596589566470275)	(8,15,0.498036789854914)	(9,15,0.404374228070767)	(10,15,0.319331517837884)	(11,15,0.245265449811771)	(12,15,0.183217619176433)	(13,15,0.133117259821857)	(14,15,0.094067104376561)	(15,15,0.0646513286328791)	(16,15,nan)	(17,15,nan)	(18,15,nan)	(19,15,nan)	(20,15,nan)

(1,16,0.987805871939086)	(2,16,0.974181189968945)	(3,16,0.934424352921542)	(4,16,0.871735695638039)	(5,16,0.790973226715866)	(6,16,0.698031481295689)	(7,16,0.599134748680002)	(8,16,0.500161524259554)	(9,16,0.406099377401566)	(10,16,0.320693856276042)	(11,16,0.2463118061254)	(12,16,0.183999265807626)	(13,16,0.133685167308927)	(14,16,0.0944684153180115)	(15,16,0.0649271454099729)	(16,16,0.043401252259261)	(17,16,nan)	(18,16,nan)	(19,16,nan)	(20,16,nan)

(1,17,0.990937724657488)	(2,17,0.977269845437283)	(3,17,0.937386958766447)	(4,17,0.874499546193769)	(5,17,0.793481018702778)	(6,17,0.700244600142548)	(7,17,0.601034314014291)	(8,17,0.501747293562906)	(9,17,0.407386921315999)	(10,17,0.321710620757894)	(11,17,0.247092741247879)	(12,17,0.184582638125177)	(13,17,0.134109018053835)	(14,17,0.0947679287869243)	(15,17,0.0651329978579353)	(16,17,0.0435388565535171)	(17,17,0.0283067000508986)	(18,17,nan)	(19,17,nan)	(20,17,nan)

(1,18,0.993270694323764)	(2,18,0.979570636746807)	(3,18,0.939593853595404)	(4,18,0.876558384871192)	(5,18,0.795349115053675)	(6,18,0.70189318952456)	(7,18,0.602449332120974)	(8,18,0.502928559738258)	(9,18,0.408346034392652)	(10,18,0.322468025689477)	(11,18,0.247674472930652)	(12,18,0.185017201957951)	(13,18,0.134424751589161)	(14,18,0.0949910413980342)	(15,18,0.0652863408021942)	(16,18,0.0436413603023564)	(17,18,0.0283733426571168)	(18,18,0.0179415103222197)	(19,18,nan)	(20,18,nan)

(1,19,0.995006096807971)	(2,19,0.981282103043136)	(3,19,0.94123547406905)	(4,19,0.878089872317012)	(5,19,0.79673871693961)	(6,19,0.703119508987834)	(7,19,0.603501907288596)	(8,19,0.503807256227479)	(9,19,0.409059480109465)	(10,19,0.323031428814181)	(11,19,0.248107199777478)	(12,19,0.185340456548803)	(13,19,0.134659613091851)	(14,19,0.0951570059132078)	(15,19,0.0654004064629052)	(16,19,0.043717608726392)	(17,19,0.0284229154167016)	(18,19,0.0179728570052151)	(19,19,0.0110535527433294)	(20,19,nan)

(1,20,0.996295630106907)	(2,20,0.982553849971706)	(3,20,0.942455320349234)	(4,20,0.879227881554788)	(5,20,0.797771294638723)	(6,20,0.704030755685598)	(7,20,0.604284049034176)	(8,20,0.504460193164484)	(9,20,0.409589623414655)	(10,20,0.323450079298228)	(11,20,0.248428748054262)	(12,20,0.185580658785881)	(13,20,0.134834132680887)	(14,20,0.0952803299090564)	(15,20,0.0654851657444497)	(16,20,0.0437742669844503)	(17,20,0.0284597516692636)	(18,20,0.0179961499253888)	(19,20,0.0110678782076455)	(20,20,0.0066204180310302)

};


 \end{axis} \end{tikzpicture}

		\label{microblogRM}
	\end{subfigure} 
	~
	\begin{subfigure}[]{0.5\textwidth}
		\caption{MBRM effects of $\alpha$ on each fold.}
		%
%\begin{tikzpicture}[thick,scale=0.7, every node/.style={transform shape}] \begin{axis}[
% %title={},
% %y dir=reverse, 
% %x dir=reverse, 
% ylabel={Split},
% xlabel={Alpha},
% zlabel={P@30},
% every axis/.append style={font=\large\bfseries},
% max space between ticks=25pt,
% view={-50}{20}
%% yticklabels={0k,100k}
% ] 
%
%		\addplot3[surf] coordinates { 
%%patch,patch type=biquadratic, shader=faceted,patch refines=3
%(0,1,0.4071)	(0.05,1,0.3895)	(0.1,1,0.3966)	(0.15,1,0.3975)	(0.2,1,0.3955)	(0.25,1,0.3918)	(0.3,1,0.3841)	(0.35,1,0.3751)	(0.4,1,0.3656)	(0.45,1,0.3539)	(0.5,1,0.3227)	(0.55,1,0.2764)	(0.6,1,0.2304)	(0.65,1,0.1951)	(0.7,1,0.1681)	(0.75,1,0.1434)	(0.8,1,0.1181)	(0.85,1,0.1043)	(0.9,1,0.0985)	(0.95,1,0.0905)
%
%(0,2,0.2188)	(0.05,2,0.2241)	(0.1,2,0.2322)	(0.15,2,0.2324)	(0.2,2,0.2317)	(0.25,2,0.2281)	(0.3,2,0.2275)	(0.35,2,0.2253)	(0.4,2,0.2191)	(0.45,2,0.2102)	(0.5,2,0.197)	(0.55,2,0.1859)	(0.6,2,0.1599)	(0.65,2,0.1417)	(0.7,2,0.1324)	(0.75,2,0.113)	(0.8,2,0.1025)	(0.85,2,0.0908)	(0.9,2,0.0876)	(0.95,2,0.0832)
%
%(0,3,0.1814)	(0.05,3,0.1878)	(0.1,3,0.1978)	(0.15,3,0.2)	(0.2,3,0.2014)	(0.25,3,0.2026)	(0.3,3,0.2027)	(0.35,3,0.2023)	(0.4,3,0.1993)	(0.45,3,0.193)	(0.5,3,0.1859)	(0.55,3,0.1707)	(0.6,3,0.1521)	(0.65,3,0.1322)	(0.7,3,0.1187)	(0.75,3,0.1006)	(0.8,3,0.0853)	(0.85,3,0.0748)	(0.9,3,0.0716)	(0.95,3,0.0685)
%
%(0,4,0.3427)	(0.05,4,0.3334)	(0.1,4,0.3351)	(0.15,4,0.3361)	(0.2,4,0.3366)	(0.25,4,0.3366)	(0.3,4,0.3364)	(0.35,4,0.3339)	(0.4,4,0.3273)	(0.45,4,0.315)	(0.5,4,0.2927)	(0.55,4,0.2695)	(0.6,4,0.2312)	(0.65,4,0.1978)	(0.7,4,0.1686)	(0.75,4,0.1426)	(0.8,4,0.1141)	(0.85,4,0.0998)	(0.9,4,0.0929)	(0.95,4,0.0874)
%
%(0,5,0.3696)	(0.05,5,0.382)	(0.1,5,0.3856)	(0.15,5,0.3867)	(0.2,5,0.3857)	(0.25,5,0.3833)	(0.3,5,0.379)	(0.35,5,0.3763)	(0.4,5,0.367)	(0.45,5,0.3549)	(0.5,5,0.3346)	(0.55,5,0.3068)	(0.6,5,0.2717)	(0.65,5,0.2308)	(0.7,5,0.2057)	(0.75,5,0.1727)	(0.8,5,0.1506)	(0.85,5,0.1338)	(0.9,5,0.1251)	(0.95,5,0.1178)
%
%
%}; \end{axis} \end{tikzpicture}
\begin{tikzpicture}[thick,scale=0.7, every node/.style={transform shape}]
\begin{axis}[
	xlabel={$\alpha$},
	ylabel={$P@30$}
]
\addplot coordinates {
	(0,0.4071)	(0.05,0.3895)	(0.1,0.3966)	(0.15,0.3975)	(0.2,0.3955)	(0.25,0.3918)	(0.3,0.3841)	(0.35,0.3751)	(0.4,0.3656)	(0.45,0.3539)	(0.5,0.3227)	(0.55,0.2764)	(0.6,0.2304)	(0.65,0.1951)	(0.7,0.1681)	(0.75,0.1434)	(0.8,0.1181)	(0.85,0.1043)	(0.9,0.0985)	(0.95,0.0905)
};

\addplot coordinates{	
	(0,0.2188)	(0.05,0.2241)	(0.1,0.2322)	(0.15,0.2324)	(0.2,0.2317)	(0.25,0.2281)	(0.3,0.2275)	(0.35,0.2253)	(0.4,0.2191)	(0.45,0.2102)	(0.5,0.197)	(0.55,0.1859)	(0.6,0.1599)	(0.65,0.1417)	(0.7,0.1324)	(0.75,0.113)	(0.8,0.1025)	(0.85,0.0908)	(0.9,0.0876)	(0.95,0.0832)
};

\addplot coordinates{
	(0,0.1814)	(0.05,0.1878)	(0.1,0.1978)	(0.15,0.2)	(0.2,0.2014)	(0.25,0.2026)	(0.3,0.2027)	(0.35,0.2023)	(0.4,0.1993)	(0.45,0.193)	(0.5,0.1859)	(0.55,0.1707)	(0.6,0.1521)	(0.65,0.1322)	(0.7,0.1187)	(0.75,0.1006)	(0.8,0.0853)	(0.85,0.0748)	(0.9,0.0716)	(0.95,0.0685)
};

\addplot coordinates{
	(0,0.3427)	(0.05,0.3334)	(0.1,0.3351)	(0.15,0.3361)	(0.2,0.3366)	(0.25,0.3366)	(0.3,0.3364)	(0.35,0.3339)	(0.4,0.3273)	(0.45,0.315)	(0.5,0.2927)	(0.55,0.2695)	(0.6,0.2312)	(0.65,0.1978)	(0.7,0.1686)	(0.75,0.1426)	(0.8,0.1141)	(0.85,0.0998)	(0.9,0.0929)	(0.95,0.0874)
};

\addplot coordinates{
		(0,0.3696)	(0.05,0.382)	(0.1,0.3856)	(0.15,0.3867)	(0.2,0.3857)	(0.25,0.3833)	(0.3,0.379)	(0.35,0.3763)	(0.4,0.367)	(0.45,0.3549)	(0.5,0.3346)	(0.55,0.3068)	(0.6,0.2717)	(0.65,0.2308)	(0.7,0.2057)	(0.75,0.1727)	(0.8,0.1506)	(0.85,0.1338)	(0.9,0.1251)	(0.95,0.1178)
};
\legend{$1$,$2$,$3$,$4$,$5$}
\end{axis}
\end{tikzpicture}
		\label{microblogRM-param}
	\end{subfigure} 
	\caption{MBRM: A Microblog Retrieval Model}
\end{figure} 

In terms of $DL$ we produce a soft increasing slope to account for increasing value assigned to more informative documents. Unlike $DFRee$, the slope is always incremental. The idea behind it being that the more terms in the microblog the more comprehensive it should be, as more information is encoded regardless of the character limitation.

In order to find the optimal value for the pondering value of $\alpha$ we divided the all the collections into 5 folds. For each of the folds we produced a P@30 result for a number of $\alpha$ values in the 0-1 range. These can be found in Figure \ref{microblogRM-param}. It can very easily be observed that the most optimal values for the mixing parameter $\alpha$ are near $0.20$.

Finally Table \ref{MBRMPerformance} shows the evaluation results obtained for MBRM in terms of Precision at different levels in comparison with IDF and DFRee. As it can be observed, the performance is always significantly superior than the baselines. The main difference with respect to IDF is obviously that it takes advantage of document statistics, where IDF does not. However the main difference with respect to DFRee is that documents longer than 15 terms are not penalised following the aforementioned rationale. 

These results not only demonstrate that we can make effective use of document statistics unlike previously thought by other authors \cite{naveed2011searching}, but also that the scope hypotheses still holds for small documents. In other words, the authors of the documents will attempt to encode as much information as possible even with the obvious document limitations. 

%This contradicts our findings in Subsection \ref{bm25case} however we believe that in the particular case of BM25, document length has a much more aggressive effect on the scores, thus resulting in a misleading behaviour.

\begin{table}[] 	
	  	  	\centering
	  	  	\caption{Performance of MBRM on all collections (Where * $p<0.05$ and ** $p<0.01$ respectively, with respect to IDF and DFRee)} 
	  	 	\begin{tabular}{l|c|c|c|c|c} 	  	 	
	  	 	\cline{2- 6}
	  	 	\multicolumn{1}{c}{}&\multicolumn{5}{c}{Precision} \\ 
	  	 	\cline{2- 6} &
	  	 	\textit{\textbf{@5}} & 
	  	 	\textit{\textbf{@10}} & 
	  	 	\textit{\textbf{@15}} & 
	  	 	\textit{\textbf{@20}} & 
	  	 	\textit{\textbf{@30}} 
	  	 	\tabularnewline
	  	 	\hline
	 	 	 DFRee  & 0.62 & 0.57 & 0.54 & 0.51 & 0.46 \\
	 	 	 IDF  & 0.62 & 0.57 & 0.53 & 0.51 & 0.46 \\
	 	 	 \hline
 	 	 	 \hline
  	  	 	 MBRM ($\alpha=0.20$)  & \textbf{0.64*} & \textbf{0.59*} & \textbf{0.56**} & \textbf{0.53**} & \textbf{0.48*} \\
	  	  	\hline
	  	  	\end{tabular}
	  	  	\label{MBRMPerformance}	
\end{table}


The verbose hypotheses however seems not to hold, as authors are simply capped by the character limitation with very little length variations. Thus documents are not generally longer due to style differences, or the verbosity of the author, but it is rather a reflection of the author's capacity to encode rich information in such limited constraints, which again aligns better with the scope hypotheses. And this is what we ultimately attempted to capture with our MBRM retrieval model.

%Final note, the previous experiment where we apply a logarithmic function to the scores of the retrieval models, reduce the effect of the possible values of DL. This can also be interpreted as being closer to the behaviour of MBRM where increasing DL also increases the score, which provides the best experimental results.


%%!TEX root = JournalChapter1.tex
\section{Understanding microblog documents}
\label{discussion}

In this Section we first study the structure of microblog documents in order to define a hypotheses that captures their relevance. Subsequently, we test our hypotheses through the implementation of a number of approaches that capture microblogs' structure and their evaluation with respect to our DFR baseline. Additionally, we evaluate the relation of the order of the different dimensions within the microblogs, and determine how to utilise this evidence for ad-hoc retrieval.

\subsection{Informativeness of Microblogs}

For web and similar documents, relevance is modelled by the inclusion of statistical measures extracted both from the collection as a whole, and the documents themselves. Most retrieval models take into consideration document based statistics, such as document length and term frequency, in an attempt to capture the relevance of the documents according to the scope and verbosity hypotheses (or similar assumptions). For the purposes of this work, we can think of each retrieval model as a delicate relationship ``\(\circled{?}\)'' between document length \(|D|\) and term frequency \( P(q\, \cap\, D | Q)\) amongst other components. We pay attention to those components as they are most likely affected by the structure of microblogs.

\begin{equation}
 P(I|Q,D) = |D|\, \circled{?}\, P(q\, \cap\, D | Q)
  \vspace{0.5cm}
\end{equation}

Microblog documents are however very short as they have a fixed maximum size. Additionally, authors tend to optimise their content to fit within the character limits and constraints set by the platform, leading to a more or less constant document length ( \(\sim15\) terms in the case of Twitter). Moreover, due to these limitations, the value of term frequencies revolve around \(\sim1.5\). Thus in-document statistical information is limited.

Both the \textbf{scope} and \textbf{verbosity} hypotheses are defined within the assumption that authors may write as much as they desire. As a result it is logical to assume that when this condition is broken unexpected behaviour may follow. Fortunately, microblog documents contain other inherent features which encode extra information in the same message following an organic community-agreed vocabulary. In our work we draw inspiration from the ideas behind the scope and verbosity hypotheses and we are set to describe a new hypotheses tailored to microblog retrieval, which highlights and relies on characteristics of microblog documents' structure. 

Firstly, we assume that microblog documents (\textbf{D}) are 4-dimensional entities comprised of \textbf{Text \(T(D)\);} a \textbf{URL \(U(D)\)} ( Linking to an external resource); \textbf{Hashtags \(\#(D)\)} (Terms preceded by \#) indicating a topical context and \textbf{Mentions \(@(D)\)} (Terms preceded by @) indicating an intended audience. We believe that the amount of space in a microblog document dedicated to each of the dimensions may have a connection with how likely it is to be relevant to the searcher. Having these characteristics in mind, we define (\textbf{H1}) \textbf{Microblog Informativeness} (MI) as the probability for a Microblog document \(D\) being informative given a query \(P(MI|Q,D)\), which depends on an optimal unobserved combination ``\(\circled{?}\)'' of the aforementioned dimensions:

\begin{equation}
 P(MI|Q,D) =  T(D)~ \circled{?}~  U(D)~  \circled{?}~  \#(D)~  \circled{?}~  @(D) \circled{?}\, P(q\, \cap\, D | Q) %,
\end{equation} \\

\noindent where \(T(D)\), \(U(D)\), \(\#(D)\) and \(@(D)\) are the ratios in terms of number of characters spent in the document for each of the dimensions considered \footnote{URL's are automatically shortened by Twitter, thus their length is constant.}. For example, the ratio for the text dimension  \(T(D)\) is given by:

\begin{equation}
	T(D)= \frac{\# of Chars for Text Dimension}{ Total \# of Chars},
\end{equation}\\

In order to test our hypotheses and learn about what are the most prominent characteristics that make up relevant microblog documents, we analyse retrieval runs produced by the state of the art baseline DFR because it is the best performing model as shown in Table \ref{traditional}. We use the documents in the runs instead of all documents in the relevance judgements in order to analyse the documents that are most likely to contain query terms and find differences amongst those documents.

We take into consideration the TREC Microblog topics 1 to 110 so that we can confirm our findings through an evaluation on the newer 111 to 170 topics which belong to TREC's 2013 iteration of the microblog search task.



\begin{table*}[]
\begin{smaller}
\vspace{0.5cm}
\caption{Ratio of each dimension for relevant (Rel) and non-relevant (Non-Rel) documents at different cutoffs.}
\label{ratiosTable}
\vspace{0.30cm}
\begin{subtable}[b]{0.32\textwidth}
\caption{Cutoff @ 10}
\vspace{-0.5cm}
\begin{center}
\begin{tabular}{|c|c|c|}

\hline  & Rel & Non-Rel \\ 
\hline Hash & 1.960 &  	1.619  \\
\hline Ment & 2.750 &  	2.444  \\
\hline Urls & 17.32 &  	14.16 * \\
\hline Text & 77.95 &  	81.77 * \\ 
\hline
\hline DocLength & 97.47 &  	100.2  \\
\hline 
\end{tabular} 
\end{center}
\label{ratio10}
\end{subtable}
~
\begin{subtable}[b]{0.32\textwidth}
\caption{Cutoff @ 20}\vspace{-0.5cm}
\begin{center}
\begin{tabular}{|c|c|c|}


\hline  & Rel & Non-Rel \\ 
\hline Hash & 2.626 &  	1.861 * \\
\hline Ment & 2.453 &  	2.402  \\
\hline Urls & 17.54 &  	13.54 * \\
\hline Text & 77.37 &  	82.18 * \\
\hline
\hline DocLength & 96.50 &  	97.38  \\
\hline 
\end{tabular} 
\end{center}
\label{ratio20}
\end{subtable}
~
\begin{subtable}[b]{0.32\textwidth}
\caption{Cutoff @ 30}\vspace{-0.5cm}
\begin{center}
\begin{tabular}{|c|c|c|}
\hline  & Rel & Non-Rel \\ 
\hline Hash & 2.514 &  	1.999  \\
\hline Ment & 3.061 &  	2.671  \\
\hline Urls & 17.13 &  	14.28 * \\
\hline Text & 77.29 &  	81.04 * \\
\hline
\hline DocLength & 96.21 &  	95.76  \\
\hline 
\end{tabular} 
\end{center}
\label{ratio30}
\end{subtable}

\hspace{1.8cm}
\begin{subtable}[]{0.35\textwidth}

\caption{Cutoff @ 50}\vspace{-0.45cm}
\begin{center}
\begin{tabular}{|c|c|c|}

\hline  & Rel & Non-Rel \\ 
\hline Hash & 2.820 &  	2.518  \\
\hline Ment & 2.968 &  	3.136  \\
\hline Urls & 17.19 &  	14.32 * \\
\hline Text & 77.01 &  	80.01 * \\
\hline
\hline DocLength & 95.90 &  	94.45  \\
\hline 
\end{tabular} 
\end{center}
\label{ratio50}
\end{subtable}
~
\begin{subtable}[]{0.35\textwidth}
\caption{Cutoff @ 100}\vspace{-0.45cm}
\begin{center}
\begin{tabular}{|c|c|c|}
\hline  & Rel & Non-Rel \\ 
\hline Hash & 2.638 &  	2.514  \\
\hline Ment & 2.893 &  	3.315 * \\
\hline Urls & 17.69 &  	14.13 * \\
\hline Text & 76.77 &  	80.03 * \\
\hline
\hline DocLength & 93.96 & 92.56  \\
\hline 
\end{tabular} 
\end{center}
\label{ratio100}
\end{subtable}


\end{smaller}
\vspace{0.70cm}
\end{table*}







Tables \ref{ratiosTable}(a...e) introduce the mean ratios for each of the dimensions for all documents at the cut-offs @10, @20, @30, @50 and @100 respectively. The asterisk indicates statistically significant differences between relevant and non-relevant documents for that dimension. The last row on each table on the other hand, indicates the average document length in number of characters for both relevant and non-relevant documents.

First we look at ``DocLength''. As we can observe in Tables \ref{ratiosTable}(a...e), the differences between relevant and non-relevant documents are not significant. Furthermore, we can see how relevant documents tend to be shorter than non-relevant documents for cut-offs @10 and @20, whereas then they become longer than non-relevant documents for any cut-off after @20. It is evident that the behaviour of this feature is unstable, and the differences between both groups of documents change wildly depending on the cut-off point, contradicting each other.

Based on this observation we can conclude that the document length feature, popular amongst retrieval models, is ineffective in estimating the relevance of a microblog document. Therefore, this helps us to confirm that the scope and verbosity hypothesis do not hold for microblog documents, as differences should have followed a more clear trend if the hypotheses were true. (i.e. One relevance group should have remained higher than the other for all cut-off cases.). Therefore we can confirm that in the case of microblog documents, longer (or shorter) does not have a connection with a document being relevant. 

Secondly, we look at the \textbf{Urls} and \textbf{Text} dimensions of microblog documents in Figure \ref{urltext}. In the case of \textbf{Urls}, this dimension tends to be significantly larger on relevant documents than in their non-relevant counterparts. This is in line with previous works suggesting that the presence of URL's increases the likelihood for a document to be relevant \cite{massoudi2011incorporating}. Figure \ref{Urls} shows the changes in space dedicated to the URL dimension as we go down the results list. An interesting behaviour that can be observed is that, relevant documents behave in exactly the opposite way to non-relevant documents. As we traverse the results list the space for the URL's in relevant documents increases whereas, it slowly decreases for non-relevant documents.

The \textbf{Text} dimension on the other hand, is significantly smaller for relevant documents, across all cut-offs. However, as observed in Figure \ref{Text}, the behaviour as we traverse the list towards lower cut-off points is similar for both relevant and non-relevant documents. Thus the differences in characters dedicated to this dimension remain stable between relevant and non-relevant documents.

The stability in the differences of both the \textbf{Urls} and \textbf{Text} dimensions make them especially interesting feature candidates to be studied, and possibly employed to improve the behaviour of retrieval systems.

%!TEX root = main.tex



\begin{figure*}
\begin{smaller}
    \centering
       \hspace{-0.5cm}
        \begin{subfigure}[b]{0.4\textwidth}
        	        \centering
                    \begin{tikzpicture}
        	\begin{axis}[
        		height=7cm,
        		width=8.5cm,
        		grid=major,
        	]
        	\addplot coordinates {
        		(10,17.32)
        		(20,17.54)
        		(30,17.13)
        		(50,17.19)
        		(100,17.69)
        	};
        	\addlegendentry{Rel}
        			
        	\addplot coordinates {
        		(10,	14.16)
        		(20,	13.54)
        		(30,	14.28)
        		(50,	14.32)
        		(100,	14.13)
        		
        	};
        	\addlegendentry{Non-Rel}
        
        
        	\end{axis}
        	
        \end{tikzpicture}
        \caption{Urls}
        \label{Urls}
        \end{subfigure}
		~
		\hspace{1.5cm}
        \begin{subfigure}[b]{0.4\textwidth}
        	        \centering
                    \begin{tikzpicture}
        	\begin{axis}[
        		height=7cm,
        		width=8.5cm,
        		grid=major,
        	]
        	\addplot coordinates {
        	(10,77.95)
        	(20,77.37)
        	(30,77.29)
        	(50,77.01)
        	(100,76.77)
        	};
        	\addlegendentry{Rel}
        			
        	\addplot coordinates {
        		(10,	81.77)
        		(20,	82.18)
        		(30,	81.04)
        		(50,	80.01)
        		(100,	80.03)
        	};
        	\addlegendentry{Non-Rel}
        
        
        	\end{axis}
        	
        \end{tikzpicture}
        \caption{Text}
        \label{Text}
        \end{subfigure}
\end{smaller}
\caption{Rate (\%) of space dedicated to Urls and Text in Relevant and Non-Relevant documents at different cut-off points.}
\label{urltext}
\vspace{0.5cm}
\end{figure*}








Figure \ref{hashtagMention} shows the behaviour for the Hash and Mention dimensions. In terms of the \textbf{Hash} dimension, differences are only significant when looking at the @20 cut-off. Additionally, relevant documents seem to have a higher portion of the content dedicated to this dimension than non-relevant documents. This behaviour can be observed in Figure \ref{Hashtags}, as relevant documents seem to dedicate more space for hashtags regardless of the cut-off chosen. Another observation that can be made, is that as we traverse the result list, the presence of hashtags become more pronounced for both relevant and non-relevant documents, thus the increased (or decreased) presence of hashtags does not serve as a discriminative factor in microblog ranking.

Finally, we observe the behaviour of the \textbf{Mention} dimension in Figure \ref{Mentions}. For the three first cut-offs @10; @20 and @30, relevant documents seem to spend more space in defining an audience than non-relevant documents. After the @30 cut-off the roles are swapped and non-relevant documents spend more space in referring to the target users than relevant documents. This makes sense if we assume that many non-relevant documents may be conversational in nature, instead of introducing facts interesting to a wider audience. In fact the differences in terms of the space dedicated to the \textbf{Mentions} dimension is only significant once we are much lower in the ranking at the @100 cut-off.

One could argue that our conclusions may be biased since the result lists are produced with respect to the retrieval model inherent features (e.g. document length). However, we can see that the differences in the observations between relevant and non-relevant documents for the good dimensions (Urls, Text and Hash) are relatively constant, thus independent from the rank for our purposes.

\subsection{Modelling Microblog Informativeness}
In the previous section we observed that relevant Microblog documents present different characteristics to those non-relevant in terms of the aforementioned dimensions (Figure \ref{dimensionaltweet}). More specifically, relevant documents tend to use less space for text, and more space to contain the URLs, and hashtag dimensions than non-relevant documents. An important note is that we cannot assume that the less space dedicated to text the more relevant the document will be, as that would make a text-less document the one with the highest likelihood of being relevant. 

Therefore, we estimate that a relevant document has an optimal amount of space dedicated to the text dimension which ranges from 76\% to 78\% as observed in Figure \ref{Text}. Thus we model informativeness in terms of the retrieval model score \(P(q \cap D|Q)\) for document \(D\) given query \(Q\) and its Text dimension as: 

\begin{figure}[]
	\centering
	\includegraphics[trim = 30mm 175mm 58mm 22mm, clip, width=11.5cm]{kiviat.pdf}
	\caption{Dimensional differences between relevant and non-relevant documents. Statistically significant differences are exaggerated for easier visualization.}
	\label{dimensionaltweet}
\end{figure}

\begin{equation}
P(MI|D,Q) = P(q \cap D|Q) + \lambda [1- \abs{T(D)-0.76}],
\label{eqText}
\end{equation}\\

\noindent where we give a lower score to those documents diverging from the optimal text dimension rate 0.76\footnote{The optimal 76\% rate of presence for the text dimension specified above, which we normalise between 0 and 1.}. We test this formulation using DFR to produce the \(P(q \cap D|Q)\) score over the microblog 2013 collection, which was \textbf{not} used in producing the analysis results in the previous section. We retrieve the first 500 documents using DFR and re-rank them using our first model (Equation \ref{eqText}) with \(\lambda\) set to 1. The results are shown in the RR-text\footnote{``RR-'' stands for ``Re Ranking'', and precedes the features utilised in the operation} row within Table \ref{dimResults}. As we can observe, the performance of DFR is enhanced by taking into account the textual dimension of the microblog documents, being statistically significantly better in terms of P@20.

\begin{figure}

\begin{smaller}
        \begin{subfigure}[b]{0.2\textwidth}
	        \centering
               \begin{tikzpicture}
               	\begin{axis}[
               		height=7cm,
               		width=8.5cm,
               		grid=major,
               	]
               	\addplot coordinates {
              		(10.0,1.96)
               		(20.0,2.626)
               		(30.0,2.514)
               		(50.0,2.82)
               		(100.0,2.638)
               	};
               	\addlegendentry{Rel}           			
               	\addplot coordinates {
               		(10.0,	1.619)
               		(20.0,	1.861)
               		(30.0,	1.999)
               		(50.0,	2.518)
               		(100.0,	2.514)
               	};
               	\addlegendentry{Non-Rel}
               \end{axis}       	
               \end{tikzpicture}
               \caption{Hashtags}

               \label{Hashtags}

        \end{subfigure}%
        ~
        \hspace{4.5cm}
        \begin{subfigure}[b]{0.2\textwidth}
	        \centering
            \begin{tikzpicture}
        	\begin{axis}[
        		height=7cm,
        		width=8.5cm,
        		grid=major,
        	]
        	\addplot coordinates {
        		(10,2.75)
        		(20,2.453)
        		(30,3.061)
        		(50,2.968)
        		(100,2.893)
        	};

        	\addlegendentry{Rel}
        	\addplot coordinates {
        		(10,	2.444)
        		(20,	2.402)
        		(30,	2.671)
        		(50,	3.136)
        		(100,	3.315)
        	};
        	\addlegendentry{Non-Rel}
        	\end{axis}
        \end{tikzpicture}
        \caption{Mentions}
        \label{Mentions}
        \end{subfigure}

\end{smaller}

\caption{Rate (\%) of space dedicated to HashTags and Mentions in Relevant and Non-Relevant documents at different cut-off points.}
\label{hashtagMention}

\end{figure}

Similarly, we combine the URL dimension expressed as a rate with the score of the retrieval model as follows:

\begin{equation}
P(MI|D,Q) = P(q \cap D|Q) + \omega U(D),
\label{eqUrl}
\end{equation}\\

\noindent where we set the free parameter \(\omega\) to 1. The results obtained for the experiments with this model are shown in Table \ref{dimResults} in row RR-Url. The use of the URL dimension on its own also improves the performance over the DFR itself, most significantly for P@10 and P@20. Furthermore, it produces slightly better results than the RR-Text approach. Additionally we combined both models to produce: 

\begin{equation}
P(MI|D,Q) = P(q \cap D|Q) + \lambda [1- \abs{T(D)-0.76}] + \omega U(D),
\label{eqComb}
\end{equation}\\

The results for this combination are shown in Table \ref{dimResults} as row RR-text-url. Further improvements with respect to previous approaches are introduced at all cut-offs except P@10, where RR-url performs slightly better than the combined approach. Finally we also added components to account for the hash and mention dimensions, producing the following two models:

\begin{equation}
\begin{split}
P(MI|D,Q) = P(q \cap D|Q) + \lambda [1- \abs{T(D)-0.76}] \\
 + \omega U(D) + \gamma \#(D),
\end{split}
\label{eqHash}
\end{equation}\\

\begin{equation}
\begin{split}
P(MI|D,Q) = P(q \cap D|Q) + \lambda [1- \abs{T(D)-0.76}] \\
 + \omega U(D) + \gamma \#(D) + \delta @(D),
\end{split}
\label{eqMent}
\end{equation}\\

\noindent where the free parameters are set to 1\footnote{Parameter optimisation would be beneficial in the future, although it was not necessary to evaluate the hypothese of this work}.

The results for both models (Equations \ref{eqHash} and \ref{eqMent}) are shown in Table \ref{dimResults} as RR-text-url-hash and RR-text-url-hash-ment respectively. The performance achieved by adding the hash component over the previous models is further increased specially for P@10, whereas it performs slightly worse than RR-text-url in terms of P@30. The addition of the mentions component in RR-text-url-hash-ment reduces retrieval performance across P@10, P@15 and P@20 with respect to the last model.

If we consider Figures \ref{Urls}, \ref{Text}, \ref{Hashtags} and \ref{Mentions} and Table \ref{dimResults} we can see how the dimensions that showed constant differences across all cut-offs are the features enhancing the performance of the baseline. The only feature which results in poorer retrieval performance is the mentions dimension, which as observed in Figure\ref{Mentions} follows an erratic behaviour (For earlier cut-offs more space is dedicated to the mentions in relevant documents, and then after the cut-off 40 is the opposite case).

\begin{table}[]
\caption{Experimental results when considering different dimensions, using the 2013 TREC Microblog collection (*\(p <0.05 \) over the DFR baseline).}
\centering
\begin{tabular}{|c|c|c|c|c|c|}
\hline Model & P@5 & P@10  & P@15  & P@20  & P@30  \\
\hline
DFR & 0.65 & 0.59 & 0.54 & 0.51 & 0.45 \\
\hline
text & 0.65 & 0.59 & 0.54 & 0.52* & 0.45 \\
url & 0.65 & 0.61* & 0.54 & 0.52* & 0.46 \\
text-url & 0.66* & 0.61* & 0.55* & 0.52* & \textbf{0.47} \\
text-url-hash & \textbf{0.66*} & \textbf{0.62*} & \textbf{0.56*} & \textbf{0.53*} & 0.46 \\
text-url-hash-ment & 0.66* & 0.61* & 0.55 & 0.52* & 0.46 \\
\hline 
\end{tabular} 
\label{dimResults}
\vspace{0.30cm}
\end{table}

Based on our experimental results, we can assert that there are structural differences between relevant and non-relevant documents in terms of the dimensions defined in this work. More specifically, we have come up with a possible instantiation which captures Microblog characteristics in the shape of a model given by Equation \ref{eqHash}. The implications of these findings and experiments are that users produce Microblog documents in different ways, with certain formats more likely to satisfy the information need of a searcher. In the following subsection, we expand our analysis by taking into consideration the order of the dimensions.
%
%%!TEX root = JournalChapter1.tex
\subsection{Dimensions Interaction.}

To further our analysis in the structure of microblog documents we studied how the different dimensions interact with each other. Apart from the presence of the dimensions above discussed, we believe that the order in which they appear, and the interactions between them are also important. In fact, there are several documents on the web \footnote{http://blog.hubspot.com/marketing/tweet-formulas-to-get-you-started-on-twitter} which are meant to assist in writing the perfect tweet to grab the attention of readers.


\subsubsection{State Machine Structure}
To properly model such interactions is no simple task. In our study we utilised all documents in the relevance judgements from the Tweets 2013 collection as our training set. Each tweet is tokenised, and each token is categorised as representing each of the ``text'', ``hashtag'', ``mention'' and ``url'' dimensions, with the help of simple regular expressions matching. Moreover we quantify the frequency that a dimension is followed by another one. For example, we count the number of times when text leads to a hashtag, or a mention leads to a url. The frequencies of each dimensions leading to another dimension of the microblog documents are then utilised to build a simple state machine (or automata). Figure \ref{automataexample} shows an example, denoting how state 1, can transition to other states, such as state 2, with the probabilities stated above the arrows \footnote{ Notice that all transition probabilities for a node add up to 1.}. 



\begin{figure}[h!]
\vspace{0.5cm}
\hspace{3.5cm}
\includegraphics[scale=0.6]{example.png}
\caption{State machine example.}
\label{automataexample}
\vspace{0.5cm}
\end{figure}



Figures \ref{relevantautomata} and \ref{nonrelevantautomata} show state machines for both relevant and non relevant documents respectively. Both these figures contain a node to represent each of the dimensions studied in previous sections. Additionally they contain a ``\textbf{start}'' and ``\textbf{end}'' nodes, to denote the beginning and ending of the microblog document. Consequently, every existing tweet can be characterised by a particular path from the \textbf{start} to the \textbf{end}.



While both figures look very similar, there are some differences that are worth noting. Firstly, looking at the transition from mentions to the end of the document, we can see that the probability for relevant documents is more than double (+21\%) than that for non-relevant documents. This means that relevant documents are more likely to finish mention than non-relevant microblogs.

Likewise the probability of ending a relevant document with a token of text is 12\% less than for non-relevant documents. Moreover the chance of transitioning from a text token to a url token is 13\% higher for relevant documents compared to non-relevant microblogs. Finally the chances to start a document with a mention is half ( 6\% less) for relevant documents with respect to non-relevant ones.

In order to test whether we can use this evidence for producing better rankings, we devised our \textbf{``State''} approach. The State approach is a re-ranking method that linearly combines the score given by any retrieval method with the aggregation of probabilities from start to end nodes w.r.t a microblog's structure.

%!TEX root = main.tex
\begin{figure}
\begin{smaller}

        \begin{subfigure}[b]{\textwidth}
        \vspace{-2.5cm}
        \hspace{-1.5cm}
        \includegraphics[width=17cm]{automatarelevant.pdf}	 
        \vspace{-4cm}  
        \caption{Relevant documents}
        
        \label{relevantautomata}
        \end{subfigure}
		
        \begin{subfigure}[b]{\textwidth}
        \vspace{-2.5cm}
        \hspace{-1.5cm}
        \includegraphics[width=17cm]{automatanonrelevant.pdf}
        \vspace{-4cm}
        \caption{Non-Relevant documents}
        \label{nonrelevantautomata}
        \end{subfigure}      
\end{smaller}

\caption{Tweet automatas for the 2013 collection}
\label{automatas}
\end{figure}




As an example, consider the following tweet: \textit{``Astronomers discover ancient system with five small planets. Details: http://go.nasa.gov/1wCpkJn  @NASAKepler''}. Following the approach described above, we can infer the following structure: ``\([start]->[text]->[url]->[mention]->[end]\)''. If we take the automata for relevant documents (Figure \ref{relevantautomata}) as the source of probabilities it would produce the score: \(0.89 + 0.60 + 0.01 + 0.37 = 1.87\). 



The ``State'' score therefore is given by the following equation:



\begin{equation}
\begin{split}
   State(D,Q) = (1-\alpha)P(q \subset D|Q) \\
   +\alpha * (R\_Score(D) - NR\_Score(D)),
\end{split}
\end{equation}\\



where \(R\_Score(D)\) and \(NR\_Score(D)\) are the scores computed by traversing the automatas in Figures \ref{relevantautomata} and \ref{nonrelevantautomata} respectively and \(\alpha\) is a weighting factor which balances the linear combination. Notice the subtraction of the score given by the automata based on non-relevant documents with respect to the score based on relevant documents. The intuition is that, we want documents that agree with the structure observed for relevant documents, whilst diverging from that of non-relevant documents.







\begin{table}
	\caption{Experimental results for the State retrieval method on the 2011 and 2012 collections. (* \(p<0.05\) and \(\dagger\) \(p<0.01\))}
	\centering
		\begin{tabular}{c|c|c|c|c|c|c}
			& P@5 & P@10 & P@15 & P@20 & P@30 & MAP \\
			\hline
			Baseline & 0.458 & 0.432 & 0.399 & 0.382 & 0.362 & 0.109 \\
			\hline
			State\_0.02 & 0.451 & 0.434 & 0.408 & 0.396* & 0.358  & 0.108 \\
			State\_0.03 & 0.475 & 0.452\(\dagger\) & 0.414* & 0.395* & 0.362  & 0.108  \\
			State\_0.05 & 0.478 & \textbf{0.469\(\dagger\)} & \textbf{0.428\(\dagger\)} & 0.395* & \textbf{0.369}  &\textbf{ 0.110} \\
			State\_0.07 & \textbf{0.481} & 0.454 & 0.416 & \textbf{0.398*} & 0.361 & 0.107 \\
			State\_0.10 & 0.458 & 0.424 & 0.397 & 0.377 & 0.349  & 0.103  \\
			\hline
		\end{tabular} 

	\label{AutomataResults}
	\vspace{0.5cm}
\end{table}





Table \ref{AutomataResults} shows the retrieval results for our re-ranking approach over the 2011 and 2012 collections. P@5 to P@30 represents Precision at the different cut-off points, whereas MAP denotes Mean Average Precision at cut-off 30. The first column contains the model being evaluated. Baseline represents a simple retrieval run using DFR only for ranking, whereas ``State\_n'' contain the results for our ``State'' approach with different values of \(\alpha\). 



As we can observe, retrieval effectiveness is improved significantly for a number of measures. Specifically the ``State\_0.05'' configuration achieved a \(p\) value below 0.01 for both P@10 and P@15. We can see how the most prominent improvements are achieved at the top cut-off points. This result suggests that taking into consideration the structure of documents, helps in bringing more relevant documents to the very first few documents, which is a highly desirable product due to the fast-paced environment that is microblog search.



%\subsubsection{Tree Structures}
%Whilst the state machine managed to effectively capture the tweets structure as evidenced by the statistically significantly improved retrieval performance reported in \cite{rodriguezPerez}, there are a number of issues. By design the state machine model is only aware of how \textbf{A} can transition to \textbf{B}, and previous transitions have no effect in the model itself. In other words, the fact that text following a mention is in the middle of the tweet or near the end, has no effect in the edge values. Furthermore, the values on an edge between \textbf{B} and \textbf{C} are not affected by a previous transition \textbf{A} to \textbf{B}.
%
%This limitation motivated the design of \textbf{TwTree} short for ``Tweet Tree''. As the name suggests, we decided to encode the structure of tweets in a tree-like structure. Our intuition is that by utilising a tree-like structure we may represent such sequences of elements in full, thus achieving a higher level of granularity. 
%
%%Moreover we build a tree for relevant tweets and another for non-relevant tweets. 
%Similarly to the work by \cite{rodriguezPerez} we tokenize the tweets thus decomposing them into their building blocks: text, mentions, hashtags and urls. We then specify the root of the tree and the ``start'' of the tweet. All tweets from either the relevant or non-relevant set are then added to the tree one element at a time in order thus becoming a path from the root of the tree specified as ``start'' and ending in the leaf node specified as ``end''. Every time we add a new tweet to our structure we note the number of times that such structure has occurred as \(t\). Concurrently, we keep the total number of tweets \(T\) that have been employed to build the tree at the root node ``start''.
%
%\begin{figure}[h!]
%\centering
%\includegraphics[scale=0.50]{treeStructure.png}
%\caption{Tree structure to represent sets of tweets.}
%\label{treeExample}
%\end{figure}
%
%
%Figure \ref{treeExample} shows a simplified example of such tree structure. Let's assume we have a tweets which structure is defined as: 
%
%\begin{equation}
%[start]->[text]->[mention]->[end]
%\label{sequence}
%\end{equation}
%
%\noindent the score is computed by traversing the tree following the sequence \ref{sequence} and extracting the \(t\) value at the leaf, and dividing by the total number of tweets that have been used to create the model \(T\). Thus the score is simply obtained by \(t/T\) and it represents how common is this structure amongst all the tweets that have been used to form the model. Notice that the score will always be a normalised value between 0 and 1, which simplifies later computations.
%
%Similarly to the state machine based system created by \cite{rodriguezPerez} we will create a tree representing relevant documents and another for non-relevant documents. Consequently we may produce an score from each model R\_TwScore(D) and NR\_TwScored(D) for any given document \(D\). Finally the scores are to be merged as follows:
%
%\begin{equation}
%merge(D) = \frac{R\_TwScore(D)}{R\_TwScore(D) + NR\_TwScore(D)}
%\label{merge}
%\end{equation}
%
%\noindent This equation always returns a value between 0 and 1. The closer the value is to 1 the closer the structure of document \(d\) resembles the relevant set of tweets and vice-versa. As an example consider:
%
%\begin{equation}
%\begin{split}
%R\_TwScore(D)= 0.75 \\
%NR\_TwScore(D) = 0.60
%\end{split}
%\end{equation}
%
%Consequently the score obtained by $merge(D)$ will be 0.55, which means tweet $D$ resembles a bit more the set of relevant tweets than the non-relevant set.
%
%Finally, the score \(TwTree(D)\) is given by its linear combination with a retrieval model \(P(q \subset D|Q)\) as follows:
%
%\begin{equation}
%\begin{split}
%   TwTree(D,Q) = (1-\alpha)P(q \subset D|Q) \\
%   +\alpha * merge(D),
%\end{split}
%\end{equation}\\
%
%
%\noindent{\bf Loose sequence matching. } As an additional note, there is a possibility that a particular sequence may not yet exist within a TwTree model. To address this problem, we perform a loose matching as we consider that returning the score for a closely related structure is better than not returning anything at all. 
%
%As an example, if we look up the following sequence:
%
%\begin{equation}
%[start]->[text]->[mention]->[url]->[text]->[end]
%\end{equation}
%
%but only the following sequence is avaiable: 
%
%\begin{equation}
%[start]->[text]->[mention]->[url]->[end]
%\end{equation}
%
%then the score associated with the later sequence will be returned, and the last text element in the former sequence will be ignored.
%
%%design
%%
%%scores computation 
%%
%%scores merging
%%
%%TreeTweet score linear combination
%
%In the following section we will introduce our experimental setting which precedes our discussions section where we present our findings.
%
%
%
%
%\mentalnote{ADD RESULTS AND DISCUSSION for TREE-STRUCTURE}
%


We can conclude from these experiments that the structure of tweets can be extracted and leveraged to produce better rankings. We can confirm that not only it is the relative space in terms of characters dedicated to each dimension that links to relevance, but also how these dimensions relate to each other within the document.



\subsection{Additional notes}

The simplicity of the state modelling allows for it to be conveniently stored and re-used in real-time. The states are stored as a set of precomputed heuristics which include the dimensions in the transition and its associated probability based on the observed data. The model itself should be updated from time to time to accommodate any shifting in the structuring and style of micro-bloggers.



%!TEX root = JournalChapter1.tex
\section{Conclusions}

\label{conclusion}

In this work, we verified whether the scope and verbosity hypotheses still hold for microblog document retrieval. We initially hypothesise that since microblog documents have a character limit the scope and verbosity hypotheses could not hold, as it is assumed that the author of the document is able to produce documents of any length. 

We then proceeded to analyse the behaviour of a number of state of the art retrieval models. The chosen models were BM25, HLM, DLM, DFRee and IDF. Our experimentation led to a better understanding of what could be the shortcomings experienced by such models under microblog retrieval constraints. Particularly, we isolated the fact that longer documents should be promoted to account for the effort of microblog authors to encode their messages into the character limit. Then we identified that higher term frequencies than 1-2 should be penalised as they are more likely to be less informative and more reminiscent of spam documents. Based on these observations we concluded that the scope hypotheses does still hold in microblog documents, however verbosity does not.

Finally we built a retrieval model optimised for microblog retrieval, namely MBRM, which takes intro account the observations extracted from the experimentation with aforementioned retrieval models. Our evaluation results demonstrate how MBRM significantly outperforms the best baselines (IDF and DFRee), by making better use of document-encoded evidence.

Future work will demonstrate how MBRM can be used to push further the current performance of approaches that rely on the initial results such as Automatic Query Expansion.


% Bibliography
\bibliographystyle{ACM-Reference-Format-Journals}
\bibliography{bibtexshort}
                             
\received{x}{y}{z}


\end{document}
