%!TEX root = JournalChapter1.tex
\section{Introduction}
\label{introduction}

Microblogs have grown in popularity in recent years, gradually transforming the way we find out about the latest events and communicate. Twitter is the most prominent service \footnote{\url{https://twitter.com/}}, as it is used by millions, posting over 500 million tweets every day\footnote{\url{https://www.dsayce.com/social-media/tweets-day/}}. Microblog services are used for various purposes including: (i) self promotion, (ii) advertising, (iii) real-time news broadcasting, (iv) social discussions etc. It provides unique insight into real-time events, such as first hand reports of events as they are developing, along with the opinion of those discussing them. This information makes Twitter a uniquely valuable media source, and hence obtaining the attention from the research and industrial communities. 

Searching through such resources is performed by millions on a daily basis, however retrieving documents in Twitter can be extremely challenging because of their morphology. The content is limited to 140 characters per messages (known as \emph{Tweets}), which in fact leads to varied linguistic quality \cite{teevan2011twittersearch} due to colloquialisms and users efforts to fit their content within the limitations. Due to these properties tweets pose new challenges for which the state of the art retrieval models were not designed for\footnote{Models such as: Okapi BM25 \cite{robertson2009probabilistic}; Divergence From Randomness (DFR) \cite{amati2003probabilistic}; Hiemstra's Language Model (HLM) \cite{model}; and Dirichlet Language Model (DLM) \cite{zhai2001study}}. 

Whilst few recent works have identified some features as being detrimental in microblog ad-hoc retrieval \cite{naveed2011searching}, no study has been carried out to determine the concrete effect of the different features on the state of the art retrieval models. Therefore we are set to investigate the connection of the structure of microblog documents with their relevance in an ad-hoc search task. This whole work revolves around the following main question: 

\begin{quotation}\begin{quote}\textbf{What are the reasons behind the underperformance of the state of the art retrieval models in the context of microblogs?}\end{quote}\end{quotation}

To this end, firstly we observe the performance of the state of the art retrieval models in the context of Twitter corpora selecting the best retrieval model as a baseline. Then we perform a series of experiments which simulate the behaviour of a number of state of the art retrieval models in order to identify possible shortcomings in their design with respect to microblog documents. This initial experiment is completed with the creation of a retrieval model, which takes into account all previous findings, namely MBRM. MBRM demonstrates that the scope hypotheses still holds within microblog documents, and that microblog document statistics can be leveraged to significantly improve ad-hoc retrieval performance. This work will be driven by the following research questions:

\begin{itemize}
\item[] \textbf{RQ1.} What is the role of document length in connection with the informativeness of microblog documents?
\item[] \textbf{RQ2.} Does term frequency of query terms relate to the informativeness of microblogs? 
\item[] \textbf{RQ3.} Can we adapt state of the art retrieval models to better handle microblogs?
\item[] \textbf{RQ4.} Can we devise a retrieval model to better capture the relevance of microblogs?
\end{itemize}

The rest of the work is organised as follows. First, we cover literature relevant to microblog retrieval and other concepts utilised throughout this work (Section \ref{background}). Section \ref{experiment} sets the evaluation environment in which our experimentation is carried out, giving way to our analysis (Section \ref{RMinvestigation}). Section \ref{MBRM-section} then introduces our MBRM retrieval model, whereas Section \ref{conclusion} concludes the work and points to future research directions.

